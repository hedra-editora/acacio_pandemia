
\chapter{Prefácio}

O novo coronavírus chegou ao Brasil em fevereiro de 2020, anunciado pela
Organização Mundial da Saúde (\textsc{oms}), governos nacionais, mídias, redes
digitais e os passageiros desembarcados de voos internacionais. Veio
acompanhando uma nova doença, a \textsc{covid}"-19, e as recomendações sanitárias
para o combate à chamada pandemia.

Nós do Nu"-Sol (Núcleo de Sociabilidade Libertária), da Soma, uma Terapia
Anarquista, e o raro José Maria Carvalho Ferreira, vivemos muito
próximos desde 1992, quando aconteceu o encontro libertário \emph{Outros
500}, em São Paulo. Em uma conversa pela internet decidimos produzir uma
página libertária conjunta com nossos pontos de vista. Nela registramos
como ``contestamos a realidade capitalista, o Estado e seus modos
contínuos de governar em função de um reordenamento normalizador'' e
afirmamos ``mais uma luta libertária no presente''\footnote{Ver \textsc{urgentes}
  em \emph{http://www.somaterapia.com.br/atividades/urgentes/} e
  \emph{https://www.nu-sol.org/wp-content/uploads/2020/07/dossie-urgentes.pdf}}.
Em seguida veio a possibilidade de organização de um livro com
libertários interessados e suas análises inéditas, redigidas até o
início de outubro de 2020.

O livro \emph{Pandemia e Anarquia} está composto de 15 capítulos
encadeados em ordem alfabética, sugerindo ao leitor tanto a leitura
segundo seu interesse, como a sequencial com as surpreendentes
implicações trazidas pelo acaso, mostrando e realçando adjacências,
conexões, complementações e diferenças. É o que buscamos com a anarquia,
uma produção interminável de práticas de liberdade contra o absoluto e o
definitivamente verdadeiro que jazem nos discursos realistas ou
utópicos.

Muitos artigos a nós endereçados não constam desta coletânea, mas
estarão em outras publicações, ou até mesmo em possível livro sobre 2021
e esta situação extraordinária tornada ordinária de suposta reescritura
da normalidade. Para nós, a normalização dos normais já ocorria nesta
sociedade"-Estado capitalista, computo"-informacional e cheia de \textsc{tic}s, que
se pretende restauradora do planeta com suas receitas de \emph{melhorias
sustentáveis}, invocando condutas resilientes.

Nossa perspectiva é a dos resistentes na invenção cotidiana de uma
realidade sem a organização hierarquizada da autoridade em um planeta
intenso de práticas de liberdade anarquistas, ultrapassando os
ativismos, essa insistente e obsessiva maneira de mirar a renovação e a
inovação, democrática ou não, produzidas pela racionalidade neoliberal.

\chapterspecial{Apoio mútuo ou ajuda humanitária?}{Notas sobre o militantismo
anarquista\\ em meio à crise planetária}{Acácio Augusto}

\epigraph{A vida é perigosa e exige que encaremos a sociedade que está
morrendo. Escancaremos as potências da vida destemida e avessa a
castigos, fronteiras, comunidades, normalizações, produtividades,
política, polícias\ldots{} A destruição de algo é sempre a construção de
uma invenção. É inventar um espaço outro, uma vida outra. Viver a vida é
não servir a nada nem a ninguém.}{\textsc{nu"-sol}, 2020}

\noindent{}O surgimento do que foi classificado pela Organização Mundial da Saúde
(\textsc{oms}) como pandemia, em março de 2020, foi imediatamente recebido e
difundido pelas autoridades estatais e organizações internacionais como
uma guerra. A começar por manifestações do Secretário Geral das Nações
Unidas (\textsc{onu}), António Guterres, que declarou guerra ao vírus. A
disseminação planetária das infecções provocadas pelo novo coronavírus
mobilizou, de forma inédita, um aparato de controle social e contenção
de danos pessoais que reunia desde profissionais de saúde, passando por
uma série de recomendações de novas condutas por especialistas
midiatizados, até a atuação ostensiva das forças de segurança, das
forças armadas, das polícias ou do regime dos ilegalismos.\footnote{Sobre
  a recepção da chamada pandemia como uma guerra, ver: \textsc{augusto}, 2020.}
Soma"-se a essa mobilização uma produção discursiva e comunicacional que,
além de buscar encaixar, definitivamente, toda vida social nos fluxos
computo"-informacionais, se precipitou em anúncios variados de uma ``nova
era'' que se abriria após a pandemia. Anúncios por vezes pessimistas e
catastróficos, a partir de problemas que já estavam postos para os
viventes no planeta e que seriam ampliados pelas medidas de contenção da
\textsc{covid}"-19. E anúncios, em direção oposta, que projetaram a possibilidade
de um futuro idílico ou utópico, esperançoso nas mudanças de percepção
das pessoas a partir do alerta trazido pelo novo coronavírus em relação
às questões ambientais, à vida social e econômica nas grandes cidades e
à vida no planeta Terra de forma geral. Essas duas projeções se apoiam
na ideia de que a situação imposta pela disseminação das contaminações e
infecções, segundo variados especialistas, tende a se repetir e/ou se
estenderá por mais tempo do que o esperado pelas pessoas e até mesmo por
autoridades médicas e governamentais.

Entre os anarquistas,\footnote{Para análises singulares a partir do
  território brasileiro e de uma perspectiva anarquista, ver: \textsc{nu}"-\textsc{sol},
  2020.} desde o primeiro momento, produziram"-se análises e intervenções
diversas, que tinham um ponto mais ou menos em comum: a situação
desencadeada pela declaração da pandemia e as formas de lidar com ela
foram produzidas pelas vidas criadas pelo capitalismo e o
Estado.\footnote{Para análises anarquistas de diversas partes do planeta
  sobre a chamada pandemia e lutas relacionadas a ela, ver: \textsc{facção}
  \textsc{fictícia}, 2020.} Da mesma maneira, a condução das soluções e a própria
gestão da crise"-guerra se orientam pela manutenção dessa forma de vida e
a sobrevivência desse mesmo capitalismo e da forma Estado como governo
dos viventes e distribuidor de mortes. Sete meses após a instalação
planetária da doença, não é exagero dizer que nada parou e que pouca
coisa mudou, ou seja, até o momento estamos sob a gestão dessa
sobrevivência e variados ``palpites'' sobre o que se colocava caíram por
terra. Diante da normalidade assassina do capitalismo planetário que se
ocupa da ``segurança do vivo no planeta'', uma normalização do
normal\footnote{Cf. \textsc{passetti} et al., 2019.} em pouco tempo se instalou
recebendo o título de ``nova normalização'', um mais do mesmo, mas com
esperanças de melhora após se alcançar o controle das infecções por meio
da vacina ou de algum meio de imunização do rebanho de humanos.

Em março de 2020, quando as mortes e infecções atingiam altos números no
continente europeu e começavam a rumar para o hemisfério sul, os
militantes anarquistas da Federação Anarquista de Turim alertavam, de
forma direta, que a chamada epidemia (se referiam ao território
italiano) era, de fato, um massacre de Estado. Como anarquistas,
afirmaram a recusa em aderir à morte gerida pelos que governam. ``Não.
Não estamos `prontos para morrer'. Não queremos morrer e não queremos
que ninguém adoeça e morra. Não estamos nos alistando na infantaria
destinada ao massacre silencioso. Somos desertores, rebeldes,
\emph{partisanos}'' (\textsc{federação}, 03/2020). A recusa das formas de gestão
pela crise"-guerra e da distribuição racional das mortes, entre escolhas
de quem deveria viver e quem poderia morrer, era uma afirmação da luta
social e da vida fora da normalidade capitalista. O texto da Federação
de Turim ainda alertava para o temor dos governantes em se disseminar um
ciclo de lutas e revoltas, como as desencadeadas pelas rebeliões nas
prisões italianas.

Em abril, no território espanhol, o anarquista Tomás Ibáñez escreve
sobre uma outra recusa. ``Esta crise também nos chama a dizer NÃO à
autoilusão praticada por um amplo setor desse espectro revolucionário e
antiautoritário no qual me encontro. Este autoengano consiste em
acreditar, e nos fazer acreditar, que o capitalismo pode ser atingido
até a morte pela crise e que a pandemia dará origem a um intenso ciclo
de lutas capazes de transformar o planeta e que, ao final, as classes
populares perceberão nitidamente a necessidade de virar as costas ao
sistema. E esta é a nossa vez de contribuir para dar o golpe final em um
capitalismo moribundo'' (\textsc{ibáñez}, 2020). Essa outra recusa se refere ao
fatalismo da crise como janela de conveniência para afirmar, ao
contrário, a multiplicidade das lutas no presente, segundo as
contingências que sempre se colocam, com ou sem pandemia. Contra uma
teleologia revolucionária que quer ver nos momentos de crise uma
oportunidade para agir e levar as classes populares ao paraíso, Ibáñez
lembra que, para um anarquista, o que existe são as lutas, sempre
múltiplas e sem pretensão de unidade, a serem travadas aqui e agora.

Esse breve escrito parte dessas duas recusas para analisar como algumas
associações anarquistas lidaram com a intensificação da crise"-guerra
planetária devido à emergência da \textsc{covid}"-19. Nos textos de análise e
relatos de intervenções, instala"-se uma tensão entre releituras das
práticas de apoio mútuo, a partir de Kropotkin, como forma política de
resistência e uma aposta em como a situação confirma críticas já
colocadas pelos anarquistas, combinada com recomendações de autocuidado.
Na chave de reafirmação do apoio mútuo, algumas análises acabam
esbarrando em práticas de gestão da crise, aos moldes das ajudas
humanitárias, em contraste com afirmações da vida que não querem se
entregar à gestão planetária dos viventes e à corrida pela sobrevida.

Em meio às disputas políticas e investidas sanitário"-securitárias dos
governos, as infecções e as mortes se acumulam na escala dos milhões.
Recusar a leitura normalizante e estatal não significa adotar uma
conduta negacionista em relação à \textsc{covid}"-19. Mas essa recusa também evita
abraçar a defesa da (sobre)vida como dado biológico e se diluir no
consenso democrático de gestão da crise"-guerra capitalista e estatal.
Diante de um consenso emergente de defesa da vida a qualquer custo, cabe
questionar qual vida se quer viver, e lembrar que a vida não se resume a
um fato biológico quantificável e mensurável.

Diante de tanta política, tanta ciência e tanta comunicação, essa
análise se referencia na crítica de Mikhail Bakunin ao governo da
ciência que objetiva tosquiar os rebanhos populares em todo o planeta,
busca"-se, aqui, ``a revolta da vida contra a ciência, ou melhor, contra
o governo da ciência'' (\textsc{bakunin}, 2000, p. 15). Em tempos de uma
emergência, declarada como crise"-guerra sanitária"-securitária, acreditar
na disputa complementar entre negacionismo e defesa da ciência médica é
entregar o próprio couro para ser tosquiado pelos governos planetários,
institucionalizados ou não. A partir dessa mirada, interessa a uma
atitude anarquista não só reafirmar as duas recusas expostas acima, mas
também se colocar no mundo \emph{contra} mundo.

\section{apoio mútuo: onde está a revolta?}

Em todo o planeta, as condutas diante da chamada pandemia foram
discursivamente organizadas entre ``negacionistas'', que não acreditavam
nem seguiam as regras da \textsc{oms} e das diversas juntas médicas nacionais,
negando"-se a adotar cuidados mínimos ou diminuindo a gravidade da
doença; e ``salvacionistas'', que, em nome da não contaminação e como
única forma de conter infecções, abraçaram com fé as inúmeras
recomendações de especialistas e de autoridades, como a \textsc{oms}, chegando a
defender decretos de estado de sítio (chamado de \emph{lockdown})
garantido por forças de segurança, militares e policiais, para
efetivação autoritária do isolamento social. Essa oposição, assim
exposta, pode soar exagero, mas foi a partir de uma imagem próxima a
essa que se traduziram politicamente as condutas entre
``negacionistas'', à direita ou extrema direita, e ``salvacionistas'',
progressistas ou à esquerda. Embora essas condutas sejam mais
complementares do que se pode supor num primeiro momento, foi no Brasil
que essa disputa ganhou contornos mais evidentes, confundindo"-se com a
situação político"-institucional do país.

Ao largo ou à margem dessa imagem de disputa midiatizada e
institucionalizada pelo melhor governo da crise"-guerra, animada por
ativistas de cada parte, surgiram discursos e intepretações sobre ações
locais de enfretamento das contaminações e infecções. Essas ações, ora
estavam associadas à histórica proposição anarquista de apoio mútuo e às
práticas de autocuidado desvinculadas de recomendações e ações de
governos e autoridades médicas, ora eram relatadas como organização de
comunidades negligenciadas pelos governos de Estado ``negacionistas''
que se uniam para travar a própria luta contra a \textsc{covid}"-19. Mantendo a
metáfora da guerra contra o vírus, essas ações comunitárias são lidas
como uma guerrilha dos moradores de bairros pobres e prova da capacidade
de resiliência dessas pessoas, capacidade acumulada em toda uma vida de
privações e adversidades.\footnote{Essa interpretação foi amplamente
  veiculada na imprensa brasileira e chegou a ganhar destaque na
  imprensa internacional, como no \emph{Washington Post}. Cf. \textsc{lopes},
  10/06/2020.}

As práticas de apoio mútuo entre anarquistas, para além da extensa
elaboração de Piotr Kropotkin em livro que leva o mesmo nome (1989),
podem ser compreendidas em dois sentidos: a) como um impulso ``natural''
e/ou ``instintivo'' de colaboração entre os viventes que garante a
sobrevivência do conjunto;\footnote{Nesse sentido, fiel às elaborações
  do anarquista russo, ``o instinto de sociabilidade que se desenvolveu
  lentamente entre os animais e entre os homens no transcurso de um
  período de evolução extremamente longo, desde os estágios mais
  elementares, ensinou, igualmente, muitos animais e homens a ter
  consciência dessa força que adquirem praticando a ajuda e o apoio
  mútuos, também por ter consciência do prazer que podem encontrar na
  vida social'' (\textsc{kropotkin}, 1989, p. 32). (Minha tradução do espanhol).}
b) como ação deliberada para interferir e interromper as cadeias
hierárquicas de competição, características das relações de produção no
capitalismo.\footnote{Esse sentido aparece nas intepretações
  contemporâneas que agregam esse fator de inciativa deliberada ao
  determinismo natural argumentado por Kropotkin. Ver, nesse sentido,
  \textsc{grubacic} \& \textsc{graeber}, 2020.} Essas práticas também são invocadas em
momentos de lutas entre os ``mais fracos'' que se associam para resistir
aos ataques de um oponente mais poderoso, como eram os fundos de greve
no século \textsc{xix} e começo do século \textsc{xx}, por exemplo. Nesse sentido, é
evidente que seu uso corrente, seja para nomear uma prática, seja para
declarar afinidade com uma determinada forma de prática anarquista,
ultrapassou a referência às elaborações de Kropotkin, mas sem apagá"-las
ou refutá"-las.

No interior da história política das lutas libertárias, o apoio mútuo,
reivindicado sob esse nome inicialmente pelos anarco"-comunistas, pode
ser compreendido numa sequência de diferenciação do que foi o
coletivismo bakuninista, mais focado nas relações de produção modernas.
Há, também, uma associação mais comum do apoio mútuo à fórmula da
Associação Internacional dos Trabalhadores (\textsc{ait}, 1864): ``para cada um
de acordo com suas necessidades, de cada um de acordo com suas
possibilidades''. Nesse sentido, reitera"-se a filiação ao
anarco"-comunismo e, para além desse sentido, há intepretações que
associam as proposições de apoio mútuo com a autogestão, uma aproximação
mais controversa. Especialmente quando se restringe o apoio mútuo não a
uma prática exclusiva do campo das relações de produção, mas de
realização de qualquer atividade entre pessoas associadas, como a
supracitada interpretação recente de Grubacic e Graeber. A partir de
Kropotkin também se argumenta, nesta interpretação, que o apoio mútuo é
uma orientação ética baseada na liberdade e no antiautoritarismo.
Contudo, deve"-se notar que a argumentação a partir da determinação
biológica que o conceito pressupõe e mobiliza afasta"-o de uma ética
libertária, para além da formulação anarco"-comunista. Todas essas
possibilidades tratam de práticas voltadas à transformação do mundo e
das pessoas envolvidas, mas, ao manterem uma perspectiva teleológica de
projeção de um futuro pós"-revolução, elas nem sempre implicam
transformação de si em associação ou afirmam uma prática libertária no
presente. Diferente da determinação pela história da biologia, uma ética
libertária implica relações livres com generosidade e reciprocidade,
ignorando, também, os cálculos de proporcionalidade e
equivalência.\footnote{Para uma análise mais ampla da ética libertária,
  que se aparta de formulações baseadas no determinismo biológico do
  apoio mútuo, ver: \textsc{passetti}, 2003.}

No entanto, ainda que haja diferentes usos e interpretações sobre o
apoio mútuo entre os anarquistas e, ao final, esta seja uma noção
própria dos anarco"-comunistas, nota"-se que há uma confusão no que é
nomeado como apoio mútuo hoje, seja em práticas diretamente vinculadas
às associações anarquistas, seja em ações nomeadas a partir da
referência à elaboração de Piotr Kropotkin. Um exemplo seria o breve
texto da jornalista Zoe Smith (02/06/2020) sobre ações comunitárias
durante a chamada pandemia, tomando alguns exemplos retirados da
Argentina. Ela parte precisamente de Kropotkin, para aproximar suas
descrições a uma ``forma anarquista de se organizar''. Contudo, muitas
das ações tomadas como exemplo se confundem com ajuda humanitária, como
ações de negócios sociais ao estilo dos Médicos Sem Fronteiras, ou
simplesmente soluções emergenciais encontradas por pessoas passando por
dificuldades materiais. Para se diferenciar desse tipo de atuação de
\textsc{ong}s, usa"-se a justificativa da ``ajuda mútua'' pelo fato de serem ações
efetivadas por membros de uma mesma comunidade, mas não se questiona
sobre a capacidade de transformação de tais iniciativas, que muitas
vezes apenas reiteram a condição de servidão e pauperismo dos habitantes
dessas comunidades. Não se trata de juízo de valor sobre as ações,
necessárias do ponto de vista da sobrevivência, mas cabe se perguntar
sobre seus efeitos de manutenção da ordem e perpetuação dessa
(sobre)vida, mesmo sob condições excepcionais.

As ações nas comunidades argentinas usadas por Smith como exemplos de
apoio mútuo e organização anarquista poderiam ser comparadas com ações
em favelas no Brasil. Uma reportagem publicada no \emph{El País Brasil}
sobre as ações na favela de Paraisópolis, em São Paulo, em meio à
chamada pandemia, embora não se refira ao apoio mútuo, celebra a
``auto"-organização'' dos moradores diante do ``abandono'' das
autoridades governamentais (\textsc{gortázar}, 04/10/2020). Seguindo o léxico de
guerra ao vírus e repondo a dicotomia entre negacionismo negligente do
governo diante de uma realidade inescapável, a solução, segundo a
reportagem, vem por meio da atuação de ``ativistas de bairro e pequenos
empresários locais''. Assim, contam as histórias de pessoas como a
``presidente de rua'', Isabel, e do presidente da ``União dos Moradores
e do Comércio de Paraisópolis'', Gilson Rodrigues. Eles se empenham em
distribuir cestas básicas, álcool em gel, cuidados médicos e
disseminação de informações sobre os riscos da nova doença, já que ``a
primeira batalha que os ativistas da favela tiveram de travar foi contra
a falsa crença de que os pobres estavam a salvo do coronavírus'' (Idem).
Assim, as ações apenas se revelam como contingências em favor da ordem e
da (sobre)vida como gestão da crise"-guerra e projeção de lideranças
locais por meio dos chamados negócios sociais, sem efeito de
transformação na vida das pessoas, que são projetadas por meio do
reconhecimento de suas vulnerabilidades como alvo das ações
assistenciais de outros moradores do mesmo bairro, muito mais uma gestão
compartilhada que um apoio mútuo.

Mesmo com algumas diferenças em relação às ações descritas por Smith,
como a associação com as premissas de apoio mútuo, são ações bem
semelhantes, como nesse comentário que ela faz: ``é apenas um exemplo
entre milhares de atos de compaixão, solidariedade e cooperação
voluntárias que vêm ganhando as manchetes em todo o mundo. Esta onda de
atividades --- caem sob a bandeira da `ajuda mútua' {[}\emph{mutual
aid}{]} porque vem de dentro das próprias comunidades e é voltada a
longo prazo, como Barrios de Pie colocou, à justiça social e à
transformação social --- em muitos casos, ultrapassou tentativas de
voluntariado lideradas pelo Estado'' (\textsc{smith}, 02/06/2020). De um lado,
seria possível argumentar que há imprecisão de Smith ao associar esse
tipo de ação comunitária às práticas anarquistas de apoio mútuo; de
outro lado, o simples fato de essa associação existir revela um problema
nas tentativas de autores e associações anarquistas contemporâneas em
atualizar o conceito proposto em 1902, por Kropotkin. Afinal, sob a
contemporânea racionalidade neoliberal,\footnote{Cf. \textsc{foucault}, 2008.}
cooperação voluntária, solidariedade social e compaixão cívica são
formas características do empreendedorismo de si em seu formato de
negócios sociais. Assim, quando relacionadas ao apoio mútuo,
independentemente das intencionalidades, as ações propriamente
anarquistas ora se confundem com esses negócios sociais, ora são
tragadas por disputas políticas de território com organizações,
governamentais e não governamentais, financiadas por agentes de mercado
ou subsidiadas por políticas sociais individualizadas. Nessas disputas,
os anarquistas podem ser tragados pela oposição complementar,
politicamente orientada, entre ``negacionistas'' e ``salvacionistas''.

Um exemplo um pouco diferente de ações anarquistas é a plataforma
brasileira anônima nomeada precisamente de ``Apoio Mútuo''. Segundo o
site, ``apoio mútuo é uma iniciativa que tem o objetivo de compartilhar
ferramentas e ampliar as redes de solidariedade entre as pessoas que são
divididas e classificadas por longas cadeias de opressão e violência.
Por isso, queremos incentivar e fornecer mecanismos de apoio a ações que
conectam demandas ao fortalecimento de pessoas, grupos, coletivos e
organizações que têm em comum princípios de inspirações anárquicas e
anarquistas''\footnote{Cf. \emph{https://apoiomutuo.com.br/sobre/}.
  Acesso em: 18/10/2020.}. No site são encontradas ações muito
diferentes entre si, muitas delas com formato meramente assistencial,
como distribuição de comida e insumos de proteção pessoal. Ainda que
menos ambígua, no que diz respeito à vinculação das ações aos
anarquismos que o texto de Smith, a mobilização do conceito de apoio
mútuo também se mostra problemática. Em primeiro lugar, cabe questionar
sobre qual seria a diferença de uma plataforma de trocas de experiências
declaradamente anarquista como esta e uma rede de ajuda humanitária ou
de solidariedade, como as que atuam nas favelas relatadas na reportagem
do \emph{El País Brasil}. Seria a plataforma digital um \emph{think
tank} solidário que funciona à margem dos grandes negócios sociais? Se
for isso, não há como escapar de ser tragado ou mesmo neutralizado pelos
negócios sociais que possuem mais capacidade de alcance (leia"-se: mais
dinheiro e logística de atuação) entre os denominados ``vulneráveis'', o
público"-alvo das assistências. Ao se analisar um pouco mais
demoradamente quais os possíveis efeitos de transformação dessas
práticas, as ações compiladas na plataforma ``Apoio Mútuo'' mostram"-se
frágeis diante do extenso investimento em negócios sociais do
empreendedorismo neoliberal hoje.

Nota"-se, portanto, haver uma disputa a partir da noção e do emprego da
expressão ``apoio mútuo''. De um lado, busca"-se ler as ações
emergenciais de solidariedade entre os mais fracos, mais pobres e vistos
como vulneráveis, como prova de que o governo e as instituições estatais
não são necessários e não atendem, propositalmente, às necessidades e
aos interesses das pessoas. Seria no mínimo precipitado ver isso como
algo próximo dos anarquismos, mesmo que vagamente. De outro lado,
busca"-se promover ações de solidariedade que funcionem como uma espécie
de ``propaganda pela ação''\footnote{Essa é a argumentação que perpassa
  algumas ações relatadas na plataforma ``Apoio Mútuo''.}, mostrando, às
pessoas que estão mais expostas às assimetrias de poder e às
desigualdades sociais e econômicas, que apenas as ações construídas por
elas mesmas, entre elas e sem interferência de governos e empresas, vão
efetivamente produzir um resultado satisfatório, especialmente em meio
às crises"-guerras, como a que se impôs com a declaração da pandemia.
Esse entendimento, de forma um pouco diversa, também é encontrado em
outros campos não vinculados aos anarquismos, em geral traduzido na
expressão ``nóis por nóis'', muito usada por coletivos de periferia ou
ações de \textsc{ong}s e Fundações voltadas para esses territórios. Bom, se, como
exposto, essas ações estão atravessadas por engajamentos ativistas
afeitos à racionalidade neoliberal, evidencia"-se um problema para uma
plataforma que visa a compilar ações anarquistas de apoio mútuo. O que
se nota, nos relatos veiculados pela ``Apoio Mútuo'', é a tentativa de
disputar politicamente os sentidos dessas ações, traduzindo"-as como
formas ou mesmo ``provas'' da efetividade do apoio mútuo como prática
identificada com o anarquismo. Nessa disputa política, como já foi
anotado, a tendência é a neutralização.

A entrada nessa disputa parece ignorar o caráter cooperativista do
capitalismo e os elementos de fobia ao Estado do neoliberalismo que
habitam as formas de empreendedorismo e se traduzem como eficiência e
eficácia de ações autônomas contidas nos negócios sociais. Nesse caso,
os efeitos de transformação radical de si, dos outros e do território em
que se está agindo ficam bloqueados ou mesmo são diversamente modulados,
ao passo que as ações dos coletivos ou associações que se declaram
anarquistas ficam disponíveis às capturas.

Em um tempo no qual a racionalidade neoliberal produz liberdades
cercadas pelas modulações de segurança para estimular uma ética de
competitividade por meio da democratização da forma"-empresa, que passou
a ser métrica de toda e qualquer organização social, o apoio mútuo até
poderia ser visto como uma potente prática dos que resistem, fosse
dentro ou fora do contexto da chamada pandemia. No entanto, quando essa
prática se avizinha às ações assistenciais, cooperativistas e de
empreendedorismo social, elas acabam por produzir efeitos apenas de
ajuda emergencial como em ações humanitárias ou colaboram, ainda que
involuntariamente, com a expansão dos negócios sociais. E não se trata
de mensurar o quanto esta ou aquela ação é ou não anarquista, pois esses
efeitos se produzem independentemente das ``intenções'' e/ou vontades
dos sujeitos envolvidos. Sendo assim, se as ações compiladas pela
plataforma ``Apoio Mútuo'' se dizem orientadas para a transformação das
pessoas envolvidas, essas são questões que não devem ser colocadas
somente pelos anarquistas relacionados com essas formas de ação e
intervenção.

Por fim, é importante registrar, nessas idas e vindas das práticas de
apoio mútuo em meio à chamada pandemia, que as inscrever no campo das
disputas políticas é o avesso da potência antipolítica da revolta. Essa
ação direta própria dos anarquistas que, como posto por Bakunin, não se
dobra nem diante da autoridade da ciência para afirmar a vida livre, não
implica negar a ciência em bloco. Inscrever práticas anarquistas nessa
disputa acaba funcionando a favor das táticas de assimilação e
neutralização contemporâneas que veem na anarquia apenas a expressão
mais radical do amplo campo político de disputa pelo governo chamado de
esquerda ou força progressista. Nesse sentido, lembrar da história de
atuação dos anarquistas nos sindicatos pode nos informar sobre como a
inscrição nessa disputa é deletéria aos anarquismos e como, nessas
ocasiões, fomos massacrados por forças que, em momentos de
recrudescimento autoritário, foram vistas como aliadas pontuais. Por
analogia, essas ações comunitárias podem apenas ser a forma
contemporânea da atuação sindical, transpondo o espaço da fábrica para a
cidade, conforme as propostas de municipalismo libertário de Murray
Bookchin\footnote{A respeito de Bookchin e sua inspiração em Kropotkin
  para as proposições do municipalismo libertário e a ecologia social,
  ver: \textsc{augusto}, 2012.}, que pode ser visto como continuador da obra de
Piotr Kropotkin, pois as proposições tanto de ecologia social, quanto de
democracia direta local, do municipalismo libertário, foram inspiradas
na idealização que Kropotkin fez das guildas medievais como referenciais
de comunidades sem controle estatal.

Não se trata de emitir juízo a respeito de práticas de resistências, mas
de alertar que, ao se perderem em disputas políticas, elas se confundem
com os negócios sociais e se veem disponíveis às capturas neoliberais
nas tentativas de restaurar um sentido contemporâneo para o apoio mútuo
de Kropotkin. Diante dessa situação, resta uma questão: onde está a
revolta?

\section{na luta contra o mundo: militantismo anarquista}

Dizer o quê ou como fazer não corresponde ao conjunto de práticas
anarquistas que compõem a \emph{cultura libertária.}\footnote{Ver:
  \textsc{passetti} \& \textsc{augusto}, 2008.} Nela, cada vivente ou associação trava
suas batalhas segundo os modos de fazer e usar que dão forma à vida
libertária. No entanto, não há nada de prescrito em ressaltar,
analiticamente, o que não faz parte da anarquia como vida militante.
Sobretudo diante de um acontecimento inédito que, numa situação de
defesa da vida em abstrato pelos controles sanitários"-securitários,
coloca precisamente a questão: qual vida se quer viver?

Essa vida militante, que a maneira anarquista de dar forma à liberdade
afirma, seguramente não é o ativismo contemporâneo que organiza as
identidades políticas por autodeclaração. Vivemos um tempo em que a
vinculação da existência a um conjunto de práticas parece se resumir a
declarar"-se algo, seja o que for. Essa é uma forma discursiva
especialmente evidente nas redes sociais digitais. Nelas, qualquer
pessoa se sente impelida a se declarar pertencente a uma identidade
política qualquer e nela se fecha como em um \emph{bunker}, ela pode
ficar por ali ou saltar em variações, mas sempre fechada do ponto de
vista da construção da própria subjetividade. A partir dessa
fortificação, cada pessoa se defende e/ou ataca as alteridades que ela
encontra pelo caminho dessas redes. Os próprios governos institucionais,
hoje em dia e em quase todo o planeta, sobrevivem desse ativismo
autodeclaratório, uma espécie de planetarização e democratização dos
atos de fala como enunciados de autoridade. Essa é uma das vias para
compreender o porquê de, mesmo após vencer as eleições, muitos governos
seguirem em campanha eleitoral por meio da atuação de seus ativistas nas
redes sociais digitais. Como colocou Gilles Deleuze (1992), nas
sociedades de controle, nada acaba, estamos sob o signo do inacabado e
do contínuo. Isso vale também para as disputas políticas em meio à
chamada pandemia, se um presidente diz que se trata de um vírus chinês e
a pessoa se declara vinculada a esse presidente, pronto, está posta uma
verdade. Da mesma maneira, se a \textsc{oms} diz que só há uma forma de lidar com
as contaminações e as infecções e a pessoa quer ser vista como alguém
que respeita as autoridades científicas, pronto, está posta uma verdade
oposta à primeira. Tudo se resume à crença e declaração.

Todo esse emaranhado de disputas declaratórias, seja de autoridades,
seja de um cidadão qualquer em seu ativismo político, é o extremo oposto
do que Michel Foucault vai chamar de militantismo a partir da
experiência trans"-histórica do cinismo antigo, na qual ele inclui,
modernamente, os anarquistas. Na penúltima aula do curso ``A coragem da
verdade'', em 21 de março de 1984, ele oferece uma descrição muito
precisa desse militantismo: ``seria a ideia de uma militância de certo
modo em meio aberto, isto é, uma militância que se dirige a
absolutamente todo mundo, uma militância que não exige justamente uma
educação (uma \emph{paideía}), mas que recorre a meios violentos e
drásticos, não tanto para formar as pessoas e lhes ensinar, quanto para
sacudi"-las e convertê"-las, convertê"-las bruscamente. É uma militância em
meio aberto no sentido que pretende atacar não somente este ou aquele
vício, defeito ou opinião que este ou aquele indivíduo poderia ter, mas
igualmente as convenções, as leis, as instituições que, por sua vez,
repousam nos vícios, defeitos, fraquezas, opiniões que o gênero humano
compartilha em geral. {[}\ldots{}{]} Um militantismo aberto, universal,
agressivo, um militantismo no mundo, contra o mundo'' (\textsc{foucault}, 2011,
p. 251). Um militantismo que, por meio da ação direta, age como revolta
antipolítica. Assim, buscar provar para as pessoas que o apoio mútuo é
eficaz, esperando que com isso elas afastem suas vidas das formas do
Estado e do capitalismo, é abrir mão da revolta que dá forma à anarquia
como antipolítica.

Nesse sentido, com ou sem crise sanitária"-securitária, é possível ver
essa revolta antipolítica em outras ações, como no militantismo que se
expressa em ações da tática \emph{black bloc} em diversas cidades do
mundo, mesmo com medidas de restrições de circulação, nas ocupações de
prédios convertidos em centros sociais e moradias coletivas, nas ações
de autodefesa de grupos antifa e/ou anarco"-queer e junto a toda forma de
viver o prazer do sexo que não reivindica reconhecimento, seja do
Estado, do mercado ou de movimentos organizados. Ações que puderam ser
observadas na atuação de pequenos grupos, em meio aos protestos
antirracistas que pipocaram em todo o planeta no meio dessa
crise"-guerra, que foram além do protesto e bradaram pela abolição da
polícia. Enfim, toda ação direta que, no momento em que é executada, não
reconhece a pacificação da política de negociação e que, fatalmente,
será acusada, por forças da esquerda e da direita, de ser radical
demais. O que deriva de cada uma dessas ações é uma outra história e não
pode mais ser vinculada a elas em uma relação de causa e efeito. Há uma
imagem, na dissertação de mestrado de Matheus Marestoni (2019), na qual
os \emph{black blocs}, em junho de 2013 no Brasil, ao retirarem as
pedras portuguesas das calçadas para resistir às investidas das tropas
de Choque da Polícia Militar, estavam levantando a poeira dos mais de
500 anos de pacificação dos selvagens dessa terra. Essa imagem dá a
dimensão desse militantismo que não busca disputar o político. Basta ver
que, depois junho de 2013, nada mais no campo das lutas sociais e das
disputas políticas foi o mesmo que antes. Se para melhor ou para pior,
não cabe aqui dizer. Isso é agonismo, não tem batalha final, é apenas
fogo.

A declaração de pandemia rapidamente se impôs como um acontecimento
inédito aos viventes humanos na Terra; ou foi percebida, sobretudo por
estes, como uma maldição. Mas tão rápida quanto a disseminação das
contaminações e das infecções, foi a criação de uma imagem política de
disputa em torno dela que, até o momento, serviu para intensificar os
controles sanitários"-securitários já existentes e animar as competições
midiatizadas do ativismo autodeclaratório. Aos anarquistas, cabe se
apartar dessa disputa ou correr o risco de serem tragados por ela. Está
em jogo a revolta da vida contra o governo da ciência, tal qual ela se
dá no momento em que nascemos. Este é um momento em que tudo é inédito,
muito mais desconhecido do que tudo que se impôs com a chegada desse
vírus no planeta; nessa vinda única à vida que cada um experimentou, não
há declaração possível, apenas um grito, um choro, que anuncia nossa
chegada, um grito que é, de certa maneira, também um grito contra o
mundo tal qual ele se encontra neste momento singular. É por saber disso
que, para os anarquistas, não cabe entrar em disputas, mas apenas ter
saúde e anarquia no planeta e contra o mundo com suas guerras"-crises.


\begin{bibliohedra}
\tit{AUGUSTO}, Acácio. ``Guerra e pandemia: produção de um inimigo invisível
contra a vida livre'' In: \emph{Pandemia Crítica}. São Paulo: n"-1, março
de 2020, vol. 18. Disponível em
\emph{https://www.n-1edicoes.org/textos/51}. Acesso em: 23/09/2020.

\titidem. Municipalismo libertário, ecologia social e resistências. In:
\emph{Revista Ecopolítica}. São Paulo: Nu"-Sol, janeiro/abril de 2012, n.
2, p. 64--98. Disponível em:
\emph{https://revistas.pucsp.br/index.php/ecopolitica/article/view/9076}.
Acesso em: 21/09/2020.

\tit{BAKUNIN}, Mikhail. \emph{Deus e o Estado}. São Paulo:
Nu"-Sol/Imaginário/\textsc{soma}, 2000.

\tit{DELEUZE}, Gilles. Post"-scriptum sobre as sociedades de controle. In:
\emph{Conversações}. Tradução de Peter Pal Pelbart. São Paulo: Editora
34, 1992.

\tit{FACÇÃO FICTÍCIA} (org.). \emph{A luta é pela vida. Escritos anarquistas
sobre capitalismo, Pandemia e a luta pela vida}. Março e Abril de 2020,
vols. 1 e 2. Disponíveis em
\emph{https://faccaoficticia.noblogs.org/files/2020/03/LUTA\_PELA\_VIDA\_F.pdf}
e
\emph{https://faccaoficticia.noblogs.org/files/2020/04/Luta-Pela-Vida-f-2.pdf}.
Acesso em: 21/09/2020.

\tit{FEDERAÇÃO ANARQUISTA DE TURIM}. ¿Epidemia? Masacre de Estado. In:
\emph{nu"-sol}, março de 2020. Disponível em
\emph{https://www.nu-sol.org/blog/epidemia-masacre-de-estado/}. Acesso
em: 15/10/2010.

\tit{FOUCAULT}, Michel. \emph{A Coragem da verdade}. Tradução de Eduardo
Brandão. São Paulo: Martins Fontes, 2011.

\titidem. \emph{Nascimento da Biopolítica}. Tradução de Eduardo Brandão.
São Paulo: Martins Fontes, 2008.

\tit{GORTÁZAR}, Naiara Galarraga. Paraisópolis, uma favela contra o vírus. In
\emph{El País Brasil}. 04/10/2020. Disponível em:
\emph{https://bit.ly/3u5yh3s}. Acesso em: 23/10/2020.

\tit{GRUBACIC}, Andrej; \textsc{graeber}, David. Introduction to Mutual Aid. An
Illuminated Factor of Evolution. In: \emph{The Anarchist Library}, 2020.
Disponível em:
\emph{https://theanarchistlibrary.org/library/andrej-grubacic-david-graeber-introduction-to-mutual-aid}.
Acesso em: 22/10/2020.

\tit{IBÁÑEZ}, Tomás. Não! In: \emph{Revista verve.} São Paulo: Nu"-Sol, 2020,
n. 38, pp. 25--30. Disponível em:
$\emph{http://www.nu-sol.org/blog/dt\_portfolios/v-e-r-v-e-38/}$. Acesso em:
11/09/2020.

\tit{KROPOTKIN}, Piotr. \emph{El apoyo mutuo}. Cali: Ediciones Madre Tierra,
1989.

\tit{LOPES}, Marina. Brazil's favelas, neglected by the government, organize
their own coronavirus fight. In \emph{Washington Post}, 10/06/2020.
Disponível em:
https://www.washingtonpost.com/world/the\_americas/coronavirus-brazil-favela-sao-paulo-rio-janeiro-bolsonaro/2020/06/09/8b03eee0-aa74-11ea-9063-e69bd6520940\_story.html.
Acesso em: 18/10/2020.

\tit{MARESTONI}, Matheus. \emph{No fogo de 2013: ação direta anarquista,
autonomismo e a democracia contemporânea}. Dissertação (Mestrado em
Ciências Sociais) --- Programa de Estudos Pós"-Graduado em Ciências
Sociais, \textsc{puc"-sp}, São Paulo, 2019.

\tit{NU-SOL} (org.). Dossiê \textsc{covid}"-19: afirmações da vida. In: \emph{nu"-sol},
2020. Disponível em:
\emph{https://www.nu-sol.org/blog/covid-19-afirmacoes-da-vida/}. Acesso
em: 15/10/2020.

\tit{PASSETTI}, Edson et al. \emph{Ecopolítica}. São Paulo: Hedra, 2019.

\titidem. \emph{Ética dos amigos: invenções libertárias da vida}. São
Paulo: Imaginário/\textsc{capes}, 2003.

\tit{PASSETTI}, Edson \& \textsc{augusto}, Acácio. \emph{Anarquismos e educação}. Belo
Horizonte: Autêntica, 2008.

\tit{SMITH}, Zoe. Mutual aid is sweeping the world. Here's how we make this
anarchist way of organising last. In: \emph{The Correspondent},
2/6/2020. Disponível em:
\emph{https://thecorrespondent.com/504/mutual-aid-is-sweeping-the-world-heres-how-we-make-this-anarchist-way-of-organising-last/553195377504-846fc559}.
Acesso em: 15/10/2020.
\end{bibliohedra}

\chapterspecial{O inimigo não é invisível}{}{Adriana F.\,Martinez}

\noindent{}Vírus, do latim: veneno. Sistema biológico muito simples e pequeno
formado por uma cápsula proteica que serve de invólucro ao material
genético. Parasitas microscópicos sem célula cuja reprodução só é
possível quando invadem o interior das células de seres vivos, como o
novo coronavírus (coV) ou \textsc{sars}"-Cov"-2, desencadeador da doença \textsc{covid}"-19.
Esse vírus, como qualquer um, forma parte da natureza, e a natureza não
obedece a leis, finalidades, controles nem fronteiras. O vírus não é o
inimigo assim como não é amigo. Por isso, os discursos fundados na
guerra para combater o ``inimigo invisível'' não passam de
antropomorfismos convenientes para produzir novas técnicas de governo
que atendem à racionalidade neoliberal,
por conseguinte, a uma economia de
livre mercado que regula e organiza o governo do Estado em toda a sua
espessura. E numa economia regrada pela concorrência e num modo de vida
empresa como poder enformador da sociedade, a política social versa em
cada um assumir a responsabilidade pelos riscos que venham a (o)correr
durante a sua existência. Não à toa prioriza"-se tanto a formação do
capital humano como sujeito econômico ativo, provedor de seus
rendimentos. Um projeto que é, em si, o próprio crescimento econômico
requerido pelo capitalismo (\textsc{foucault}, 2008).

A atual pandemia é efeito do capitalismo. Um capitalismo em nível
planetário no qual se destaca a capacidade de mobilidade do capital
humano em termos de empreendimento
individual para obter melhores
posições sociais, aprimorar o conhecimento e aumentar as chances
profissionais. Motivo pelo qual a diferença dessa pandemia com outras de
contextos históricos anteriores, a exemplo do coronavírus H1N1, chamado
de ``gripe espanhola'' (1918), é que o vírus não infectou primeiro os
pobres, famélicos, moradores de lugares insalubres considerados
\emph{vulneráveis}, mas a contaminação e propagação inicial da \textsc{covid}"-19
ocorreu pelo deslocamento das camadas sociais mais abastadas. A doença
foi importada pela \emph{elite} social.

Como medida de \emph{segurança}, ativada sob a justificativa de conter a
transmissão do vírus, as fronteiras territoriais foram fechadas e foram
instaurados limites entre os corpos, ampliando as técnicas de
monitoramento. E as práticas constantes de monitoramento virtual, ou
não, colaboram com o exercício de governo sobre todos os processos da
vida, proporcionando a possibilidade de ``acompanhar uma atividade,
conduta ou ambiente sem a necessidade de interferir em sua pretensa
continuidade infinita'' (\textsc{passetti} et al., 2019, p. 259). Não se sabe
ainda o que irá ocorrer após o novo coronavírus. Porém, com o
alastramento do contágio, admitiam"-se apenas as viagens rotuladas
\emph{essenciais} ou o ingresso de concidadãos fora dos seus territórios
nacionais, sem bloquear o fluxo de dados, produtos e transações
financeiras. Ao blindar as fronteiras estatais com base no argumento de
proteger seus cidadãos de possíveis infecções, escolhe"-se a dedo quem
pode ingressar e quando, com o propósito de salvaguardar a saúde do
capital humano local. Tais critérios reforçam o nacionalismo virulento
restaurado exponencialmente na última década.

Duas regras sanitárias foram de imediato instituídas: o ``isolamento
social'' e o ``distanciamento social''. A primeira implantou"-se sob o
preceito de evitar a proliferação da \textsc{covid}"-19 e a segunda com a
finalidade de restringir o contato entre as pessoas visando a amortecer
a velocidade de transmissão. Configurações estas replicadas em quase
todos os países do planeta, e antes que se interprete a questão como
parâmetro a favor ou contra, já é bom responder que não se trata disso,
tampouco se trata de estar deste ou daquele lado. Trata"-se, sim, de
questionar as medidas de segurança produzidas particularmente desde a
última década do século passado, em que a noção de segurança nacional se
espraiou para segurança universal em nome da \emph{segurança humana} com
o objetivo de barrar os deslocamentos de pessoas avaliadas como virtuais
ameaças. Estratégia esta assimilada pelas condutas individuais que
\emph{compartilham} junto ao Estado o governo das condutas, no intuito
de resguardar seus ambientes. Estratégia utilizada para manter os
chamados \emph{vulneráveis} fixos em suas regiões mediante programas de
melhorias. Estratégia usada para criminalizar, punir e identificar as
possíveis \emph{ameaças}.

As diretrizes de isolamento e distanciamento tomadas por conta da atual
pandemia expõem como a \emph{segurança humana} diz respeito a ações
governamentais bem precisas, encarregadas de monitorar fluxos, gerir
processos, forjar condutas submissas, capturar revoltas em benefício da
economia fundada na racionalidade neoliberal. Ademais, as disposições
estipuladas devido à \textsc{covid}"-19 expressam como toda e qualquer
\emph{crise} (política, econômica, sanitária, etc.) constitui a forma
corrente de governar na racionalidade neoliberal. A função consiste em,
por meio de protocolos de prevenção e precaução, proporcionar opções às
políticas vigentes com o objetivo de tornar o que agora é provisório em
algo definitivo.

\section{cálculo do custo"-benefício na saúde}

O vírus, após ter visitado as coberturas dos estratos sociais
superiores, desceu para os andares inferiores da sociedade que ficaram
mais expostos ao contágio, ao adoecimento e à morte. Nos \textsc{eua}, são os
pretos e latino"-americanos pobres os primeiros a morrerem em casa, na
rua, na porta do hospital ou apinhados em prisões, algumas construídas
especialmente para imigrantes \emph{ilegais}, outras reservadas para o
encarceramento em massa, preferencialmente, da população preta. Também
morrem os refugiados abarrotados em campos, assentamentos, abrigos ou
barcos aqui e acolá. Enquanto isso, o Alto Comissariado das Nações
Unidas para os Refugiados (\textsc{acnur})\footnote{Cf.
  https://nacoesunidas.org/coronavirus-e-refugiados-o-que-o-acnur-esta-fazendo-no-brasil-e-no-mundo/.
  Acesso em: 19/09/2020.} monitora continuamente as fronteiras e os
aeroportos para, segundo a Organização das Nações Unidas (\textsc{onu}), conter
potenciais \emph{riscos} adicionais envolvendo a chegada de mais
solicitantes de refúgio. As ações \emph{humanitárias} promovidas pela
agência abrangem informar, aos refugiados, sobre a doença e distribuir
máscaras, luvas, sabão em lugares onde são disputados o uso da água e o
espaço.

Morrem os indígenas na América do Sul. No Brasil, o vírus chega a eles
pelos perdigotos de garimpeiros, fazendeiros, grileiros, ``sojeiros'',
pecuaristas, madeireiros, militares e missionários, muitos deles
dedicados a expandir o agronegócio. A propósito, como todo momento de
\emph{crise}, esta é uma \emph{oportunidade} para dilatar lucros, uma
estratégia que cumpre, no mínimo, com um dos três pilares do
desenvolvimento sustentável ao transformar"-se em \emph{economicamente
viável}. Nos 54 países do continente africano, as autoridades locais e a
Organização Mundial da Saúde (\textsc{oms}) dizem estar surpresas pelo baixo
impacto do novo coronavírus. Falam, inclusive, que são países
acostumados com epidemias e, por isso, souberam tomar medidas adequadas,
mas não levam em consideração o quase nulo índice de testes aplicados na
maior parte dos países do continente. Morrem os pobres sem assistência
médica nos confins de regiões e cidades. Morrem pelo vírus. Morrem pela
fome. Atalho próspero para aproximar"-se dos dois primeiros objetivos da
Agenda 2030: acabar com a pobreza e com a fome no
planeta.

Outra morte promissora nessa pandemia é a dos velhos. Velhos pobres
principalmente, porque os outros abastecem um largo mercado e não só no
âmbito da saúde. Quanto maior o número de velhos mortos menos
aposentadorias precisam ser pagas e mais rápido ficam vagos os leitos
para serem ocupados por corpos economicamente produtivos. Por vezes, o
processo consistiu em sequer permitir que ocupassem lugares nos
hospitais, foram abandonados em asilos ou em suas casas. Em muitas
circunstâncias são mortes não registradas, ocultas embaixo do tapete (de
terra?) para difundir a baixa taxa de letalidade, como na Alemanha que,
apesar de ter uma população com 25\% acima dos 60 anos, nas estatísticas
médicas são somente 20\% do total de infectados. Os números seriam 11\%
a mais na Grã"-Bretanha, se tivessem incluído os asilos nas estatísticas,
ou na Suécia, onde a morte de velhos constitui 50\% do total, sem contar
a não realização de testes nessas pessoas em vários países.

Dito de outra forma, todos aqueles
que carecem de condições econômicas, não apresentam \emph{eficiência}
para assegurar a sua sobrevivência, nem conseguiram investir na sua
saúde, morrem nas filas de hospitais, em casa, nas ruas\ldots{} Nenhuma
novidade dentro da racionalidade neoliberal, posto que o projeto social
incorre em o sujeito obter rendimentos suficientes para
``se garantir por si mesmo contra
os riscos que existem, ou também contra os riscos da existência, ou
também contra essa fatalidade da existência que são a velhice e a
morte'' (\textsc{foucault}, 2008, p. 197). O mercado da saúde funciona igual a
qualquer outro mercado, corroborando o jogo de desigualdades próprio da
concorrência. Quem não adquiriu renda suficiente para arcar com o custo
de sua saúde perece ou espera para ter a chance de ocupar um leito e
usufruir dos equipamentos hospitalares públicos sem custo adicional,
desde que o Estado os ofereça. A saúde redundou num amplo mercado e na
exoneração da responsabilidade estatal. Nos \textsc{eua}, por exemplo, 30 milhões
de pessoas encontram"-se sem cobertura médica alguma; na China, o auxílio
público não estabelece que o atendimento seja gratuito; na Alemanha, as
modalidades pública e privada são pagas, isto só para mencionar alguns
dentre tantos países com esses modelos de saúde. A atual pandemia trouxe
à tona como o direito universal à saúde, baseado no plano Beveridge
(1948) que indicava ser de incumbência do Estado a saúde da população
para ter certa paridade nos tratamentos de cura e prevenção de doenças,
passou a ser, no cerne da racionalidade neoliberal, um cálculo previsto
no orçamento individual, ao invés de estimativas de receitas estatais.

Resultou mais proveitoso, ao Estado, repassar benefícios ou proporcionar
incentivos fiscais à rede privada de \emph{prestação} de \emph{serviços}
de saúde e ao setor empresarial, que oferece assistência médica aos seus
\emph{colaboradores}, do que comprometer a sua arrecadação nesse
quesito. ``Vemos, assim, que a esperada igualdade de consumo médico
mediante a seguridade social é pervertida em favor de um sistema,
tendente cada vez mais a reestabelecer as grandes desigualdades da
doença e da morte que caracterizavam a sociedade do século \textsc{xix}. Hoje, o
direito a uma saúde igual para todos é capturado em uma grande
engrenagem que o transforma em uma desigualdade'' (\textsc{foucault}, 2016, p.
391). Isso não exprime almejar a volta do Estado de Bem"-Estar em que a
interferência do Estado se expressava em manter saudáveis a força de
trabalho e a força física nacional, apenas como capacidades militar e de
produção. Trata"-se de assinalar aqui como a saúde recai na
responsabilidade individual, uma vez que alcançou valor econômico e foi
inserida no mercado. Nesse sentido, tendo em vista que o sistema de
saúde dos países é reservado para poucos e em face da mortalidade
provocada pelo vírus, o discurso difundido pelos quatro cantos do
planeta é a higiene, o uso de máscara, o isolamento e distanciamento
como alternativas para contar, pelo menos, com assistência médica caso o
indivíduo adoeça. Não obstante, grupos clamam pelas suas liberdades
civis e pela liberdade econômica, requerem suas \emph{liberdades}
liberais garantidas pelas constituições nacionais, até mesmo exigem o
\emph{direito} de contagiar"-se. Manifestações estas amparadas pela
democracia numa política pluripartidária.

Basta ver como os princípios democráticos foram proferidos, solicitados
e advogados pelo supremacista branco presidente dos \textsc{eua}, Donald Trump, e
por seu bajulador, o capitão reformado do exército presidente do Brasil,
Jair Bolsonaro. Os dois declaradamente racistas, xenofóbicos,
nacionalistas, homofóbicos, machistas, autoritários. Eles, sustentando
que a \textsc{covid}"-19 é mais uma \emph{gripezinha}, uma versão alarmista da \textsc{oms}
e da mídia ou uma conspiração chinesa, divulgam que a economia não pode
parar. Desde a moral do protestantismo, ambos julgam que quem fica em
casa não quer trabalhar, pouco importa a falta de vagas hospitalares
para os indivíduos de baixa renda ou serem os países com o maior
registro de mortes em números absolutos. Daí o empenho em defender o
princípio liberal de ir e vir, embora não tenham hesitado em fechar as
fronteiras, curiosamente, para impedir a entrada de pessoas infectadas
pelo novo coronavírus, especialmente pobres e refugiados. Nesses termos,
a \emph{gripezinha} passa à categoria de segurança nacional.

A despeito de esses mandatários terem insinuado \emph{superioridade}
imunológica, no dia 7 de julho de 2020, o capitão reformado anunciou
estar contaminado pelo vírus e o supremacista branco foi internado em 2
de outubro depois de ter testado positivo para a \textsc{covid}"-19. Bolsonaro, na
época, disse que, poucas horas após a administração do fármaco
hidroxicloroquina, já se sentia bem. Vale lembrar que Trump figura entre
os acionistas da empresa Sanofi, uma das maiores no ramo farmacêutico, a
qual detém a patente dessa droga. Um medicamento, segundo a \textsc{oms} e
pesquisadores, sem evidências científicas que comprovem bons resultados
contra a doença. Talvez por esse motivo o presidente dos \textsc{eua} não incluiu
a hidroxicloroquina em seu tratamento. Tal ocorrência evidencia, não
apenas o selo de acordos lavrados entre os dois Estados, como a
confirmação de que o Estado brasileiro acata as coordenadas políticas
oriundas dos \textsc{eua}. Isto não é novo, basta revisitar a subordinação
irrestrita dos países latino"-americanos aos \textsc{eua} no período das ditaduras
civil"-militar nas décadas de 1960 a 1980, quando a América do Sul foi o
laboratório para realizar a implementação da racionalidade neoliberal.
Aliás, hoje, no Cone Sul, predomina a ala da direita, seguidora de
preceitos análogos aos do capitão reformado e do supremacista branco.
Será essa região novamente eleita para serem processados os
experimentos? Já estão sendo aprimoradas as estratégias para, quando a
\emph{crise} amainar, transformar as reformas em alterações permanentes?
O vírus é invisível, não o inimigo.

\section{isolamentos e distanciamentos de corpos}

De modo complementar às reivindicações das liberdades liberais, já é
possível vislumbrar como práticas de monitoramento acirram"-se ainda mais
sob o discurso de controlar a vida dos outros para \emph{proteger} a
própria saúde e a da população global: todos são suspeitos de
\emph{portar} e transmitir o vírus. Por isso, o \emph{cidadão de bem}
sente"-se na \emph{obrigação} cívica de delatar reuniões com muita gente,
aglomerações em locais públicos, o uso inadequado de equipamentos de
proteção individual, a falta de higiene. A saber: no território
brasileiro, qualquer reunião com mais de seis pessoas é considerada
aglomeração e passível de ser enquadrada como crime contra a saúde
pública. A saúde, além de ter sido incorporada ao mercado, é uma questão
política cujo efeito pretende"-se totalizador e de governo enquanto
gestão das condutas dos indivíduos.

De um lado, reclama"-se das medidas de isolamento e distanciamento,
exigindo a garantia dos direitos liberais. De outro, solicita"-se punição
para quem deixa de cumprir tais regras. Duas faces da mesma moeda. É à
sombra de diretrizes globais que agem os sujeitos, alguns sequer
conseguem imaginar"-se fora delas. Para os que assim pensam, qualquer
ação diferente é inaceitável, condenável e precisa ser punida, adequada
ou extinta. Essa é a regra de corte: mais punição, regulamentações,
prisões, legalismo. Em vez disso, por que não se ocupar da sua
existência? Dedicar"-se, por conta própria, aos cuidados pessoais para
fortalecer"-se, bem como não expor a saúde do outro, sem precisar da
imposição de regulamentações. É curioso como tanto quem está contra
quanto quem está a favor das disposições sanitárias não questionam
algumas estratégias de isolamento e distanciamento relacionadas ao
chamado \emph{novo normal} que tornam \emph{natural} certos
procedimentos; pelo contrário, são aceitas com entusiasmo. Provavelmente
porque tais estratégias se apresentam, para o indivíduo, como garantia
de \emph{segurança}, sobretudo as correlacionadas com áreas chamadas de
\emph{automação inteligente}.

O afrouxamento das medidas de isolamento ou distanciamento projeta um
\emph{novo normal} configurado num \emph{ambiente} moderado, conformado,
adaptado, pacificado. Medidas estas elásticas, capazes de estabelecer, a
qualquer momento, barreiras restritivas em países, cidades, regiões ou
bairros com base numa outra ``onda viral''. E após a vacina ser
descoberta, testada, patenteada, produzida e comercializada, os sujeitos
já estarão habituados a esse \emph{novo normal}. A atual pandemia
catalisa sobre os corpos mortos o \emph{futuro melhor} para corpos
distanciados. O quanto esse \emph{distanciamento} vai emplacar no dia a
dia de cada um ainda é incerto, mas possivelmente seus efeitos irão
modificar o governo das condutas pós"-pandemia. O distanciamento entre os
corpos direciona"-se para uma governamentalidade --- compreendida
enquanto condução da própria conduta e a dos outros (\textsc{foucault}, 2008) ---
em que se busca limitar o contato físico dos indivíduos mediante um
conjunto de ações com as quais se sedimente a conduta de manter
proximidade, de preferência, com as pessoas que convivem sob o mesmo
teto.

Basta ver a propensão de afastar gradativamente crianças e jovens entre
si fixando"-os no espaço virtual com a ``educação a distância'' ou no
estímulo de interações via redes sociais. Com isso, evitam"-se os toques
corporais a partir de uma dimensão moral e inibe"-se a força da revolta
característica da juventude. Sem mencionar estarem permanentemente
monitorados e persuadidos a trocar atitudes rebeldes por práticas
obedientes. Essa modalidade educativa abastece um vasto banco de dados,
convoca à participação democrática moderada e favorece a politização
orientada para o governo de condutas inerentes à racionalidade
neoliberal, por conseguinte, sujeitos econômicos empresários de si
(Idem). A sua implementação é um negócio rentável para \emph{startups}
ligadas ao \emph{lobby} de empresas (Microsoft, Google, Facebook, etc.)
que disponibilizam suas plataformas, ferramentas e programas de ensino
remoto. Negócio bem"-sucedido para as instituições de ensino públicas ou
privadas, na diminuição de custos e na ampliação da formação do capital
humano. O único inconveniente é, para alguns pais, que não suportam a
convivência com os filhos em tempo integral. Para a mulher empoderada é
inconcebível cuidar de seus rebentos cotidianamente, para o homem de
negócios é ultrajante realizar os serviços domésticos, e ambos se sentem
importunados por esses corpos alvoroçados quando precisam responder às
suas responsabilidades laborais, primordiais em suas vidas. Quais serão
os recursos \emph{inovadores} a serem utilizados pelos progenitores ou
indicados por especialistas, se persistirem as aulas a distância?

Separam"-se, também, os trabalhadores em suas casas na execução do
``trabalho remoto'', por meio dos equipamentos computacionais,
\emph{familiarizando} o trabalho e \emph{domesticando} ainda mais o
trabalhador. As vantagens relacionam"-se à redução das despesas fixas nas
empresas, aumento de produtividade, melhores condições para obter
\emph{feedback}, comunicação instantânea graças aos aplicativos e gestão
virtual de tarefas. Por sua vez, o indivíduo que consegue desempenhar
suas atividades em seu \emph{lar} sente"-se satisfeito por manter seu
emprego, pela sua eficácia, pela sua capacidade produtiva e pelo bom
\emph{investimento} em si mesmo. Deverá alternar"-se o distanciamento com
aproximações corporais nos espaços físicos? No fim das contas, os
encontros podem servir de requisitos para verificar a adequação ou não
às mudanças. A conformação do distanciamento ligado ao trabalho divide
quem permanece detrás das telas e quem precisa trabalhar de forma
presencial. Será esta uma nova divisão do trabalho? O que ela acarreta?
Imprime"-se, aos poucos, o distanciamento entre os corpos e quiçá perdure
mais tempo do que se espera.

As áreas ligadas ao consumo de equipamentos culturais também indicam
alterações. A tecnologia de \emph{streaming}, capaz de transmitir
instantaneamente dados de áudio e vídeo pelas redes, tem angariado
mercado a ponto de as plataformas de conteúdo \emph{personalizado,} como
Netflix, Amazon, entre outras, faturarem, em 2019, mais que as
indústrias cinematográficas tradicionais. À semelhança das aulas e do
trabalho remotos, a área de consumo cultural certamente irá apresentar
mudanças no \emph{novo normal}, aliás, nesses três setores, nem tão novo
assim. Os entusiastas de tais inovações sentem"-se satisfeitos e preveem
a adoção das atividades remotas como um caminho sem volta. Sem embargo,
conclusões apressadas geralmente são válidas para os que naturalizam as
soluções encontradas por outrem como se elas fossem inevitáveis.

E não para por aí. Recomendações
para o sexo \emph{saudável} já foram elaboradas. A exemplo de um guia
organizado por médicos estadunidenses, no qual \emph{aconselham}:
masturbação, desde que com as mãos limpas ou com os \emph{brinquedos}
sexuais higienizados; encontros virtuais, sempre e quando haja segurança
de não serem vazadas as imagens. Caso as pessoas decidam ter uma relação
sexual física, os \emph{peritos} indicam: em primeiríssimo lugar,
relacionar"-se com quem está em isolamento na mesma casa; reduzir o
número de parceiros; usar máscara; evitar beijar; dispensar fazer sexo
oral ou anal; não ter contato com sêmen ou urina; limpar antes e depois
o lugar da relação, assim como tomar banho previamente e após o ato
sexual (\textsc{turban} et al., 2020). Obviamente que, para os especialistas, o
menor risco é a abstinência sexual. Nada de cheiros, roces epidérmicos,
troca de fluídos. Explicita"-se, com isso, como se reafirmam a
instituição família e os relacionamentos
monogâmicos. É bom reiterar: o
inimigo não é invisível, ele é sempre muito visível.

Afora essas configurações, incute"-se, com o alastramento da chamada
``tecnologia avançada'' e da quinta geração de telecomunicação móvel
(5G) --- mediante a qual os dispositivos irão interconectar"-se
permanentemente na rede, desde drones até artefatos domésticos ---, que
as máquinas são indispensáveis porque não instituem um perigo biológico
como os humanos. Uma ocasião favorável para acelerar a automação no
âmbito cotidiano. Para uma parcela da população considerada
privilegiada, quase tudo pode ser realizado a partir dos computadores
pessoais ou dos dispositivos móveis, \emph{compartilhando}, por meio das
plataformas digitas interligadas, todo o registro de suas vidas. Para
aqueles que inevitavelmente precisam deslocar"-se --- e são os que estão
atendendo quem ficou \emph{seguro} em casa ---, a tecnologia também
colabora para decodificar cada movimento, situar a posição geográfica,
ranquear a eficiência. Com isso, facilita"-se a centralização de dados
sofisticando controle, persuasão e condução de condutas, numa parceria
frutífera entre organizações governamentais e empresas de tecnologia.

Sistemas de tecnologia denominada \emph{hightech} são testados para
restringir o contato humano, dando cada vez mais lugar à Inteligência
Artificial (\textsc{ia}). Antes da \textsc{covid}"-19, a \textsc{ia}, embutida nos aplicativos de
celulares, era oferecida em prol da personalização do serviço ou do
produto e da simplificação do dia a dia. Hoje está sendo comercializada
em defesa de manter a \emph{salvo} as pessoas, protegê"-las desta ou de
outra potencial pandemia. A expectativa de uma sociedade intermediada
mais por vínculos digitais e menos por contato físico vinha gestando"-se
a largos passos em empreendimentos integrados entre Estados e empresas
privadas. Redesenhar o futuro pós"-\textsc{covid}"-19 implica investir em \textsc{ia}, não
apenas para atender a finalidades econômicas, mas com o propósito de
afirmar o chamado \emph{distanciamento} físico.

Gilles Deleuze, ao sublinhar que na sociedade de controle se operava por
computadores, mostrou como o ``perigo passivo {[}era{]} a interferência,
e, o ativo, a pirataria e a introdução do vírus'' (\textsc{deleuze}, 2004, p.
223). Nesse sentido, se o vírus biológico propiciou a corrida para
estender a \textsc{ia} e intensificar o controle, as resistências no chamado
\emph{novo normal} talvez consistam em propagar vírus virtuais
direcionados a \emph{infectar} os algoritmos da \textsc{ia} para quebrar códigos
de segurança, invadir redes operacionais, destruir banco de dados e
todos os sistemas utilizados para controlar initerruptamente cada um. E
por que não pensar também como resistência a redução do uso de aparelhos
portáteis ou dos contatos mediados só por telas? Inventar de que modo
transgredir essa maneira controlada de viver.

Na \emph{nova normalidade,} o imperativo é o distanciamento físico. A
pandemia opera como o laboratório de práticas a serem prolongadas nos
setores de telemedicina, comércio e prestação de serviços. As diretrizes
de distanciamento irão redefinir ainda as instalações públicas, as
acomodações das casas, o traçado urbano e enformar uma outra moral.
Decerto, não será a primeira vez na história que medidas sanitárias
reorganizam as cidades e consolidam o exercício de poder que estava em
curso, como bem assinalou Michel Foucault referindo"-se ao fenômeno
urbano em Paris no final do século \textsc{xviii}. Segundo ele, nessa época,
surgiu um medo urbano relacionado à aglomeração da população, à
concentração das fábricas, às tensões políticas, às grandes edificações
e às epidemias. Assim, o planejamento de Paris, na ocasião, foi disposto
mediante diretivas médicas com objetivos bem"-definidos para avaliar
lugares de ``amontoação, desordem, perigos'' na zona urbana; o controle
da circulação de coisas e de elementos naturais como água e ar; a
organização de distribuições concernentes a esgoto, água potável, etc.
Em suma, ``a medicina passou da análise do meio à dos efeitos do meio
sobre o organismo e, por fim, à análise do próprio organismo''
(\textsc{foucault}, 2016a, p. 415--418). Apesar de cada período histórico ter a
sua especificidade, Foucault suscita certos indícios para compreender o
presente.

Hoje ainda se mantém a preocupação do meio sobre o organismo, haja vista
a hipótese levantada pela Universidade de Harvard\footnote{Cf.
  \emph{https://www.hsph.harvard.edu/c-change/subtopics/coronavirus-and-pollution/}.
  Acesso em: 20/09/2020.} concernente à correlação entre a poluição do
ar e o aumento de mortalidade por \textsc{covid}"-19. Tal pesquisa adere a outras
efetuadas na China, na Itália, na Espanha, na França e na Alemanha.
Esses estudos já começaram a interferir nos projetos urbanos, e as
cidades \emph{inteligentes}, ancoradas na sustentabilidade ambiental,
sinalizam a opção desejada para se viver com \emph{qualidade}, em paz e,
quem sabe, constituir uma família \emph{feliz}. Trata"-se de lugares onde
tudo é rastreado e onde a população consegue, por meio de aplicativos,
saber o que acontece na cidade em tempo real, contribuindo com o
monitoramento. Ou seja, se algo por acaso escapou do controle efetuado
pela tecnologia, o \emph{cidadão"-de"-bem} encarrega"-se de denunciar. O
que essas cidades precisam ofertar é segurança, ordem e a certeza de que
todos vão ser \emph{responsáveis} pelas suas vidas e pelo ambiente. A
proposição é que nenhum fluxo escape de um governo ensimesmado em
eliminar revoltas.

No entanto, embora as pesquisas se atenham à análise do meio pelos
efeitos dele sobre o organismo, atualmente prevalecem os discursos
direcionados aos efeitos do organismo de um indivíduo sobre o organismo
do outro. A empresa Google lançou um aplicativo de realidade aumentada
para identificar, por meio da projeção de um círculo em volta do
usuário, a distância com os outros. Sem contar as marcações desenhadas
para delimitar a separação entre as pessoas em ruas, parques, praias,
escolas\ldots{} Dentro do espaço \emph{compartilhado}, encontram"-se os
conhecidos, a família, aqueles ligados por laços estreitos e, claro,
\emph{saudáveis}. Aos poucos vai introjetando"-se o afastamento com os
desconhecidos apontados, em geral, como potenciais perigosos. Aliás, a
reorganização de espaços e a produção de objetos anunciam outras
configurações de relacionamentos. Em Amsterdã, um restaurante construiu
cabines com lugares para três pessoas recomendadas para quem mora junto,
e, segundo os planejadores dessas acomodações, pretende"-se incentivar a
experiência de reinterpretar o conceito de \emph{união} (\textsc{harrouk},
19/05/2020). As adaptações espaciais ocorrem em todos os lugares, e, com
elas, evidencia"-se como se formam as subjetividades nessa
governamentalidade em curso. Contudo, parafraseando Nietzsche (1999),
uma perspectiva imensa sempre se abre ante nós que faz cambalear
verdades e crenças.

Há tanto para pensar! Há tanto para lutar! Como nos tornarmos outros
nesse contexto outro que já se desenha à nossa frente? O distanciamento
físico prenunciado parece sedimentar"-se como algo ordinário, frequente,
rotineiro. Demarcações instauram"-se entre os corpos em defesa da vida,
enquanto se rearranjam técnicas de controle e se naturaliza desde agora
o \emph{novo normal}. Projeção de um estado permanente de estar dentro.
Dentro de governamentalidades, do que é recomendado, do decretado, das
adaptações, do discurso do modelo universal de humano. Muitos desejam
sair de suas casas com \emph{segurança} para continuarem dentro de todo
esse circuito. Não se ignoram aqui as pessoas que precisam amontoar"-se
nas ruas ou nos transportes coletivos para conseguir o ganha"-pão diário.
Tampouco se omitem os procedimentos apropriados para diminuir o
contágio. Muito menos se apaga o querer aproximar"-se dos corpos
necessários em nossas vidas para um longo abraço e/ou o sexo livre.

Todavia, é vital encarar a morte ocupando"-se da própria vida. Não se
trata de renunciá"-la em nome de uma salvação futura nem a deixar em mãos
de outrem para tomarem conta. Ocupar"-se da própria vida exige uma
prática constante capaz de limitar o governo de condutas e impedir que
alguns tentem conduzir como devemos fazer, pensar, dizer. Cada um
precisa agir para que, no final das contas, saibamos o que é a grande
saúde: um sim, como atitude afirmativa perante a vida. Uma atitude que
nos permita experimentar saltar fora desta ou de qualquer outra situação
por nós mesmos. Enfrentar o mais pesado dos pesos: o eterno retorno. E
se é necessária a distância, que seja para afirmar a diferença. E se é
necessário estar sozinho, que seja a \emph{solitude} para experimentar
estar consigo e se deliciar com o silêncio. Sim, vamos pôr o nariz para
fora e no meio ``da própria engrenagem'', inventar a ``contra mola que
resiste''\footnote{``Primavera nos dentes'', de João Apolinário e João
  Ricardo, 1973, com o grupo Secos \& Molhados.}.

\pagebreak
\begin{bibliohedra}
\tit{DELEUZE}, Gilles. \emph{Conversações.} Tradução de Peter Pál Pelbart. Rio
de Janeiro: Ed. 34, 2004.

\tit{FOUCAULT}, Michel. Crise da medicina ou crise da antimedicina? In: \textsc{motta},
Manoel Barros (org.). \emph{Arte, Epistemologia, Filosofia e História da
Medicina}. Tradução de Vera Lucia Avellar Ribeiro. Rio de Janeiro:
Forense Universitária, 2016, p. 334--393. (Ditos e Escritos; \textsc{vii})

\titidem. ``O nascimento da medicina social''. In: \textsc{motta}, Manoel Barros
(org.). \emph{Arte, Epistemologia, Filosofia e História da Medicina}.
Tradução de Vera Lucia Avellar Ribeiro. Rio de Janeiro: Forense
Universitária, 2016a, p. 402--424. (Ditos e Escritos; \textsc{vii})

\titidem. \emph{Nascimento da biopolítica: curso no Collége de France
(1978--1979)}. Tradução de Eduardo Brandão. São Paulo: Martins Fontes,
2008. (Coleção tópicos).

\tit{HARROUK}, Christele. Serres Séparées Proposes a Socially"-Distant Dining
Experience in Amsterdam. In: \emph{ArchDaily}, 19/05/2020. Disponível
em:
https://www.archdaily.com/939806/serres-separees-proposes-a-socially-distant-dining-experience-in-amsterdam.
Acesso em: 20/09/2020.

\tit{NIETZSCHE}, Friedrich. \emph{Genealogia da moral: uma polêmica.} Tradução
de Paulo César de Souza. São Paulo: Companhia das Letras, 1999.

\tit{PASSETTI}, E. et al. \emph{Ecopolítica}. São Paulo: Editora Hedra, 2019.

\tit{TURBAN}, J. L.; \textsc{keuroghlian}, A. S.; \textsc{mayer}, K. H. Sexual Health in the
\textsc{sars}"-CoV"-2 Era. In: \emph{Annals of internal medicine}, 2020, 173(5), p.
387--389. Disponível em: https://doi.org/10.7326/M20-2004. Acesso em:
16/09/2020.
\end{bibliohedra}

\chapterspecial{Rompendo com a lógica capitalista de uma pandemia\footnotemark}{}{Allan Antliff}
\hedramarkboth{Rompendo com a lógica\ldots}{}

\footnotetext{Tradução do inglês por Eliane Carvalho.}

\noindent{}Assim como o vírus da \textsc{covid}"-19 varreu o globo durante os primeiros meses
de 2020, ele também o impactou de diferentes maneiras. Minhas reflexões
focam no anarquismo em Vitória, Canadá, que envolveu protestos e ações
diretas, a manutenção de um centro/livraria social, o funcionamento de
um círculo de leitura e a condução de uma feira anual do livro.

Até 2020, apoiar a luta do povo indígena Wet'suwet'en para afirmar sua
soberania tinha sido, por algum tempo, o foco de anarquistas de Vitória.
O território Wet'suwet'en encontra"-se dentro da província canadense da
Colúmbia Britânica (C. B.), e os Wet'suwet'en nunca assinaram acordos
com nenhuma entidade colonial: nem com os britânicos durante os anos do
primeiro contato (anos 1800), nem com o governo federal do Canadá
(fundado em 1867), ou com a província da C. B. (fundada como colônia da
Colúmbia Britânica em 1868 e que se uniu ao Canadá em 1871).\footnote{Mapa
  de território do \emph{Office of the Wet'suwet'en}. Disponível em:
  \emph{http://www.wetsuweten.com/territory/}} Em 2009, o clã Unist'ot'en
(C'ihlts'ehkhyu) --- um dos cinco clãs dentro do sistema de autogoverno
dos Wet'suwet'en\footnote{Desenho do sistema de clãs, do \emph{Office of
  the Wet'suwet'en}. Disponível em:
  \emph{http://www.wetsuweten.com/culture/clan-system/}} --- resolveu
construir um posto de controle no caminho dos dutos projetados que
deveriam transportar o chamado gás ``natural'' (fraturado) do norte da
Colúmbia Britânica até a costa do pacífico, onde seria liquefeito para
exportação em um navio"-tanque. Em 2010, o posto de controle tornou"-se um
acampamento com construções para habitação e atividades
culturais.\footnote{Cf. \emph{https://unistoten.camp/}.} Eu e outros
anarquistas de Vitória apoiamos a reivindicação dos Unist'ot'en pela
soberania territorial por meio de arrecadação de fundos, organização de
palestras realizadas por Unist'ot'en e levando pessoas ao acampamento
para ajudar com as construções. Os anarquistas estiveram envolvidos no
apoio aos Wet'suwet'en daí em diante (recebendo palestrantes, visitando
o acampamento, auxiliando com a promoção de mídia e arrecadação de
fundos, comprometendo"-se com as ações de resistência, etc.), ao mesmo
tempo que os Wet'suwet'en construíam fortes redes de solidariedade com
indígenas e não indígenas em todo o país.

Em outubro de 2018, os governos da Colúmbia Britânica e do Canadá
coanunciaram que um duto pertencente a um consórcio de investidores, o
CoastalGaslink, seria construído através do território Wet'suwet'en no
caminho onde fora construído o acampamento Unist'ot'en. Membros do clã
Wet'suwet'en Gidimt'en responderam construindo o segundo posto de
controle para dificultar o acesso ao trajeto projetado para os
dutos.\footnote{Cf. \emph{https://www.yintahaccess.com/}} A vigilância
sobre os Wet'suwet'en e seus apoiadores pela \textsc{rcmp} (Royal Canadian
Mounted Police, a força policial do governo federal) se intensificou, e
a polícia começou a reunir forças no final de 2018. Em janeiro de 2019,
a \textsc{rcmp}, altamente armada, invadiu o posto de controle do clã Gidimt'en e
prendeu 14 pessoas. Protestos irromperam em Vitória, pelo Canadá, e
internacionalmente (mais de 70 cidades). Dez dias depois da invasão,
anarquistas de Vitória organizaram uma conferência com o anarco"-indígena
soberanista Mel Bazil (Gitxsan e Wet'suwet'en), um dos cofundadores do
acampamento Unist'ot'en, em Vitória. Em fevereiro, participamos da
organização de uma segunda conferência com Molly Wickham, uma porta"-voz
que fora detida no posto de controle do clã Gidimt'en. No dia 29 de
fevereiro, o posto foi reocupado pelos Gidimt'en. A \textsc{rcmp}, então, criou
um destacamento no território Wet'suwet'en para manter a repressão
contínua dos defensores indígenas enquanto a construção da
CoastalGasLink estava em curso. Em abril, as acusações contra aqueles
detidos no posto de controle do clã Gidimt'en foram retiradas, mas a
\textsc{rcmp} continuou a reprimir e espionar os defensores das terras
Wet'suwet'en durante o verão e o outono de 2019, enquanto os dutos eram
introduzidos no território Unist'ot'en.

Antecipando os confrontos iminentes, em novembro, eu e outras pessoas
organizamos um novo grupo, Wet'suwet'en Solidarity Victoria
(Solidariedade aos Wet'suwet'en de Vitória), para realizar manifestações
e ações diretas em Vitória.\footnote{Ver nossa página no facebook,
  Wet'suwet'en Solidarity Victoria.} Quando a \textsc{rcmp} lançou um cerco
prolongado e uma ofensiva contra os acampamentos Wet'suwet'en's, em
meados de janeiro de 2020, protestos e ações diretas disruptivas
irromperam por todo o Canadá. No dia 11 de fevereiro, a \textsc{rcmp} invadiu o
último posto de controle e forçosamente prendeu defensores da terra no
acampamento Unist'ot'em, e a resistência pelo Canadá intensificou"-se
exponencialmente. Em Vitória, envolvi"-me em uma série de ações entre
janeiro"-fevereiro"-março, trabalhando com uma extensa aliança de não
indígenas e um grupo de militantes indígenas que, periodicamente,
ocupavam os degraus da frente cerimonial do legislativo provincial.
Assumindo a liderança de Wet'suwet'en e de outros indígenas que tinham
conexões diretas com os Wet'suwet'en, milhares de pessoas participaram
de uma série de manifestações, bloqueios em ruas e pontes, ocupação de
gabinetes, suspensão de faixas e outros distúrbios que levaram o governo
da Colúmbia Britânica a um desconcerto total (em certo ponto, fechamos
todo o governo provincial por um dia). Em 27 de fevereiro, os governos
da C. B. e do Canadá concordaram em iniciar um diálogo com os chefes
hereditários dos Wet'suwet'en e, três dias depois, em primeiro de março,
anunciou"-se um acordo em que se reconheceu a soberania territorial e o
sistema tradicional de autogoverno dos Wet'suwet'en's. O projeto dos
dutos, no entanto, foi excluído desse acordo. Com a submissão do acordo
para discussão e aprovação do povo Wet'suwet'en, o ímpeto diminuiu. Em
Vitória, indígenas que ocupavam os degraus do legislativo (amparados por
um acampamento com cozinha e mais de uma centena de apoiadores todos os
dias)\footnote{O coletivo Victoria Anarchist Bookfair possui uma grande
  tenda que montamos nos degraus do legislativo, junto com outros
  abrigos improvisados.} decidiram realizar uma última ação em que se
encontraram com um ministro do governo e se recusaram a deixar o prédio
do legislativo, levando à sua prisão no dia 4 de março (foram soltos
rapidamente). Levantamos acampamento em frente ao legislativo\ldots{} e então
veio a \textsc{covid}"-19.

De longe, estávamos cientes da quarentena da \textsc{covid}"-19, primeiro em Wuhan
e, no final de janeiro, por toda a província de Hubei. A mídia
acompanhou a chegada do vírus no Irã em meados de fevereiro, sua
disseminação pela Coreia do Sul e o dramático \emph{lockdown} no norte
da Itália no início de março. Com a declaração de uma pandemia global,
no dia 11 de março, e a disseminação exponencial do vírus nos Estados
Unidos, casos também surgiam no Canadá. Em resposta, governos
provinciais declararam estados de emergência, a fronteira com os Estados
Unidos foi fechada pelo governo do Canadá, para tudo que não fosse de
deslocamento essencial (bens e serviços), e voos para o Canadá foram
rigorosamente restringidos e depois suspensos. Em 17 de março, o governo
provincial da Colúmbia Britânica fechou universidades, escolas,
gabinetes do governo e a maior parte das operações comerciais (com
exceção daquelas consideradas essenciais). Todas as aglomerações
públicas foram banidas para evitar o contato físico. A mobilização de
pessoas para mais ações nas ruas em apoio aos Wet'suwet'en tornou"-se
praticamente impossível.

Vitória não é somente um núcleo de militância. Nós mantemos um espaço
social de voluntários e sem fins lucrativos a partir da venda de livros
novos e usados (Camas Books, fundado em 2007), um círculo de leitura que
se encontra duas vezes por semana (Victoria Anarchist Reading Circle,
fundado em 2005), e organizamos uma Feira de Livro Anarquista de Vitória
anualmente, sempre no mês de setembro (fundada em 2006). Quando foi
declarada a emergência sanitária, nosso sentido de
solidariedade/responsabilidade entrou em ação e fechamos o Camas Books
no dia 19 de março. O coletivo Camas foi então confrontado em como
sustentar o projeto pelo tempo necessário, até que as restrições
sanitárias fossem atenuadas e que pudéssemos abrir a loja novamente. O
fato de estar fechado não significou que o Camas tivesse uma pausa no
pagamento de aluguel ou outros gastos. Ao mesmo tempo, o número de
voluntários caiu drasticamente, uma vez que muitos tinham empregos
precários e/ou eram estudantes que se viram forçados a lidar com suas
questões pessoais. Ao fim, sobramos eu, minha companheira, Kim Croswell,
e mais seis outros para manter o Camas funcionando. Nesse contexto, a
criatividade engenhosa intensificou"-se. Primeiro, Kim trabalhou com
outro voluntário para renovar nosso website (camas.ca) e facilitar as
vendas \emph{on"-line}. Dois outros voluntários providenciaram entregas
por bicicleta para os compradores, o que se encaixou muito bem em nossos
princípios ecológicos radicas. Kim também se atirou na fabricação de
máscaras para vendas \emph{on"-line} ou para distribuição gratuita para
os povos indígenas que necessitassem (remessas foram enviadas por
correio para várias comunidades). A venda de máscaras mostrou"-se popular
e começamos a vender livros em um ritmo constante. Simultaneamente,
recebemos doações de livros usados e dinheiro, sem que estas fossem
solicitadas. Em resumo, descobrimos que a reputação do Camas enquanto
uma presença duradoura era valorizada e, em tempos de adversidade, foi
extremamente prazeroso perceber o quão profundamente nossas raízes
anarquistas foram cravadas em Vitória. Reabrimos o Camas por meio
período em 16 de maio e, quando a notícia se espalhou, inúmeras pessoas
pararam para comprar livros, zines, máscaras e camisetas, mostrar apoio
ou trazer doações. Gradualmente expandimos o horário de funcionamento à
medida que antigos voluntários voltaram e novos se inscreveram para se
juntar a nós (temos um rigoroso processo de inscrição e verificação para
eliminar os voluntários em potencial, que não estejam em acordo com
nossa declaração de missão ou não sejam, de alguma forma, receptivos ao
envolvimento em um projeto social anarquista tocado coletivamente). No
momento deste texto, já estávamos praticamente de volta ao horário
normal, com a aplicação de alguns protocolos em relação a quantas
pessoas poderiam estar no Camas de uma vez, higienizador para as mãos na
porta da frente para a limpeza antes de entrar, uma barreira de acrílico
na mesa do caixa, separando os voluntários dos clientes, e uma política
segundo a qual o voluntário poderia usar uma máscara se quisesse (e a
maioria usa). O coletivo agora olha para o futuro, quando poderemos mais
uma vez realizar exibição de filmes, grupos de discussão, lançamento de
livros e encontros de ativistas e anarquistas.

Com o Camas fechado e os protocolos de saúde em relação ao
distanciamento social, as discussões quinzenais cara a cara do Victoria
Anarchist Reading Circle (\textsc{varc} --- Círculo de Leitura Anarquista de
Vitória) não foram mais viáveis. O \textsc{varc} mantém seu próprio website
(victoriaanarchistreadingcircle.ca), no qual são postadas as próximas
leituras para discussão. Ao longo dos anos, acumulamos uma extensa lista
de contatos de e"-mails e, quando novas leituras são disponibilizadas
\emph{on"-line}, as pessoas recebem um alerta; dessa maneira, se
quiserem, podem participar do próximo encontro. Temos, também, uma
política aberta em relação aos que chegam sem aviso: qualquer um e todos
podem participar, desde que tenham lido o texto daquela semana e estejam
prontos a tomar parte em uma discussão respeitosa (do contrário, é
solicitado à pessoa que saia). Em todos os encontros, a pessoa que
sugere a leitura daquela semana coordena a discussão e, ao fim do
encontro, as pessoas propõem qual será a próxima leitura. Dessa forma a
responsabilidade de ``organizar'' é alternada, e a variedade de temas é
garantida. Ao menos, é assim que as coisas funcionam em princípio. Na
prática, a participação pode aumentar e diminuir drasticamente.
Consequentemente, um ``núcleo'' de participantes tende a sugerir as
leituras, escaneá"-las, postá"-las no website e coordenar as discussões.

Em qualquer evento, dado seu hibridismo operacional, o \textsc{varc} parecia
estar bem"-preparado para levar as discussões \emph{on"-line}. Anunciamos
a implementação virtual dos encontros do \textsc{varc}, utilizando uma plataforma
de conferência de vídeo totalmente criptografada (que não coletasse o
número de endereço do \textsc{ip}). As leituras foram postadas normalmente e um
\emph{link} disponibilizado para que as pessoas pudessem participar da
discussão agendada previamente. O processo parecia familiar, mas não
era. Alguns tinham uma capacidade limitada no computador e só conseguiam
participar por mensagens de texto. Alguns novos participantes decidiram
não ficar visíveis, o que era desconcertante para os visíveis. Tínhamos
inscrições de pessoas de outras localidades, e isso não era um problema,
mas não estávamos mais partilhando nossas perspectivas pessoalmente. Com
o tempo, ficou claro que encontros cara a cara são inestimáveis para
construir uma comunidade, especialmente quando a sua comunidade está em
fluxo constante: a conferência de vídeo \emph{on"-line} não substituiu
isso. O núcleo do grupo (por volta de 12 pessoas) que estava se
encontrando antes da pandemia forçar o \textsc{varc} \emph{on"-line} sustentou
isso por um tempo, mas a participação foi diminuindo gradualmente. O
\textsc{varc} manteve"-se até o início de setembro, quando decidimos suspender o
círculo até podermos nos encontrar pessoalmente de novo.

A Feira de Livro Anarquista de Vitória (\textsc{vabf}) acontece em meados de
setembro e atrai aproximadamente mil participantes durante os dois dias
(sábado/domingo). Durante a semana que culmina na feira do livro, o
coletivo da \textsc{vabf} geralmente organiza conversas, exibição de filmes,
\emph{performances}, exposições de arte, etc., na Camas Book e em outras
localizações pelo centro de Vitória. A acomodação de editoras
anarquistas, distribuidoras, e a vendas de camisetas, zines,
\emph{bottons}, bijuterias e outros itens de artesãos de orientação
militante são complementadas com uma série de oficinas apresentadas por
indígenas e não indígenas durante o sábado e o domingo (sempre que
possível, as viagens de apresentadores indígenas de fora da cidade são
subsidiadas). Apresentadores de oficinas que vêm de fora (e vendedores
de livros anarquistas) se hospedam com os membros do coletivo \textsc{vabf} e
outras pessoas em nossa comunidade. Dessa forma, a feira de livro
possibilita redes de relações que nos ligam a comunidades anarquistas em
Vitória e além. Ademais, a \textsc{vabf} é um espaço para mobilizar o
envolvimento das pessoas com as lutas ecológicas, sociais e indígenas
locais, assim como causas internacionais (o movimento anarquista em
Rojava, por exemplo). Esse sucesso pode ser avaliado, em parte, pelas
atividades da força policial da província e federal com a aproximação
das Olimpíadas de Inverno de fevereiro de 2010. Organizada na região de
Vancouver, C. B., com o início do revezamento da tocha olímpica,
realizado em 30 de outubro de 2009, em Vitória, a nossa feira do livro
de 2008/2009 mobilizou as pessoas para agitações e protestos em massa,
sob o \emph{slogan} ``Nenhuma Olimpíada em Terra Nativa Roubada'' (esse
\emph{slogan} foi cunhado pelo indígena anarquista Kwakwaka'wakw, Gord
Hill, que teve um papel essencial no movimento \emph{No2010
Olympics}).\footnote{Fomos bem"-sucedidos em prejudicar o lançamento do
  revezamento da tocha, obtendo atenção nacional e internacional. Ver:
  \textsc{no}2010, 1º/11/2009.} A polícia conduziu uma vigilância sobre nossa
feira do livro (assim como da Camas Book e dos ativistas locais
\emph{No2010 Olympics}, incluindo eu e Kim)\footnote{Cf. \textsc{olympic},
  24/08/2009.} e tentou fechar a feira, informando, aos diretores do
espaço em que organizamos o evento desde 2006, que estávamos sendo
observados (soubemos da interação da polícia com os diretores --- que
adotaram uma postura de cooperação --- por meio de um
informante).\footnote{O gerente do espaço tentou dificultar a preparação
  da feira do livro de 2009 e anunciou, posteriormente, que não
  poderíamos mais realizar o evento lá. Mudamos para outro local.}

O envolvimento no coletivo da feira do livro se dá por convite e, antes
que um convite seja estendido, os membros vigentes avaliam a proposta de
novo membro, que papéis ele poderia exercer e como ele poderia se
encaixar com os princípios anarquistas da feira do livro (ou não: nós
rejeitamos algumas propostas de indicação). Em março de 2020, havia
cinco de nós. Naquele verão, uma sexta pessoa do Wet'suwet'en Solidarity
Victoria foi convidada a participar. Conforme ponderávamos a situação em
março, tornou"-se bastante claro que não iríamos realizar uma feira do
livro cara a cara em setembro de 2020, então decidimos fazer a feira
\emph{on"-line}. É aí que o apoio mútuo e a construção de relações
durante as 14 feiras do livro entraram em evidência. Eu contatei o From
Embers\footnote{Cf. https://fromembers.libsyn.com/}, um coletivo de
rádio anarquista sediado em Kingston, Ontário, Canadá, que concordou em
nos ajudar com a produção e veiculação de uma série de \emph{podcasts}
de entrevistas realizadas pelos membros da \textsc{vabf}. Nós então nos
comunicamos com os contatos indígenas e não indígenas por todo o Canadá
e internacionalmente. As gravações foram realizadas e editadas para
veiculação. A entrevista inicial com um membro do From Embers sobre a
história da Feira --- \emph{A Feira de Livro Anarquista de Vitória está
aqui e todos podemos ir (virtualmente)!} --- foi veiculada em 23 de
setembro de 2020. As outras foram ``liberadas'' durante os sete dias (14
a 20 de setembro) na nossa página\footnote{Cf.
  http://victoriaanarchistbookfair.ca/index.php/category/bookfair-2020/}
e no \emph{website} do From Embers.\footnote{Respondendo às últimas
  oportunidades, algumas entrevistas estão disponíveis apenas no
  \emph{website} da \textsc{vabf}.} As entrevistas refletem a amplitude de nossos
contatos e as preocupações em nível local, regional e internacional.

\emph{O porta"-voz do posto de controle Gidimt'en, Molly Wickham, sobre
Resistência antes, durante e depois da pandemia (4 de set.); Kathy
Ferguson sobre as mulheres de Emma Goldman e anarquismo como ``um
movimento do livro'' (1º de set.); John Zerzan sobre sua vida como
anarquista e o estado das coisas nos \textsc{eua} (16 de set.); o ativista
Nuu"-chah"-nulth \& Costa Salish, Queen Sacheen (Ancestral Pride), reflete
sobre sua vida e a soberania indígena (17 de set.); a anarquista Ann
Hansen e a trans ativista Naphtali discutem a (in)justiça da prisão (18
de set.); a teórica do Reino Unido Ruth Kinna (editora de Anarchist
Studies) sobre capitalismo, anarquismo, e os comuns (19 de set.); Sem
tetos na pandemia (20 de set.); a matriarca Ma'amtagila (Kwakwaka'wakw),
Tsastilqualus, da House of Umbas, sobre soberania indígena e rematriação
(20 de set.) e a bloqueadora Suzanne (Metis) sobre a série de ações de
Fairy Creek /Yews convida; um colono anarquista anônimo da costa oeste
reflete sobre a defesa autônoma da floresta de Kaxi:ks (Walbran) a
Be:tadt (Fairy Creek) (20 de set.).}

Além disso, um coletivo hip hop anarquista, \emph{\textsc{rymth}i\textsc{nk}}, doou uma
coleção de canções para \emph{download} no \emph{site} da \textsc{vabf}. Também
promovemos e disponibilizamos um \emph{link} para um painel e discussão
ao vivo organizado pelo Institute for Social Ecology no dia 20 de
setembro: ``Rojava hoje: onde está o movimento agora?''. Finalmente,
qualquer interessado em comprar livros, zines ou camisetas era
direcionado à Camas Books (e, com certeza, houve um aumento nas vendas).
Os \emph{podcasts} foram muito fáceis de fazer, e estamos pensando em
integrá"-los em futuras feiras, assim podemos dar voz a quem não pode
participar de nosso evento, devido aos custos da viagem ou a problemas
na fronteira (todos conhecemos anarquistas impedidos de entrar neste ou
naquele país).

Durante o verão, eu terminei um capítulo de livro sobre três artistas
(ativos em tempos diversos, dos anos 1940 até os anos 2000) que
promoveram modelos inspiradores de economia anarquista por meio das
artes.\footnote{Escrevi a biografia de um desses artistas. Ver Allan
  Antliff, \emph{Joseph Beuys} (London: Phaidon, 2014).} Enquanto
escrevo, me impressiona que, em um tempo em que a economia capitalista
estava em queda livre (necessitando da intervenção do Estado com
resultados variados), nosso sucesso comparativo com a livraria, o
círculo de leitura e a feira do livro também diz muito, economicamente
falando. Essas ``instituições'' (e aqueles que vieram em nosso auxílio)
estabeleceram relativa autonomia do
trabalho-assalariado/venda-para-o-lucro da economia capitalista muito
antes da \textsc{covid}"-19 chegar, e isso foi chave para sua resistência. As
estruturas anarquistas de autogoverno encorajaram o apoio mútuo, a
iniciativa inventiva, a solidariedade e a responsabilidade coletiva:
``modos de ser'' que se provaram inestimáveis para a sustentabilidade
durante tempos confusos. Isso criou um imenso contraste com a economia
predatória em profundo declínio, sublinhando que nossos projetos
constituem uma ``ruptura no capitalismo\ldots{} espaços na vida cotidiana em
que o capitalismo \emph{não} está presente'' (\textsc{shannon} et al., 2012, p.
5).


\begin{bibliohedra}
\tit{NO2010} Victoria Statement on Torch Relay Disruption. In:
\emph{Anti"-Olympics Archive}, 01/11/2009. Disponível em:
\emph{http://vancouver.mediacoop.ca/olympics/no-2010-victoria-statement-torch-relay-disruption/5759}.
Acesso em: 27/08/2020.

\tit{OLYMPIC} Cops Harass Victoria Activists. In: \emph{Anti"-Olympics
Archive}, 24/08/2009. Disponível em:
\emph{http://vancouver.mediacoop.ca/olympics/olympic-cops-harass-victoria-activists/5801}.
Acesso em: 27/08/2020.

\tit{SHANNON}, D.; \textsc{nocella ii}; A. J.; \textsc{asimakopoulos}, J. Anarchist Economics: A
Holistic View. In: \emph{The Accumulation of Freedom: Writings on
Anarchist Economics.} Oakland, \textsc{ca}: \textsc{ak} Press, 2012.
\end{bibliohedra}


\chapterspecial{O inimigo invisível\footnotemark}{}{André Liohn}

\footnotetext{Tradução do francês por Martha Gambini.}

\epigraph{Nós estamos aqui para ajudar caralho! Se vocês preferirem, nós o
levamos! O levamos como um saco de batatas jogado na ambulância\ldots{} para
nós será muito melhor porque poderemos ficar na ambulância, no calor, e
fumar um cigarro enquanto ele morre! Vocês estarão melhor, mas ele, vai
estar muito pior.}{}

\noindent{}Leves calafrios, inesperado cansaço, fastidiosa dor por todo o corpo e,
finalmente, início de uma inexplicável falta de ar piorada pela tosse
seca e febre aguda que se seguiram. O mal"-estar de Giuseppe
Guardabascio, um homem tipicamente baixo como as pessoas daquela região,
mas, por toda a vida, um homem muito forte, agricultor aposentado de 84
anos, havia começado por volta do dia 21 de março, pouco depois que o
primeiro caso de \textsc{covid}"-19 foi detectado em uma mulher de 59 anos de
idade na pequena cidade de Ariano Irpino, região da Campânia, Sul da
Itália, onde ele vive com o filho, a nora e suas duas netas.

Novas doenças contagiosas são assustadoras porque são desconhecidas e
imprevisíveis. Os noticiários do mundo todo já praticamente não tratavam
de outro assunto, situação vista poucas vezes antes, as manchetes e
debates em qualquer jornal, canal, revista e nas mídias sociais tratavam
quase que unicamente de assuntos relacionados à nova doença,
principalmente, o crescimento diário de infecções e mortes. Apesar dos
repetidos esforços de suas netas em tentar mantê"-lo em casa, teimoso,
como as duas o definiam, porém, mais que tudo, forte apesar da idade e
independente pela clareza de suas razões, Giuseppe nunca se preocupou
com os noticiários e não aceitava as recomendações de sua família. Todos
os dias, fazia pequenas visitas à loja de equipamentos agrícolas onde
podia jogar conversa fora com velhos amigos, de tempo e, assim como ele,
também de idade.

Os sintomas que surgiram fracos, rapidamente pioraram. A febre, condição
que Giuseppe raramente tinha enquanto saudável, subiu para quase 40
graus, jogando"-o imediatamente na grande cama onde dormia sozinho, desde
que sua esposa havia falecido há mais de 10 anos. Durante a noite, mesmo
que ininterrupta, depois de todos os dias naquela condição, sua tosse já
quase não emitia qualquer som, apenas um chiado fino que se misturava
aos seus também incessantes lamentos.

Atentas, mas emocionalmente fragilizadas, amedrontadas pelo pior que
imaginavam estar acontecendo com seu avô e genro, suas netas e nora
praticamente não podiam mais dormir durante a noite, revezavam"-se como
podiam nos cuidados do avô doente, evitando acordar seu pai, o filho
mais velho de Giuseppe que, por ser um homem bastante comum, quieto e
fechado, não sabia bem como lidar com a dor do pai e que, mesmo com o
isolamento sanitário, imposto a todos os moradores da cidade pelo
governo local e implementado pelo próprio exército nacional, era ainda o
único que saía para trabalhar e precisava acordar cedo todos os dias
para cuidar sozinho das terras que não podiam deixar de ser preparadas
para o plantio de feno, que aconteceria em algumas semanas.

Em meio às alucinações que tinha, meio dormindo ainda meio acordado,
Giuseppe não tinha mais forças para se levantar nem mesmo para ir ao
banheiro. Sua respiração, que já havia se tornado curta e sofrida,
ficara cada vez mais pesada, assim como a dor que dizia sentir em todo o
seu corpo e a responsabilidade de mantê"-lo vivo. Exaustas, temendo que o
novo coronavírus, causador da \textsc{covid}"-19, recém"-chegado da China, até
aquele momento, já responsável pela morte de 76.190 pessoas ao redor do
mundo e 14.681 pessoas somente na Itália, tivesse o infectado e certas
de que Giuseppe poderia morrer se continuasse naquelas condições, na
noite de 3 de abril, suas netas e nora ligaram para o número 118 de
emergência e, depois de minutos esperando para serem atendias, do outro
lado da linha, uma atendente perguntou sobre a temperatura corporal,
respiração e o apetite de Giuseppe. A partir das respostas, que deixaram
claro o quanto a situação era grave, concluiu não ser necessário ou
possível o envio de uma equipe médica à casa, pois não havia ambulâncias
ou leitos disponíveis no hospital local naquele momento.

A atendente recomendou que Giuseppe se hidratasse, descansasse e
mantivesse o máximo de distância de todos. A noite e o dia seguinte
foram insuportáveis para todos, Giuseppe já não se comunicava
claramente, e as três mulheres, consumidas pelo esforço e tristeza
daqueles dias e noites, começavam elas também a demonstrar pequenos
sintomas que poderiam indicar que todos na casa pudessem estar doentes.
Durante a tarde do dia seguinte, decidida que não poderiam mais esperar
por ajuda, Ângela, a neta mais velha de Giuseppe, voltou a entrar em
contato com o serviço médico que, desta vez, enviou uma ambulância com
um médico e enfermeira vestidos com macacões brancos e luvas cirúrgicas,
óculos brilhantes que cobriam quase todo o rosto, deixando apenas espaço
para uma grossa máscara que protegia suas bocas e narizes.
Imediatamente, o médico e a jovem enfermeira mediram sua temperatura e
capacidade pulmonar. O resultado foi alarmante.

Os instrumentos indicavam que a febre de Giuseppe havia pelo menos
temporariamente baixado, os medicamentos mais comuns, como paracetamol,
haviam feito algum efeito, mas sua capacidade respiratória variava em
níveis próximos a um colapso de seus órgãos vitais. Seu pulmão não era
capaz de oxigenar seu sangue que indicava apenas 82\% de saturação, mas,
ainda assim, os médicos foram informados que não poderiam transferir o
agricultor para o hospital. Depois de uma rápida conversa com a central
que orienta as ações dos médicos depois de intervenções como aquela, o
médico, evidentemente frustrado, foi informado que não havia leitos
disponíveis para pacientes com suspeita de \textsc{covid}"-19 em toda a região.

O hospital da cidade de Ariano Irpino, Sant'Ottone Frangipane, sede do
distrito sanitário n. 1, compreendendo 29 municípios para uma população
total de 87.993 habitantes, havia sido recentemente reaberto, depois que
dezenas de seus funcionários haviam sido contaminados pelo mesmo vírus,
deixando, inclusive, o próprio médico infectologista internado em
condições graves no maior hospital da região, na cidade de Nápoles.
Apesar de o hospital estar praticamente vazio, além de uma tenda
inflável, montada na área externa do estacionamento, onde alguns
enfermeiros ficavam de prontidão, apenas 5 leitos haviam sido preparados
para receber pacientes demonstrando sintomas da nova doença, e todos
estes já estavam ocupados há dias.

Questionada, pelo médico, a caminho da porta de saída, se já havia
administrado oxigênio em algum paciente, Ângela respondeu insegura que
sim e foi então instruída a buscar um cilindro de oxigênio para seu avô.
Foi também informada que uma das farmácias locais poderia fornecer"-lhe
um cilindro de 20 litros que seriam suficientes para garantir que o
sangue de Giuseppe pudesse ser oxigenado artificialmente pelo menos até
o dia seguinte. Na farmácia, confusa com toda a situação e depois de
compartilhar com os funcionários a situação em que seu avô se
encontrava, ela foi informada, com severidade, que o velho poderia
morrer se não fosse rapidamente transferido para um hospital.

Ariano Irpino é uma cidade sonolenta e melancólica, com poucas
indústrias e pequeno comércio, é um pequeno centro agrícola com 22.000
habitantes que plantam feno entre montanhas a uma hora de carro ao lado
leste da costa mediterrânea, onde está a cidade de Nápoles. O passatempo
preferido de seus moradores, as poucas crianças, alguns jovens, muitos
adultos e os tantos velhos e idosos, é caminhar em passos infinitamente
lentos em torno das ruínas de um antigo forte militar normando,
carinhosamente chamado de ``Castello'', construído por volta do oitavo
século depois de cristo.

Quando o governo italiano impôs um toque de recolher nas províncias mais
duramente atingidas no Norte, em 8 de março, milhares de pessoas
provenientes do Sul que se encontravam naquela região viajaram
repentinamente para suas cidades de origem. Foram trabalhadores,
estudantes e pessoas que buscaram refúgio com suas famílias. Muitos
levaram consigo o vírus. Um jovem que já apresentava leves sintomas
quando deixou a região da Lombardia, após retornar a Ariano Irpino,
desconhecendo ou mesmo ignorando os riscos de difundir o contágio, foi a
uma festa de carnaval onde estavam pelo menos outras 500 pessoas. Um
médico local, que também havia estado no Norte a passeio com sua esposa,
sem demonstrar sintomas, mas já portador da doença, infectou tantos de
seus colegas, médicos, enfermeiros e técnicos hospitalares, que o
hospital local teve de ser fechado pelo período de um mês para ser
higienizado. Logo, o ``Castello'', local em que os arianeses se
encontram e discutem as alegrias e as fortunas da vida, foi um dos
primeiros locais públicos onde a convivência social foi proibida para
evitar a propagação do vírus. Então veio o toque de recolher.

Durante os dias de isolamento mais severos, os ``vecchi cinghialoni''
ou, os ``os velhos porcos do mato'', definição jocosa, irônica, mas
subliminarmente depreciativa, usada pelo governador da região da
Campânia, Vincenzo de Luca, para descrever, segundo ele, os
irresponsáveis. Mulheres e homens vestidos em apertadas calças de yoga
ou aquelas ``pula brejo'', que, apesar da crise mundial, dos milhares de
contaminados, da falta de leitos e de equipamentos hospitalares
necessários, de uma vacina ou mesmo de um tratamento efetivo e, por
isso, apesar das já milhares de mortes, em apenas míseras duas semanas
de crise, e da explícita ordem de recolhimento vigente, tentando evitar
isso tudo, aqueles entre estes que insistiam em sair nas ruas, passear
pelos parques ou pela orla, fazer \emph{jogging}, vivendo em seus
próprios mundos, enfim, qualquer pessoa que fosse pega fora de sua casa
sem um motivo válido, como ir ao supermercado ou farmácia, ir ao
hospital ou ter que trabalhar em funções definidas como ``tarefas
socialmente críticas'', no melhor dos casos, deveria cumprir 15 dias de
isolamento domiciliar forçado, e, se isso não fosse respeitado,
cumpriria 3 meses de prisão e arcaria com uma multa de, no mínimo, 200
euros. E, então, no dia 15 de março, sem nenhum aviso prévio e por um
decreto anunciado pelo governador da região, toda a cidade de Ariano
Irpino e seus moradores foram isolados do mundo exterior.

O medo que até então era interno e pessoal se espalhou quando,
inesperadamente, um cidadão comum, um homem de meia idade, gordo e
malvestido, com voz áspera, aguda e nasal, sozinho, num pequeno carro
cinza equipado com autofalantes de baixa qualidade, mas estridentes,
cortantes como as cornetas de vendedores de pamonha, improvisadamente
presas por cordas sobre o teto, reproduzindo uma fala alarmante de tom
funesto, decidiu por conta própria dirigir lentamente pelo centro e
outros bairros pregando que a situação era muito grave, que as crianças
deveriam ser mantidas dentro de casa, que os idosos deveriam permanecer
em suas casas, que toda a população deveria permanecer dentro de suas
casas e que os sintomáticos não deveriam, sob qualquer condição, ir por
conta própria ao hospital, mas procurar ajuda médica através do telefone
com seus médicos de família. Imediatamente, as ruas ficaram vazias e,
logo mais, soldados do exército e forças policiais de outras partes do
país se enfileiraram nas estradas de acesso e saída da cidade.

Com seus braços cruzados e com seu corpo recolhido, triste e impotente,
Ângela procurava acalmar sua mãe dizendo da sala vizinha ao quarto de
seu avô, lugar de onde ela mesma chorava sozinha, que alguma solução
seria inevitavelmente encontrada. Sua mãe, magra e visivelmente exausta,
cuidando de seu sogro, tossia secamente e sem força, consecutivamente
tentando esconder sua boca entre a dobra de seu braço fino. Seu olhar
era caído e seu nariz estava ligeiramente vermelho, úmido e com a pele
levemente ferida. O ar dentro da casa era estático e quente, grosso,
quase palatável e perceptível na pele de quem ali entrasse. A televisão
permanecia ligada mesmo que muda. O clima montanhoso continuava muito
frio e muito úmido, mesmo que não houvesse mais neve pelas ruas ou pelos
campos, a não ser os poucos e pequenos montes brancos que restavam nos
locais onde o sol, todavia fraco, ainda não chegava. Entre as lágrimas,
gotas de suor escorrendo pelo rosto faziam Ângela parecer ainda mais
frágil e confusa, seu medo era que a condição de seu avô piorasse, caso
as janelas ou as portas da casa fossem abertas para que o ar pudesse se
renovar.

O oxigênio paliativamente recomendado pelo médico na noite anterior não
havia sido capaz de melhorar a condição de Giuseppe até aquela manhã.
Ele continuava sentindo a mesma lancinante dor física, o mesmo obstante
mal"-estar que o impedia de sequer poder se sentar na beira da cama com
suas próprias forças e, sempre que dormia, demonstrava continuar penando
com as mesmas alucinações, falando de forma confusa e angustiada com
alguém presente apenas em seus sonhos. Sua família já não tinha mais
condições emocionais, físicas ou de saúde para atendê"-lo e, num
desolamento que se aprofundava cada vez mais, as três mulheres decidiram
chamar novamente pela ajuda médica, mas, quando chegaram, mais uma vez,
com o paciente mal conseguindo respirar, os médicos se recusaram a
levá"-lo para um hospital.

Dessa vez, o novo médico em serviço, menos atento ou preocupado com a
condição de toda a família e com evidente pouquíssima aptidão para o
relacionamento direto com o público, fez desabar sobre as três quase
moribundas um inesperado e incompreensível tsunami de acusações contra
tudo e todos. Mesmo com o pulmão seriamente comprometido, depois que
essa equipe decidiu aumentar fortemente a carga de oxigênio administrada
artificialmente através de suas narinas, a saturação sanguínea de
Giuseppe melhorou chegando a críticos, porém não mais desastrosos 90\%.
Com esse argumento em mãos e sem oferecer outras alternativas, sentado
na última cadeira ao lado da cabeceira da longa mesa de jantar onde a
família Guardabascio sempre se reuniu, protocolarmente paramentado com
todo o equipamento de proteção hospitalar, macacão médico para ambientes
contaminados, inteiriço de proteção biológica nível 6, com botas e gorro
branco, largos óculos de proteção ocular e facial, máscara cirúrgica
N95, com filtro, capaz de filtrar pelo menos 95\% das partículas virais
existentes no ar, mas constrangedoramente imprópria para permitir que
sua voz pudesse ser ouvida com clareza, fazendo"-o se mover e ouvir como
se fosse um \emph{alien} distante e imune ao ambiente em que se
encontrava. Imperativo, porém, distanciando"-se da responsabilidade que
tinha sobre a recomendação que fazia, o médico definiu que Giuseppe não
deveria ser levado para o hospital naquele momento, pois caso fosse,
imediatamente seria enviado para casa, pois, com os hospitais
superlotados, os médicos de plantão não receberiam casos que, como o
dele, ainda não estavam próximos ao terminal.

Descontentes com a situação e ainda mais confusas sem saber o que
decidir, se podiam ou deveriam aceitar que o doente não fosse
hospitalizado, as três demonstraram sua insatisfação, que foi
agressivamente repelida com ameaças vindas do médico que as acusava de
terem votado mal nas últimas eleições, de estarem pensando em si mesmas
tentando se livrar do avô e de, finalmente, virem a ser as responsáveis
pela sua morte caso ele viesse a falecer durante o tempo de deslocamento
até o hospital.

Tentando encurtar intencionalmente sua permanência na casa, o médico
ensaiava sua saída pela porta que ainda se mantinha fechada. Abria"-a,
deixando metade de seu corpo fora e metade de seu corpo dentro, enquanto
dizia que o velho seria transportado como um saco de batatas dentro da
ambulância. Dizia também que, entre elas, uma deveria assinar um
documento comprovando que a família estava de acordo com a decisão, algo
que foi imediatamente negado por todas, apesar de todo o constrangimento
e aflição.

O clima que, desde o início, não era amigável, piorou com a recusa da
família em assinar o documento exigido pelo médico:

--- Vocês me fazem gastar toda a minha energia e boa vontade. Caralho! Eu
estou aqui para ajudar vocês. Estou perdendo tempo fazendo vocês
entenderem como lidar da melhor maneira possível com essa situação,
gritava o médico.

E continuava.

--- Posso levá"-lo comigo como um saco de batatas, mas com licença,
tentem se controlar, essas explosões emocionais me deixam nervoso --- e
sugeriu que voltaria no dia seguinte para verificar as condições de
Giuseppe, ou isso, ou então, pegá"-lo, transportá"-lo até um hospital e
permanecer ali por horas dentro da ambulância, fumando um cigarro
enquanto esperavam por um leito disponível.

--- Querem assinar ou não? Insistia o médico.

--- Não, respondeu a nora, já praticamente em prantos, pedindo conselhos
às suas duas filhas, ambas, tão cansadas, confusas e emocionalmente
abaladas quanto ela.

--- Eu não aguento mais. Mama Mia, exclamava o médico dizendo que não
tinha mais tempo para perder e que tinha outros pacientes precisando de
ajuda e esperando por sua visita.

--- Ângela, agora você tem que usar a cabeça! Completou o médico.

--- Você acha que ele vai melhorar se eu o levar daqui? Acho que ele vai
piorar, disse o médico, girando suas costas e caminhando de volta para o
quarto onde Giuseppe permanecia deitado em sua cama.

--- Nonno! O que o senhor prefere? Prefere ser levado para o hospital ou
prefere estar aqui com a família? Perguntou o médico ao mesmo tempo em
que ajudava o velho quase inconsciente a se sentar na beira da cama;
que, depois de tudo, disse que preferia ficar em casa e que assinaria
ele mesmo o documento assumindo a responsabilidade por qualquer coisa
que pudesse acontecer consigo.

--- ``Bravo, Bravo'', exclamou o médico antes de sair, enquanto o velho
doente finalmente pegou a caneta e, tremendo, assinou o papel.

A iminente escassez de ventiladores durante uma onda de infecções virais
evoca a cena no romance de William Styron, de 1979, \emph{A escolha de
Sophie}. Ao chegar a Auschwitz, Sophie, interpretada no cinema, em 1982,
por Meryl Streep, uma jovem mãe católica polonesa deve escolher qual de
seus dois filhos seria gaseado imediatamente e qual teria permissão para
viver. A decisão a perseguirá pelo resto de seus dias.

Assim como Giuseppe Guardabascio, na Itália, milhares de pessoas doentes
foram deixadas em suas casas durante o pior período da pandemia; um
número ainda desconhecido dessas pessoas morreu sem ter recebido do
Estado a ajuda de que elas precisavam. Naquele momento de incertezas
científicas e processuais, de grande demanda de hospitais, o acesso a
equipamentos médicos, especialmente respiradores, que eram vitais para
os casos mais graves como o de Giuseppe, era excepcionalmente escasso e
em situações limites como as vividas na região Norte do país,
principalmente nas cidades de Brescia, Bergamo e Cremona, médicos
intensivistas foram forçados a decidir quais pacientes deveriam ou não
ser salvos. ``Decidimos quais pacientes salvar pela idade e pelas
condições de saúde. Como em todas as situações de guerra''. Testemunho
assustador do médico Christian Salaroli, 48, anestesista do hospital
Papa Giovanni \textsc{xxiii} em Bergamo. Um médico civil que se via na linha de
frente de uma guerra contra o que se definia como um ``inimigo
invisível''.

Ângela é cortês com as pessoas. Como a maioria dos italianos e
habitantes de Ariano Irpino, é católica, é também tímida e romântica e,
explicitamente, muito afetuosa com as pessoas de sua família. Ela tem 27
anos de idade e um namorado com quem passa muito tempo. Estudou ciências
da educação e trabalha como \emph{baby"-sitter} enquanto se prepara para
prestar concursos públicos, e seu objetivo é lecionar nas escolas
públicas da Itália. Sua família tem posses, mas não é rica, e ela e sua
irmã mais nova, Laura, são as primeiras da família Guardabascio a terem
frequentado e concluído um curso universitário.

Até então, viveu toda sua vida na mesma casa em que seus avós e descreve
seu avô como um bom homem, que definitivamente não merecia ser ignorado
num momento tão delicado. Seu pior temor, naquele momento, era não poder
ajudá"-lo e, também, talvez ainda pior que isso, não poder estar presente
para ajudar sua mãe que, mesmo não sendo filha de Giuseppe, o havia
sempre amado como um verdadeiro pai.

Na última noite, depois que os médicos o haviam praticamente abandonado,
assistindo à deterioração da grave condição de sua saúde, sozinhas, as
duas permaneceram ao lado do avô doente até o dia clarear. Além de
medicá"-lo e cuidar para que se alimentasse, com muito esforço físico,
precisaram carregá"-lo até o banheiro onde ele precisou ser limpo e,
então, não com menos esforço físico, novamente carregá"-lo até sua cama.
Em exasperação, sentindo ela própria alguns dos sintomas e dores que
poderiam ser da \textsc{covid}"-19, como se estivesse caindo, apoiando seu ombro
esquerdo na parede ao lado da cama em que o avô delirava, repetindo
palavras sem um sentido claro, Ângela começou a chorar com dissabor.
Também visivelmente abatida, mas concentrada no que fazia, sua mãe
tentava girar o corpo pesado do velho que já praticamente não respondia
a qualquer comando. Com um pequeno termômetro de vidro e mercúrio em
suas mãos, tentava medir sua temperatura. Giuseppe estava com 39 graus
de febre e assim permaneceu durante todo o dia até que, novamente, uma
nova equipe médica fosse chamada.

A nova equipe médica fez apenas os exames rápidos e protocolares,
algumas poucas perguntas aos familiares e, em poucos minutos depois, já
havia conseguido acomodar Giuseppe numa maca para transferi"-lo
imediatamente até o hospital local da cidade. Um paciente havia morrido
exatamente no mesmo tempo em que a ambulância havia se deslocado para
atender ao chamado e, ao chegar no hospital, as duas macas praticamente
se chocaram, enquanto o corpo morto de um saía e o corpo do outro, ainda
vivo, entrava.

Desde o início da crise até hoje, 09 de outubro de 2020, 30 pessoas
morreram em Ariano Irpino em decorrência do vírus e, segundo dados
estatísticos produzidos pelos governos regional e federal, uma nova onda
de contaminação está tomando forma em toda a Itália e, dessa vez, a
região Sul, principalmente a zona urbana em torno à cidade de Nápoles,
local da maior concentração habitacional de toda Europa, corre o risco
de se tornar o grande epicentro dessa nova fase de contágio e mortes.

As fotos e vídeos que produzi acompanhando o caso da família
Guardabascio foram publicados ao redor do mundo, inclusive no Brasil, no
jornal \emph{Folha de S. Paulo} no dia 04/03/2020 (\textsc{liohn}, 04/03/2020).

Desde que Donald Trump, presidente dos Estados Unidos da América, fez
seu \emph{mea"-culpa} em relação à crise do novo coronavírus em meados de
março, admitindo a possibilidade de um desastre sanitário eminente, ele
tentou rotular o \textsc{sars}"-CoV"-2, responsável pelo contágio da doença
\textsc{covid}"-19, como um ``inimigo invisível'' e, desde então, líderes
políticos conservadores ou mesmo os que se denominam ``progressistas'',
representantes da sociedade civil, profissionais da saúde e segurança,
professores, grande parte da imprensa e, finalmente, indivíduos ao redor
do mundo passaram a observar a atual crise sob a fachada artificial de
uma guerra.

Trump usou a frase pela primeira vez durante uma coletiva de imprensa da
força"-tarefa contra o novo coronavírus no dia 16 de março, dizendo:
``Não importa para onde você olhe --- é um inimigo invisível''.
Obviamente, uma metáfora cunhada conscientemente. Dias depois, também
durante uma nova coletiva de imprensa, dessa vez acompanhado de seu vice
e do presidente da Agência Federal de Gestão de Emergências, Trump fez
questão de retornar à frase, adicionando: ``Sou um presidente servindo
em tempo de guerra''.

A associação da atual crise com uma guerra rapidamente se popularizou e
passou a ser explorada amplamente por qualquer um interessado em
antropomorfizar o vírus, um fato concreto, em um ser senciente, com más
intenções, contra a humanidade.

Pode parecer estranho alguém
atribuir inteligência a um emaranhado de ácidos nucléicos dentro de uma
casca de proteína revestida por uma capa de gordura, mas essa estratégia
foi, é e ainda será empregada amplamente. O discurso de guerra serve
para que o Estado justifique sua incapacidade de observar o indivíduo,
algo fundamental para que o encontro coletivo seja formulado, e
literalmente, campos de guerra são o resultado de encontros mal
formulados, feitos de perdas coletivas.

Em julho de 2017, descrevi a última batalha contra militantes do Estado
Islâmico na cidade de Mossul no norte do Iraque da seguinte forma:

\begin{quote}
Apesar do rosto sério e abatido, de sua expressão de terror e
cansaço, quando foi capturado pelos soldados da primeira divisão da
polícia federal, ao sair do rio, o homem estendido, morto a poucos
metros de nossos pés, aparentava estar em boas condições físicas. Ao
contrário dos milhares de civis que haviam deixado aquela mesma parte da
cidade, esqueléticos e doentes, seu corpo estava bem nutrido, seus
braços eram fortes e suas costas eram largas como as de alguém que se
exercita especificamente para isso. Seu cabelo era curto e sua barba
suficientemente longa para que um soldado o tirasse da água puxando"-a.
Ele vestia uma calça escura que chegava até pouco abaixo de seus joelhos
e usava uma camiseta branca com a palavra `Florida' estampada sobre o
desenho de uma paisagem muito diferente daquela em que ele se encontrava
naquele momento. Ele tentava desesperadamente se comunicar com os
soldados que o espancavam, gritavam e filmavam toda a situação com seus
celulares. Gritava repetidamente não ser um combatente; ``Ana last Daesh,
Ana last Daesh Saidi!'' Suas forças e suas esperanças de
sobreviver duraram apenas os poucos metros e minutos que os soldados
iraquianos precisaram para arrastá"-lo, enquanto o despiam, rasgando suas
roupas violentamente. Sem suportar a dor e a humilhação dos golpes e
ameaças que recebia, em total desengano, o homem permitiu que seu corpo
caísse sem nenhuma resistência sobre o chão para ser imediatamente
executado com diversos disparos de fuzis e pistolas.

[\ldots{}]

Uma nuvem de poeira fina se levantou, irritando ainda mais os
olhos e as narinas de todos. Um jovem soldado usando uma bandeira
iraquiana como se fosse uma capa de super"-herói e outros dois
companheiros seus escalavam as ruínas dos prédios destruídos para chegar
no ponto recém"-conquistado. Braços, pernas, cabeças, centenas de corpos
inteiros ou destroçados, decompondo há muito e há pouco tempo estavam
espalhados por todas as partes, por cima e embaixo de um enorme
amontoado de escombros. A cena em que os últimos combatentes do Estado
Islâmico e suas famílias foram mortos só poderia ser comparada a um
inferno. --- A menos de dois metros de tocar as águas do rio Tigres, o
corpo sujo de um bebê vestindo apenas uma camiseta camuflada ainda
estava inteiro jogado debaixo de pedaços de metal e arame farpado. Seu
nascimento e sua morte haviam acontecido há pouco. Sob seus pés, um
retrato colorido de um homem jovem que não aparentava ser iraquiano. O
corpo ressecado pelo sol direto de uma mulher estendida em forma de
crucifixo e os corpos mutilados de pelo menos outras quatro crianças
mais velhas estavam há poucos metros de distância. Seriam eles seu pai,
sua mãe e seus irmãos e irmãs? Como se tivessem marcado um encontro para
caminhar de mãos dadas, a gestação daquela criança e as ofensivas
militares que derrotaram o \textsc{ei} em Mossul haviam começado há exatamente 9
meses antes e, agora, continuam aqueles que nascem e vivem com as
consequências dos crimes cometidos para que o feto da paz possa também
tentar um dia existir no Iraque.
\end{quote}

Associar a atual crise a uma guerra é tão ou mais oportunista que o
próprio vírus que, em si, não tem uma agenda política, religiosa,
racial, social ou histórica. O vírus é um fato concreto que precisa ser
administrado e eventualmente solucionado, mas que, dentro da narrativa
de guerra, serve como álibi para que o Estado controle, não apenas as
ações das pessoas, mas, finalmente, tutele o próprio sentimento de medo
imposto sobre cada uma delas.

Até o presente momento, um milhão e setenta mil pessoas morreram
infectadas pelo vírus da \textsc{covid}"-19 em todo o planeta. Giuseppe
Guardabascio ficou internado 16 dias e sobreviveu podendo retornar à sua
família.

\begin{bibliohedra}
\tit{LIOHN}, André. Na Itália, uma família tentava internar avô com
coronavírus. Conseguiu quando alguém morreu. In: \emph{Folha de S.
Paulo}, 04/03/2020. Disponível em:
\emph{https://www1.folha.uol.com.br/mundo/2020/04/na-italia-uma-familia-tentava-internar-avo-com-coronavirus-conseguiu-quando-alguem-morreu.shtml}.
Acesso em: 10/08/2020.

\titidem. Retomada de Mossul das mãos do Estado Islâmico deixa
rastro de mortos. Disponível em:
https://www1.folha.uol.com.br/mundo/2017/07/190\\5598-retomada-de-mossul-das-maos-do-estado-islamico-deixa-rastro-de-mortos.shtm.
Acesso em: 10/08/2020.
\end{bibliohedra}

\chapterspecial{Coronavírus}{}{Claire Auzias}

\noindent{}Às vezes tenho medo, mas não percebo. Às vezes me sinto estrangulada, só
porque não posso sair. Mas, de fato, posso sair. Talvez, então, seja só
porque não posso divagar. Mas, habitualmente, há muito tempo não divago
mais pelas ruas. Então, por quê? Mental, abstrato, psíquico. Não é a
prisão. Sou invadida pela compaixão por meus camaradas encarcerados em
revolta contra a proibição das visitas. Sinto o sufocamento que se fecha
ao redor deles, mas, na verdade, as visitas também podem trazer o vírus.
Faz muito tempo que aprendi o confinamento, esse louco corte em relação
aos outros. Será que atualmente isso é mesmo tão difícil assim? Sim, há
um grau a mais. Não há como negar. Quase ficamos melhor sozinhos em
nossas casas, sem riscos. De repente, mergulhados na Idade Média\ldots{} será
que consigo compreender isso? Não, acho que não consigo entender. É
profundo demais, é antigo demais, inacessível à minha compreensão. A
Idade Média está muito longe. O que me acontece hoje, nesses dias de
hoje, está longe demais. Não é possível comparar. Todos os dias,
aprendemos um pouco mais sobre isso. Agora, dizem oficialmente que as
portas e maçanetas, teclados e outras superfícies comuns são portadores
do vírus. É preciso desinfetar, mas nem sempre os produtos --- nem o
álcool 90º, nem os lenços umedecidos antissépticos, nem os álcoois em
gel --- são encontrados nas farmácias. Se entendi bem, parece que as
entregas por correio não estão funcionando. Não é mais possível
encomendar livros pelo correio. Vou realmente acabar sendo obrigada a
ler a \emph{Enciclopédia}. Eu teria gostado de ler \emph{Decameron} e
\emph{A peste em Londres} de Daniel Defoe. \emph{Niet!} Além disso, vou
ter que mudar meus hábitos de abastecimento e planejar bem minhas
saídas. Fazer compras a cada dois dias, ou três. Os amigos e amigas
mandam \emph{links} de internet para passar o tempo, filmes e música
para escutar grátis, livros para serem lidos \emph{on"-line}, mas, como
não tenho talento para a internet, não acho grande coisa.

19 de março de 2020. Esta noite sonhei com um exame médico num
laboratório cujo resultado teria sido de alguma forma falsificado,
interpretado malevolamente (sic) num sentido abusivo, e que então seria
preciso repetir para obter dados corretos. Ou ele foi mal transmitido
(no sonho) por uma mulher, nascida de um pai vietnamita. Lembro disso ao
despertar, pois subitamente pensei que a figura de uma mulher asiática
estava no sonho para me indicar a ameaça. A ameaça do vírus asiático. E,
então, lembrei"-me de Charlotte Beradt e de seu livro \emph{Sonhar sob o
\textsc{iii} Reich}, os sonhos sob o império nazista. Esse livro causou surpresa
quando Annelise apresentou a nós. Sonhar um acontecimento mundial
histórico como aquele é distorcer o sonho, espaço privativo por
excelência, na esfera pública, política e mundial. Foi o que fiz esta
noite. Sonhei com o coronavírus.

20 de março. Caio toda hora no choro, a emoção está à espreita; não
longe. Ontem à noite, às 20 horas, aplaudiram nas varandas. A rua estava
animada. Ufa!

21 de março. Não tenho pressa de me atirar aos miasmas da rua e dos
outros. Não sofro muito por estar isolada. Em primeiro lugar, todo mundo
está. Parece que há um bilhão de confinados na Terra. Inacreditável.
Tudo bem, pelo menos enquanto durar meu vinho. E, na verdade, posso sair
quando quiser, munida de meu atestado. Mas, como não tenho vontade de
ficar passeando lá fora, só dou breves pulos ao supermercado e volto
rapidamente para o aconchego do lar.

24 de março. Nossos governantes perderam a cabeça: convocaram os
desempregados e os confinados para virem trabalhar de graça pelas
atividades essenciais para o país (a agricultura). Em vez de criar
empregos para os desempregados, eles chamam as pessoas para virem
gratuitamente formar o ``exército da sombra''. Como se estivéssemos num
drama vital que pudesse justificar um alistamento sem qualquer
remuneração. A irracionalidade vigente no mundo atual é inconcebível.
Parece que eles não entenderam o que foi o nazismo (``o exército das
sombras'' era a resistência ao nazismo).

Alguns amigos --- amigas --- de longe atendem o telefone se ligo para eles.
Outros, mais raros, dão uma passada na minha casa. A sociabilidade ainda
não está superativada. O isolamento predomina. Decidi encomendar o filme
\emph{O cavaleiro no telhado}, na Amazon, sem uma ponta de vergonha.
\emph{A bela França}, de Darien, me faz bocejar um pouco. Estamos
esperando minha sobrinha do Chile, que está voltando para casa. Não
gosto de sair para fazer compras, pois tenho a impressão de cruzar com
muitos curiosos, muita gente sem noção, desorientados, que não acreditam
no que se passa. Em casa, o ambiente é muito mais relaxado. Acho que
escutei no rádio que somos agora dois bilhões de seres humanos
confinados, ou seja, um terço do planeta. Será que escutei bem? À noite,
às 20 horas, aplausos para as equipes médicas, o que nos dá, por alguns
minutos, sensações de sociabilidade.

26 de março. Escutando a bela ideia dos atores de \emph{Lundimatin} (o
jornal esquerdista) que irão ler diariamente um conto do
\emph{Decameron}, imaginei inventar um jogo de escrita coletiva para
criar vínculos em tempos de isolamento. Duvido que cheguem respostas
favoráveis, por enquanto um único amigo se recusou. Vai ser preciso
procurar em outros grupos ou eu mesma inventar, mas pelo jeito a escrita
confinada não cria vínculos. Retorno ao ponto de partida. Imaginei uma
fábula coletiva, uma colcha de retalhos, de acordo com o que cada um
trouxesse. Que nada, não parece que a ideia inspire nossos amigos.
Paciência, esses tempos exigem muita paciência!

27 de março de 2020. Acordei esta manhã com um torcicolo danado do lado
esquerdo. Tentei massagear, tratar com calor, mas, em vez de se
dissipar, ele se espalhou por toda a parte inferior da cabeça e subiu
até o meio do crânio. Será que fui pega por um tipo de corona virulento
e devo esperar pelo pior? Não tenho febre alguma nem tosse. Não estou
com coragem de contar para ninguém, achariam que estou paranoica. Como
não se conhece exatamente o que acontece com essa doença, tudo fica
perigoso. Ontem, uma jovem de 16 anos morreu, mesmo com dois testes
negativados, antes de ser testada novamente pela terceira vez, quando
perceberam que sim, ela era positiva. E foi o fim. Tomei dois Doliprane
(medicamento para dor), para ver se ele resolvia a questão.

28 de março. Despertei às 3 horas da manhã e foi impossível dormir
novamente. Tenho tanta dor de cabeça que não consigo me mexer, nem para
o lado, nem no sentido da altura. Nem me virar na cama. O Doliprane é
pouco eficaz, as ondas de calor da almofada elétrica também, não sei
mais o que fazer. Não tenho febre, só 37.5 ao meio"-dia. Um pouco de dor
de garganta, pequenos gânglios, dificuldade para engolir quando a
cabeça, ou melhor, a nuca, está doendo muito. A nuca está totalmente
endurecida. Postei no Facebook uma mensagem dizendo que estou doente, o
que é verdade, e não sei o que tenho, o que também é; os amigos médicos
consultados por telefone parecem concordar com o diagnóstico de
contração muscular em razão do estresse. E estão também de acordo quanto
às formas de combatê"-la, ou seja, até aqui sem sucesso. À noite, a febre
aparece, 38.1.

29 de março. O pico do pânico passou, 37.2 hoje de manhã. Portanto, não
há infecção. A nuca continua endurecida, mas um tantinho menos dolorida.
Dei uma volta pelo bairro para tomar um pouco de ar, conforme aconselhou
um amigo médico. Esta noite, temos uma hora de verão a mais. O Facebook
funciona muito bem para manter unido nosso grupo de amigos. Sinto que
vai faltar leitura para me distrair. O isolamento foi prorrogado até 15
de abril. Na minha opinião, vai durar o mês inteiro. Depois, ficaremos
medrosos, não vamos ter coragem de nos abraçar e beijar, iremos evitar
perdigotos. Pela rua, muitos entregadores de bicicleta, negros ou
árabes. Triste, é o novo proletariado.

Segunda feira, 30 de março. Continuo sem febre. Preciso me tranquilizar,
pois esse ataque continua parecendo muito estranho. Tomei o metrô várias
vezes na última semana antes do isolamento, há oito dias meu resfriado
aumentou, depois apareceram gânglios e dificuldade de engolir,
finalmente me senti febril muitas vezes antes de tirar a temperatura
corretamente. Minha nuca, todos os músculos do pescoço continuam doendo.
Massageio, a dor alivia, mas não desaparece. De qualquer forma, e
independentemente das sequelas e sua reabsorção, não corro perigo. As
mensagens de amizade afluíram no Facebook. Isso aquece o coração, faz
companhia, afinal de contas, os amigos virtuais são amigos. Eles estavam
lá, respondendo ``presente''. Obrigada a todos. O instrumento internet
funciona, o que é vital nesses nossos tempos.

31 de março. Continuo sem febre, mas as dores não passam. Após uma
retrospectiva de minhas sucessivas casas anteriores, fica claro que
nunca morei por muito tempo em um lugar tão pequeno. Montparnasse, com
seus 13 metros quadrados, destinava"-se a servir de promontório a curto
prazo. Exatamente o que ele foi. Meus quartinhos (de empregada, no sexto
andar) de antigamente eram ambos tão vastos quanto minha habitação
atual. Eu nunca teria imaginado isso, o que é sinal de uma pauperização,
nem mais nem menos, em minha vida. Terminar tão cedo num lugar tão
exíguo.

1º de abril. Nenhuma febre quando acordo, e não preciso tirar a
temperatura para saber disso. Mas as dores dos músculos inflamados
continuam, alastrando"-se por todo o pescoço a partir da nuca. A
inflamação não desaparece magicamente. Mas minha capacidade de virar a
cabeça aumenta a cada dia, embora com incômodo e dor. Os sintomas formam
um conjunto que já experimentei outras vezes, mas não a crispação
muscular. Espero poder ser testada um dia para saber se estou imunizada.
O ataque se manifestou sexta"-feira à noite por meio de angústia e de
``alucinações'' olfativas. Tive a impressão de que havia fogo no prédio.
Saí no corredor e não senti nada. A não ser, claro, que fossem odores
efetivamente insinuados através das paredes da vizinhança. Se não for
uma forma benigna de corona, talvez seja uma forma ativa de estresse em
razão da angústia pelo vírus invisível e onipresente. Não consigo
produzir nada intelectualmente, não estou estudando coisas sérias, quase
não leio, minha escrita é totalmente rasa. Sem elevação onírica,
imaginativa, sem entusiasmo. Chumbo nas asas. A dor de cabeça muda.
Hoje, ela subiu um grau acima do nível da orelha; isso é novo. A dor na
nuca fica mais forte à noite. Tudo bem que os sintomas circulem, desde
que um dia isso passe. A Coisa.

2 de abril. Minha cabeça dói sem parar. Sinto a nuca afogueada. Nunca
passa. O intelecto está confinado, desorientado. Nenhum sopro de
inteligência me atravessa. Muita internet, pouca leitura, nada de
escrita. Estou sufocada. O consumo cultural disponível graças à internet
é apenas consumo no sentido estrito. Ontem eu vi Beckett,
\emph{Esperando Godot} no teatro filmado, notável; agora estou
assistindo \emph{O sr. Deligny, vagabundo eficaz}. Tudo isso é belo e
bom, mas não ativa meu espírito, não anima meu intelecto nem minha
imaginação. Sinto"-me árida, estagnada. Apenas consumindo o tempo, que se
desenrola de modo ordinário. E isso é tudo.

4 de abril. Vivaldi na estação de rádio de música clássica, sempre a
mesma dor na nuca, embora eu tenha escrito que melhorou e que a dor
diminuiu. Sim, ela diminuiu, mas ainda sinto dor, de fato. Nem febre,
nem Doliprane, mais nada. Mas a dor se agarra ainda à minha nuca. Entro
devagarinho em \emph{Glória incerta} de Joan Sales, traduzido do catalão
por Bernard Lesfargues.

5 de abril. Todos os sinais discretos de uma doença de inverno
continuam: sempre a dor de cabeça, nuca, músculos do pescoço. Sempre o
desarranjo intestinal, sempre a garganta arranhando e o resfriado ativo.
Não tomo mais Doliprane nem meço a temperatura, não tenho febre. Mas o
corpo sofre, há um claro ataque sendo combatido por minhas defesas
imunológicas. Ontem, fui arrastada por um amigo para o problema dos
ciganos de Perpignan. A relação com o vírus. Os observadores e
cuidadores locais não entendem as reações dos ciganos, que são tratados
como ``outros'' distantes. Na realidade, segundo a imprensa, esses
ciganos demonstram ser bem"-informados em tempo real, além de revelarem
uma intensa angústia, como todos nós. Lamento por eles que a população
francesa em geral manifeste tanto desconhecimento a respeito de seus
vizinhos ciganos.

E lamento amargamente, uma vez mais, que os espíritos esclarecidos e
emancipadores não se aproximem mais dessas pessoas que, finalmente,
poderiam partilhar raciocínios sensatos e modernos, em vez de
confiná"-los --- devemos dizer assim --- numa tradição com a qual ninguém
se importa e que é subestimada. Esses homens e mulheres só pedem uma
coisa: ser poupados do contágio. Para isso, interlocutores minimamente
corretos poderiam aperfeiçoar a prevenção, em vez de abandoná"-los a
charlatães que os expõem aos piores riscos. A situação parece melhor no
Nordeste com os Ciganos e Viajantes, se deixarmos de lado o fato de que
as caravanas não permitem medidas de distanciamento suficientes. E que,
para alguns, permanecem precárias as condições para imprimir as
autorizações de saídas ou para ter acesso a redes de wi"-fi, muitas vezes
caras, além de outros problemas específicos dos terrenos de
estacionamento.

6 de abril. Continuo doente. Os sintomas não desaparecem. Não estou bem
e não tenho coragem de contar sobre isso, pois vão achar que estou
sofrendo de uma angústia delirante. Ontem clonaram meu cartão de
crédito, então, a partir de agora, estou sem cartão; vou precisar
recorrer aos amigos para ter dinheiro líquido. Não posso mais fazer
transações \emph{on"-line} por internet. Sinto o peso do isolamento, como
todo mundo. Passagem difícil por essa quarta semana de isolamento que
está começando. E, no entanto, não tenho nenhuma vontade de sair, de ir
ao encontro da \textsc{covid}! Nesse caso, o objetivo não é não cair doente, mas
como fazer para se tratar? Aonde ir? Chamar ``\textsc{sos} Médico'' por causa de
uma doença benigna quando comparada à pandemia? Não; vou reunir minhas
forças e resistir. É espantoso que a inteligência não funcione, que o
espírito esteja paralisado. Trata"-se apenas de sobreviver ao fenômeno,
sem qualquer outra ambição. Gralhas enormes, e pombas, vêm ciscar em meu
gramado. É a primeira vez que as vejo. Elas estão com fome, diz uma
amiga. Não há mais piqueniques na cidade, nos jardins, nenhuma migalha
nas varandas dos restaurantes. Nossos animais urbanos sofrem. É com as
flores, e pela primeira vez na vida, que tenho a mais íntima
proximidade. Eu as observo várias vezes por dia, avalio se estão
precisando de água, observo como elas se estendem para o sol e a luz,
aprendo tudo. Eu as amo. Elas são a presença viva sensível mais próxima
a mim, nesses tempos confinados. Há quem converse com os animais. Eu
converso com as plantas.

9 de abril. A dor na cervical continua terrível. Um sofrimento concreto
que protege do sofrimento abstrato do coronavírus, que me impede de
galopar nas restingas imaginárias de uma peste bubônica inapreensível.
Ontem celebramos, no Facebook, o Dia Internacional dos Roms, com algumas
controvérsias atuais sobre os benefícios e malefícios da \textsc{uri} (União
Romani Internacional). Continuo me arrastando pelo meu romance catalão
de uma burguesia em plena guerra que, na verdade, não me inspira muito.
Converso com minhas flores que se abrem uma depois da outra. Descubro os
mistérios vegetais. As tulipas chegam à maturidade envolvidas por uma
capa verde que se transmuta em um dia em cores brilhantes e abrem"-se ao
sol. Mas elas duram pouco, sua longevidade não ultrapassa a marca de
poucos dias. Quando acabar o isolamento, vou me abastecer de mudas e
sementes, bulbos e plantas para cultivar na minha varanda. Entrei em
hibernação, em isolamento, ensimesmada, totalmente encolhida e sem
qualquer pressa de sair. Já me sinto ansiosa com a ideia de voltar a
sair diante das ameaças infecciosas. Acho que nunca mais vou ter coragem
de beijar alguém nem de trocar apertos de mãos\ldots{} Será uma fronteira a
atravessar.

10 de abril. O heliotropismo das tulipas é extraordinário: durante o dia
elas estão na horizontal, completamente oferecidas ao sol e à luz; à
noite, fecham"-se como ostras. Quantas lições vegetais tenho tido por
esses dias! Nenhum filme realmente bom para assistir, nenhum livro
realmente bom nas mãos. Como tudo isso é insípido e, de alguma forma,
sofrido\ldots{} Não consigo retomar a iniciativa, a criatividade, o
sentimento de agir por mim mesma. Trata"-se apenas de fazer com que as
horas passem. Olho o despertador com alívio: ufa, finalmente já são 21
horas! Logo vou poder dormir. Estamos vivendo uma mera caricatura do
normal. Quanto à música, é preciso muita sorte para conseguir descolar
alguma que seja agradável nas ondas públicas!

14 de abril. Ao que parece, produzi um ataque de artrose cervical para
me proteger da angústia do corona, uma dor contra a dor irrepresentável
do corona. Assim, pelo menos, sei onde e como dói, e isso me ocupa: isso
ocupa o primeiro plano, embora se trate da parte de trás de minha nuca.
Uma muralha de dor contra outra dor. O futuro não tem rosto, não tem
perfil nem contornos. O presente nunca foi tão plano, atordoante. O
passado ficou para trás como uma anedota. Como se meu passado tivesse se
tornado anedótico, ele murchou.

15 de abril. Estamos mergulhados na hipótese invasora da morte e, no que
me diz respeito, de uma velhice certeira, na qual vejo desaparecer
muitos dos meus entusiasmos de antigamente. Um verdadeiro umbral foi
ultrapassado, estou perdida para a vida. Vou assistir novamente ao filme
sobre Eric Clapton, para tentar fazer reviver os anos quentes. No
entanto, acolhi vivamente a autobiografia de Keith Richard, com
entusiasmo; eu a havia devorado há dez anos.

18 de abril. Dor na nuca. Estou cheia disso, será que essa sensação de
prensa na nuca vai durar até a vacina? Boas"-vindas à somatização! De
Ponta"-Negra, no Congo, Lionel nos envia charadas pelo Facebook, mas não
consigo resolver nenhuma por completo. Minha mente está lerda.
Finalmente, os velhos não ficarão mais confinados que os outros, mas
para mim isso é indiferente. Tenho que reaprender a me interessar pelo
mundo exterior; tive um sonho muito emocionante neste 20 de abril.
Fiquei profundamente emocionada e reagi com lágrimas.

21 de abril. Hoje levei a ousadia até a loja do Monoprix no meio da
avenida Gambetta, na altura da rua Pelleport. Sinto necessidade de sair
do torpor protetor no qual me aninhei esse mês inteiro. Preciso
reescrever, retrabalhar, re"-refletir e andar até os confins de meu
perímetro autorizado. Chegou a hora de me mexer. Livrarias e
floriculturas reabriram. Ou seja, soou o pré"-desconfinamento. No nicho
de livros grátis de meu bairro, descolei um Pontalis. Que leio com mais
fervor do que Hampaté Bâ, que descobri com cinquenta anos de atraso. Se
eu tivesse lido Hampaté Bâ quando tive a felicidade de conhecer o Mali,
teria adorado. Mas hoje, o gosto requentado dessa leitura me parece
insípido.

23 de abril. Lá fora o sol brilha forte nesta tarde. Saí apressada para
comprar algo para beber e vi muita gente pelas calçadas. É inegável que
as pessoas voltaram a passear. Relaxamento. Se as coisas continuarem
nesse ritmo, atenção à segunda onda! Em casa já está escuro, minha gruta
entra na obscuridade às 15 horas, devido à posição do prédio.
Geralmente, essa precocidade também me coloca fora dos comportamentos
gerais: eles estão ainda brilhando ao sol, e eu já estou inclinada sob
lâmpadas elétricas. Isso muda os modos de pensar.

24 de abril. Primeiro pesadelo do confinamento nesta noite. Entre outras
coisas, iam me guilhotinar. Alguém me segurava pelo pescoço e passava
gordura nele para que a lâmina da guilhotina cortasse melhor. Ignoro
qual faísca pôde me lançar a um pesadelo como esse. Vou me tratar com
cinema. Quero assistir a tudo o que houver de Visconti. Vou compensar
todo meu atraso de cinéfila. É o que mais me agrada nestes dias. Ao som
do adágio de Mahler.

1º de maio. Tudo vai mal. Passei seis dias em refúgio climático na casa
de uma amiga, por causa da devastação em meu apartamento causada por um
curto"-circuito no relógio de luz. Inabitável. Foi um parêntese muito
simpático, caloroso e tenso de meu lado, tenso devido à ansiedade. Agora
já voltei a meus domínios, é 1º de maio, meu modem de internet está
novamente em pane, tenho algumas questões materiais dos seguros para
resolver. Isso me ocupa. Estou me sentindo meio para baixo. Ouvi uma
leitura de Charlotte Delbo, com Louis Jouvet, muito forte. Sinto muito
cansaço; é a ressaca das panes elétricas. Tenho a sensação de estar
confinada para sempre. Não imagino como vou conseguir voltar a beijar os
amigos, as amigas, tomar o metrô. Alguns amigos reagem melhor que eu,
menos aturdidos, relutam em seguir as ordens autoritárias do governo. No
meu caso, deixo acontecer, deixo"-me levar. Quanto à minha casa, nunca
antes tinha sofrido tanto por causa de seu tamanho, mas agora estou
vivendo como um caracol em sua concha. Não consigo abrir as asas. Estou
quebrada. Sinto estar caminhando rápido para o fim. Qual energia devo
aspirar para reaprender o gosto das coisas, de fora, dos outros? Onde
encontrar forças?

8 de maio. Esta noite, eu estava preparando a saída de Didier (meu
marido) da prisão, com sua família, e disse de repente: ``Didier não vai
sair da prisão, porque ele está morto''. Pela primeira vez na vida sonho
explicitamente com a morte.

11 de maio. Sonhei com a morte de minha mãe há alguns dias. Mas hoje é
sobretudo o dia 1 do des"-confinamento. Não imaginem que isso me faz
ficar exultante. Meu livro \emph{Um fato de verão} ainda não tem editor.
Um livro sem editor sofre de uma orfandade sem fim. Dois editores o
leram e o recusaram. Dois outros o receberam (há muito tempo), mas nem
leram, nem acusaram o recebimento. Sobra um, que me garante que quer
lê"-lo e quer publicá"-lo, mas ainda não o leu. Isso parte inefavelmente
meu coração: abandonar um manuscrito escrito, e perdido. Sem
posteridade, sem leitores, sem futuro, sem existência. Apagar o esforço,
a dinâmica, a força que impulsionou a escrever. E subitamente nada mais
disso existe. O livro não existe. Havia entre essa escrita, esse levante
psíquico, e os leitores, um vai e vem, uma troca, um movimento que teria
feito a existência do livro, sua partilha e nossos comentários, sua
força e sua pouca existência. Mas agora ele fracassou, não existe
absolutamente, é um natimorto. Na verdade, é terrível e eu não saberia
expressar o que estou sentindo com essa imolação. Se ao menos fosse um
romance que pudesse suportar alguma espera, algum amadurecimento. Mas
não é o caso. Trata"-se de um diálogo entre contemporâneos, com meus
contemporâneos, e tal diálogo é não advindo. É essa a cruel evidência de
minha realidade. Mais um encontro fracassado. Escrever e clamar no
deserto só podem acontecer em certas circunstâncias e certas condições.
Não era assim que eu entendia esse livro. Estou ferida.

26 de maio. Escrevi sobre G. e para G., meu primeiro amor. Li Giono e me
senti reconfortada. Finalmente saí de meu estupor confinado e pude
saborear as alegrias da literatura. De novo, mergulho no tédio. Com
exceção da cinemateca, a falta do que fazer é o que me domina,
ferozmente. Há quinze dias, meus semelhantes já estão livres do
confinamento, mas eu quase não saio, dou um passo para frente e logo
recuo. Os pedestres da minha rua parecem arrogantes, descuidados,
perigosos. Mas estou entediada, vai ser preciso inventar alguma coisa.

30 de maio. Estou encantada com os filmes antigos da cinemateca e vou
continuar. Encontrei alguns magníficos. Até recebi como presente um
filme muito bom sobre Antonin Artaud que conseguiu me agradar. Só saio
de meu abrigo pouco a pouco. Ainda não estou usando o transporte
público, a não ser uma vez há uma semana, para ir ao hospital. Não foi
uma boa experiência! Senti uma grande vontade de ir para o campo neste
fim de semana, mas, diante de minhas incapacidades, a ideia me
abandonou. O jardim de infância da minha rua está aberto. Quem sabe ele
recebe pessoas como eu. Tenho que me convencer a retomar os costumes de
sociabilidade de antigamente.

18 de junho. Voltei esta semana para a psicanálise presencial, no
consultório. Retomei imediatamente o trabalho e redigi dois artigos, um
após o outro. Decidi que eu mesma vou publicar \emph{Um fato de verão}.
O caso está sendo estudado por meu camarada O.

22 de julho. Depois de um mês de correspondência regular com G., mais
nada. Pena, foi um reencontro feliz, embora um pouco delirante. Era uma
companhia. Novamente, eis"-me aqui comigo própria. O livro está sendo
fabricado pelas mãos de O., que conhece muito bem seu negócio, mas está
soterrado de trabalho. A coisa se arrasta um pouco, e isso também é
pena, mas vamos conseguir, se eu aceitar ter paciência.

Ainda dez dias em Paris, depois uma viagem no Sul. Quanto ao corona, ele
volta a galope depois de breve pausa. As pessoas não tomam cuidado, não
acreditam mais, habituaram"-se, pensam que vão atravessar incólumes. O
metrô é uma provação especialmente penosa, as pessoas se sentam em
qualquer lugar, e os trens estão sempre cheios, mesmo fora do horário de
pico. É enlouquecedor. Como o vírus oficialmente voltou em alta, não sei
mais se devo circular ou não. Não tenho alternativa para os transportes
públicos.

24 de agosto. Como previsto, é a segunda onda do vírus. Os veranistas
demonstraram que não conseguiam manter os gestos de proteção e que não
conseguiam sustentar uma prevenção do contágio. Minhas férias foram
muito agradáveis na casa de amigos. G. escreveu todos os dias nesses
últimos tempos, mas hoje, nada. Estou lendo Ursula le Guin.

O fim das férias está anunciado para 31 de agosto, as crianças voltarão
para as escolas da França. Os deslocamentos em transportes públicos são
uma catástrofe, tem gente demais por toda parte. Todo mundo parece ter
se acostumado ao corona, e ninguém se esforça mais. São cada vez mais
numerosos os recalcitrantes que cruzo, na rua ou mesmo no metrô, sem
máscara.

Li várias tentativas de análise por parte de meus contemporâneos, muitas
delas por parte de esquerdistas. Todos questionam a ordem liberal, até
mesmo protofascista, tanto econômica quanto social, no tempo do novo
coronavírus. O que eu poderia pessoalmente acrescentar? Não estou
sofrendo em razão das medidas sanitárias impostas pelas autoridades.
Minha rebelião não está focada nessas ordens, certamente autoritárias, e
até mesmo infantilizantes. Pois o perigo existe e é preciso se proteger
e proteger os outros. Essa máxima não sofre qualquer contradição de
minha parte. Enquanto não houver vacina, somos vulneráveis. Ora, mesmo
que a vida não seja tão interessante assim, não estou disposta a
perdê"-la devido à incivilidade de meus semelhantes. As recusas
obstinadas de se conformar às prescrições provêm principalmente da
extrema direita e dos idiotas iluminados que enxergam complôs por toda
parte. A esquerda e a extrema esquerda criticam principalmente o
autoritarismo e a grave perspectiva de que as medidas provisórias de
coerção sanitária possam se perenizar e instalem uma nova ordem de
marcação cerrada contra os cidadãos. Na medida em que a pandemia atingiu
abertamente o mundo todo, é fácil imaginar que uma resposta local será
insuficiente como resistência à tentativa autoritária. Além disso, os
comportamentos irresponsáveis de certa parcela da população fazem com
que as práticas repressivas sejam toleradas e mesmo apoiadas. E, ainda
que essa infecção fosse mínima, ninguém faz questão de ser agraciado com
ela. Mesmo no caso de resfriados de inverno, deveríamos usar máscaras
para não contaminar o vizinho.

Como sou aposentada, vou escapar em parte das calamidades econômicas que
nos espreitam nos próximos meses e anos em consequência da pane
econômica da ordem capitalista. E é verdade que algumas das propostas
que li \emph{Para o Mundo de Depois} chegam a se mostrar sensatas e
encorajadoras pela inovação, preconizando uma retomada mais igualitária
e ecológica para o futuro. Mas não acredito nisso. Na minha opinião,
elas pertencem ao campo da ficção e dos bons sentimentos. Não estou
dizendo que essas projeções sejam inválidas, longe disso. Estou dizendo
que não acredito, nem por um minuto, que haja alguém, dentre os agentes
políticos e econômicos do capitalismo mundial, que tenha a menor
intenção de melhorar a vida na Terra. Numerosas proposições já poderiam
ter sido concretizadas para amortecer o choque para os mais frágeis,
entre os quais se encontram os pobres esfomeados do mundo. Mas, hoje,
nada é feito. E é de se esperar que esse nada continuará a ser feito
numa escala ainda mais ampla. Não consigo acreditar. Talvez seja uma
questão de crença pessoal. E como não coloco meu narcisismo no esplendor
das teorias abstratas, a exemplo de tantas falsas celebridades,
permaneço na superfície dos fluxos da cotidianidade, na minha opinião,
muito mais importante.

Anuncia"-se uma magistral recessão econômica, desemprego maciço, etc. Mas
quem é que propõe que os recursos sejam partilhados, as horas de
trabalho repartidas para que cada um tenha um pouco? Ninguém! Quem
propõe alimentar os cidadãos a partir de agora privados de recursos, por
exemplo num país como a Índia, onde os ``bicos'' do mercado negro
permitem que toda uma família possa viver? Ninguém. Todo mundo considera
que isso é normal numa ordem capitalista. Como anarquista, ouso
proclamar que o texto mais inteligente que li sobre a situação foi o de
Dominique Strauss"-Kahn, \emph{O Ser, o ter e o poder}. Claro que não
estou afirmando que se trata de um texto anarquista ou feminista. Mas,
deixando de lado os pressupostos que não dizem respeito aos anarquistas,
como o exercício do poder, por exemplo, ou a representatividade política
parlamentar, fora dessas evidências, no que se refere à análise material
da situação gerada pela \textsc{covid}"-19 nos próximos anos, nenhum esquerdista
teria feito melhor.

Portanto, quero dizer que nós, anarquistas, somos como todo mundo: fomos
pegos de surpresa por um acontecimento imprevisível para a grande
maioria da população mundial. Sim, eu sei, também li textos anunciando a
chegada da pandemia, há mais de dez, e até vinte anos, por parte de
certos observadores. Ninguém acreditou. E ninguém tomou providências
dada a previsão da explosão sanitária. De fato, teria sido possível.
Lembrem"-se dos agentes, como por exemplo da \textsc{cia}, que avisaram há muito
tempo que esse risco era atual. Vou repetir o que dizemos há muitas
décadas: este mundo, tal como é, caminha para sua perdição, e nós
poderíamos lhe dar um pouco de harmonia, caso ouvíssemos as propostas
dos anarquistas do mundo inteiro. Sim, posso dizer isso mais uma vez,
mas o que é que isso traz para o debate de hoje?

Hoje, as pessoas querem saber
quando teremos a vacina e quando é que vamos poder novamente ficar nas
baladas até a manhã seguinte. O resto não passa de peripécias de
especialistas, não é problema nosso, dos vivos. É essa a mensagem que a
rua nos envia, todos os dias. As pessoas estão se lixando para o resto.

8 de setembro. Foi oficializada a notícia de que a segunda onda grassa
no mundo inteiro, mas que é preciso aceitar ser infectado, pelo bem da
economia mundial. Nesse sentido, poderíamos certamente nos ofender e
indignar. E administrar a pandemia de outra forma em benefício da
humanidade, e não para a sua perda, como escolhe o capitalismo. Os
efeitos devastadores ainda não são evidentes: por exemplo, nossos amigos
livreiros conseguiram compensar seu atraso devido aos dois meses de
isolamento social e de fechamento de seu comércio.

Reencontrar o gosto da liberdade nesses tempos? Vai ser difícil, mas
felicito aqueles que conseguirem. Que nossos sucessores possam desejar
que a alegria ainda se encontre no tumulto!


\chapterspecial{Pandemia, saúde pública e liberdades}{}{Edson Passetti}

\noindent{}Começou em dezembro de 2019, na China. Chegou no Brasil entre fevereiro
e março de 2020 com contaminações e mortes trazidas pelos segmentos mais
ricos dos humanos. É o novo coronavírus, depois batizado de \textsc{sars}"-CoV"-2,
produtor de uma doença chamada \textsc{covid}"-19. Veio pelo ar de longínquo
oriente, por avião. Ocupa o ar por meio de perdigotos lançados por
contaminados, atinge preferencialmente os que estabelecem relações de
proximidade ou em aglomerações. Chegou pelo transporte aéreo e se
expandiu pelo ar; invisível e alojado em um condutor, contamina os
lugares por onde se transita: escolas, empregos, áreas de transportes,
espaços de lazer\ldots{} Parece que nenhum lugar lhe escapa. Em 11 de março,
a Organização Mundial da Saúde (\textsc{oms}) decretou pandemia.\footnote{Cf.
  \emph{https://www.gov.br/pt-br/noticias/saude-e-vigilancia-sanitaria/2020/03/oms-classifica-coronavirus-como-pandemia}.
  Acesso em: 02/10/2020.}

Dizia Hipócrates que as doenças vêm de ares, águas e lugares em um livro
em que buscava a harmonia entre o homem e o seu ambiente. Na Grécia já
se constatava a presença de doenças endêmicas e de outras, nem sempre
regulares, mas com uma capacidade de expansão surpreendente, chamadas de
epidêmicas. Duas epidemias deixaram marcas na história desde a Idade
Média. Foi a peste de Justiniano, em 543, e a peste negra em 1348.
Entretanto, a lepra que incidia preponderantemente entre pobres nos
séculos \textsc{vi} e \textsc{vii}, trazida para Europa pelas Cruzadas, foi a doença mais
temida por provocar o ``isolamento''. O leproso passava a ser o
morto"-vivo, e seu isolamento servia, ao mesmo tempo, para a proteção dos
sadios e para combate ao mal. Foi com a peste negra que se instituiu a
prática da quarentena\footnote{Foram dois os modelos de organização
  média no Ocidente, o da lepra e o da peste. Neste ``o poder político
  da medicina consistia em repartir os indivíduos uns ao lado dos
  outros, isolá"-los, individualizá"-los, vigiá"-los um a um, controlar seu
  estado de saúde, verificar se ainda viviam ou se estavam mortos e em
  manter, assim, a sociedade em um espaço compartimentado,
  constantemente vigiado e controlado por meio de um registro, o mais
  completo possível, de todos os acontecimentos sobrevindos'' (\textsc{foucault},
  2011, p. 414).}, noção que se fundamenta nos escritos bíblicos (o
dilúvio durou 40 dias) e na crença, desde o século \textsc{xiii}, que o 40º dia
era o que separava formas agudas e crônicas das doenças. Doenças
epidêmicas estavam diretamente relacionadas com mal, punição divina,
risco à natureza e aos humanos moralmente sãos; com a necessidade de se
viver sob boas condições higiênicas para que ares, águas e lugares não
fossem infestados.

A medicina grega não se pautou na cura, mas na higiene que acompanhou a
história da medicina sanitária e de saúde pública. Trata"-se da busca
pelo ideal do organismo são em oposição ao organismo doente. Discute"-se
a natureza do mal, mas não o ideal de bem. À sociedade importa ``saber
qual é seu estado ideal ou a sua norma'' (\textsc{canguilhem}, 2005, p. 76). A
natureza dos males sociais é pouco questionável e a discordância se dá
em função dos remédios a serem aplicados, a necessidade de
medicalização. Por isso, as vias de reformas são variadas, as regulações
conturbadas nas institucionalizações e sobressaem as discordâncias não
só quanto ao enfrentamento com epidemias, pois o que está em jogo é como
equacionar a crise, conceito que veio da medicina para o governo da
sociedade.

Sabemos, modernamente, que saúde pública é o nome que se dá à gestão das
doenças endêmicas, epidêmicas, pandêmicas e pestilentas no interior de
um território"-Estado e/ou disseminadas pelo planeta. Para Canguilhem
(2005), o mais correto seria chamar a saúde pública de sanitarismo, pois
o que está em questão é o governo das doenças. A higiene atravessa a
noção de saúde médica, com ambição sociopolítica e médica de
regulamentar a vida do indivíduo, e transforma a vida em cálculo na
definição de saúde pela \textsc{oms}.\footnote{Conceitos:
  \emph{https://conceitos.com/organizacao-mundial-de-saude-oms/};
  definição de saúde pública:
  \emph{https://conceitos.com/organizacao-mundial-de-saude-oms/}. Acesso
  em: 02/10/2020.}

Para muitas doenças há vacinas, recomendações de precauções, medidas
sanitárias enquanto, para outras, incluindo o \textsc{hiv}"-\textsc{aids}, aguarda"-se a
descoberta de uma vacina, e não se deixa de constatar os interesses e
moral em se descobrir ou não uma vacina, seus limites técnicos, morais e
políticos, as maneiras de administrar a contenção da doença e os
respectivos medicamentos. Enfim, cada vez que o humano se vê
surpreendido por uma doença nova que coloca em risco sua continuidade e
reprodução, múltiplas medidas são buscadas pelas ciências e nas práticas
de laboratórios e de higiene para contê"-las, fazendo da medicina um
saber que absorve uma pluralidade de disciplinas.

Para a gestão planetária das doenças e das respectivas contaminações,
considera"-se central o desempenho da Organização Mundial da Saúde (\textsc{oms}),
fundada em 7 de abril de 1948 e vinculada às Nações Unidas, pautando
suas ações em fundamentações científicas e traduzidas em recomendações a
todos os Estados no planeta. A \textsc{oms} foi uma das primeiras
institucionalizações planetárias desde quando o exercício de poder
passou da biopolítica (\textsc{foucault}, 1978) das populações e meios
controlados pela estatística, economia política e políticas públicas
compensatórias, para a ecopolítica (\textsc{passetti} et al., 2019) em que o
planeta e os ambientes são controlados pelos monitoramentos eletrônicos
e pessoais, a economia política, as políticas públicas e as específicas,
com variadas políticas de direitos de minorias, para o meio ambiente e
medidas e programas de segurança.

Notícias vão e vêm de todos os lugares por todos os meios disponíveis.
Uma das primeiras matérias que marcaram a surpreendente contaminação
pelo novo coronavírus e as medidas sanitárias necessárias indicadas de
imediato foi a da escritora polonesa Olga Tockarczuk, traduzida e
publicada no jornal \emph{Folha de S. Paulo}, como ``Janela indiscreta''
(11/04/2020): ``para mim, já havia um bom tempo, o mundo estava em
demasia. Por demais, rápido demais, barulhento demais''. Indo contra a
maré, a escritora saúda o isolamento, lamenta pelas pessoas que perderam
empregos em coisas e espaços que ela prescinde e pergunta: ``não
teríamos, por acaso, voltado para o ritmo normal da vida? O vírus não
seria, então, um desvio da norma, muito pelo contrário --- aquele mundo
frenético antes do vírus teria sido anormal?''. O vírus teria deixado
claro a qualquer mortal que há pessoas mais fracas e que precisam de
nossos cuidados, e que, apesar de tudo, voltamos para nossas casas e
nelas fazemos nossos aquartelamentos. A casa é o quartel; a epidemia
exige uma ação de polícia e medidas repressivas. Foi assim que, segundo
a autora, a Europa mostrou sua face mais frágil com o fechamento das
fronteiras entre Estados. Todos que vêm do estrangeiro são tidos como
suspeitos. Será mesmo que o vírus nos mostrará, como supõe a escritora,
quanto somos desiguais? E o que acontecerá depois de passada a chamada
pandemia?

Seis meses depois do artigo e dez ou onze da detecção do novo
coronavírus, muitas empresas buscam a vacina: China, Rússia Inglaterra e
\textsc{eua} disputam a dianteira. No Brasil, seu presidente decidiu negar a
existência da chamada pandemia, ridicularizando os demais, enfatizando
ser necessário não parar a economia, como se algo pudesse parar a
economia capitalista que não fosse a mobilização dos
trabalhadores"-empreendedores desse capitalismo neoliberal em que parecem
estar quase todos satisfeitos ou sofrendo, em conjunto, de embotamento
de imaginação, de insensibilidade epidérmica para as misérias, de
desconhecimento proposital das desigualdades, de anestesias contínuas
com direitos de minorias ofertados, empoderamentos prometidos, qualidade
de vida no horizonte ou um \emph{novo} normal, redutores maquiados de
racismo, etc. Há um \emph{pharmacón} contemporâneo, destinado menos à
cura e mais ao envenenamento gradativo chamado democracia.

\section{desiguais, democratas, moderados conservadores}

Não é difícil constatar, hoje em dia, os modos e as modulações que se
produzem em função da gestão da dinâmica das desigualdades
socioeconômicas que, entre outras coisas, cria a lista anual de
bilionários, incluindo gente deste país em que eu vivo; que desova mais
e mais crentes na economia capitalista, no trabalho, no emprego, como se
tudo ficasse azul depois de passada mais uma crise, nesta vida que é só
de crise, medo, dissimulações e demonstrações contínuas de felicidades
\emph{by selfies}; que reproduz o ideal de democracia equalizando as
diferenças de direitos e fomentando mais formalizações
jurídico"-políticas, judicializações e medicalizações. Não há como olhar
para o planeta sob a chamada pandemia e não se perguntar como os súditos
se governam, ou seja, deixar de lado a transcendentalidade humanista e
nos voltar à luta diária contra as desigualdades, principalmente contra
o exército (não só das forças armadas e policiais de Estado) de súditos
que amam, enaltecem e gozam com a economia capitalista para si e para
sua vitória como membro da manada em estado de engorda para o abate e o
corte ou como o tísico e o nanico em estado de inveja de ser a carne
tenra para os superiores.

Desde as Declarações dos Direitos do Homem e do Cidadão de 1789 e 1793,
está em jogo a defesa dos homens livres e iguais, sua emancipação
segundo direitos e modos de acesso e conservação da liberdade,
propriedade e segurança. Direito de resistir à opressão do soberano
quando este exacerba sua autoridade no exercício de seu governo e se
torna um anormal. Somos todos governados pela iluminação do Estado e
pelas variadas formas de exercício da soberania em regimes políticos
fundados na autoridade centralizada, explicitando o exercício do
entendimento na maioridade alcançada pela emancipação. E assim será,
como o escrito nas velhas escrituras.

Contra o nocivo para sociedade, haverá a lei (via constituição e direito
penal, e com os livros sagrados das mais variadas religiões), como
vacina e medicamento. E está claro que o nocivo para a sociedade é
decorrente do humano passível de punição, por isolamento, asilamento (de
refugiado, de velhos, de doentes mentais) ou encarceramento. Mas como
será diante da morte anunciada por um novo coronavírus quando não há,
nem jamais haverá, lei a ser edificada contra essa ou outras tantas
invisibilidades? Não houve, há ou haverá uma guerra entre o visível
humano e o invisível vírus, a não ser na mente metafórica e
dissimuladora de certos analistas ou de agitação regular de noticiários.
Diante de uma normalidade a ser almejada, de um novo normal a ser aceito
ou atingido, não há também punição efetiva capaz de conter aqueles que
duvidam, desafiam e se arriscam desconsiderando as recomendações da \textsc{oms},
repassadas por parte da imprensa, das redes digitais e de uma infinidade
de alertas constantes. Não há vida da norma sem a possibilidade da
aplicação, ainda que tênue, de sanção que sustente a própria norma. A
norma é que endireita o torto, o desviante potencial, que esquadrinha os
espaços (\textsc{canguilhem}, 2019; 2012; 2005). A vida sob o coronavírus
permanece realizando a continuidade do normal (ainda que para adocicar a
população se fale de um novo normal em trânsito para o qual todos são
convocados a participar da mesma vida normativa como se fosse a
procriação de uma nova).

A propriedade é o direito inviolável e sagrado em ambas as declarações,
assim como a necessidade de força pública para garantir os direitos é
garantida nas duas. A declaração de 1793 promete a felicidade pelo
governo que garante o gozo dos direitos naturais imprescindíveis
(igualdade, liberdade, segurança, propriedade) porque os homens são
iguais perante a lei (o que pode ser retocado, traduzindo homens por
humanos e dando os ``mesmos'' direitos às mulheres, mais de dois séculos
depois). Afinal, é o ``não faças ao outro o que não quiseres que te
faça'', fundindo razão e religião. Preservar direitos é garantir o uso
legítimo da punição, garantindo a segurança da pessoa, dos seus direitos
e de suas propriedades, da mesma maneira que as autoridades que excedem
de modo arbitrário no uso da lei deverão ser punidas. Direitos que só
existem porque há punições à vista e um suposto nivelamento hierárquico
por onde todos punem: os pais, os patrões, os juízes, os políticos; e
todos podem punir no retorno, com fiscalização sobre como os pais
ultrapassaram o tolerável como ``maus"-tratos'', os patrões com suas
normas de funcionamento e pagamento de salários; os juízes com as
sentenças e suas legítimas interpretações da lei; os políticos punidos
pelo não cumprimento de seus programas na próxima eleição, ou por meio
de um impedimento imediato mediante uma fraude constatada, etc. E toda e
qualquer outra profissão de educação, saúde, arte até a repressão.
Punição pela lei, pela norma, pelos direitos, em favor da segurança.
Pela saúde do Estado e da sociedade. Nenhuma pessoa pode ser propriedade
alheia (direitos contra a escravidão, segundo as especificidades e
legislações parciais de libertação, de acordo com a respectiva maneira
de exercício da dominação branca superior sobre indígenas, pretas e
qualquer outra etnia).

Cabe ao Estado o auxílio público, também conhecido como dívida sagrada
aos infelizes. Pode ser com os descendentes de escravos pelas peculiares
políticas compensatórias; pode ser com populações pobres e miseráveis
sob a situação atual de pandemia por meio de auxílio emergencial. A
garantia social está na conservação de direitos baseada na soberania
nacional. Por conseguinte, a revolta só é tolerável quando o governo
viola o direito do povo. Por isso, medidas como as brasileiras, de
redução de salários, jornadas de trabalho, dispensas do trabalho durante
este ano da chamada pandemia, são legais! Ainda no caso brasileiro,
quando em agosto deste 2020, certas forças antifascistas se organizaram
para tomar as avenidas, rapidamente a imprensa liberal e similar, o
judiciário, o legislativo e até mesmo o executivo que financiou, entre
outras, várias encenações de derrubada do judiciário, decidiram
apaziguar tudo, em nome do consenso em torno do conservadorismo moderado
na democracia. O investimento foi em alijar certas forças antifascistas
das ruas, orquestrado pela proeminência de muitas forças à direita e à
esquerda como a verdadeira força antifascista capaz de garantir a
democracia e o governo do Estado, reconhecido legitimamente nas urnas na
última eleição. O governo deu em troca o silêncio de sua força de
atuação e apoio declaradamente fascista, sua extrema direita, no dizer
pluralista. A aproximação do executivo do chamado centro (o antigo
Centrão), dentro do que esse governo chamou de nova política, evidenciou
que a política democrática pluralista vai da extrema esquerda à extrema
direita (por isso mesmo que nunca capitalistas, liberais e socialistas
deixaram de compor ou pretender compor com forças fascistas até a
recomposição da democracia liberal), sacrifícios saudáveis em nome da
continuidade do \emph{regime}.

Mas há alguma coisa dissonante nessa composição de imbecis certificados.
Na tradição moderna, a epidemia e a peste\footnote{``Sabemos que
  diferentes doenças infecciosas desapareceram do Ocidente antes mesmo
  da introdução da grande quimioterapia no século \textsc{xx}. A peste, ou o
  conjunto das doenças ao qual os cronistas, os historiadores e os
  médicos darão esse nome, apagou"-se ao longo dos séculos \textsc{xviii} e \textsc{xix}
  sem que se conheça de fato as razões nem os mecanismos desse fenômeno
  que merece ser estudado.'' (\textsc{foucault}, 2011, p. 402).} estão
relacionadas às doenças que vêm dos de baixo, dos que vivem embaixo, de
pobres e miseráveis. A \emph{sua} sujeira contamina o meio urbano
destinado à circulação livre de pessoas e mercadorias; contamina as
águas com esgoto a céu aberto, fonte de mananciais de águas, nascedouro
de espécies, polui o meio ambiente onde vivem, os ares e lugares,
incluindo os asilos, os hospitais, os leprosários: os espetáculos da
doença. Mesmo que o higienismo os transforme em pobres"-limpinhos,
asseados e obedientes, ainda são vistos como o bando de pobres sujos,
empesteados, doentes, nocivos tanto com sua presença física maltrapilha
e encardida quanto por meio de suas ilegalidades políticas que
sobressaem, desde o fim do século \textsc{xviii}, quando já são acusados de
ameaçar acabar com a \emph{boa} sociedade. Todas as práticas corajosas
desde então se voltam para a morte dessa sociedade fundada na
desigualdade socioeconômica e na ilusão dos direitos fundados na
diferença. A sociedade deve morrer (\textsc{stirner}, 2004). Isso é normal. ``É
possível que o normal seja uma categoria do pensamento popular porque o
povo sente --- de maneira profunda, apesar de confusa --- que sua
situação social não é justa'' (\textsc{canguilhem}, 2012, p. 187). Anormal é a
vida que nos impõem; é o superior que pretende nos governar; é o súdito
que se espelha no superior; é o eterno retorno \emph{do mesmo}; é
preciso arruinar essa sintaxe.

Todos aqueles que se aproximam desses corajosos e que convivem contando
com o apoio mútuo (\textsc{carneiro}, 2020) na autogestão de suas vidas são
nocivos, são mais do que disseminadores de vírus (novo ou velho), são a
peste (\textsc{oliveira}, 2020). A história da saúde pública é sempre a história
dos efeitos de pobres, sujos e nocivos sobre os bons, saudáveis, ricos e
ilustrados (aristocratas ou classe). Desde o século \textsc{xviii} ``a pandemia é
uma tessitura mais compacta de fenômenos constantes e mais homogêneos,
{[}\ldots{}{]} um modo autônomo, coerente e suficiente de ver a doença''
(\textsc{foucault}, 1980, p. 24). Contagiosa ou não, a epidemia tem sua
individualidade histórica, pois como fenômeno coletivo ela exige ``um
olhar múltiplo; processo único; é preciso descrevê"-la no que tem de
singular, acidental, imprevisto; {[}\ldots{}{]} só pode haver medicina das
epidemias se acompanhada de uma polícia'' (Idem, p. 27). E, assim,
formaram"-se não só as comissões de saúde nacionais como foi sendo
esboçado o jogo das relações internacionais entre variadas instituições
(até chegar à \textsc{oms}), transformando a consciência coletiva dos fenômenos
patológicos, ou seja, aqueles que estão fora da normalidade. Assim como
é a história dos humanos à espera de um condutor de consciência capaz de
levar a manada ao paraíso e reiterar o mito do desaparecimento total da
doença em uma sociedade sem distúrbios e paixões, restituída a saúde de
origem. Mas isso nada mais é que idealizar a vida no paraíso, no sonho
justo dos mortos, e jamais matar a sociedade que morre, talvez
restaurá"-la na misericórdia dos reformistas melancólicos e saudosistas
do \emph{welfare"-state}. Como os padres aliviam a dor da alma, os
intelectuais racionais abrandam o sofrimento (incluindo os médicos).
Normal é o matrimônio religião"-razão. Normal é até mesmo o
revolucionário porque sinaliza para a perpetuação do Estado; patológico
será o anarquista, assim como foram os combatentes das ilusões dos
direitos nos primeiros 40 anos seguintes da Revolução Francesa que
definiram a unidade política dos ilegalismos populares.

Mas ainda somos governados pela razão liberal, para a qual a primeira
tarefa dos médicos, e dos demais racionais, é sempre política: o combate
à doença é uma luta contra os maus governos, para que uma vida humana
com figuras positivas de saúde, virtude e felicidade a serviço do Estado
se realizem como norma. Há um pêndulo do justo que deve oscilar para
encontrar estabilidade no meio; deparar"-se com a necessidade de ver tudo
em \emph{processo}, um processo que avista a utopia tanto quanto a
consolidação imediata do real; portanto, simplesmente kafkiano. Todo
médico vive e viverá assim em uma bipolaridade entre o normal e o
patológico, pois a medicina é o \emph{corpus} de técnicas de cura e do
saber, e, ao mesmo tempo, conhecimento do homem saudável (o não doente e
também definição do homem modelo) (\textsc{foucault}, idem). O médico, o político
e o cidadão respiram \emph{direitos}.

A pandemia reiterou as desigualdades e, no Brasil, as colocou em um
plano (ou seria uma plataforma?) distante, ofuscadas pelas discussões
acerca das diferenças sociais e acessos a direitos (muitas notícias
sobre situações consideradas patológicas de subjugação de mulheres e
pretos), possibilidades de auxílios emergenciais, a saúde de todos por
meio de ameaças de \emph{lockdowns} (ou estado de sítio?),
aquartelamentos, controle mais acentuado de populações de riscos
(maiores de 60 anos e portadores de comorbidades), trabalhos deslocados
do presencial para \emph{home office}, escolas para crianças e jovens
fechadas, universidades particulares e algumas estatais funcionando por
acesso remoto, trocas de ministros da saúde, descaso do presidente com a
pandemia reduzida a gripezinha ou coisa de fracos (até ele mesmo ser
pego). Depois de contaminado, ele espertamente reduziu seus
pronunciamentos aos devotos em busca certo consenso nas sondagens, com
vista às eleições de final de ano, restaurando sua figura de mito
justiceiro em uma democracia em exercício\ldots{} O uso de máscaras
permaneceu recomendado, assim como a assepsia de mãos, distanciamento de
contatos próximos e de aglomerações, mas, com a chegada do calor, ainda
no início de setembro, as praias e ruas/avenidas foram tomadas de jovens
(e nem tão jovens) à procura de algo, de alguém, de uma presença\ldots{} E
muitas notícias! Foi o isolamento e não o trabalho, que segundo muitos,
provocou o desejo de suicídio, depressão, doenças da alma para pastores
laicos e religiosos. Tudo normal em direção ao novo normal, ou não?

\section{libertárias}

Na produção de notícias não é imediato estabelecer a incidência regular
de assuntos. Para efeitos deste artigo, basta relembrar por que as
prescrições permanecem nítidas para cada uma, ainda que nem sempre sejam
seguidas (ele/ela não usa máscara na rua, no transporte, na praia; mas
não deixa de usar no \emph{shopping center}, no hospital\ldots{},
simplesmente porque nesses lugares não se entra sem máscara). Não
faltaram ou faltam comunicações sobre a contaminação pelo novo
coronavírus e sobre o atendimento dos casos médicos em hospitais, leitos
suplementares, hospitais de campanha, segundo a gravidade dos casos em
relação direta com o até então desconhecimento dos medicamentos
eficazes, o que ocorreu segundo as \emph{n} tentativas de conter a
morte. Médicos, paramédicos, equipes de enfermagem, auxiliares, por
vezes voluntários, em um esforço enorme para garantir a continuidade da
vida de infectados. Testagens em série foram acionadas principalmente no
Oriente; atenção com os ambientes fechados após a infecção se alastrar
pela Europa; medidas interesseiras de alguns governos indicando o uso
``preventivo'' de cloroquina e hidroxicloroquina aos que acreditam mais
na autoridade do Estado fundada em si mesma e em seus interesses do que
nas recomendações científicas da \textsc{oms} e nos efeitos iniciais devastadores
da \textsc{covid}"-19, divulgados pelas agências de comunicação. Curvas
estatísticas são produzidas diariamente por universidades estrangeiras
e/ou por conglomerado de imprensa no Brasil (que explicitou o mau uso da
estatística pelo Ministério da Saúde) sobre mortes, contaminações em
alta e depois em estabilidade, em baixa. Medicamentos combinados são
encontrados para o atendimento imediato. Não há mais notícias
insistentes sobre morte de médicos e equipe de enfermagem. Hospitais de
campanha são desativados, a redução do uso de leitos de \textsc{uti} constatável,
há muita gente nas ruas, de volta ao trabalho presencial. Só não
voltaram ainda em grande parte dos municípios as escolas para crianças e
o lazer cultural (outubro de 2020). Falam do crescimento de suicídios,
de pessoas que estão confusas com o isolamento, de como é \emph{natural}
as pessoas voltarem às praias e ruas, porque ``ninguém aguenta mais'',
enfim, um rol de temas sobre o novo normal, que se confirmará após o fim
da pandemia (ou pelo menos sua contenção regular e regulamentada, a fase
de testagem de vacinas, gestão sobre os novos ``picos''). Novo normal?

Normal é uma condição de ajuste moderador de condutas, a condução
esperada de condutas que não firam a continuidade da ordem: o normal é
tolerante, solidário, humanista, democrata\ldots{} Com as normas, o normal
vai se ajustando segundo as situações para que ocorra a prevenção de uma
revolta ou insurreição, o retorno à ordem democrática sem a ameaça de
uma contenção de uma convulsão social, corrigindo a descontinuidade
circunstancial, com a contínua educação punitiva aos iracundos. Há
sempre um normal até mesmo nas pandemias, assim como há um normal no
patológico. O que está em questão é justamente a implosão simultânea do
desejo e da efetivação do normal, da normalidade, do normativo. Ou, como
situam os anarquistas, escapar da normalização é a sua saúde também.
Como disse certa vez um poeta: ``de perto ninguém é normal'', verso
contundente da canção ``Vaca Profana''\footnote{``Vaca profana'' de
  Caetano Velosos. Disponível em:
  \emph{https://www.letras.mus.br/gal-costa/153885/}. Acesso em:
  03/10/2020. Curioso que o contundente verso acabou sendo apropriado
  pelos lados menos imaginados da indústria cultural e editorial. Mas
  não perde sua força estranha.}. Ainda relembrando Canguilhem, o único
normal convincente é o que diz respeito a como o povo vive e constata a
sua situação real, não justa.\footnote{``Quando eu falava dessas cores
  mórbidas/ Quando eu falava desses homens, sórdidos/ Quando eu falava
  desse temporal/ Você não me escutou/ Mas isso é tão normal (Você não
  quis acreditar)''. ``Paisagem na janela'' de Lô Borges e Fernando
  Brandt, 1972, \emph{Clube da esquina}.} Um povo e não o povo como
totalidade, posto que neste \emph{mundo} em que vivemos não são
episódicas as maiorias populares que se espelham na existência
aristocráticas ou elitistas e se voltam contra uma quantidade ou
qualidade de seu próprio povo. Os anarquistas sempre souberam bem traçar
esse governo dessa maioria do povo contra as suas minorias capazes de
subversão (\textsc{goldman}, 2008).

O início da pandemia provocou o entrecruzar de informações e de
inquietações libertárias como o documento da Federação Anarquista de
Turim (2020), o de Ian Alan Paul, ``Dez premissas sobre a pandemia''
(2020), seguidos de muitos outros de pessoas e associações voltados para
ressituar a urgência de lidar com o risco de morte e, ao mesmo tempo,
com a realização de uma vida livre, livre de governos por hierarquias,
condutores de consciência, superiores, platônicos, plantonistas
humanistas, e os que se autodefinem capazes de falar da espécie para
espécie, pastores eruditos e \emph{populares}. Um outro olhar para a
peste embutida no discurso pandemia. A peste vista como a guerra moral
contra o que é capaz de destruir o humano hierarquizado e consolidado
daquela ou desta época. O anunciado perigo da extinção apocalíptica da
espécie, da degradação do planeta, da morte iminente que deve ser
combatida a qualquer custo pelas reformas. O perigo do retorno da peste
anarquista (\textsc{dossiê covid}"-19, \textsc{nu"-sol}, 2020).

O medo da morte, curiosamente a única certeza do humano desde seu
discernimento sobre as coisas e as gentes. É o que fortalece os pilares
do edifício da ordem. No presente, essa ordem fundada na racionalidade
neoliberal põe para crescer edifícios assustadoramente altos,
gigantescos falos arquitetônicos, pontos móveis de acesso para além das
nuvens como ilusão e proximidade do céu. Vida explicitamente
hierarquizada e governada pelo alto, com o apoio de todos os andares
inferiores da construção, incluindo suas garagens e subsolos. Vida que
começa na origem divina e na comunicação com o terreno que bate e volta.
A morte real e visível está imantada ao invisível dos céus. A morte de
cada um, seu esquife e sua sepultura, privada ou coletiva, e a
invisibilidade do vírus. O terrível instante que se torna normal quando
o médico na \textsc{uti} deve decidir para quem ele destinará o que há de
equipamento respiratório restante aos casos extremos; ele decidirá e
decidiu quem deve viver. Não se viu ou ouviu, todavia, um médico se
pronunciando contra a falta de equipamentos que o levava a se assenhorar
explicitamente da vida do \emph{paciente}.

A pandemia tende a \emph{perdoar} os equívocos da administração
lucrativa dos hospitais, as desatenções corriqueiras das instituições de
saúde pública, as vantagens dos planos de seguros privados e enaltecer
qualquer coisa visível no chamado sistema público diante do
extraordinário. Um extraordinário que comprime o ordinário, fazendo com
que hospitais funcionem apenas para o grande problema chamado pandemia,
havendo retração generalizada de exames e cirurgias até então
consideradas cotidianas e necessárias. Há diretamente uma escolha de
quem deve morrer! Há indiretamente o exercício de uma nova clínica
submetida às exigências da medicina diante da pandemia. E, mesmo assim,
não entra em questionamento o esperado zelo do Estado e das instituições
de saúde pública pela vida da população (dos pobres e miseráveis).
Aliás, essa população se satisfaz com qualquer declaração de que o
Estado, a província e o governo da cidade estão fazendo seu melhor para
o bem de todos. Entretanto, se essa população se visse, ainda que
momentaneamente, como povo provocaria um susto no Estado. Mas, quando
ela se vê como povo, é para se posicionar diante do que \emph{falta} e o
que sempre lhe falta é mais uma política pública, mais Estado. E se a
falta de política pública é um problema, essa parte do povo exerce sua
compaixão pela solidariedade e, muitas vezes, com mais eficiência que a
solidariedade entre empresários. Ambas funcionando para conter a
convulsão social e atendendo, minimamente, às necessidades dos pobres e
miseráveis, também contemplados pelo auxílio emergencial em dinheiro
pelo Estado.

A pandemia produz a aversão à possibilidade de alteração da vida normal,
seja pela compaixão individual, cívica ou empresarial, seja pela
obliteração da atração de forças antifascistas que são também antiordem,
seja pela crença de que todos caminhamos para o novo normal em um futuro
próximo, quando, além de medicamentos de primeira hora, além das medidas
de proteção recomendadas, haverá a vacina. A economia capitalista,
enfim, estará salva porque \emph{todos nós ajudamos a nos salvar}. E,
dessa maneira, a prática do apoio mútuo anarquista vira palavra"-chave em
um discurso fundado na compaixão religiosa convencional e racional como
solidariedade S/A.

A peste está relacionada aos ``de baixo''. A chamada pandemia provocada
pelo novo coronavírus (\textsc{sars}"-CoV"-2,) espalhou"-se pelos ``do alto'' (pelas
camadas superiores da sociedade, viajando em aviões e, também, em
transatlânticos turísticos). A peste está sempre relacionada no
imaginário dos \emph{acusados} à sentença do juízo divino, anúncio do
juízo final ou causa do juízo da ciência por meio da economia política,
medicina, estatística e ciência política dizendo quem é são. Há vírus
para os quais não há vacina, como o \textsc{hiv"-aids}, vírus governáveis a partir
de um certo momento, assim como há bactérias (principalmente as
hospitalares) para as quais não há medicamento para a cura. Estão e
estamos enquanto população, sob o governo do Estado, da ciência, da
religião e de nossa crença na punição. E não estamos libertaria"-mente!

Estamos em um planeta em degradação onde forças políticas reacionárias e
liberais (e outras) voltadas para a sustentabilidade (crença em
capitalismo civilizado) se enfrentam em função da salvação do planeta.
Há uma pandemia à vista gerada pelas queimadas (para além daquelas
esperadas nas estações climáticas, ainda que seus efeitos, ano a ano,
tenham se tornado mais prejudiciais pelo acúmulo de condescendências com
as devastações anteriores)? No Brasil, a crença dos normais na economia
viva exige exploração extensa de terras ocupadas por florestas, e nelas
indígenas e populações ribeirinhas, segundo interesses agrários que, vez
por outra, pressionam o atual governo para que o Ministério do Meio
Ambiente esteja fundido ao da Agricultura (o que não é novidade na
pandemia, mas regra do atual governo desde sua posse em janeiro de 2019,
rejeitando desde o Acordo de Paris à estrutura institucional de controle
herdada sobre a Amazônia e o Pantanal).

A pandemia atual, com várias modulações, fortalece as forças liberais e
de sustentabilidade e condena o governo e seus parceiros (os miseráveis
que se prestam a devastar as florestas e matar indígenas ou mesmo
contaminá"-los, em nome do emprego e da economia \emph{viva}) a serem
combatidos pelos antibióticos e/ou os anti"-inflamatórios da democracia.
Resta aos demais a pecha de agentes da peste. A pandemia, assim,
fortalece os ativismos de saúde planetária em que o humano ainda vive e
ajuda a produzir mais jovens lideranças. O \emph{novo} normal funda"-se
na mesma regra da normalidade entre as forças políticas em disputa pelo
governo do Estado e de suas naturezas e gentes. Novo normal é apenas a
atualização do normal.

Quando se imaginava que tudo ficaria muito mais ampliado com a
consolidação da Europa, vieram o \emph{brexit} da Grã"-Bretanha, os
governos nacionalistas no leste europeu, o acomodamento de forças
fascistas nos parlamentos europeus do oeste, a aversão a estrangeiros, a
defesa da nacionalização de empregos, as proteções ao nacional, o
governo Trump nos \textsc{eua}, a continuidade governamental na China e Rússia
(que não são mais comunistas), o saber tratar com os procedimentos
democráticos que se estenderam do regime político para o governo das
empresas por meio do empreendedorismo e da seguridade da
saúde.\footnote{Impossível não lembrar de \emph{Você não estava aqui},
  2018, e de \emph{Eu, Daniel Blake}, de 2016, ambos de Ken Loach.}

A pandemia ocupou as redes digitais, e novas plataformas apareceram para
incentivar o aparecimento de mais pastores digitais (os que têm milhões
de seguidores e os que têm milhares; os que pretendem ter mais do que
dezenas e centenas, ou seja, um local de produção de pastores que
complementam os religiosos e os laicos do Estado moderno e de suas
filantropias). A chamada pandemia precisa de mais pastores zelando pela
circulação de pessoas e coisas, reiterando precauções, mostrando
alternativas de vida saudável aos aquartelados: como melhorar a dieta
alimentar, os exercícios para o corpo, as meditações, as leituras e
blogs recomendáveis, etc., tudo dentro do normal e da normalidade
anteriores. A comunicação computo"-informacional é o grande centro da
produção de verdades, estejam elas rubricadas de veracidade ou de
etiquetas \emph{fake news}. Mostra a vida no espaço da comunicação pelos
ares em qualquer lugar. Nela também há vírus incontornáveis e que levam
à ampliação de seus dispositivos de segurança. Trata"-se de uma forma de
produção por modulações que corre o risco de estar sob uma pandemia a
qualquer momento, mesmo porque nada mais acontece de modo endêmico por
muito tempo. Somente pandemia e peste se comunicam. Não há Estado
democrático que não produza miséria. A miséria é endêmica a qualquer
Estado. Assim como a epidemia é a política e o Estado é a peste para um
povo.

\begin{bibliohedra}
\tit{ALAN PAUL}, Ian. Dez premissas sobre a pandemia. 2020. Disponível em:
\emph{https://www.ianalanpaul.com/ten-premises-for-a-pandemic/} e também
em \emph{https://www.nu-sol.org/blog/dez-premissas-para-uma-pandemia/}.
Acesso em: 15/09/2020.

\tit{CANGUILHEM}, Georges. \emph{Escritos sobre a Medicina.} Tradução de Vera
Avellar Ribeiro. Rio de Janeiro\emph{:} Forense Universitária, 2005.

\titidem. O normal e o patológico. In: \emph{O conhecimento da vida}.
Tradução de Vera Lucia Avellar Ribeiro. Rio de Janeiro\emph{:} Forense
Universitária, 2012, p. 169--185.

\titidem. \emph{O normal e o patológico}. Tradução de Maria Thereza R.
de C. Barrocas. Rio de Janeiro\emph{:} Forense Universitária, 2019.

\tit{CARNEIRO}, Beatriz. A prática anarquista da ajuda mútua e seu sequestro
na atualidade. In: \emph{Revista verve}. São Paulo: Nu"-Sol, 2020, n. 38,
pp. 10--24 Disponível em:
\emph{http://www.nu-sol.org/blog/dt\_portfolios/v-e-r-v-e-38/}

\tit{FEDERAÇÃO ANARQUISTA DE TURIM}. Epidemia, massacre de Estado? 2020.
Disponível em:
\emph{https://www.nu-sol.org/blog/epidemia-masacre-de-estado/}. Acesso
em: 14/09/2020.

\tit{FOUCAULT}, Michel. \emph{O nascimento da clínica}. Tradução de Roberto
Machado. Rio de Janeiro: Forense Universitária, 1980.

\titidem. \emph{História da sexualidade I: A vontade de saber}. Tradução
de Maria Thereza da Costa Albuquerque e J. A. Guilhon Albuquerque. Rio
de Janeiro: Graal, 1978.

\titidem. O nascimento da medicina social. In: \textsc{motta}, M. B. da (org.)
\emph{Arte, epistemologia, filosofia e história da medicina}. Tradução
de Vera Lucia A. Ribeiro. Rio de Janeiro: Forense Universitária, 2011,
p. 402--424. (Ditos e escritos; \textsc{vii}).

\tit{GOLDMAN}, Emma. Minorias versus maioria. Tradução de Eliane Carvalho. In:
\emph{Revista verve}, São Paulo: Nu"-Sol, 2008, n. 13, p. 123--133.
Disponível em:
\emph{http://www.nu-sol.org/wp-content/uploads/2018/02/Verve13.pdf}

\tit{NU"-SOL}. Dossiê \textsc{covid}"-19: afirmações da Vida\emph{.} In: \emph{nu"-sol},
2020. Disponível em:
https://www.nu-sol.org/blog/COVID-19-afirmacoes-da-vida/. Acesso em: 14/09/2020.

\tit{OLIVEIRA}, Salete. urge arruinar a cultura do castigo. vestígios de
anotações. In: \emph{Revista verve}, São Paulo: Nu"-Sol, 2020, n. 38, pp.
31--54. Disponível em:
\emph{http://www.nu-sol.org/blog/dt\_portfolios/v-e-r-v-e-38/}. Acesso em:
15/09/2020.

\tit{PASSETTI}, Edson et al. \emph{Ecopolítica}. São Paulo: Hedra Editora,
2019.

\tit{STIRNER}, Max. \emph{O único e a sua propriedade}. Tradução de João
Barrento. Lisboa: Antígona, 2004.

\tit{SZYMBORSKA}, Wislawa. Reciprocidade. In: \emph{Para o meu coração num
domingo}. Tradução de Regina Przybycien e Gabriel Borowski. São Paulo:
Companhia das Letras, 2020, p. 336--337.

\tit{TOKARCZUK}, Olga. Janela indiscreta. In: \emph{Folha de S. Paulo}. São
Paulo: Folha de S. Paulo, 11/04/2020, p. B12.
\end{bibliohedra}

\chapterspecial{Sexo em tempos de \textsc{covid"-19}}{}{Eliane Carvalho e Flávia Lucchesi}

\noindent{}O sexo está relacionado ao risco. É considerado a porta de entrada de
várias doenças assim como de comportamentos ditos perigosos. Ao mesmo
tempo, como um dos efeitos da chamada liberação sexual nos anos 1960,
proliferou o discurso do sexo como elemento essencial para uma vida
saudável, desde que seguro.

A situação colocada pela propagação em nível planetário do novo
coronavírus --- isolamento social, medidas de higiene, uso de máscara,
trabalho remoto quando possível, educação a distância, intensificação do
uso dos meios eletrônicos --- salientou outras questões relativas à
saúde física e mental da população, como os efeitos nos relacionamentos
afetivos e sexuais em meio ao confinamento.

De um lado, casais monogâmicos que vivem juntos são tidos como a forma
mais segura de se relacionar, ainda que a convivência confinada tenha
levado ao ``desgaste das relações'' em alguns casos e, no limite,
acentuado a chamada violência doméstica. De outro lado, aqueles que, por
opção ou acaso, não se encontram em um relacionamento monogâmico ou não
vivem no mesmo espaço, enfrentam dificuldades de encontrar novos
parceiros ou de se adaptarem a um relacionamento a distância, o que para
alguns é considerado uma fonte de ansiedade ou solidão.

Obviamente, o problema passa pelo cálculo de risco do contágio. Até o
momento, início de outubro de 2020, considera"-se ampla e consensualmente
que o vírus é transmitido por mucosas orais, pela saliva e também pela
respiração. De maneira que o beijo, a troca de salivas, é tido como uma
forma de alto risco de contaminação. Assim como a chamada ``Zona T'':
nariz, olhos e boca.

Há poucas pesquisas que analisam uma possível transmissão sexual do
vírus. A pesquisa ``Clinical Characteristics and Results of Semen Tests
Among Men With Coronavirus Disease 2019'' (\textsc{li}, D. et al., 2020),
produzida na China, epicentro da doença, é uma das pioneiras na
investigação do contágio via fluídos genitais.

Realizada entre janeiro e fevereiro de 2020, a pesquisa coletou e
analisou via \textsc{rt"-pcr} amostras seminais de 38 homens hospitalizados,
infectados pelo vírus da \textsc{covid}"-19. Os pesquisadores concluíram que,
apesar da pequena amostragem, era provável que a contaminação ocorresse
também por meio do sêmen. Indicaram a necessidade de maiores estudos com
esse material, abrangendo também a temporalidade do contágio mesmo após
a recuperação sintomática. De antemão, orientaram o uso de camisinha e a
abstinência. Essas orientações foram replicadas por outras pesquisas,
por guias estatais e de organizações da sociedade civil, por médicos e
profissionais da saúde, pela imprensa, e produzem verdades sobre as
práticas sexuais e relacionais de acordo com cálculos de riscos
atualizados.

Entre janeiro e março, uma pesquisa semelhante foi realizada pelo Comitê
de Revisão Institucional da Universidade de Medicina de Nanjing, a
partir de amostras seminais de 13 infectados (\textsc{song} et al., 2020). Ainda
que os testículos sejam abundantes na enzima \textsc{ace}2 --- reconhecida como
porta de entrada para a infecção do vírus --- e que seja constatado
maior número de homens contaminados do que mulheres, os pesquisadores
relataram que nenhuma das amostras coletadas apresentou carga viral do
novo coronavírus. Concluíram: ``nossos dados sugerem que 2019-nCov está
ausente no sêmen e nos testículos de homens infectados pela \textsc{covid}"-19,
tanto em fase aguda quanto em recuperação. Assim, é altamente improvável
que o 2019-nCov possa ser sexualmente transmitido por homens'' (Idem).

Essas duas pesquisas seguem como referências em relação às
probabilidades de transmissão via fluídos sexuais masculinos. Repercutem
em orientações para condutas seguras durante a chamada pandemia que, em
grande parte, levantam a suspeita do contágio pelo sêmen. Mas há também
as que alegam não haver comprovação dessa forma de disseminação do
vírus. No Brasil, ``recomenda"-se o banho antes e depois das práticas
sexuais. Masturbação, carícias no corpo, masturbação a dois (hétero ou
homo) e relação sexual pênis"-vagina apresentam menos risco, com
camisinha e sem beijos'' (\textsc{federação}, 22/04/2020). Estas são as
orientações da Federação Brasileira das Associações de Ginecologia e
Obstetrícia, em abril, possivelmente o primeiro documento oficial a
tratar desse tema no país. Depois, algumas secretarias municipais de
saúde publicaram informações e orientações para o sexo seguro.

\textls[-25]{A primeira investigação sobre a presença do novo coronavírus em fluídos
vaginais ocorreu em fevereiro, também na China (\textsc{qiu} et al., 2020). Foi
realizada a partir de amostras de 10 mulheres em fase pós"-menopausa e
acometidas por problemas respiratórios graves, internadas com a doença
na \textsc{uti}. A hipótese que levou à pesquisa foi a permanência dos vírus da
Ebola e do Zika nos fluídos vaginais, também após o período de
recuperação sintomática dessas enfermidades (o que também ocorre em
relação aos fluídos seminais).}

\textls[-5]{O interesse não era apenas aferir uma possível transmissão sexual, mas
também vertical, de mãe para bebê. Uma pesquisa chinesa anterior (\textsc{chen}
et al., 2020), identificou presença viral no líquido amniótico e no
sangue do cordão umbilical. Contudo, testes feitos nos bebês não foram
positivos para \textsc{covid}"-19. Concluiu"-se que o nível de transmissão vertical
do novo coronavírus é muito baixo. A pesquisa com os fluídos vaginais
não detectou carga viral, mesmo após 40 dias do início da infecção,
sinalizando baixo risco de transmissão por essa via. Em função do número
reduzido de pesquisas, não há ainda uma conclusão sobre o contágio da
\textsc{covid}"-19 pelo sexo.}

Mas, se os resultados são inconclusivos e, até o momento, considera"-se
pouco provável o contágio pela troca de fluídos genitais, não se ignora
a relevância do tema no atual contexto. Tanto pela produção de
documentos de orientações para uma prática sexual segura --- ou mais
segura\footnote{O termo ``mais seguro'' vem do inglês \emph{safe/r sex}
  --- sexo (mais) seguro --- e surgiu no contexto do controle da
  disseminação do vírus do \textsc{hiv} na comunidade gay, reconhecendo que há
  sempre um risco nas práticas sexuais.} --- em meio à chamada pandemia,
quanto pelo investimento em pesquisas que comprovem maior possibilidade
de transmissão por meio de práticas sexuais específicas e atreladas ao
``comportamento'' de grupos de ``alto risco''.

\textls[-10]{A primeira pesquisa sobre ``comportamento sexual'' relacionada à
\textsc{covid}"-19 também veio da China. Em março, 270 homens e 189 mulheres
chineses participaram voluntariamente de um questionário \emph{on"-line}.
Os investigadores concluíram que ``a atividade sexual, a frequência e os
comportamentos de risco diminuíram significativamente entre homens e
mulheres jovens {[}\ldots{}{]} a saúde sexual sofreu impactos durante a
pandemia da \textsc{covid}"-19, o que representa uma área potencial a ser
reconhecida e tratada por especialistas da saúde sexual'' (\textsc{li}, W. et
al., 2020). Em números gerais, menos da metade dos entrevistados alegou
redução no número de parceiros e na frequência sexual.}

Os resultados foram muito semelhantes aos de outra pesquisa chinesa
realizada em maio, na qual 3.500 jovens (cujas idades não foram
informadas) foram recrutados via WeChat e Weibo (redes sociais chinesas)
para responder questionários sobre desejo, frequência e satisfação
sexual (\textsc{li}, G., 2020). Do total de participantes, apenas 967 foram
considerados na pesquisa\footnote{\textls[-20]{Não há nenhuma referência ao motivo de
  se utilizar as respostas de apenas 967 participantes. Houve coleta de
  dados demográficos incluindo ``orientação sexual'', mas não foram
  divulgados esses resultados específicos. Vale ressaltar que, na China,
  a chamada homossexualidade deixou de ser considerada crime somente, em
  1997, e como doença mental em 2001. Entretanto, até hoje, há uma
  censura nas redes sociais relativas a qualquer conteúdo gay.}} que
contemplou também questões sobre o consumo de álcool antes de relações
sexuais (com queda de 20\%, segundo os voluntários) e a ``deterioração''
com o parceiro sexual (31\%).

O uso de álcool e outras ``drogas'', associado à prática sexual durante
os períodos de quarentenas e afins, foi uma questão recorrente nessas
pesquisas. No contexto da \textsc{covid}"-19, o consumo de bebidas alcoólicas e,
principalmente, de ``drogas'' é classificado como uma porta de entrada
para o sexo menos seguro ou ``comportamentos de risco''\emph{. }

Entre março e abril, ápice da disseminação do novo coronavírus em parte
da Europa, uma parceria entre pesquisadores espanhóis, italianos e
iranianos indicou impactos nas relações entre casais e nos indivíduos,
na chamada saúde mental, refletindo na libido e saúde sexual. Destacaram
o aumento de diagnósticos de ansiedade, depressão, estresse, por vezes
associados à disfunção erétil. Consideram que ``a saúde sexual e
reprodutiva é um estado de bem"-estar físico, emocional, mental e social
em relação a todos os aspectos da sexualidade e reprodução, não é
meramente a ausência de doenças, disfunções ou enfermidades'' (\textsc{mehrad},
2020). O argumento reflete a definição de saúde sexual da Organização
Mundial da Saúde: ``A saúde sexual é um estado de bem"-estar físico,
emocional, mental e social, em relação à sexualidade. Não é a simples
ausência de doença, disfunção ou enfermidade'' (\textsc{who}, 2006, p. 5). A \textsc{oms}
não produziu recomendações específicas voltadas ao sexo em meio à
\textsc{covid}"-19.

Os cientistas também postularam o uso de camisinha e a abstinência como
os métodos mais seguros para fins preventivos. Declararam não haver
evidências de transmissão vaginal, mas consideraram a possibilidade de
transmissão seminal, reiterando a necessidade de pesquisas mais
abrangentes no futuro. Destacaram que, apesar de inexistirem pesquisas
que comprovem transmissão anal, há evidência de infecção por meio de
material fecal de doentes. Portanto, para eles, a constatação de
transmissão oral"-fecal da \textsc{covid}"-19 implica que a \emph{anilingus} pode
representar um alto risco infeccioso. Este é o tom da maioria das
pesquisas realizadas nesse âmbito até o momento. Ainda de acordo com os
pesquisadores, ``para homossexuais, a propagação por meio de relações
anais e via oral"-fecal é possível'' (\textsc{mehrad}, 2020). Chama a atenção a
conexão direta entre sexo oral"-anal como ligado especificamente aos
gays, uma vez que independentemente do gênero, sexo, orientação, etc.,
todos (ou quase todos) têm cu e boca. Essa associação parece remeter às
primeiras pesquisas relativas à \textsc{aids} e ao vírus do \textsc{hiv}. No entanto, até
o momento, a propagação da \textsc{covid}"-19 não é associada a nenhum grupo
identitário. Apesar de uma primeira onda racista contra asiáticos, que
permanece entre nacionalistas e supremacistas brancos (``vírus
chinês''), e dos alardes apocalípticos de alguns religiosos enquanto
castigo divino (``homovírus'').

Relativo à pesquisa, concluíram: ``as recomendações gerais que podem ser
feitas são as de que começar um novo relacionamento é muito arriscado
porque talvez uma das pessoas esteja infectada, e fazer sexo não
monogâmico também é um risco. A única forma segura de fazer sexo é a
primária ou monogâmica, caso nenhuma das duas pessoas tenha saído de
casa ou possua um emprego de risco'' (Idem). Resultado muito semelhante
ao de outro estudo desenvolvido, quase simultaneamente, por
pesquisadores espanhóis sobre as recomendações consensuais para o sexo
seguro durante a chamada pandemia, com o objetivo de evitar o contágio e
manter ativa a prática sexual, devido às ``múltiplas vantagens que uma
sexualidade saudável traz, de acordo com evidências científicas''
(\textsc{cabello}, 2020).

Endossaram como principal recomendação: ``você é seu parceiro sexual
mais seguro'', em conformidade com o guia ``Sexo mais seguro e a
\textsc{covid}"-19'', publicado pelo Departamento de Saúde de Nova York no início
de junho (\textsc{departamento}, 2020). Mas, assim como o guia de recomendações
nova"-iorquino, consideraram que as pessoas ``vão, e devem, fazer sexo''
--- com exceção dos doentes e em recuperação, e profissionais da saúde e
da ``linha de frente'' (\textsc{cabello}, 2020). Alguns governos foram pioneiros
na produção de guias para evitar o sexo casual e as relações não
monogâmicas, como medidas de controle da disseminação do novo
coronavírus. Os Estados da Alemanha, Holanda e Bélgica orientaram os
indivíduos solteiros a terem um ``amigo sexual'' (\textsc{bbc}, 15/05/2020).

Contudo, foram as recomendações do Departamento de Saúde de Nova York as
que mais repercutiram para além do território estadunidense, sendo
mencionadas em pesquisas científicas, traduzidas por organizações \textsc{lgbt}+,
replicadas em jornais de outros países. ``Sexo mais seguro e a
\textsc{covid}"-19'' oferece estratégias de ``redução de danos'', buscando
auxiliar nas ``decisões sobre sexo e sexualidade {[}que{]} precisam ser
equilibradas com a saúde pessoal e pública'' (\textsc{departamento}, 2020).

Os pesquisadores espanhóis dividiram as orientações de segurança em dois
grupos: para casais que vivem na mesma casa e para casais que não moram
juntos ou que têm relacionamentos ``poliamorosos''.

Para o primeiro grupo, as recomendações resumem"-se em guardar quarentena
de ao menos 28 dias, caso alguém tenha sido infectado ou saído do
isolamento. Pessoas com \textsc{covid}"-19, ou que manifestem sintomas da doença,
devem ser isoladas e voltar suas práticas sexuais para o ambiente
virtual e fazer uso de ``jogos eróticos criativos'', combinados com
masturbação. A segunda orientação é lavar bem as mãos antes de ter
relações sexuais ou se masturbar. Preferencialmente não tocar nas
``zonas T'' do parceiro, evitar beijo, sexo oral e anal, preferir
posições em que uma das pessoas fique de costas (\textsc{cabello}, 2020).

Para o segundo grupo, determinam que ``sob nenhuma circunstância'' se
deve fazer ``sexo \emph{in vivo}'' com um novo parceiro, a menos que
esteja imunizado. ``Para aqueles que não têm um parceiro ou que são
muito `erotofílicos', casais não heteronormativos e poliamorosos ou
aqueles que vivem separadamente, a recomendação é a abstinência de
relações sexuais, até que passe o período de incubação {[}de 28 a 33
dias{]} sem sintomas e depois de começarem a coexistir. Somado a isso,
pode ser funcional usar \emph{sexting}\footnote{Prática de
  compartilhamento de conteúdo erótico/sexual por meio de aplicativos de
  mensagens instantâneas e nas redes sociais.} ou sexo virtual'' (Idem).

Neste ``desafio de manter a vida sexual saudável e segura'', os
pesquisadores destacaram que homens que fazem sexo com homens e
trabalhadores do sexo são os ``grupos mais vulneráveis'' e enfatizam que
a pesquisa cobriu apenas uma parcela da diversidade sexual:
heterossexuais, homossexuais e bissexuais. Porém, há ``peculiaridades de
minorias sexuais durante a pandemia que ainda não são conhecidas,
portanto, essas recomendações podem não ser generalizáveis'' (Ibidem).
Logo, para serem \emph{mais bem} assistidas pela ciência e saúde
pública, essas minorias sexuais devem \emph{participar}, se
disponibilizar e confessar.

Não muito diferente dos pesquisadores espanhóis, o guia nova"-iorquino
estabelece que a variação de sexo mais segura é a masturbação com mãos e
brinquedos eróticos bem higienizados. A segunda forma mais segura é com
a pessoa que vive com você, mesmo que não seja um casal. Propõe que haja
um círculo pequeno de pessoas com quem se tenha contato, também sexual
(\textsc{departamento}, 2020). As recomendações não se restringem ao sexo, mas
também aos encontros entre amigos, sempre prezando pelo menor número de
pessoas, assim como as orientações alemãs, belgas e holandesas.

Caso a pessoa desconsidere todas essas orientações e vá fazer sexo com
alguém desconhecido ou não muito próximo, enfatizam que tenha atenção a
possíveis manifestações sintomáticas, converse a respeito com a outra
pessoa, considere a convivência dela com pessoas do ``grupo de risco''
como fator determinante, não negligencie as medidas de proteção e faça
um teste para \textsc{covid}"-19, se possível antes e depois (idem). Os
``cuidados'' durante o sexo, além das ponderações gerais, incluem o uso
de dique de borracha\footnote{O dique de borracha é uma folha de
  borracha utilizada na prática ortodôntica. No Brasil, a recomendação
  mais comum para a mesma finalidade é o ``papel filme''.}, que reduzem
os riscos durante sexo anal e oral (Ibidem).

\textls[-5]{A masturbação, outrora condenada e perseguida constituindo o interdito
maior (\textsc{foucault}, 2015), passou a ser a recomendação mais recorrente,
considerada mais segura e \emph{universalizável}, porque aplicada a
todos independentemente do estado civil e da sexualidade. É atrelada ao
bem"-estar, à saúde sexual e mental. Outras práticas sexuais também
condenadas, perseguidas e criminalizadas, no passado, agora são
toleradas e, em um contexto monogâmico, até aceitas. Estimula"-se sua
inclusão. Contudo, há práticas que passam, ou voltam, a ser consideradas
mais arriscadas, perigosas sob chancela científica. Procura"-se conhecer
e mapear outras práticas sexuais e modos de se relacionar para controlar
seus riscos. Sob a categoria de ``vulneráveis'' ou grupos de ``alto
risco'', homens que fazem sexo com homens\footnote{É interessante notar
  que o uso do termo ``homens que fazem sexo com homens'' também surge
  no contexto da disseminação do \textsc{hiv}, como resposta à luta contra a
  identificação da comunidade gay masculina, enquanto uma categoria
  específica, com a doença, e ampliando o alcance dos estudos, uma vez
  que nem todos os homens que se relacionavam sexualmente com outros
  homens se identificavam como gays.} foram objeto de pesquisas
\emph{on"-line} por meio de redes sociais e aplicativos de encontros, que
se inserem no rol das que investigam as formas sexuais de transmissão da
\textsc{covid}"-19 e das que procuram rastrear as mudanças de condutas.}

Talvez um dos primeiros estudos voltados para esse grupo tenha sido
desenvolvido em Israel, entre março e abril. Participaram 2.562 homens
israelenses, maiores de 18 anos. A maioria diminuiu o número de
parceiros, de relações sexuais e o consumo de álcool e outras
``drogas''; aumentou a prática de \emph{sexting}, o uso de preservativos
e evitou beijar. Mas 39,5\% disseram continuar a encontrar novos
parceiros no período da quarentena. Relatou"-se, entre os mais jovens e
solteiros, maior propensão a ``distúrbios mentais''. Para os
pesquisadores, o sexo casual durante o distanciamento social está
atrelado a ``sentimentos negativos de angústia''. Concluíram: ``ter sexo
casual, desafiando as regulações de distanciamento social, esteve
associado a distúrbios mentais. Sentimentos negativos de solidão devido
ao isolamento social são vistos como uma das sérias consequências da
\textsc{covid}"-19, especificamente entre populações vulneráveis, e deveriam ser
considerados uma barreira potencial à aderência às regulamentações entre
outras populações vulneráveis, como os \textsc{hsh}'' (\textsc{shilo} et al., 2020).

O aumento de ansiedade e solidão entre homens que fazem sexo com homens,
especialmente jovens, também foi observado em pesquisas realizadas nos
Estados Unidos. Assim como a redução do número de parceiros, de relações
sexuais e o consumo de alteradores de consciência. Uma pesquisa
\emph{on"-line} realizada na primeira quinzena de abril com 1051 homens
que fazem sexo com homens, de todas as faixas etárias a partir de 15
anos, constatou que os mais jovens foram os que mais aumentaram esses
usos e os que indicaram menor acesso a preservativos (\textsc{sanchez} et al.,
2020). Entre abril e maio, outra pesquisa \emph{on"-line} com 728 homens
adultos bissexuais e gays estadunidenses indicou que nove entre dez
tiveram relações sexuais durante a quarentena apenas com um parceiro ou
não tiveram relações (\textsc{mckay} et al., 2020). Os pesquisadores qualificam
as mudanças como importantes para reduzir a propagação do novo
coronavírus e, também, de infecções sexualmente transmissíveis (Idem).

Outra pesquisa realizada no mesmo contexto contou com a adesão de 696
homens que se identificam como gays e bissexuais, metade deles estavam
solteiros. Os participantes mostraram"-se cientes das possibilidades de
contágio elencando o beijo, o sexo oral"-retal e o sexo anal,
respectivamente, como as práticas mais arriscadas. Contudo, houve
aumento médio no número de relações sexuais durante a chamada pandemia,
incluindo essas mesmas práticas, exceto o sexo sem preservativo. Os
cientistas vincularam o uso de ``drogas'' com o aumento no número de
parceiros, concluindo que esses ``comportamentos'' podem ser mecanismos
de resposta ao estresse da \textsc{covid}"-19 (\textsc{stephenson} et al., 2020).

Ainda que alguns pesquisadores indiquem uma tendência de homens gays a
terem mais parceiros e relações sexuais ou ``comportamentos de risco'',
evitam condenar explicitamente a promiscuidade. As pesquisas científicas
na área da saúde mostram"-se tolerantes e buscam enfatizar como condutas
menos cautelosas ou consideradas arriscadas estão vinculadas ao contexto
social e histórico de perseguições às minorias sexuais. Reiteram que são
grupos mais ``vulneráveis'' às violências (também no âmbito familiar),
aos problemas financeiros e às doenças, por recorrerem menos ao sistema
de saúde devido aos tratamentos majoritariamente ``excludentes e
preconceituosos''. Em meio às medidas de governo com suas modulações de
quarentena, essas situações se agravaram ainda mais para os solteiros.
Acredita"-se que homens gays casados têm uma redução no grau de
vulnerabilidade. Nesses contextos, frequentemente, sinaliza"-se para os
impactos na saúde mental, podendo produzir \emph{quadros} de ansiedade,
depressão, estresse, entre outros \emph{transtornos}. O que, por sua
vez, seria intensificado pelo uso de substâncias classificadas como
drogas e de álcool. Em todas as pesquisas, foram propostas medidas para
amenizar os danos causados pela \textsc{covid}"-19 na vida dessas pessoas, por
meio de assistência psicológica e de saúde. Documentos assinados por
organizações \textsc{lgbt}+, em diversos países, reiteram o discurso de sua
``vulnerabilidade'', de seu ``alto risco'' e da necessidade de um melhor
tratamento na saúde pública. Reivindicam \emph{cares} e se dispõem a
serem monitorados e cientificamente (re)conhecidos.

A chamada pandemia acentuou um modo de vida que já estava posto e, com
isso, ampliou também os problemas que o acompanhavam. Constatou"-se que,
com a situação colocada pelo vírus, houve um aumento significativo no
uso de \emph{sites}/\emph{apps} de encontro e no consumo de pornografia
e \emph{sex toys}. O \emph{sexting} e a masturbação com brinquedos
sexuais ou vídeos pornôs foram amplamente divulgados na mídia.
Combinando a segurança com a circulação financeira no mercado. O
incentivo ao \emph{sexting}, além de ser propagado por alguns governos,
como o da Argentina, foi defendido pela International Society for the
Study of Women's Sexual Health, ``o novo sexo `realmente seguro' em
muitos casos deve requerer `e"-sex''' (\textsc{international}, 24/04/2020, p. 1).
Assim como a European Society for Sexual Medicine ``acredita que a saúde
e"-sexual pode oferecer oportunidades para melhorar a saúde sexual da
população'' (\textsc{döring}, 2020).

Nesse sentido, o mercado do sexo acompanhou a tendência de renovação e
ampliação do mercado pelas vias digitais. Os produtos sexuais, por seu
caráter ``privativo'', já vieram recobertos com o \emph{know"-how} e
experiência do uso das redes para impulsionar os negócios.

O exacerbado uso das redes sociais, somado ao distanciamento, colocou
ainda a questão dos efeitos psicológicos, e o sexo como forma de
``alívio''. Muito se falou sobre ``cibercrimes'', enfatizando os
sexuais, como exposição de fotos e vídeos íntimos, e a chamada
pedofilia\footnote{Ainda que se reconheça que grande parte dos nomeados
  abusos aconteçam dentro de casa e por pessoas próximas, a ênfase na
  discussão midiática girou em torno do risco proporcionado pela maior
  utilização da internet por crianças e jovens nesse período.},
configurando a internet e as redes sociais --- e o sexo no meio disso ---
como uma saída para o isolamento social e uma ameaça, classificadas no
âmbito jurídico como criminal. Como é praxe, essa questão permaneceu
intocada, apenas se reiterou o punitivismo direcionado àqueles
identificados como ``monstros''.

\textls[-5]{O sexo em si não é tido como um perigo na atualidade. No entanto, o sexo
em meio à ``pandemia'' evidencia o problema do governo das condutas, por
cada um ou cada par, e como este se mostrou fundamental para garantir o
sucesso da manutenção da normalidade, o que retoricamente se chama de
\emph{novo normal}. Se a epidemia do vírus do \textsc{hiv} teve como efeito um
controle de condutas por meio da identidade, a disseminação do novo
coronavírus levantou uma questão econômica. Pois, apesar de ter se
difundido primeiramente por meio da população mais abastada, são os
pobres, miseráveis e as minorias ``vulnerabilizadas'' que enfrentam os
efeitos mais devastadores.}

\textls[-15]{Mas a atenção às práticas sexuais parece retomar de alguma maneira a
questão da identidade, quando ela é associada ao que escapa das condutas
desejáveis, ou seguras. Mais do que a identificação e estigmatização dos
gays, como ocorreu com a \textsc{aids}, as orientações e recomendações parecem
ter como alvo modos de vida que não foram enquadrados e, muitas vezes,
procuram defini"-los por parâmetros de saúde mental de modo a tentar
normalizá"-los. Não se trata de uma sexualidade específica, mas de modos
de se relacionar e de práticas sexuais que não se governam a partir de
prescrições de segurança.}

\textls[-10]{Michel Foucault alertou, em 1982, que a cultura gay, que não se confunde
com a produção de aparatos culturais pelos homossexuais, é: ``uma
cultura que inventa modalidades de relações, modos de vida, tipos de
valores, formas de troca entre indivíduos que sejam realmente novas, que
não sejam homogêneas nem se sobreponham às formas culturais gerais.
{[}\ldots{}{]} É preciso inverter um pouco as coisas, e, mais do que dizer o
que se disse em um certo momento: `Tentemos reintroduzir a
homossexualidade na normalidade geral das relações sociais', digamos o
contrário: `De forma alguma! Deixemos que ela escape na medida do
possível ao tipo de relações que nos é proposto em nossa sociedade, e
tentemos criar no espaço vazio em que estamos novas possibilidades de
relação {[}\ldots{}{]}. Creio que uma abordagem interessante seria fazer com
que o prazer da relação sexual escape do campo normativo da sexualidade
e de suas categorias, e por isso mesmo fazer do prazer o ponto de
cristalização de uma nova cultura'' (\textsc{foucault}, 2012, p. 119--120).}

\textls[-25]{Diante da \emph{profusão de direitos inacabados}\footnote{Cf.: \textsc{passetti}, 2013, p. 2--37; \textsc{passetti} et al., 2019.}, as minorias que os
\emph{portam} tendem a prosseguir participando para melhorar suas
condições de vida. Constata"-se, pelas reivindicações atuais de
organizações \textsc{lgbt}+ acerca de seus direitos em meio à \textsc{covid}"-19, que a
maioria continuará melhorando o possível, bem adequada ao normal e
aderindo às regulamentações de saúde pública. Pessoas que experimentam
outros modos de vida em suas relações afetivas e sexuais, livres de
identidades, enfrentarão o governo das evidências científicas?}

A chamada pandemia do novo coronavírus anunciou uma ``crise social'',
mas foi tomada como uma grande oportunidade para se intervir de forma
incisiva nos modos de vida da população. Confinados entre quatro
paredes, a maior parte do tempo atrás de telas e nos comunicando, é
quase imperceptível um espaço vazio. Talvez ainda precise ser criado,
construído em meio a escombros. Se há um investimento em preencher esses
espaços por meio de controles milimétricos sobre o corpo e seus fluidos,
e a vida em seu descompasso, é preciso rasgar e ruir com a ordem para
que possa entrar o ar. Sem nos esquecer que a vida não é mero fato
biológico.

\pagebreak
\begin{bibliohedra}
\tit{BBC}. Coronavirus: Dutch Singletons Advised to Seek ``Sex Buddy''. In:
\emph{\textsc{bbc} News}, 15/05/2020. Disponível em:
\emph{https://www.bbc.com/news/worl3europe-52685773}. Acesso em:
25/09/2020.

\tit{CABELLO}, F.; \textsc{sánchez}, F.; \textsc{farré}, J. M.; \textsc{montejo}, A. L. Consensus on
Recommendations for Safe Sexual Activity during the \textsc{covid}"-19 Coronavirus
Pandemic. In: \emph{Journal of Clinical Medicine}, 2020. Disponível em:
\emph{https://pubmed.ncbi.nlm.nih.gov/32698369/}. Acesso em: 25/09/2020.

\tit{CHEN}, H.; \textsc{guo}, J.; \textsc{wang}, C.; \textsc{luo}, F. Clinical characteristics and
intrauterine vertical transmission potential of \textsc{covid}"-19 infection in
nine pregnant women: a retrospective review of medical records. In:
\emph{The Lancet}, 2020. Disponível em:
\emph{https://www.researchgate.net/publication/339216834\_Clinical\_characteristics\_and\_intrauterine\_vertical\_transmission\_potential\_of\_COVID"-19\_infection\_in\_nine\_pregnant\_women\_a\_retrospective\_review\_of\_medical\_records}.
Acesso em: 25/09/2020.

\tit{DEPARTAMENTO} de Saúde de Nova York. Sexo mais seguro e a \textsc{covid}"-19. In:
\emph{\textsc{nyc} Government}, 2020. Disponível em:
https://www1.nyc.gov/assets/doh/downloads/pdf/imm/Covid-sex-guidance-pt.pdf.
Acesso em: 25/09/2020.

\tit{DÖRING}, N. How Is the \textsc{covid}"-19 Pandemic Affecting Our Sexualities? An
Overview of the Current Media Narratives and Research Hypotheses. In:
\emph{Archives of Sexual Behavior,} 2020. Disponível em:
\emph{https://www.ncbi.nlm.nih.gov/pmc/articles/PMC7405790/}. Acesso em:
25/09/2020.

\tit{FEDERAÇÃO} Brasileira das Associações de Ginecologia e Obstetrícia. \textsc{faq
febrasgo --- covid} 19 --- Perguntas e Respostas que o \textsc{go} precisa saber.
In: \emph{\textsc{febrasgo}}, 22/04/2020. Disponível em:
https://www.febrasgo.org.br/pt/Covid19/faq.
Acesso em: 25/09/2020.

\tit{FOUCAULT}, M. O saber gay. Tradução de Eder Amaral e Silva e Heliana de
Barros Conde Rodrigues. In: \emph{Revista Ecopolítica}, São Paulo:
Nu"-Sol, 2015, n. 11, p. 2--27.

\titidem. O triunfo social do prazer sexual: uma conversação com Michel
Foucaut. In: \emph{Ditos \& Escritos V}: Ética, Sexualidade, Política.
Manoel Barros da Motta (Org). Tradução de Elisa Monteiro e Inês Autran
Dourado Barbosa. Rio de Janeiro, Forense, 2012, p. 116--122.

\tit{INTERNATIONAL} Society for the Study of Women's Sexual Health. Position
Statement: Sexual Activity and \textsc{covid}"-19. In: \emph{\textsc{isswsh}}, 24/04/2020.
Disponível em:
\emph{https://www.isswsh.org/images/COVID\_Position\_Statement\_5-08-20\_-\_revised.pdf}.
Acesso em: 25/09/2020.

\tit{LI}, D.; \textsc{jin}, M.; \textsc{bao}, P. et al. Clinical Characteristics and Results of
Semen Tests Among Men With Coronavirus Disease 2019. In: \emph{\textsc{jama} Netw
Open}, 2020. Disponível em:
\emph{https://jamanetwork.com/journals/jamanetworkopen/fullarticle/2765654?utm\_source=For\_The\_Media\&utm\_medium=referral\&utm\_campaign=ftm\_links\&utm\_term=050720}.
Acesso em: 25/09/2020.

\tit{LI}, W.; \textsc{li}, G.; \textsc{xin}, C.; \textsc{wang}, Y.; \textsc{yang}, S. Challenges in the Practice
of Sexual Medicine in the Time of \textsc{covid}"-19 in China. In: \emph{The
Journal of Sex Medicine}, 2020, p. 1225--1228. Disponível em:
\emph{https://pubmed.ncbi.nlm.nih.gov/32418751/}. Acesso em: 25/09/2020.

\tit{LI}, G.; \textsc{tang}, D.; \textsc{song}, B; \textsc{wang}, C.; \textsc{qunshan}, S.; \textsc{xu}, C.; \textsc{geng}, H.; \textsc{wu},
H; \textsc{he}, X.; \textsc{cao}, Y. Impact of the \textsc{covid}"-19 Pandemic on Partner
Relationships and Sexual and Reproductive Health: Cross"-Sectional,
Online Survey Study. In: \emph{Journal of Medical Internet Research},
2020. Disponível em: \emph{https://pubmed.ncbi.nlm.nih.gov/32716895/}.
Acesso em: 25/09/2020.

\tit{MCKAY}, T.; \textsc{henne}, J.; \textsc{gonzales}, G.; \textsc{quarles}, R.; \textsc{gavulic}, K. A.;
\textsc{gallegos}, S. G. The \textsc{covid}"-19 Pandemic and Sexual Behavior among Gay and
Bisexual Men in the United States. In: \emph{\textsc{ssrn}}, 2020. Disponível em:
\emph{https://papers.ssrn.com/sol3/papers.cfm?abstract\_id=3614113}.
Acesso em: 25/09/2020.

\tit{MEHRAD}, M.; \textsc{mauro}, M.; \textsc{godoy}, M. F. P.; \textsc{cruz}, E. G.; \textsc{nilforoushzadeh}, M.
A.; \textsc{russo}, G. I. Impact of the \textsc{covid}"-19 pandemic on the sexual behavior
of the population. The vision of the east and the west. In:
\emph{International braz j urol}. Rio de Janeiro, 2020, vol. 46, supl.1.
Disponível em:
\emph{https://www.scielo.br/scielo.php?pid=S1677-55382020000700104\&script=sci\_arttext}.
Acesso em: 25/09/2020.

\tit{PASSETTI}, E. Transformações da biopolítica e emergência da ecopolítica.
In: \emph{Revista Ecopolítica}, São Paulo: Nu"-Sol, 2013, n. 5, p. 2--37.
Disponível em:
\emph{https://revistas.pucsp.br/index.php/ecopolitica/article/view/15120/11292}.
Acesso em: 02/10/2020.

\tit{PASSETTI}, E. et al. \emph{Ecopolítica}. São Paulo: Hedra, 2019.

\tit{QIU}, L.; \textsc{liu}, X.; \textsc{xiao}, M.; \textsc{xie}, J.; \textsc{cao}, W.; \textsc{liu}, Z; \textsc{morse}, A.; \textsc{xie},
Y.; \textsc{li}, T.; \textsc{zhu}, L. et al. \textsc{sars}"-CoV"-2 Is Not Detectable in the Vaginal
Fluid of Women With Severe \textsc{covid}"-19 Infection. In: \emph{Clinical
Infectious Diseases}, 2020, vol. 71, n. 15, p. 813--817. Disponível em:
\emph{https://academic.oup.com/cid/article/71/15/813/5815295}. Acesso em:
25/09/2020.

\tit{SANCHEZ}, T. H.; \textsc{zlotorzynska}, M.; \textsc{rai}, M.; \textsc{baral}, S. D. Characterizing
the Impact of \textsc{covid}"-19 on Men Who Have Sex with Men Across the United
States in April, 2020. In: \emph{\textsc{aids} and Behavior}, 2020. Disponível
em: \emph{https://doi.org/10.1007/s10461-020-02894-2}. Acesso em:
25/09/2020.

\tit{SHILO}, G.; \textsc{mor}, Z. \textsc{covid}"-19 and the Changes in the Sexual Behavior of
Men Who Have Sex With Men: Results of an Online Survey. In: \emph{The
Journal of Sexual Medicine.} 2020, vol. 17, issue 10. Disponível em:
\emph{https://www.sciencedirect.com/science/article/pii/S1743609520308365}.
Acesso em: 25/09/2020.

\tit{SONG}, C.; \textsc{wang}, Y.; \textsc{li}, W.; \textsc{hu}, B.; \textsc{chen}, G.; \textsc{xia}, P.; \textsc{wang}, W.; \textsc{li}, C.;
\textsc{diao}, F.; \textsc{hu}, Z.; \textsc{yang}, X.; \textsc{yao}, B.; \textsc{liu}, Y. Absence of 2019 novel
coronavirus in semen and testes of \textsc{covid}"-19 patients. In: \emph{Biology
of Reproduction}, 2020, vol. 103, n. 1, p. 4--6. Disponível em:
\emph{https://academic.oup.com/biolreprod/article/103/1/4/5820830}.
Acesso em: 25/09/2020.

\tit{STEPHENSON}, R.; \textsc{chavanduka}, T. M. D.; \textsc{rosso}, M. T.; \textsc{sullivan}, S. T.;
\textsc{pitter}, R. A.; \textsc{hunter}, A. S.; \textsc{rogers}, E. Sex in the Time of \textsc{covid}"-19:
Results of an Online Survey of Gay, Bisexual and Other Men Who Have Sex
with Men's Experience of Sex and \textsc{hiv} Prevention During the \textsc{us} \textsc{covid}"-19
Epidemic. In: \emph{\textsc{aids} and Behavior}, 2020. Disponível em:
\emph{https://doi.org/10.1007/s10461-020-03024-8}. Acesso em: 25/09/2020.

\tit{WHO}. Defining sexual health: Report of a technical consultation on
sexual health, 28--31 January 2002, Geneva. In: \emph{who.int}, Genebra:
\textsc{who}, 2006. Disponível em: \emph{https://bit.ly/3fwRjLU}. Acesso em: 25/09/2020.
\end{bibliohedra}

\chapterspecial{Amor selvagem}{}{Gustavo Ramus}

\epigraph{amor é o jeito que você pisa no chão.}{arrudA, 2010}

\noindent{}O inesperado aconteceu: o mundo parou. Ninguém esperava que o
capitalismo fosse capaz de cessar suas atividades, mesmo que por alguns
instantes. A crise causada pelo novo coronavírus mostrou, para algumas
pessoas, que suas vidas valem mais do que o lucro de seus patrões, e
viver em função do trabalho não faz sentido algum. Ao mesmo tempo também
evidenciou, para outras pessoas, que a vida de seus empregados não vale
mais do que seu próprio lucro.

No início da pandemia, pareceu haver um movimento de repensar a
sociedade e o nosso estilo de vida. Impactamos brutalmente no planeta.
Por onde passamos, deixamos nossas pegadas de sangue e rastros de
destruição. No entanto, esse pensamento não teve muito fôlego e logo foi
engolido pelo desejo da volta à normalidade e ao ritmo frenético das
máquinas. Mesmo à beira do abismo e de um cenário próximo de fim do
mundo, a Humanidade insiste no que não deu certo. Estamos aprisionados
em uma estrutura de pensamento que produz verdades universalizantes, nas
quais se fundamentam as práticas e as relações de poder.

Em \emph{Do governo dos vivos}, Foucault (2014) narra a passagem do
imperador romano Sétimo Severo, que reinou entre 193 e 211 d.C. Sétimo
Severo pintou no teto do seu palácio uma representação da conjunção
astrológica no dia de seu nascimento, cuja finalidade era legitimar suas
decisões pelo \emph{logos} que presidia esse dia, fundando, assim, o seu
reinado na astrologia. Assegurado pelos astros, Sétimo Severo não
somente justificava sua ascensão ao poder, como afastava a possibilidade
de quem quer que fosse de conspirar contra sua fortuna. Contudo, a
``verdade dos astros'' não era um monopólio do Império e poderia estar
ao alcance de qualquer um. Sétimo Severo então decretou a pena de morte
contra os astrólogos ou qualquer outro que pudesse afirmar uma nova
``verdade'' que ameaçasse seu poder. Nota"-se como uma ``verdade'',
fundada na astrologia, mostrava uma ordem do mundo sobre a qual o
imperador apoiava seu poder e proferia sua justiça e, também, como ele
utilizava esse dispositivo de governo para anular outras formas de
verdade que poderiam, eventualmente, contrariar aquela na qual ele se
firmava.

O saber proferido por essa ordem cosmológica tornava o exercício do
poder do imperador quase uma necessidade. Estava em jogo o governo como
a manifestação de verdade da ordem mundana. Esse exemplo evidencia que o
exercício de um poder não pode ser separado da produção de saberes
verdadeiros, não importando qual a sua procedência; mais do que isso,
expõe como é preciso produzir conhecimento para se governar. Assim,
Foucault nos apresenta a uma nova palavra: \emph{alêthourgia}\footnote{Palavra
  forjada por Foucault a partir do termo \emph{alêthourguês,} expressão
  grega utilizada por Ponticus Heraclides (388--322 a.C.), gramático
  grego e discípulo de Platão, para designar quando alguém pronuncia a
  verdade ou para designar o que é verídico.}, conjunto de procedimentos
pelo qual se manifesta a verdade. Tanto o discurso científico como o
religioso produzem um determinado conhecimento e são formas de
\emph{alêthourgia}, ferramentas essenciais para governar a conduta dos
indivíduos.

A retomada do céu estrelado de Sétimo Severo é uma tentativa de
demonstrar como a ligação entre o exercício de poder e a manifestação da
verdade é muito mais antiga do que a constituição do Estado moderno, e
anterior ainda ao próprio Sétimo Severo. Para assegurar o exercício de
um poder, é necessário combater e anular tudo que possa abalar as
verdades que o sustentam. Isso explica as diversas guerras religiosas ao
longo da história, períodos como a inquisição, em que pessoas eram
queimadas vivas, e o recorrente extermínio de etnias e povos que não
compactuam com um ``entendimento de mundo'' que foi dado e estabelecido
como ``verdadeiro'' por alguém que se diz, posiciona ou está em patamar
superior.

A partir do momento em que a cultura judaico"-cristã criou um Deus à
imagem e semelhança do Homem, a nossa espécie foi definitivamente
apartada da natureza. Esquecemos do elementar: somos animais, dependemos
de ar para respirar, de água para sobreviver e de alimentos para nos
sustentar e, como bichos, somos elementos constituintes da natureza. A
divinização do ser humano é a \emph{alêthourgia} que sustenta nosso
domínio predatório em relação aos chamados recursos naturais e às demais
espécies que habitam o planeta. É evidente que, mesmo no interior do
cristianismo, podemos encontrar outras interpretações, como a de
Francisco de Assis, que tinha o sol e a lua como irmãos, uma percepção
de mundo capaz de compreender que tudo que move é sagrado e todos nós
carregamos uma parcela de divino e do maravilhoso. No entanto,
preponderou o pensamento"-posse. O que entendemos hoje como Humanidade é
uma consequência desse Deus/Homem, fruto de um narcisismo
antropocêntrico, e de um modo de ``estar no mundo'', que prolifera
relações de dominação, propriedade, exploração e aniquilamento.

Não se trata aqui de um ataque direto ao cristianismo ou a qualquer
outra religião. Sabemos que é possível estabelecer uma perspectiva
libertária mesmo tendo como ponto de partida os ideais cristãos. Liev
Tolstoi anarquizou o cristianismo e elaborou um pensamento próximo ao de
Spinoza (1957), no qual Deus e Natureza convergem. Ao desvencilhar"-se da
soberania divina e da transcendentalidade, o anarquismo cristão
reconecta Homem e Natureza, ao mesmo tempo que dispensa intermediários
com o divino. O anarquista russo reformulou o cristianismo em práticas
cotidianas cuja finalidade era realizar o Reino de Deus na Terra,
retirando"-o do lugar de uma promessa de vida após a morte. Realizando o
Reino dos Céus no campo do possível, Tolstoi (1994) atribui ao
cristianismo uma existência próxima da heterotopia, um viver efetivo da
utopia.

As práticas que envolvem essa singular interpretação se revelam
insubmissas a toda forma de autoridade, pacifismo e desobediência civil.
Spinoza não concebe a imaterialidade de Deus e se aproxima de uma
concepção panteísta. Para ele, tudo o que existe é corpóreo, toda
substância é um corpo, e Deus está em toda forma de manifestação de
vida. Essa ideia simples da divindade arruína a imagem de Deus
semelhante aos governantes. A doutrina religiosa tornou"-se o laço
sagrado originado pelo temor que tem como consequência a obediência. A
religião é uma das formas de o homem provar sua superioridade em relação
às outras espécies e de se colocar acima da natureza. Ao impor a noção
de um Deus onipotente e onipresente a um deus imanente, criou"-se a ideia
de um Deus/Estado capaz de controlar, manipular, julgar e condenar o que
se denominou Humanidade, e que, por sua vez, também não passa de mais
uma construção.

A maioria das pessoas estranha a aproximação entre cristianismo e
anarquismo. De fato, as instituições eclesiásticas foram alvo de ataques
por diversos anarquismos. No entanto, as fundações religiosas não são a
expressão da religiosidade, são apenas matrizes defensoras de uma
interpretação que se pretende única e verdadeira. O estranhamento maior
deveria ser a associação entre cristianismo e capitalismo, o que parece
totalmente naturalizado hoje. Jesus andava com os pobres, era inimigo do
Império Romano e dos comerciantes, principalmente daqueles que ficavam
nas portas dos templos sagrados. Não fazia distinção entre as pessoas de
diferentes classes sociais e ensinou a dividir os alimentos, o que ficou
conhecido como milagre da multiplicação. Para muitos historiadores, as
primeiras comunidades cristãs foram a primeira experiência socialista
registrada na história ocidental. No evangelho de Mateus, encontramos a
seguinte passagem: ``Ninguém pode servir a dois senhores. Com efeito, ou
odiará um e amará o outro, ou se apegará ao primeiro e desprezará o
segundo. Não podeis servir a Deus e ao dinheiro'' (\textsc{mt}, 6: 24).

Tolstoi entendia que, como um filho de Deus, ele só poderia seguir suas
palavras e não devia obediência ao czar ou qualquer outro governante.
Esse é também o ponto principal de Etienne de La Boétie (1999): a recusa
em servir; ou ser cúmplice da tirania. Para La Boétie, nada é mais
contrário a Deus do que a tirania. No conto ``Os três anciãos'', Tolstoi
(1997) escreve, como um seguidor de Deus, que ele só sabe e só pode
servir e nutrir a si mesmo, uma vez que O Reino do Céu está dentro de
cada um de nós. Nessa perspectiva, a liberdade está na aproximação com a
natureza, e, uma vez que tudo e todos são igualmente criações divinas,
não há por que estabelecer relações de hierarquia entre os humanos ou de
domínio destes em relação à natureza.

No entanto, devemos admitir que o pensamento de Tolstoi, por mais que
tenha rompido relações de dominação e desestabilizado poderes, parte da
construção do pensamento ocidental que estabelece o absoluto universal
como sagrado. O sagrado já não se encontra somente no campo das
religiões, mas também se faz presente na razão e na política. Por ele
podemos compreender o Estado, o Homem, a Humanidade, a moralidade e
todas as formas de pensamento que se pretendem verdades universais. O
Estado sustenta"-se, dentre outras coisas, pela abstração do bem social.
Por isso, quando os anarquismos se posicionam combativamente diante
dessas formas de transcendentalidades, desafinam o canto do poder. Mas,
se recusar a ser propriedade do Estado já não é mais suficiente, é
preciso questionar também as transcendentalidades que compõem os
anarquismos, lançar mão de nossa animalidade e não ser proprietários da
natureza.

Reconhecer"-se como parcela do corpo divino não é uma exclusividade do
anarquismo cristão. Podemos encontrar isso em outras culturas que têm
sido abafadas, caladas e oprimidas pelo olhar ocidental. Tais
pensamentos muitas vezes se mostram mais potentes, levando em conta que
não carregam o peso da piedade e da benevolência cristã. ``Deveríamos
admitir a natureza como uma imensa multidão de formas, incluindo cada
pedaço de nós, que somos parte de tudo: 70\% de água e um monte de
outros materiais que nos compõem. E nós criamos essa abstração de
unidade, o homem como medida das coisas, e saímos por aí atropelando
tudo, num convencimento geral até que todos aceitem que existe uma
humanidade com a qual se identificam, agindo no mundo à nossa
disposição, pegando o que a gente quiser. Esse contato com outra
possibilidade implica escutar, sentir, cheirar, inspirar, expirar
aquelas camadas do que ficou fora da gente como ``natureza'', mas que,
por alguma razão, ainda se confunde com ela. {[}\ldots{}{]} Os quase"-humanos
são milhares de pessoas que insistem em ficar fora dessa dança
civilizada, da técnica, do controle do planeta. E, por dançar uma
coreografia estranha, são tirados de cena, por epidemias, pobreza, fome,
violência dirigida'' (\textsc{krenak}, 2019, p. 69--70).

Pierre Clastres empresta um novo olhar para a antropologia, que antes
tomava os povos indígenas como seres primitivos, não evoluídos,
incapazes de se organizarem estruturalmente em organizações como as
nossas. Sob essa nova ótica, ele afirma que esses povos não chegaram à
forma de Estado, não por incapacidade, mas sim por escolha: o Estado
nunca lhes interessou. Clastres inaugura o termo ``sociedades contra o
Estado'', fazendo oposição ao termo primitivo. A ideia de que um povo é
primitivo parte de um olhar eurocêntrico, que toma a nossa sociedade
como o suprassumo da evolução; enquanto os povos que não reproduzem
nosso sistema teriam sido incapazes de trilhar um percurso rumo ao
desenvolvimento, ficando no meio do caminho. Ao classificar esses povos
como primitivos, estamos decretando a sua inferioridade, o que justifica
sua catequização e seu extermínio.

Em suas anotações sobre o ``Discurso da Servidão Voluntária'', Pierre
Clastres (1999) opõe sociedades de liberdade às sociedades sem
liberdade. Onde há Estado não há liberdade. Nas ``sociedades contra o
Estado'' ou de liberdade, não há desigualdade ou desejo de submissão. O
chefe exerce uma liderança, mas ninguém lhe deve obediência. Na nossa
sociedade, por sua vez, temos a herança cristã da obediência como uma
virtude, e, para quem não obedece, opera"-se uma série de mecanismos de
punição. Os Estados existem para administrar as misérias, assegurar a
propriedade, garantir que a maior parte da riqueza produzida no mundo
fique nas mãos de apenas um por cento da população mundial e que as
indústrias consumam a natureza da forma como elas bem entenderem.
Clastres acerta ao empregar o termo ``contra o Estado'', mas classificar
esses povos como sociedade evidencia um resquício de um olhar
civilizatório. Será que eles se enxergam como sociedade? Esse conceito
não seria mais uma forma de pensamento totalizante imposta pela
civilização?

É preciso questionar toda forma de pensamento totalizante que se julgue
verdadeira e produza subjetividades, mesmo quando estas se autodenominem
anarquistas. Alguns anarquismos se aproximam de discursos humanistas
substituindo o amor a Deus pelo amor à Humanidade, e tanto um quanto
outro são ideias construídas, abstrações que servem de alicerce para a
devastação do planeta. Primeiro se criou um Deus/Homem, depois um
Deus/Estado e, agora, chegamos ao Homem/Estado. Vivemos em um mundo
doente. Em vez de lutar por liberdades, as pessoas clamam por
regulamentação, controle e policiamento. Muito pior do que isso,
incorporam a figura do fiscal, do agente e do juiz. Tudo é controlado,
medido, inspecionado e julgado. Na sociedade de soberania, tínhamos a
figura do pastor disseminada em várias profissões e ocupações. Agora,
nas sociedades de controle, todos são dispositivos voluntários. O
celular tornou"-se mais que uma arma, virou um instrumento de vigilância
permanente, e as redes sociais um tribunal constante. Vivemos na era dos
chamados por ``lacração'' e ``cancelamento''.

A expressão do autoritarismo está na tela mais próxima a você e cabe no
seu bolso. É inevitável ver as pessoas incorporarem, ao mesmo tempo, a
figura do policial e do juiz. Elas internalizam e reproduzem o Estado o
tempo todo. Apontam, julgam, condenam, mas não trazem pensamentos ou
ações libertárias e, sem perceberem, tornam"-se inimigas da liberdade.

Um exercício possível e urgente para nós hoje é tocar fogo no juízo,
incendiar o tribunal que se instaurou em nós mesmos, matar o policial
que nos habita. Precisamos estar atentos às armadilhas que a nossa época
nos impõe. O que parece nos facilitar a vida também nos submete a uma
relação de dependência e exposição. O modismo do cancelamento evidencia
como o pensamento colonizador se incrustou no comportamento humano e
tornou impossível o convívio com o diferente. Diante da incapacidade de
lidar com a diferença, a anulação surge como o caminho mais rápido.
Anular, oprimir, constranger são táticas de imposição de uma verdade,
uma forma de catequização.

A pandemia que parou o planeta, e é tratada por alguns governantes como
uma simples ``gripezinha'', já matou mais de cento e quarenta e seis mil
pessoas, no Brasil, e mais de um milhão no mundo todo, sem contar as
subnotificações. Por mais que o vírus da \textsc{covid}"-19 não tenha sua origem
totalmente comprovada, parece difícil desassociá"-lo da ação predatória
imposta pelo estilo de vida capitalista. Bastou as atividades
industriais e comerciais pararem por alguns instantes para proliferarem
notícias de animais tomando as cidades, melhora na qualidade do ar e das
águas\ldots{}

O Homem vem pisando no chão de forma bruta e com travas afiadas,
causando feridas irreparáveis. Recordes de incêndios na Amazônia, no
Pantanal, na Mata Atlântica e em outros países como Austrália, \textsc{eua} e
Ucrânia. Rejeitos de mineração lançados em rios. Incontáveis litros de
petróleo derramados nos oceanos, para não falar na quantidade de
resíduos plásticos e outros materiais descartáveis. Sem contar episódios
trágicos e inadmissíveis como: a bomba atômica em Nagasaki em 1945; o
vazamento radioativo em Kyshtym, na Rússia, em 1957; o vazamento químico
em Sevesco, na Itália, em 1976; o acidente nuclear em Chernobyl, em
1986; a explosão de setecentos poços de petróleo durante a Guerra do
Golfo no Kuwait, em 1991; Fukushima, em 2011; o rompimento de barragem
de resíduos de mineração em Bento Gonçalves, em 2015; outro rompimento
de barragem em Brumadinho, em 2019; a explosão em Beirute, em 2020, etc.
Ficou claro que, para grande parte das pessoas, a vida vale menos que a
economia. Uma árvore morta é muito mais lucrativa que uma árvore viva,
mas somente em vida ela produz oxigênio, elemento indispensável para
nossa existência. Assim como o nosso lixo, a vida na civilização é
descartável. A sociedade falhou e chegou ao seu limite.

Vivemos o que é chamado de ``novo normal'', cercados por telas, privados
das nossas relações de afetos, com medo do contágio e da morte,
conectados às notícias em qualquer lugar do mundo, mas descuidados com o
que acontece debaixo do nosso nariz. Pessoas aflitas anseiam pela volta
ao normal. Mas que normal é esse? Estima"-se que quase metade da
população global viva abaixo da linha da pobreza. No Brasil, cerca de
treze milhões e meio de pessoas se enquadram em situação de extrema
pobreza. A fome não é mais uma exclusividade dos chamados países de
terceiro mundo. Em 2019, segundo a \emph{Worldometers}, estimava"-se uma
média de vinte a quatro mil mortes por dia causadas pela fome. De acordo
com a Organização Internacional do Trabalho, o número de desempregados
no mundo é superior a cento e noventa milhões; treze milhões só aqui no
Brasil. Mais de onze milhões de pessoas estão encarceradas no mundo
todo. O índice mundial de suicídios só aumenta. Conforme publicação da
Organização Mundial de Saúde (\textsc{oms}), a cada quarenta segundos registra"-se
uma morte por suicídio. Ainda de acordo com a \textsc{oms}, trezentos e vinte e
dois milhões de pessoas no mundo sofrem de depressão. O Brasil é o país
com mais casos de depressão na América Latina, com quase doze milhões de
casos registrados. Sem falar de outros tipos de transtornos que também
aumentam a cada ano; das guerras, declaradas ou não, pelo mundo todo; e
da situação delicada em que vivem mais de sessenta e oito milhões de
refugiados. Esse é o nosso normal.

O apego à pequenez da vida do cidadão comum parece maior do que a
preocupação com a vida na sua mais ampla definição. À espera de uma
vacina milagrosa, o cidadão exemplar exime"-se de sua responsabilidade,
alivia sua consciência e segue contribuindo para a continuidade desse
mundo podre. Enquanto ele se esforça para ser um sujeito normal, fazer
tudo igual e ser só mais um boçal nessa corrida pela conquista do ouro
de tolo, o mundo que conhecemos está se dissolvendo. Seguimos
enfeitiçados, sujeitados, devotos e obedientes, acreditando na
necessidade do Estado. Essa sociedade representada pela figura do Homem
hétero e branco carrega consigo, ao longo dos séculos, um rastro de
mortes, vírus, guerras, escravidão e extermínios.

Diante desse cenário catastrófico, o que estamos esperando? Uma resposta
dos governantes? A salvação divina? O surgimento de novas lideranças
para uma remodelação do capitalismo? E por que não buscar outras
respostas? Esse mundo que inventamos não nos oferece soluções para os
problemas que nós mesmos criamos. Talvez seja o momento de olhar por
outras perspectivas e ouvir a voz de culturas que foram caladas durante
todos esses séculos.

\begin{verse}
Um índio descerá de uma estrela colorida, brilhante\\
De uma estrela que virá numa velocidade estonteante\\
E pousará no coração do hemisfério sul\\
Na América, num claro instante\\
Depois de exterminada a última nação indígena\\
E o espírito dos pássaros das fontes de água límpida\\
Mais avançado que a mais avançada das mais avançadas das tecnologias\\
Virá\\
Impávido que nem Muhammad Ali\\
Virá que eu vi\\
Apaixonadamente como Peri\\
Virá que eu vi\\
Tranquilo e infalível como Bruce Lee\\
Virá que eu vi\\
O axé do afoxé Filhos de Gandhi\\
Virá\\
Um índio preservado em pleno corpo físico\\
Em todo sólido, todo gás e todo líquido\\
Em átomos, palavras, alma, cor\\
Em gesto, em cheiro, em sombra, em luz, em som magnífico\\
Num ponto equidistante entre o Atlântico e o Pacífico\\
Do objeto"-sim resplandecente descerá o índio\\
E as coisas que eu sei que ele dirá, fará\\
Não sei dizer assim de um modo explícito\\
Virá\\
Impávido que nem Muhammad Ali\\
Virá que eu vi\\
Apaixonadamente como Peri\\
Virá que eu vi\\
Tranquilo e infalível como Bruce Lee\\
Virá que eu vi\\
O axé do afoxé Filhos de Gandhi\\
Virá\\
E aquilo que nesse momento se revelará aos povos\\
Surpreenderá a todos não por ser exótico\\
Mas pelo fato de poder ter sempre estado oculto\\
Quando terá sido o óbvio.\\
(Caetano Veloso, \emph{Um índio}, 1975)
\end{verse}

\begin{quote}
Os únicos núcleos que ainda consideram que precisam se manter
agarrados nessa Terra são aqueles que ficaram meio esquecidos pelas
bordas do planeta, nas margens do rio, nas beiras dos oceanos, na
África, na Ásia ou na América Latina. Esta é a sub"-humanidade: caiçaras,
índios, quilombolas, aborígenes. Existe, então, uma humanidade que
integra um clube seleto que não aceita novos sócios. {[}\ldots{}{]} Fomos,
durante muito tempo, embalados com a história de que somos a humanidade
e nos alienamos desse organismo de que somos parte, a Terra, passando a
pensar que ela é uma coisa e nós, outra: a Terra e a humanidade. Eu não
percebo que exista algo que não seja natureza. Tudo é natureza. O cosmos
é natureza. Tudo em que eu consigo pensar é natureza. {[}\ldots{}{]} Tomara
que não voltemos à normalidade, pois, se voltarmos, é porque não valeu
nada a morte de milhares de pessoas no mundo inteiro. Depois disso tudo,
as pessoas não vão querer disputar de novo seu oxigênio com dezenas de
colegas num espaço pequeno de trabalho. As mudanças já estão em
gestação. {[}\ldots{}{]} Não podemos voltar àquele ritmo, ligar todos os
carros, todas as máquinas ao mesmo tempo. Seria como se converter ao
negacionismo, aceitar que a Terra é plana e que devemos seguir nos
devorando. Aí sim, teremos provado que a humanidade é uma mentira
(\textsc{krenak}, 2020, posição 44--51; 98--102).
\end{quote}

Por que não experimentar outras formas de pensamento? Outros olhares que
seguem abafados, encurralados, isolados e feridos? Por que não dar
passagem a um pensamento selvagem, sem imperativos categóricos,
paradigmas, superego? Devemos parar de olhar para nós como lobos de nós
mesmos, ou como seres bons por natureza, ou até mesmo bípedes sem penas.
A vida não cabe em uma medida. O Homem não é a medida do mundo. O
planeta chegou ao limite da devastação. Esqueçamos o mal"-estar na
civilização: o mau é estar na civilização. Não existe produção de
riqueza que não cause impacto ambiental e social. Isso tudo é
consequência do capitalismo. A normalização da fome e da miséria é
apenas um dos perigosos efeitos do que entendemos por civilidade.

Hoje, o capitalismo repaginou seu enunciado, e o discurso da
sustentabilidade está estampado em todas as empresas que, por sua vez,
não cogitam parar suas atividades, tampouco cessar suas fontes de lucro.
Esse discurso empresarial preocupado com a saúde do planeta mascara a
administração ou o gerenciamento de uma morte lenta, mas em momento
algum se cogita a hipótese de não matar. Grandes corporações investem na
compra de terras com nascentes. Há um plano em curso de privatização das
águas. Se isso de fato acontecer, nos tornaremos ainda mais reféns
deles. Fazer viver e deixar morrer não é uma máxima de governo
direcionada apenas à sociedade, mas a tudo que está inserido no planeta.

Algumas lideranças de povos indígenas entendem o novo coronavírus como
uma resposta da natureza. O vírus só está colocando a espécie humana em
perigo, os outros animais não estão morrendo em decorrência da \textsc{covid}"-19.
O estilo de vida ocidental serve"-se dos recursos naturais de forma bruta
e não pensa em estabelecer um equilíbrio ecossistêmico. A natureza quer
apagar o que está desregulando o grande organismo do planeta e recuperar
novamente sua harmonia. Se para isso o ser humano tiver que sair de
cena, que saia! Talvez seja o momento de uma inversão: o momento em que
a mosca devorará a aranha. Não somos mais importantes do que o planeta.
Ele existe muito tempo antes de nós, e não seríamos a primeira espécie a
entrar em extinção.

``O que aprendi ao longo dessas décadas é que todos precisam despertar,
porque, se durante um tempo éramos nós, os povos indígenas, que
estávamos ameaçados de ruptura ou da extinção dos sentidos das nossas
vidas, hoje estamos todos diante da iminência da Terra não suportar a
nossa demanda'' (\textsc{krenak}, 2019, p. 45). Krenak sugere que o homem branco
deveria pisar mais suavemente na terra. Se o amor é o jeito como a gente
pisa no chão, devemos fazer dele uma prática revolucionária de não
dominação e de não posse. Uma prática selvagem na qual não haja espaço
para submissões e propriedades. É preciso desaquecer as ideias
hegemônicas como a de Humanidade e pôr um fim ao baile civilizatório, a
essa dança sem cadência, essa decadência de uma perspectiva que nos
aproxima do fim do mundo. Os bárbaros, a chegada dos bárbaros, para
relembrar o poeta grego Konstantínos Kaváfis, eram e continuam sendo a
solução. É o momento de dar lugar ao selvagem, bárbaro, bruto e sem
razão. Está mais do que na hora de mudarmos nossa maneira de estar no
mundo, de deixar certas convicções para trás e pôr um fim ao
pensamento"-posse. É tempo de refazermos nossas pegadas e pensarmos no
rastro que vamos deixar no amanhã. Pensar menos e pisar mais suavemente,
como o amor selvagem.


\pagebreak
\begin{bibliohedra}
\tit{ARRUDA}. \emph{As menores distâncias podem levar uma vida}, São Paulo:
Selo Edith, 2010.

\tit{CLASTRES}, Pierre, \emph{Liberdade, Mau Encontro, Inominável}. In:
\emph{Discurso da Servidão Voluntária}. Tradução de Laymert Garcia dos
Santos. São Paulo: Brasiliense, 1999, p. 109--124.

\tit{FOUCAULT}, Michel. \emph{Do governo dos vivos.} Tradução de Eduardo
Brandão. São Paulo: Martins Fontes, 2014.

\tit{KRENAK}, Ailton. \emph{O amanhã não está à venda,} São Paulo, Companhia
das Letras: 2020 {[}\emph{E"-book}{]}.

\titidem. \emph{Ideias para adiar o fim do mundo}, São Paulo: Companhia
das Letras, 2019.

\tit{LA BOÉTIE}, Etienne de. \emph{Discurso da Servidão Voluntária}. Tradução
de Laymert Garcia dos Santos. São Paulo: Brasiliense, 1999.

\tit{SPINOZA}, Baruch de. \emph{Ética.} Tradução de Lívio Xavier, São Paulo:
Atena Editora, 1957.

\tit{TOLSTOI}, Liev. \emph{A morte de Ivan Ilitch e Os três Anciãos.} Tradução
de Boris Schnaiderman, Aurélio Buarque de Holanda e Paulo Rónai. Rio de
Janeiro: Ediouro, 1997.

\titidem. \emph{O reino de Deus está em vós}. Tradução de Celina
Portocarrero, Rio de Janeiro: Rosa dos Tempos, 1994.
\end{bibliohedra}

\chapterspecial{Catástrofe, história e destino}{Da pandemia como verdade e como
representação}{Ilana Viana do Amaral}

\hfill\ \emph{Para Sandra, minha amiga"-irmã.}

\hfill\ \emph{Da \textsc{covid} nós perdemos, tá sem jeito e dói.}

\hfill\ \emph{Também no que de ti guardo e levo, continuo, insisto.}

\bigskip

\noindent{}A emergência (do verbo emergir, aparição.) pandêmica do novo coronavírus
em humanos, com a \textsc{covid}"-19, caiu sobre o globo sob os signos da
catástrofe e da emergência (aquilo cuja resolução reclama ação urgente).
Caiu, como verdadeira catástrofe, como um céu desabando sobre nossas
cabeças. Uma emergência em forma de pandemia viral no século \textsc{xxi} não
era, exatamente, inesperada, dada a extensão e a amplitude da devastação
do planeta e da vida humana, produzidas sob o domínio das formas capital
e Estado como formas dominantes de existência. De Rob Wallace, biólogo
evolucionista anticapitalista, autor de \emph{Big Farms Make Big Flu,} a
Bill Gates, passando por Obama e pelos irmãos Koch, mas também por
Sting, Raoni e Greta, não havia voz, dentre as que acompanham as
questões ambientais contemporâneas, que não reconhecesse a iminência da
aparição de um evento catastrófico ligado à degradação ambiental. Fosse
privadamente, num reconhecimento diante ``dos seus pares'', para junto
com eles articular uma negação pública de fachada,\footnote{Refiro"-me ao
  negacionismo climático orquestrado pelos irmãos Koch, traficantes de
  petróleo e articuladores da ``nova(!)direita'' e seus antecedentes
  correlatos (como o negacionismo da indústria de tabaco). Esse
  negacionismo para uso do público nunca duvidou, privadamente, da
  emergência climática, que em verdade sabe perfeitamente que existe
  desde os anos 1960. A admissão pública pela indústria petrolífera
  implicaria uma renúncia à sua identidade, preço que, definitivamente,
  eles não pagam. Dessa posição discursiva, encontramos o modelo nas
  páginas da \emph{Filosofia na alcova}, de Sade (1999). Sobre o saber
  dos irmãos Koch e o negacionismo, é possível acessar a documentação do
  relatório Stanford e seus desdobramentos e ramificações em
  \emph{https://oquevocefariasesoubesse.blogspot.com/2019/07/eles-sabiam-verdadeira-conspiracao-por.html}.
  Acesso em: 05/10/2020.} como os Koch, fosse publicamente, nos
discursos de vários matizes que apontam o aquecimento global como um
problema, não havia discurso sobre o ambiente que não tomasse a sério a
iminência de uma pandemia. Era uma questão de ``mais dia, menos dia'' e
de ``mais uma menos uma''. A catástrofe, amplamente anunciada, tinha um
signo à sua espera. Como signo, não a continha no que ela tem de real.
Por isso mesmo, seu anúncio não preparava ninguém, apenas a preparava.

O caráter de catástrofe iminente, assumido como um prenúncio e depois no
anúncio de sua chegada pelos organismos internacionais, vigilâncias
epidemiológicas e \emph{medias}, preparou as posições reativas ao
evento. As reações diante do desastre (aquilo que impõe desistir dos
astros e olhar para a Terra) produziram"-se, como só ocorre nos
desastres, enfrentados desde a Terra e desde os corpos nela situados,
com os meios e segundo as formas que os sujeitos, para isso, encontramos
já dadas. Como tais, nunca à altura de um desastre, que exige, para ser
verdadeiro desastre, formas novas, exige formar, porquanto as formas já
dadas são o que, num verdadeiro desastre, desabam. Na ausência de formas
novas, a repetição das posições típicas diante das tragédias, ainda sem
fim (sem catástrofe, o nome dado ao final das tragédias), já cotidianas
no capitalismo contemporâneo, posições via de regra --- porque típicas
--- situadas numa insistência que colabora e participa ativamente na
produção sem fim da catástrofe, sob a crença de combatê"-la.

O escamoteamento ``realista'' do que verdadeiramente estava em questão
foi, assim, a tônica. O impossível, que era exigido pela emergência
catastrófica real do vírus, foi o verdadeiramente excluído no
enfrentamento feito. O que teve de ser feito foi ``extra"-ordinário''.
Não o impossível que a catástrofe real exigia. Em razão do ``realismo''
econômico, esse impossível, aliás, sequer foi globalmente aventado: agir
a sério para suprimir a pandemia, parar tudo, frear completamente a
transmissão. Para isso seria necessário assumir, no catastrófico da
pandemia, a catástrofe do mundo presente, enfrentando"-a e dando"-lhe um
fim, inventando, de outro modo, outro mundo. O que foi feito, ao
contrário, foi o enfrentamento possível à \textsc{covid}"-19, segundo as razões
que contam para as finanças e os Estados: razões econômicas, que
articulam presentemente nossas vidas e, nelas, as formas e os meios de
enfrentamento dessa e de quaisquer tragédias.

Salvo nas raras experiências que buscaram enfrentá"-la de modo radical,
isto é, contra as formas dadas, mercantis e estatais --- porquanto
experimentassem já, segundo restritas possibilidades de isolamento,
viver contra a ordem atual das coisas, comunidades como as de Rojava e
as zapatistas, que tentaram com os meios que possuíam cessar
completamente a transmissão ---, a mitigação foi, desde o princípio, a
baliza realista das hipóteses discutidas. Esse realismo das formas
existentes desrealizava, por isso mesmo, a verdade de catástrofe,
mantendo"-a como tragédia sem catástrofe, tragédia sem fim. Era, assim,
convertida em mais uma pandemia, embora grande e grave. Apenas mais uma,
dentre as tantas tragédias com as quais convivemos, e não a verdadeira
catástrofe, o fim do mundo ou o fim de um mundo, muito embora tenha sido
efetivamente catastrófica e o fim de tantos pequenos mundos para quantos
morreram, para quantos amores perdemos para a \textsc{covid}"-19. Na medida em que
a sua excepcionalidade era enfrentada nessa via de uma gravidade
relativa, ela era situada --- como as demais tragédias presentes ---, no
manejo interno à totalidade existente dos conflitos, sob a lógica da
redução de danos.

O realismo do possível, ao não instaurar uma verdadeira exceção, uma
verdadeira catástrofe e um verdadeiro desastre, produziu o que já produz
nas condições presentes, só que em escala acelerada e concentrada:
estados de exceção e de emergência de um lado, com a montanha de
cadáveres que lhes é conexa; e, de outro, uma montanha de discursos,
dizendo o mesmo que já sempre diziam, agora ``aplicado'' à \textsc{covid}"-19. Um
tenebroso cortejo de clichês, afinal, acompanhava o evento mais mortal
que nossa geração experimentou até aqui, não suportando o silêncio de
que também era feita aquela hora, inseparável da gravidade da morte,
inseparável da angústia. Assumi"-la radicalmente talvez pudesse produzir
algo além da repetição das formas dadas, algo à altura de um verdadeiro
desastre: catástrofe. O tempo, como em tudo o mais, foi aqui decisivo.
Uma emergência que imobilizava e mobilizava ao mesmo tempo não tinha
tempo. E o cortejo de discursos veio de todos os lados, em blocos. No
bloco animista, os New Age, para quem o vírus ``traz um recado sobre a
urgência da mudança, sobre a mudança planetária que o astral prepara
para a Terra'', com versões à direita e à esquerda, com direito ao
compartilhamento eventual de argumentos (a batalha da vacina, por
exemplo). Numa versão ``sul"-epistecosmológica'' cujo animismo é
\emph{mezzo} retórico, \emph{mezzo} ato falhado, \emph{mezzo} puro
teatro, o vírus é o ``pedagogo cruel''. No bloco das teorias da
conspiração, as versões vão da
``arma"-biológica"-comunista"-chinesa"-para"-trazer"-o"-caos"-ao"-ocidente'', às
``manipulações da \textsc{nom}'', também com as versões à esquerda e à direita e
frequentemente misturadas em muitos de seus elementos. No bloco da
ciência, dos mais impactados pelo vírus em sua específica posição
discursiva, a defesa do ``saber baseado em evidências'', da pesquisa e
dos sistemas públicos de saúde, articulada ao desamparo desse instante,
resultante da ausência de tais evidências em relação à \textsc{covid}"-19,
ausência também de uma estrutura universal de socorro.

Esse bloco, no qual se encontra quem suportou, com todos os
``trabalhadores essenciais'', o enfrentamento corporal aos piores
momentos da devastação trazida pela doença, se socorria, enquanto
socorria, do saber articulado no campo das crenças já instaladas,
tentando. Do lado do bloco crítico, da denúncia do existente, ``já
sabíamos que viria e avisamos''; Biopolítica para quem é de Biopolítica,
com direito à estranha crença --- pouco desconfiada para quem faz a
crítica --- nos relatos epidemiológicos iniciais e resistência aos
seguintes; Afetos para quem é de Afetos, com direito a carta de amor não
enviada; crítica ideológica da ideologia para quem é da crítica da
fantasia alheia; fim do capitalismo
que"-já"-ia"-acabar"-mesmo"-afinal"-dizemos"-isso"-há"-décadas, para quem já diz,
há décadas, que o capitalismo está acabando. Ainda nesse polo do
cortejo, a denúncia de todos os outros discursos, ``que não estavam
prestando atenção às novidades e alterações reais, nem fazendo o que
deveriam, nem sofrendo com os filhos em casa, nem fazendo a
solidariedade, nem angustiados o suficiente com as perdas'' (olá!). Em
cada um desses \emph{topoi} discursivos, uma constante: a forma do já
sabido e do já dito sendo convocada a fazer barreira à angústia, à
desorientação e à impotência que era a de todos nós. Em cada um desses
discursos, muita defesa, alguma verdade e a tentativa de manter uno o
que ameaçava se partir.

De um outro lugar, acatando de modo não defensivo a angústia e assumindo
o que nela era verdade, discursos que roçavam a coisa diante de uma
catástrofe assumida em ato, como presente, como desastre. Aqui e ali,
somente. Raros, porquanto quase ninguém (quase porque a fixidez existe e
às vezes a insistência defensiva não cede a nada) se sustentava
completamente em um único desses \emph{topoi} discursivos e, antes,
derivávamos para cada um deles. Nesse outro lugar, discursos dos laços
insurgentes em meio à miséria pandêmica, laços assentados na
singularidade de uma palavra frágil, nas ações diretas, na
auto"-organização comunitária e, aqui e acolá, numa greve selvagem ou
levante que obrigava a parar a produção de morte. Um ato com cruzes numa
praia ou praça imprimia gravidade e cantos de resistência nos hospitais,
leveza, ao cansaço e à dor extrema daquela hora. Também as escutas das
derivas alheias e das nossas próprias derivas escutadas por alguém. Já
mais adiante, no tempo, palavras que de sua fragilidade faziam força
real nas barricadas estadunidenses, nas ruas mundo afora, na Capitol
Hill autogerida. Nesse \emph{topos} também havia choro e havia silêncio.
Nesse outro lugar ainda, diários de psicodeflação que sustentavam uma
nudez de verdade, em pérolas de desconfiança e aflição em mais questões
que respostas. A articulação da pandemia à necropolítica, no que ela
carregava já de uma verdade tão corpórea, acenava naquela hora à
abolição dos nexos de culpa, instante de perigo assumido.

A experiência de catástrofe com a \textsc{covid}"-19 trouxe a essa palavra, com a
qual a vida e os discursos, nos dias que correm, não cessam de insistir
em familiarizar, a brutalidade do real. Interpelar a tentação da
familiaridade, colocando"-a em questão, permite retomar a sua ocorrência
sob a forma de uma ameaça, anúncio ou profecia --- modo pelo qual ela
ingressou tanto neste texto, quanto nos anúncios da catástrofe nos
discursos contemporâneos --- para nela apontar uma indicação temporal
que é elisão do presente catastrófico numa ``catástrofe que nos
ameaça'', quase sempre seguida de um ``como combatê"-la''\footnote{Esse é
  o título de um escrito de Lenin, de setembro de 1917, que afirmava a
  necessidade de um Estado socialista, como condição de superação da
  catástrofe --- a fome generalizada --- que a I Guerra houvera agudizado.
  Soviets controlando o Estado e o partido controlando os soviets eram
  ``os meios de combatê"-la''. Concentrando a ``catástrofe que ameaça''
  no fenomenológico, no identificável --- a fome, a miséria e a gestão da
  produção mercantil ---, mantinha intocado o lugar de sua produção --- as
  condições já capitalistas da Rússia, com as hierarquias que lhe são
  próprias, inclusive o Estado. Era o projeto do que efetivamente foi:
  modernização capitalista, Estado e barbárie. Enfrentando os efeitos de
  miséria pela evitação do enfrentamento radical do conflito ---
  tratava"-se, para ele, da substituição de uma totalidade por outra ---,
  buscava, em verdade, evitar o real da insurreição naquilo em que sua
  sustentação interpela os fechamentos definitivos numa forma. Assumia,
  assim, sob a presunção de um saber do caminho e da forma (já fechada
  no ideal que todo vanguardista carrega), a tutela da insurgência por
  sua representação unitária, o partido. A catástrofe a evitar era,
  assim, para Lenin, na real, a própria insurreição. O esmagamento da
  autonomia real dos soviets (assentada na heteronomia múltipla e
  conflitiva de toda insurgência) e o massacre de Kronstadt foram, como
  sabemos, a primeira matança daquele projeto estatista cujo
  desenvolvimento culminou no stalinismo. A catástrofe anunciada era o
  prenúncio da catástrofe. Não exatamente a que Lenin pensava anunciar,
  mas a que ele enunciava.}. Circunscrever tal elisão a uma conjuração
fantasmática, cuja função é fazer frente a um presente catastrófico
denegado, permite localizar o caráter destinal, nela conjurado em
oposição ao histórico, ao presente assumido em sua conflitividade
radical, incontornável, verdadeiramente catastrófica. Assumir e nomear a
crise do presente como catástrofe interrompem sua produção como
destinal, é ato que abre ao que insiste como vivo, não sujeito à
forma"-palavra dada, letra da palavra a vir, poesia. Nessa assunção do
presente catastrófico, o des"-astre, redução ao estado de caco, de letra,
abre à palavra poética, a um nomear que inaugura, à invenção,
insistência como vivo. Chegamos a ela na produção da catástrofe, por um
fim, negação do já dado. Atravessando a via do anúncio da catástrofe ao
assumi"-la no presente, é possível repetir, na pandemia, o gesto
catastrófico das barricadas: quebrar a totalidade dada e recolher sua
verdade em cacos, com os quais construímos a poesia de um mundo a
formar.

Para os que experimentamos a crítica do capitalismo como crítica em ato
da representação, pela via da ação direta, esse é, a cada instante, o
desafio: sustentar a crítica em ato, desconfiando dos discursos --- dos
nossos próprios antes de mais ---, daquilo que neles clama por unidade e
apazigua contradições. Só falamos uns com os outros de formas múltiplas
--- o contrário do uníssono (do capital, do Estado, dos partidos, das
Igrejas) ---, assumindo o cindido em nós. Insistir na desconfiança da
unidade que é uníssona,desconfiança da armadilha da representação,
unitarizante, sempre à espreita. Ela permite sustentar os laços que
fazemos desde as barricadas, sustentar um lugar discursivo no qual quem
fala não se identifica inteiro no/com seu discurso. Não só no sentido
óbvio de não pretender ``ser dono da verdade'', mas, mais radicalmente,
no de não se encontrar, realmente, inteiramente nele, assumindo a falha,
a distância entre o que queremos dizer e o que dizemos e entre o que
dizemos e o que não podemos dizer, porque sem palavras, só cacos. Isso
as barricadas ensinam. Insistir no que nelas aprendemos é sempre aposta.

Assumir o presente catastrófico desconfiando do próprio discurso e, por
isso mesmo, dizer. Foi o que pôs a escrever este texto. Escrever para
falar com o Zé Maria, querido, que, com sua bela história de barricadas
e vinhos e amores e vida, me recebeu tão maravilhosamente no seu
Portugal de onde vimos chegar o horror em forma de pandemia. Mas também
com tantos outros queridos com os quais falo por outros meios, e talvez
com uns tantos que nunca vi nem li, sobre pandemia e insistência
libertária. Difícil escrever agora. A palavra insistia nas formas já
dadas, nessa inércia que protege da dor de pari"-las dos cacos, num
conflito assumido. O incômodo com os discursos sobre a pandemia e seu
caráter clichê, apontado no (olá) que me indicava nessa posição
discursiva, a que me acompanha por mais tempo nessa pandemia e na vida,
assumindo o que nela é incontornável, como estruturante do meu discurso,
me remeteu ao imenso incômodo durante a eleição de 2018, onde a palavra
era o impossível. Muitos dos que não votamos nas eleições, porque as
denunciamos como farsa, numa recusa à representação, ficamos numa sinuca
de bico na qual, nem era possível subscrever a cretinice eleitoral dos
``bons contra os maus'' (os ``bons'' eram o que sabíamos, quem havia
capitaneado a nossa criminalização em 2013, as \textsc{upp}s, o Haiti, colaborado
com as chacinas em seus anos de gestão da barbárie), nem podíamos
desconhecer, sobretudo ouvindo quem do nosso lado da barricada convivia
de perto com a truculência da máfia miliciano"-policial, as leves
diferenças entre os dois lados em disputa, ali importantes diante da
nossa desarticulação, sobretudo para os mais expostos, dentre nós, à
barbárie presente. Na ocasião, assumindo a palavra impossível, conversei
com quantos pude. Não foram muitos. Tentei conversar também em alguns
pequenos textos nas minhas redes sociais, entre eles, um no qual nomeava
minha posição como um vote e vomite!, que não fazia, do que ali
reconhecia como necessário, nem virtude, nem ideologia. Não fazia regra,
nem ideal, invocava. O que me incomodou, à época das eleições, e
retornou, sob outro aspecto, nos clichês sobre a pandemia, não foi só a
sinuca de bico, que, afinal, marca a vida histórica sob as condições
adversas, nas decisões que exige de cada um. Foi, sobretudo, o
estreitamento do espaço da conversa sustentada numa palavra frágil, que
não se tomasse como regra, uma palavra acatada em sua fratura, cindida.
Naquela hora de 2018, essa conversa era impossível até com compas, com
os quais dividi barricadas. Só foi tornada possível, e de modo muito
restrito, quando assumida em seu impossível, assumida fora dos ideais e
dos clichês. Assumida porque acatava aquele momento de perigo como
história, como o contrário do destino. Essa palavra frágil pode, naquele
momento, delineando o presente em sua específica dor, sendo luto do
efetivamente perdido (da nossa desarticulação, que era o real da nossa
derrota desde as insurreições de 2013 e 2014 até as ocupações
secundaristas de 2015, derrota que se agravara com a adesão, tenebrosa e
massiva, de muitos dos que haviam estado ao nosso lado nas barricadas ao
dualismo dos bons e dos maus; real das mortes que choramos agora na
pandemia), ser poesia e nomear, sustentando a insistência no que é
vital. Permitiu inventar, ressuscitar e liberar palavras para dizê"-la.
Agora, é numa palavra frágil, que conversa, produzida no luto, na
catástrofe assumida, que insisto.

\pagebreak
\section{catástrofe, fantasia e imagem: o tempo sem tempo}

{\setlength{\epigraphwidth}{.5\textwidth}
\epigraph{Se lembra do futuro que a gente combinou?\\
Querendo acreditar que o dia vai raiar\\
só porque uma cantiga anunciou?}{\textsc{chico buarque}}}

\epigraph{O que não tem remédio, nem nunca terá}{\textsc{chico buarque}}

A catástrofe fez"-se familiar. Conteúdo do fantasma em quase todos os
discursos sobre o futuro do mundo presente, como uma espada de Dâmocles
ou do Anjo Gabriel, pairando sobre as cabeças. Ameaça de juízo, ameaça
de final com a qual, quase universalmente, desde há tempos, se fala do
presente. A ameaça só anuncia o catastrófico como imaginado, evento a
vir. Foi assim também na chegada da pandemia. Ela apareceu e foi
recebida num anúncio do que viria (ameaça, alerta, aviso ou profecia:
vai morrer muita gente de \textsc{covid}"-19, inclusive pode ser você ou alguém
que você ama). Esse anúncio, na desorientação, angústia e desamparo que
carregava, pela verdade do que anunciava --- um vírus mortal que se
espalhava e a morte, numa rapidez incontornável porque sem remédio ---,
atualizava em cada um de nós o terror do futuro já anunciado, a morte
que cada um sabe que chegará, um dia, para si e para os que amamos. O
anúncio atualiza o recebimento do aviso, dado na linguagem, que nos faz
humanos: morrerás. Catástrofe. Ele o faz pela negativa, já que o que ele
visa é prevenir e afastar a catástrofe que nisso, em negativo,
reconhece. Essa atualização de um anúncio já ocorrido constitui e
mobiliza, no presente em que, enunciando, anuncia um nexo entre o já
sido e o futuro, nexo tecido para cobrir o que esse aviso comporta de
dor.

A função"-mapa do anúncio da catástrofe é instruir. Dar caminhos.
Traçá"-los --- para dizê"-lo num vocabulário clássico --- como um
mapeamento estratégico, capaz de orientar taticamente. Sua verdade serve
pelo que efetua: mobilizar para o caminho apontado, adiar o fim ou se
assegurar da salvação, garantir um ``bom termo''. Mapeando e indicando
caminhos, o anúncio da catástrofe visa a controlar o futuro, garanti"-lo.
Alterando seu curso pelo correto delineamento das tendências do presente
ou, alternativamente, preparando para o inevitável do melhor modo:
assegurando"-se de um além a vir, capaz de redimir a dor. Sua suposição
de partida, porque instrutiva, deve estar assentada em um saber preciso
sobre o traço que indica, no presente, a catástrofe futura, como traço
tendencial. Daí sua familiaridade, signo do já conhecido. Ao
assumir"-negando o catastrófico como futuro, o anúncio é uma hipérbole do
que reconhece, no presente, como um traço e como tendência. Mas é também
sua negação, isto é, seu reconhecimento como o ainda não realizado
inteiramente do presente, realização para a qual ele caminha, no futuro,
se nada for feito ou inevitavelmente. Nisso há o reconhecimento de um
curso, de um destino. No anúncio de algo que virá, há uma relação
ambígua do enunciador com o tempo. Pela intenção, que familiariza e
instrui, aquele que enuncia e sabe o que virá, fala do futuro,
reconhecendo no presente o traço do que há de vir, para reparar, ou para
preparar para o irreparável. Ela deve prover àquele que prevê uma
reparação imaginada, ou não prepararia para o futuro. O foco da
intenção, assim, é a salvação no futuro anunciado. Mas pelo que o
anúncio é, seu tempo é duplamente o presente: na enunciação, que se faz
no presente e também no reconhecimento, no presente, do traço da
catástrofe, que lhe permite apontar o desfecho catastrófico futuro como
hipérbole do que, presentemente, é reconhecido apenas como traço. O
anúncio, assim, tendo o futuro como intenção, ocorre no presente em que
afirma, numa elisão em que, ao mesmo tempo, ele reconhece e nega a
catástrofe enquanto presente. Afirma a catástrofe futura negando seu
caráter já presente como acabado, isto é, já catastrófico. A via do
anúncio é uma \emph{via} \emph{negationis} (que diz de uma verdade pela
sua ausência).

Enquanto articula os tempos (o presente e o futuro), o anúncio também
põe em cena o passado, fazendo um nexo temporal --- história como um
saber --- desses tempos. No anúncio da \textsc{covid}"-19, vimos esse nexo
aparecer atualizando o anúncio do passado recente, esse que não cessa de
se dizer no discurso contemporâneo, de que haveria uma catástrofe
ambiental, uma pandemia, ``O'' evento, o fim dos tempos, o fim do
capitalismo junto com o fim do gênero humano, etc. A desorientação,
angústia e desamparo daquele presente, que era de todos, foi articulada,
de modo geral, em discursos cujas balizas estavam já dadas, como disse
no início, estes, precisamente, que já conjuravam o fantasma da
catástrofe. A vantagem específica dessa articulação era estar já à mão,
permitindo, a quem a escutava"-enunciava, situar a ameaça, localizá"-la e
manejá"-la, tornando possível o seu enfrentamento. ``Adoecerão muitos,
mas nem todos'', ``morrerão muitos, mas só os suscetíveis'', ``os
sistemas ficarão sobrecarregados'', ``é preciso prevenir o mal maior''.
A busca de advertir, diante das mortes que se avolumavam, da tendência
ao pior, carregava nexos, sentidos, que permitiam, ao mesmo tempo,
cobrir e enfrentar, com o que se tinha à mão, o tenebroso no presente.
Cobriam porque remetiam a verdadeira catástrofe para o futuro. Permitiam
enfrentar, porque asseguravam de estarmos vivos. Assumiam, assim, do
presente, esse traço do catastrófico, remetendo a verdadeira catástrofe
ao futuro, recusando a \emph{via} \emph{eminentiae,} (aquela que diz
sobre uma verdade presente) num mero traço horrível do presente. A
\emph{via negationis} do anúncio \emph{``}salvava'', assim, no
enunciado, o presente da sua catástrofe, reparava imaginariamente o
presente.

O destino que aparecia como sombra --- a morte anunciada --- no anúncio
da pandemia, o passado que habitava o presente como horror foi, assim,
enfrentado como anúncio, de dois modos: o primeiro reconhecia a
catástrofe a vir e arregimentava forças para impedi"-la, minorá"-la ou
remediá"-la. Nisso assumia uma posição que parecia histórica, diante do
caráter destinal do anúncio: alterar o curso das coisas para salvar da
catástrofe a vir. Habitada pelo destinal, que em negativo afirmava, essa
negativa ao destino era, contudo, pseudo"-histórica. Negando realmente o
catastrófico já presente, encaminhava ao futuro imaginado a dor já aí da
catástrofe, já presente em seu real. O que ela negava assim não era o
destino, mas a própria catástrofe como o já presente. Nessa
pseudonegação do destino, vinda no nexo que afirmava o sentido destinal
da catástrofe a vir, a verdade era o destino nessa negação, o que ela
continha de verdade do falso: cobrir, com os sentidos já dados, o que
não tem sentido nem nunca terá. O real do presente tornado catastrófico.

A intenção de negar o destino pela negação da catástrofe futura,
acreditando nisso fazer o máximo de esforços para evitá"-la, para
interromper o curso da pandemia e ``o futuro catastrófico que ele
traria'', realmente o afirmava, assim, pela negativa da presença já aí
da catástrofe. A intenção é a obediência às formas já dadas, ao sentido
já dado do mundo, aquele que estrutura as respostas no campo dos
possíveis, assentado na matança sem lei, lei do presente. A questão,
para o bem"-intencionado, não era se haveria ou não mortes aos milhões
--- ``o mundo não pode simplesmente parar, morrerão milhões'' ---, mas a
de se, diminuindo o ritmo da produção"-circulação no mundo ao mínimo,
pelo menor tempo, sob um controle eficaz e emergencial, segundo as
formas dadas, mercado e Estado, distribuiriam os milhões, ordenadamente,
numa curva achatada, não desordenada. Contar e administrar a morte. O
outro modo de resposta à pandemia aparecia, a esses de boa intenção,
como resposta suicida e assassina, sem sentido e sem noção. Ela era,
entretanto, apenas a obediência crua e descoberta à forma que governa o
mundo, impessoal do valor e do Estado ali assumido sem questão. ``O
mundo é o que é'', ``Morre"-se''. Assumia a impessoalidade e, nisso,
trazia a verdade desnudada, terrível, da morte banalizada no seu
``acontece''. Realismo sem adorno, sem maquiagem. Tanto quanto os que
buscavam meios de manejar a pandemia, esses que recusavam o seu anúncio,
os ``negacionistas'', faziam um anúncio, embora outro. Nos nexos e no
sentido do já dado, anunciavam o caráter destinal pelo qual o presente
era, de outro modo, assumido e negado. Não disfarçavam numa face
pseudo"-histórica, de ``manejo'', o destino. Antes o afirmavam num
``todos morreremos'' universal. Esse anúncio era também
pseudo"-histórico, mas por outra via. Assumia uma verdade da morte ---
essa que funda a história, no que finitiza e, incontornável, dói ---,
sustentando o realismo resignado ao seu saber: ``nada há a fazer,
morreremos todos''. Nessa resignação, o que havia, ainda uma vez, não
era história, que sabe em ato, sem saber. Era destino, tivesse ele a
face de Deus, da imunidade de rebanho ou da descoberta do remédio
milagroso, que ``virá se Deus quiser''. O que era acatada, aqui
diretamente, era a forma do já dado como forma da morte, mas não a morte
em seu real, seu sem sentido, que interpela o sentido do já dado e por
isso é histórica. Era a morte como forma, como destino.

O nexo e o destino, no anúncio, são o saber que permite, a quem o
enuncia"-escuta, tornar navegável e garantida uma viagem de Ulisses, uma
emergência (do verbo emergir, aparição) do não sabido. Neles o não
sabido é, ao mesmo tempo, reconhecido e desconhecido. É reconhecido num
saber anexado, articulado pelo nexo, ao já presente. Saber articulado na
consciência, cuja verdade inconsciente é o nexo em verdade já feito,
nexo dado pela forma que o estrutura. Esse reconhecimento é o que lhe
permite anunciar e, do anúncio, fazer mapas, desenhar caminhos. O
problema, na emergência (do verbo emergir, aparição) do não sabido é
que, sendo realmente do não sabido, o anúncio que traz o seu
reconhecimento implica sempre certo desconhecimento do não saber, sua
relativização num já sabido e num ainda não sabido, reconhecimento que
indica o não saber e o afasta, ao mesmo tempo. Na pandemia, o que era
reconhecido como emergente era o próprio vírus, descontroladamente
espalhando a morte. Por isso, os anúncios da emergência do novo
coronavírus e da \textsc{covid}"-19 eram de uma emergência tanto como aquilo cuja
resolução reclama ação urgente, quanto de uma emergência no sentido do
verbo emergir, aparição. Nos dois sentidos, ela partia do reconhecimento
desse novo coronavírus e da \textsc{covid}"-19, isto é, de sua identificação à
família corona, reconhecimento pelo qual o não sabido se torna relativo,
pelo qual podemos reconhecê"-lo, isto é, pelo qual ele não é mais o não
sabido, mas o ainda não inteiramente sabido.\footnote{Bifo Berardi, numa
  das primeiras edições do seu diário da psicodeflação, em 2 de março,
  proferiu uma preciosa interpelação ao tamponamento do não saber no
  modo como, naquele momento, o que era reconhecido como não sabido
  sobre a \textsc{covid}"-19 e o \textsc{sars}"-CoV"-2 era negado a partir de uma
  pressuposição. Numa frase que interpelava a hipótese aventada de que,
  com a chegada do calor, a pandemia arrefeceria, Bifo perguntava,
  finamente, sobre como se podia saber das temperaturas prediletas de um
  vírus sobre o qual nada ainda se sabia. A interpelação ia ao cerne do
  modo como a suposição de que os saberes já existentes sabiam algo ali,
  afastava a perturbação, trazida pelo vírus como não sabido, ao saber.
  Ela autorizava um discurso apaziguador, que trazia a pretensão de
  aplicar ao que não se sabia o que já se sabia sobre a família do
  vírus, os corona, ao contrário do que os esforços reais para conhecer
  o não sabido do vírus, feito por pesquisadores no mundo inteiro
  faziam: tratá"-lo como o (ainda) não sabido, que verdadeiramente era. O
  momento dogmático de toda ciência, que, para sê"-lo, parte de um não
  saber relativo, segundo o já sabido, cobria, na pressuposição sobre a
  temperatura, a novidade radical daquele vírus, sobre a qual se
  debruçavam, realmente, os cientistas. O retificável de todo saber da
  ciência, seu momento crítico, assim, era ao mesmo tempo mantido e
  afastado, na pesquisa, na pressuposição do típico, dada no já sabido.
  Que esse procedimento seja mesmo a dialética própria da ciência, como
  modelar dos saberes e da produção moderna, não é uma mera face entre
  outras das formas dadas e do sentido do presente. É, aliás, seu cerne.}

O anúncio da catástrofe apresenta uma imagem de futuro pelo
reconhecimento do catastrófico como um traço do presente. Como ele, ao
mesmo tempo, reconhece e nega a catástrofe, isso significa, do ponto de
vista da imagem, um problema. Ela é uma imagem que, ao mesmo tempo,
reconhece o traço como traço, no presente, isto é, como ainda não
formado, e como a imagem já formada, no futuro. Na imagem de totalidade
do presente, o traço é apenas um informe, um não formado. Mas só pode
ser traço de uma forma imaginada no futuro se for visto e reconhecido
numa forma a vir, já formada. Exatamente o que todos reconheciam: a
forma da morte já presente, não seu real sem forma, verdadeiramente
generalizado, que era remetido a um futuro a impedir de vir a tornar"-se
real. Se pensarmos numa imagem já muitas vezes visitada, o quadro
\emph{Os Embaixadores}, de Holbein, talvez possamos ir mais rápido ao
ponto. Esse quadro ``imaja'' essa dupla aparição, contraditória, o duplo
aparecimento de um traço não formado e já formado, dependendo de que
lugar se olhe. Como ele é uma imagem, o que é realmente contraditório
aparece não exatamente numa contradição, mas numa sucessão. Por um
lado\ldots{} por outro lado\ldots{} Essa ``mágica'' de ``imajar'' o contraditório
é possível apenas como abolição do contraditório na sucessão, pelo
deslocamento no espaço, que exige também sucessão temporal. Um depois do
outro, deslocamento pelo qual a simultaneidade e o agora, impossíveis da
contradição, são contornados, para se apresentar na forma, num
deslocamento espacial e numa sucessão temporal. Essa ``mágica'' é a
inclusão de um ponto anamórfico no quadro.

Quando o observador olha para esse quadro, ele vê, desde um ponto
central de observação, dois embaixadores num ambiente com piso, paredes
(cortinas), uma mesa, o chão e vários objetos. Entre eles, nesse centro,
ponto médio de observação, uma imagem informe, estranha à totalidade em
sua ausência de nitidez. Se o observador olha o quadro a partir desse
ponto anamórfico, eis que lhe aparece, nessa mudança de foco, uma
caveira onde não havia forma, mas o que é desfocado, agora, é o resto da
pintura. O discurso que anuncia a catástrofe faz com o presente uma
operação como essa. Descobrindo a caveira no quadro onde, vista no
centro, ela era o informe que perturba a totalidade ``imajada'', aponta
em seu anúncio a catástrofe a vir: a totalidade distorcida que ele vê ao
se aproximar da caveira. A forma pela qual a totalidade desfocada é
vista é assim anunciada como o que será o futuro, quando a morte, vista,
chegar. Cada um que olha o quadro o apresenta desde sua posição exata no
ponto de anamorfose: as alterações perspectivas dadas pela posição, pelo
ponto de vista preciso de quem observa. As perspectivas --- o lugar do
qual a caveira do quadro é vista --- são articuladas, assim, de modo
diferente, relativo à posição precisa de cada um (com seu ângulo de
visão particular) no ponto de anamorfose. Os discursos aos quais me
referi, no início, como fazendo parte de um cortejo tenebroso descrevem
tantos desses possíveis pontos de observação do quadro a partir da visão
da caveira. Sendo o olhar humano situado num corpo, e sendo finitas, do
ponto de vista da materialidade do espaço, as variações em torno do
ponto anamórfico, do ponto de vista da relação entre a visão e o quadro,
considerando a objetividade material dos corpos (o olho, o quadro e o
espaço no qual se encontram), assim como são finitos os lugares ocupados
no espaço realmente existente no mundo, designado por certas
materialidades típicas (gênero, raça, etnia, profissão ou ausência dela,
habitação ou ausência dela, etc.), a caveira aparece com formas
diferentes conforme as diferenças de perspectiva a situem a partir do
ponto anamórfico, definidas pelos possíveis materiais do espaço,
delimitadas na unidade da forma, como é próprio de toda imagem, isto é,
pela corporeidade da posição dos que o examinam, do quadro mesmo e da
sala em que ele está, e pelas posições articuladas da totalidade que
realmente existe do mundo, no caso da imagem pandemia. A imagem aparece,
em qualquer caso, como unidade. Uma forma em voga para nomear essa
localização dos pontos de observação e suas diferenças é útil pelo que
de mal"-entendido carrega: podemos nomeá"-la como o lugar de observação do
qual fala quem enuncia a catástrofe, como um ``lugar de fala'' tomado
como uma singularidade já encerrada numa totalidade. Porque, atentemos,
essa singularidade é, aqui, submetida à unidade da imagem e, como tal,
não é exatamente singular, mas já tomada em sua relação com o universal.
Apresentando a caveira como traço mortal do presente, cada anúncio,
assim posicionado em alguns desses pontos materialmente possíveis,
anuncia desse ponto, uma visão cuja verdade se encontra dada em sua
imersão na totalidade, o ponto desde o qual ela é visível a quem
anuncia, e do qual ela mesma vê. O ponto anamórfico é o lugar do qual
quem anuncia vê e se vê porque faz totalidade dos dois tempos sucessivos
da visão do quadro, lugares que, articulados pela sucessão e pelo
deslocamento, unificam no anúncio da catástrofe que virá. A imagem da
morte é, aqui, como unidade, imagem do que ameaça a unidade do quadro e
do que o mantém, pois, ao mesmo tempo que desfigura a totalidade, essa
desfiguração, ainda não, de novo unificada no anúncio, a sustém. Por
isso ela só sustenta a morte desfocando (não desmontando) a totalidade,
ou, ao contrário, só sustenta a totalidade, desfocando a morte. Esses
dois lugares de observação correspondem, pode"-se ver, aos dois modos
pelos quais o anúncio é feito, dois modos de anúncio e representação da
catástrofe. Ambos se assentam na morte ``imajada'', na morte
afirmada"-negada como presente, na imagem. Outra é a assunção da morte
como o sem imagem, a morte da imagem da morte.

\section{a impossível escrita do finado}


{\setlength{\epigraphwidth}{.55\textwidth}
\epigraph{o melhor prólogo é o que diz menos coisas.\\
Ou o que as diz de um jeito obscuro e truncado.}{\textsc{machado de assis}}}

Referindo"-me aos discursos sobre a pandemia, falei dos que, ``roçando a
coisa como podiam, aqui e ali, falavam de outro lugar''. Outro lugar
situado como posição não defensiva diante da angústia, isto é, do horror
da morte como presente e como inominável. Ele era ali articulado à
gravidade do horror, assumida como a condição do presente, diante do
qual sua negação real se põe. Apenas como negação da negação da morte
que configura e conforma o presente, ela é negação real e nisso, vida,
insistência. Porque assume a morte e chora sua dor, ela é o contrário da
sua glorificação. É a posição dos corpos reais, finitos, insones,
exaustos, assombrados, revoltados, insurgentes que, durante a pandemia,
assumindo o real da morte como o incontornável catastrófico do presente,
assumiram uma palavra outra, uma palavra feita dos seus cacos. O
impossível do dizer a morte é que só há caveira real, fora do quadro,
fora da imagem. Coisa. Sendo caveira, é o impossível de ser dito que,
por isso mesmo, permite dizer. É fragmento, é parte, é pedaço que, no
vivo, insiste. Em nós, nossos mortos. Sandra. Leo, Leca, George Floyd,
Giovanni, Guy, Ernesto, Nestor, Buenaventura, Rosa, Louise,
Valdo\ldots{} Letra"-fragmento"-parte que, em cada um, parte. No que parte
e é parte, insiste, vive, faz dizer. Morremos todos, em meio a tanta
morte. E aqui estamos, vivos. Dos mortos e pelos mortos, guardando na
verdade"-letra, uma palavra"-interpelação: catástrofe! A vertigem da morte
assumida como presente há de ser a gravidade da morte do presente.
Sejamos, os vivos, a sua catástrofe. Ponhamos fim. Não é isso, afinal, o
que significa insurreição?

\begin{bibliohedra}
\tit{BIFO}, B. Crônica da psicodeflação. Tradução de João Pedro Garcez.
Disponível em: \emph{https://bit.ly/3sO4Jas}. Acesso em: 09/10/2020.

\tit{COLETIVO ANARQUISTA DE TORINO}. Apontamentos sobre a epidemia em curso.
Disponível:
\emph{https://faccaoficticia.noblogs.org/post/2020/03/24/apontamentos/}.
Acesso em 09/10/2020.

\tit{COLETIVO CHUANG}. Contágio Social --- coronavírus, China, capitalismo
tardio e o ``mundo natural''. Disponível em:
\emph{http://afita.com.br/outras-fitas-contagio-social-coronavirus-china-capitalismo-tardio-e-o-mundo-natural/}.
Acesso: em 08/10/2020.

\tit{LENIN}, V.I. A Catástrofe que nos ameaça e como combatê"-la. Disponível
em: \emph{https://www.marxists.org/portugues/lenin/1917/09/27-2.htm}.
Acesso em 06/10/2020

\tit{SADE}, D. A. \emph{A filosofia na alcova, ou, Os preceptores imorais}.
Tradução, posfácio e notas de Contador Borges. São Paulo: Iluminuras,
1999.

\tit{WALLACE}, R. \emph{Big Farms Make Big Flu: Dispatches on Influenza,
Agribusiness, and the Nature of Science}. New York: Monthly Review
Press, 2016.
\end{bibliohedra}

\chapterspecial{Solidariedade, apoio mútuo e vida anarquista}{}{João da Mata}
\hedramarkboth{Solidariedade, apoio mútuo\ldots}{}

\epigraph{Caridade é sarcasmo e burla. Diante dessa palavra, quanta
hipocrisia!}{\textsc{sílvio de figueiredo}}

\noindent{}A presença de diferentes pandemias entre humanos não é exatamente uma
novidade. Muitas delas já dizimaram quilíades de pessoas em todos os
cantos e épocas. Se há algo de surpreendente na emergência do vírus
\textsc{sars}"-CoV"-2, causador da \textsc{covid}"-19, é sua rápida e larga disseminação em
escala planetária, atingindo os mais remotos lugares. A circulação do
vírus atende à velocidade do mundo globalizado e conectado, coerente com
um tempo presente no qual as novas tecnologias de informação e
comunicação transformaram as relações entre o real e o virtual. A atual
pandemia também explicitou o que se tornou a vida contemporânea: cidades
abarrotadas de gente, níveis extremos de pobreza e um modelo de vida
predatório jamais visto. A coexistência dos humanos com outras espécies
tornou evidente sua capacidade destruidora, procedente de um outro e
mais antigo contágio: a exploração capitalista.

A célere disseminação do vírus também fez surgir diferentes protocolos,
ações e procedimentos, que incluíram a recomendação para o uso ou não da
máscara, qual o momento em que se deveria procurar atendimento médico, o
uso dessa ou daquela droga e muitas outras incertezas diante do
desconhecimento da doença. Continuamos imersos em meio à pandemia, e não
há clareza de quanto tempo ela irá durar e o que poderá vir adiante, mas
já é possível promover algumas análises críticas desde sua emergência.
Se há dúvidas sobre o tempo presente, um cenário pós"-pandemia é ainda
mais indefinido.

A partir da dispersão do vírus pelo mundo e sua chegada ao Brasil, vimos
acontecer uma série de práticas na sociedade que reivindica a
\emph{solidariedade} diante do colapso econômico e social provocado pela
\textsc{covid}"-19. Não que essas práticas sejam novidades, mas reacendem o
interesse na ajuda ao próximo, especialmente aqueles que entram nas
categorias de mais necessitados, hoje em dia também chamados de
vulneráveis, atingidos agora pelo novo coronavírus. Mobilizaram"-se redes
de apoio, que se estenderam entre governo, sociedade civil, empresas,
organizações do terceiro setor. Enfim, somos atingidos por um amplo
leque de convocação ao voluntariado, tornado necessário, senão urgente,
diante dos fatos. Vemos ainda como médias e grandes corporações
capitalistas têm sido enaltecidas pelo seu papel de responsabilidade
social diante da crise. São reconhecidas em função do apoio financeiro
que disponibilizaram, seja na produção e distribuição de Equipamento de
Proteção Individual (\textsc{epi}), seja na doação de alimentos, na capacitação
profissional e mesmo no patrocínio de \emph{lives} para artistas
\emph{entreterem} as pessoas e, assim, aliviarem os efeitos do
isolamento social.

No bojo dessas ações, vem surgindo a esperança de um novo amanhã, quando
a atual organização social, modificada pelo conjunto dos acontecimentos,
poderá, então, reconfigurar"-se em um cenário de pós"-pandemia. Dessa nova
ordem social, espera"-se, portanto, que seja mais justa e equânime,
reaquecendo algumas propostas já conhecidas, como renda universal,
desenvolvimento sustentável, desaceleração do ritmo da vida,
decrescimento, etc. A confiança no amanhã redentor costuma ser reativada
em momentos como o atual, levando a uma projeção utópica que, distante
de nós, haverá dias melhores.

Na busca por manter o funcionamento da economia, Estados em todos os
cantos do planeta, com suas alianças inequívocas com o capital, buscam
adequar"-se ao \emph{novo normal}, termo que tem sido amplamente
anunciado em função dos impactos da pandemia da \textsc{covid}"-19. A partir dele,
são elaboradas e sugeridas maneiras para manter a circulação de bens e
serviços, esgarçando a virtualização da economia e das relações e
fazendo ainda mais atuais as narrativas de um mundo viável e
ecologicamente limpo. Nessa naturalização das coisas, faz"-se crer na
continuidade do vivo a partir da manutenção das mesmas práticas em
vigor, agora renovadas em embalagens mais palatáveis e no apelo à
economia verde.

Numa ponta, a pandemia fez acelerar a fase mais cruel do capitalismo na
era digital: miscelâneas das relações de trabalho, aumento da miséria,
virtualização de toda ordem. Passados mais de dez meses de pandemia,
novos efeitos têm aparecido com força: desemprego atingindo níveis
recordes, sofrimento mental em parte significativa das pessoas, evasão e
abandono escolar; e, na outra ponta, lucros exorbitantes em setores e
segmentos que faturaram na \emph{crise}. Mais uma vez, a história mostra
como o funcionamento capitalista evidencia a produção de um modo de vida
arrogante e destruidor, pautado na miserabilidade de muitos e na riqueza
de poucos.

O modelo predatório que o capitalismo neoliberal impôs está condenado ao
fracasso, apesar de suas renovações tópicas, pois sua dinâmica de
funcionamento no contínuo aumento de acumulação de capital encontra nos
limites do planeta a impossibilidade de seguir sempre mais. A rápida
disseminação do novo coronavírus matou e continua matando muita gente.
Possivelmente seguiremos com sua presença entre nós como mais um vírus
endêmico, estratificando exponencialmente a sociedade entre miseráveis e
não miseráveis. Eles, os pobres, serão sempre o destino preferencial das
desgraças de um vírus que, num primeiro momento, se fez crer na
narrativa de ser democrático e que a todos atingia igualmente. A
ontologia do capital separa e segrega, para fazer viver e deixar morrer
vidas em recortes sociais e políticos evidentes.

Em meio a tudo que passamos, o capitalismo neoliberal procura se vender
em uma roupagem \emph{humanizada,} neologismo utilizado para incitar e
ampliar a participação no largo arco das democracias representativas,
sem produzir um real contraponto à sua lógica de funcionamento. Na mesma
direção, o amálgama de práticas sob designações solidárias reforça as
relações de dominação para se manter hierarquias e esconder a caridade
por trás da cortina de fumaça das morais redentoras.

Diante de toda apologia ao novo amanhã e de uma solidariedade
sacralizada, os anarquismos explicitam outras formas de viver. Aquelas
que surgem da mudança na relação ética de si para consigo mesmo e de si
com os demais, por meio de práticas de liberdade que garantam a
cumplicidade como forma de cuidado. Em momentos limítrofes como esse que
a pandemia nos coloca, abrem"-se possibilidades para essa inflexão ética,
que possa fazer deslocar a relação com o vivo sob uma perspectiva
antiautoritária. A crítica anarquista será sempre aquela que insiste em
mostrar as contradições entre a vida livre e a lógica de domínio dos
Estados e do capital. Afirmar uma vida \emph{outra}, para nós, é lutar
incessantemente para eliminar as práticas autoritárias, inclusive
aquelas que persistem em nós mesmos.

\section{de qual cuidado estamos falando?}

A palavra solidariedade ocupou as manchetes de noticiários já nos
primeiros momentos da pandemia, revelando práticas de pessoas, grupos e
empresas empenhados em minimizar as suas consequências. Fortemente
empenhadas em atenuar o sofrimento de quem foi atingido direta ou
indiretamente pelo novo coronavírus, muitas dessas ações ancoram"-se na
certeza de fazer o \emph{bem} ao próximo. A noção de solidariedade tal
qual é disseminada e muitas vezes exercida em diferentes ações na
sociedade capitalista está fortemente impregnada de valores morais.
Frequentemente, acabam por atualizar os resíduos da tradição
judaico"-cristã. Desde a noção do \emph{amor ao próximo}, passando pela
solidariedade humanista, somos levados a pensar e agir na relação com o
outro a partir de uma situação desigual, na assimetria entre quem ajuda
e é ajudado. O mandamento é claro: devemos amar o próximo como a nós
mesmos, ainda e a despeito das diferenças. Na moral cristã, esse é o
ensinamento de primeira ordem e o princípio perpetuado pelos monoteísmos
de forma geral. No espírito da bondade, a ascese cotidiana de uma
existência é consagrada a partir da imitação do sofrimento do Cristo
crucificado como exemplo de abnegação ao outro e que inclui o exame de
consciência.

Ponto de partida para as sociabilidades em geral, o núcleo familiar
tradicional conserva os vínculos de afeto e cuidado entre seus membros
quase sempre embebidos de jogos chantageadores, que evidenciam como o
amor continua sendo utilizado enquanto instrumento de controle e posse
das pessoas. É ali também que opera o ponto de partida da noção empatia.
Palavra que vigora entre os conceitos que se tornam modismos, o termo
empatia tem sua raiz etimológica no grego \emph{empatheia}, formado pela
junção \emph{en}, ``em'', mais \emph{pathos}, ``emoção, sentimento''.
Terminologia que foi absorvida pela Psicologia no início do século \textsc{xx},
passou a ser largamente usada nos últimos anos, sempre que se busca
``colocar"-se no lugar do outro''. Na democracia representativa, que
busca incluir tudo e todos, é sugerido que nós sejamos empáticos com os
demais e nos disponibilizemos ao outro sempre que possível. Mas será que
é realmente assim que as coisas acontecem? O uso das palavras
solidariedade e empatia tornou"-se, em muitos casos, banalizado e
esvaziado de sentido ético e político.

Funcionando como unidades basais das sociedades capitalistas e
socialistas autoritárias, as famílias ainda operam como Estados em
miniatura, a partir da manutenção da centralidade da autoridade, da
obediência e da hierarquia. Sua dinâmica ocorre tomando por base a
maneira de amar e sociabilizar quase sempre baseada no sacrifício. Esse
modelo de cuidado e afeto com os demais acaba por estender"-se para
outros âmbitos. O sacrifício ao outro e mesmo o altruísmo tornam"-se algo
que moralmente adquire um valor de respeito e admiração perante a
sociedade. O assistencialismo, a caridade ou que nome se dê, é exaltado
como valor de \emph{gente do bem}, outro neologismo comum nos dias de
hoje. No caso de ricos e abastados, serve ainda para aliviar a culpa
pelos lucros nos negócios. A solidariedade no capitalismo está atolada
na esperança de um mundo melhor e na promessa de um novo amanhã, seja
pela reforma ou pelo aperfeiçoamento do modelo, mas nunca por sua
ruptura. A palavra solidariedade acabou por ser sequestrada pelo
amálgama das moralidades redentoras, para tornar"-se uma prática
apaziguadora.

A despeito das novas configurações amorosas, o desejo ainda é vivido
como falta e carência. Busca"-se, no outro, o complemento que está
ausente em si, para então chegar à unidade. A incompletude não surge
apenas no amor romântico e narrativas, tais como cara"-metade e alma
gêmea, mas também nos vínculos entre pais e filhos, entre amigos e em
diferentes acoplagens que funcionam no diapasão dominador"-dominado.
Afirmar o desejo como excesso e transbordamento implica um
redimensionamento das relações que podemos chamar de amorosas, nas quais
o encontro com o outro não esteja pautado na complementaridade. Esse
deslocamento no campo do amor e do afeto está implicado também com as
sociabilidades de forma geral. Criar relações baseadas no excesso passa
fundamentalmente por suscitar práticas de autonomia, nas quais as
pessoas envolvidas sejam capazes de autogovernarem suas vidas. Nada nem
ninguém poderá saber o que o outro precisa mais que ele mesmo. Caso
contrário, o que resta são sobras que apenas produzem prazeres vicários,
mantendo a condição subalterna de quem é ajudado.

Uma outra noção de solidariedade libertária, mais bem"-evidenciada no
conceito de apoio mútuo, formulado pelo geógrafo e anarquista russo
Piotr Kropotkin (1842--1921), voltou a ser reivindicada por diferentes
grupos anarquistas. Suas análises incidem na noção de que seres vivos,
incluídos os humanos, tendem a se apoiar mutualmente para promover a
cooperação entre todos. Kropotkin sugere que a solidariedade é
uma lei ou fator geral da natureza e a responsável em promover
possibilidades de respostas mais eficientes diante das intempéries da
vida como fator de evolução. Será a partir do apoio mútuo entre os
indivíduos, e não da competição entre eles, que a sociedade se tornará
mais justa e igualitária.

As práticas de apoio mútuo, solidariedade e mesmo de empatia vivificadas
pelos anarquistas estão ancoradas na certeza de que qualquer associação
só poderá acontecer por reciprocidade, jamais por alguém que ajude
\emph{outrem} em condição desigual. Dessa forma, a cooperação entre
libertários busca ir além das ações de apoio a quem necessita em um dado
momento. Trata"-se mesmo de uma ruptura com a lógica hegemônica e
fatalista, que faz crer que os indivíduos se enfrentam como se
estivessem em um ringue da vida, na luta de aniquilação dos mais fracos
pelos mais fortes. Redimensionar essas práticas, a fim de estabelecer
outros arranjos associativos, será, pois, a luta daqueles que querem a
liberdade em sua dimensão coletiva.

A atualidade do apoio mútuo em Kropotkin está em mostrar que o
federalismo cooperativo se torna um contraponto a esse modelo predatório
de destruição da natureza, dos próprios humanos entre si e a si mesmos
em última estância. Segundo ele, ``quanto mais o princípio de
solidariedade igualitária se encontra desenvolvido numa sociedade animal
e mais próximo do estado de hábito se encontra, mais possibilidade essa
última tem de sobreviver e de sair triunfante da luta contra as
intempéries e contra os seus inimigos. Quanto mais sinta de cada membro
da sociedade a sua solidariedade com qualquer outro membro dela, melhor
se desenvolvem, em todos, essas duas qualidades que constituem os
principais fatores da vitória e de todo o progresso --- a coragem, por
um lado, e a livre iniciativa do indivíduo, por outro'' (\textsc{kropotkin},
2009. p. 79--80).

Em momento histórico próximo à elaboração das primeiras edições do livro
\emph{Apoio mútuo}, o poeta francês Arthur Rimbaud (1854--1891) trazia
sua provocativa constatação: \emph{sabemos que o amor está por ser
reinventado.} Rimbaud não era um anarquista, apesar de sua obra
instaurar uma ruptura estética e revolucionária fundamentais à poesia
francesa de seu tempo. Sua atitude rebelde também fez sacudir os
círculos intelectuais por onde passou e abriu espaço para pensarmos o
amor como campo de batalha por uma associação que faça detonar os pactos
de mediocridade.

A afirmativa de Rimbaud, apesar de aguda e atual, não fez eliminar as
relações de poder nos vínculos amorosos. Ainda assistimos às diferentes
práticas afetivas carregadas de gestos generosos misturados com
cobranças e culpabilidades. Suas extensões no seio das sociabilidades
autoritárias, sob designações solidárias, acabam por renovar o
sacrifício de cada um em função do ideal coletivo. Revogam novas
religiosidades que se constroem sobre a adoração de generalidades, como
o povo, a nação, a pátria, os direitos. Encarar uma vida \emph{outra},
distante desses idealismos amalgamados na ideia de amor ao próximo,
significa desmontar os placebos ontológicos do medo, do receio, da
incompletude, da impotência diante da vida. Ela situa"-se no aqui e no
agora, e não no futuro redentor; em nossos corpos, e não no plano das
ideias. Enfim, no que somos e no que fazemos de nós mesmos a despeito do
que está posto.

Diante de nós, assistimos a uma frenética corrida por vacinas ou
medicações eficazes para a cura da doença, que fez mais uma vez acirrar
as disputas econômicas ao redor do planeta. Se é dado como certo o
surgimento de uma ou muitas vacinas em breve, a volta à
\emph{normalidade} não garante que as condições de vida sejam realmente
satisfatórias. Muito pelo contrário. O modelo de organização social que
vivemos, baseado na exploração, na hierarquia e no autoritarismo,
possivelmente fará surgir novas pandemias, como também o aumento da
miséria, dos desastres ambientais, entre tantos outros efeitos
decorrentes de sua lógica genocida.

No momento em que o esgotamento emocional e psíquico das pessoas se soma
ao aumento exponencial da miséria e da pobreza, recorrer às esperanças
sacralizadas da bondade alheia acaba por tornar"-se um alento falacioso.
Se há miséria, desigualdades e explorações de toda ordem, o modelo
capitalista e suas alianças com Estados liberais e socialistas
autoritários são os responsáveis por essa produção. Qualquer prática que
vise aperfeiçoar ou minimizar seus efeitos se torna ação inócua e levará
cada vez mais a vida ao atoleiro do medo e do desalento. Assim,
naturalizar o \emph{novo normal} torna"-se uma perigosa maneira de
aceitar o sofrimento em nossos corpos, na ocupação cada vez maior do
trabalho no tempo e espaço de nossas vidas, na exclusão de pessoas pelo
desemprego e fome e, especialmente, pelas centenas de mortes diárias que
se tornam paisagem no cenário atual.

\section{cuidado de si e associação entre livres}

Criar associações em liberdade, que reciprocamente comportem apoio entre
seus membros, emerge sem que jamais se cristalize o lugar essencialista
de quem ajuda e daquele que é ajudado. A noção de solidariedade para os
anarquistas será sempre a afirmação de parceria e nunca de caridade,
porque não há ajuda mútua que não passe pelo autocuidado como condição
primeira. Uma chave possível para pensar esse arranjo de forças está nas
análises do filósofo Michel Foucault, quando afirma que cuidar de si
deve ser condição ontológica e primeira para cuidar do outro.

O cuidado de si é um constante (re)inventar"-se, ativado pelas práticas
de liberdade diante das práticas de poder como condição de estar no
mundo, uma ação, simultaneamente, solitária e coletiva. Mesmo entendendo
que ``o cuidado de si vem eticamente em primeiro lugar, na medida em que
a relação consigo mesmo é ontologicamente primária'' (\textsc{foucault}, 2004, p.
272), estar em permanente percepção e relação ao outro é a condição
necessária ao acordo ético. No exercício da vida livre, qualquer prática
que esteja desconectada ao outro pode rapidamente tornar"-se algo contra
o outro, mesmo quando esta se apresenta no lugar da ajuda.

A perspectiva libertária de associação, portanto, vai buscar
continuamente a criação de um balanceamento de reciprocidades que
possibilitem um permanente arranjo de forças. Algo difícil de ser
pensado a partir das sociabilidades que estão postas pelo capitalismo
contemporâneo e pelos resíduos das moralidades judaico"-cristãs,
impregnadas de ressentimento e da culpa. Como contraponto, o cuidado de
si e do outro está instaurado no sincronismo de ações, prática que
ocorre como obra aberta no instante de cada encontro. Os anarquismos
afirmam"-se em momentos dinâmicos e entende que não há prática de
liberdade possível sem a permanente consideração do outro.

Distante de qualquer ideia de amor ao próximo ou sentido humanista, a
relação libertária sustenta"-se na noção de que é com o \emph{outro} que
se estabelece o real sentido de si. As práticas de liberdade vividas por
cada um encontram seu significado e seu retorno quando a troca é
simétrica. Quando essa simetria se desfaz, há falta de ética e
consequentemente tendência para a dominação. Típicas das relações
assistencialistas, as formas de controle de uns sobre outros são
camufladas em atos pródigos, quase sempre para promoverem o \emph{bem}
do próximo. Nos instantes em que são instaurados e mantidos vínculos
assentados no diapasão entre aqueles que ajudam e os que são ajudados, a
dependência se perpetua. Não devemos entender a relação de domínio
apenas no momento da coesão, mas, sobretudo, na capilaridade das
práticas e ações que operam a partir dos governos das condutas.

A luta por instaurar sociabilidades livres que a anarquia propõe ocorre
a partir um princípio ético em relação aos demais, a partir do qual se
elejam os que estão mais próximos de si, daqueles que se remetem a
outros círculos mais distantes. Esta será uma escolha própria, seguindo
o próprio desígnio, jamais por imposição de uma moral universal já
preestabelecida ou por qualquer noção de Bem que se coloque \emph{a
priori}. Será a própria análise, a partir das informações que são dadas
pelos demais, num conjunto de circunstâncias, atitudes e sinais, que
cada um escolhe ou não pela possibilidade de encontro e troca solidária.
Uma articulação absolutamente singular, jamais genérica, inscrita a cada
instante de realidade e em permanente movimento.

A solidariedade e a generosidade são práticas típicas dos anarquismos.
Em acontecimentos históricos ou no tempo presente, a luta por construir
relações de parceira e cumplicidade marca a trajetória de homens e
mulheres que se colocam diante dos desafios de romper as práticas de
dominação e o fazem a partir do sentido de apoio mútuo. A liberdade na
perspectiva libertária será sempre exercida em perspectiva social,
fazendo deslocar o sentido de relação com o outro dos modelos
instaurados pelo liberalismo, pelos monoteísmos e pelos Estados. A vida
anarquista é uma luta constante pela eliminação da autoridade, seja ela
exercida pele coesão ou pela sugestão.

São ainda associações inscritas em sociabilidades que resistem para
fazer desaparecer a competição, o interesse pelo domínio e a
dependência. Buscam explodir as hierarquias e fazer vigorar práticas
livres, em um incessante combate agônico. Fundadas no princípio da
reciprocidade e na construção de uma cultura libertária, as associações
entre anarquistas procuram inventar encontros de apoio mútuo, distantes
dos pactos celebrados a partir da centralidade. A preocupação na
associação libertária supõe o uso da alteridade, a fim de promover um
bom encontro, na qual nenhuma das partes inclua a outra em seu projeto
para subjugar ou transformá"-lo em troféu. O anarquista não está
interessado também em ser mártir, em salvar o outro, pois apenas o outro
deve ser autor de sua própria existência.

Por fim, vale questionar sobre a atualização da noção de ajuda mútua
fora dos valores e modelos que nos são ofertados cotidianamente. Seria
possível aproximá"-la do individualismo radial anarquista? Acredito ser
importante resgatar aqui o conceito de \emph{único}, presente no
pensamento de Max Stirner (2004), que por muito tempo foi marginalizado
dentro do próprio movimento libertário. Situando"-se na contramão da
tradição socialista e à borda de valores defendidos e expressados pelo
anarquismo como a cooperação e a solidariedade, a defesa radical do
\emph{único} em Stirner tornou"-se um manifesto de uma forma de
individualidade que não despreza o outro. Não se trata de um egoísmo
vulgar, desconectado aos demais, mas a certeza de que nada ou ninguém
estará acima do \emph{único} para governar ou indicar caminhos. Estar
junto a Stirner e buscar a força de seu pensamento para uma análise
crítica do presente é fazer tencionar as ações acomodadas na democracia
representativa e nos saldos humanistas e religiosos de amor ao próximo,
muitas vezes presentes mesmo entre anarquistas.

Com seu pensamento, somos provocados a sustentar o sentido de
\emph{único} que há em nós como forma de ação política dentro do atual
cenário, no qual as ideologias coletivas mostraram seu fracasso. Stirner
aponta a necessária importância em se livrar das noções essencialistas,
tais como o homem, a natureza, a sociedade, etc. para então sustentar
sua dimensão estritamente singular: a singularidade"-do"-próprio. A recusa
ao Estado anda acompanhada pela recusa ao vínculo gregário tomado como
inerente à condição humana. Stirner diz que ``aqueles que exortam o
homem a ser `altruísta' acham que têm muitíssimo a dizer. E o que
entendem eles por isso? Qualquer coisa de muito semelhante a `renúncia
de si mesmo'. Mas quem é esse `si mesmo', que tem de ser renegado e não
pode ter interesses? Parece que deve ser tu próprio. E que interesse se
esconde por detrás da exortação altruísta à renúncia de ti mesmo? É
também o teu interesse e proveito, para que tu, por altruísmo, alcances
o teu `verdadeiro eu'.'' (\textsc{stirner}, 2004, p. 54).

A resistência que ele afirma é solipsista e vem junto com a recusa por
uma ordem social que não cessa em tentar moldar singularidades ao
interesse coletivo. Ao \emph{único}, o que importa é a preservação de
sua autonomia, tornando"-se um ser de difícil captura pelo jogo do poder
societário. É possível e necessário estabelecer uma justa medida que
produza um máximo de benefícios para um e para todos.

Uma associação entre libertários, portanto, difere dos arranjos
inscritos na família hierarquizada e sua moral de acomodação. Não se
trata também de um contrato social e muito menos no Estado como
atualização em larga escala da própria família. A vida libertára nos
provoca a definir nossos encontros baseando"-se na realidade que se
mostra diante de cada um, nunca numa lei já existente, que se defina em
nome de um coletivo. A construção por espaços de liberdade ocorre,
portanto, na relação direta com o real, descartando qualquer condição
que não se situa nessa esfera.

O atual momento pelo que passamos, em decorrência da presença de um
patógeno a ser mais bem"-conhecido, fez sacudir o mundo e esgaçou ainda
mais as condições de vida. Serve de alerta sobre o modo predatório que
estamos vivendo e sobre a relação que estabelecemos com o outro e com o
meio. Ao mesmo tempo, possibilita aos anarquismos explicitarem que há
outras formas de viver. A revolta como afeto; o ódio como aliado do
amor; e o incômodo como disparador de ações são formas de amar e ser
solidário entre os libertários. Surgem da luta incessante de si consigo
mesmo e de si com os demais, na afirmação da autonomia e da construção
de práticas de liberdade a fim de garantir que a cumplicidade é também
uma forma de cuidado.

\begin{bibliohedra}
\tit{FOUCAULT}, Michel. \emph{História da Sexualidade \textsc{iii}: O cuidado de si.}
Tradução de Maria Thereza da Costa Albuquerque. Rio de Janeiro: Paz \&
Terra, 2014.

\titidem. \emph{O governo de si e dos outros.} Curso no Collège de
France (1982--1983). Tradução de Eduardo Brandão. São Paulo: Martins
Fontes, 2010.

\titidem. \emph{Ética, Sexualidade, Política}. In: \textsc{motta}, M. B. (org.).
Tradução de Elisa Monteiro e Inês Autran Dourado Barbosa. Rio de
Janeiro: Forense Universitária, 2004. (Ditos e Escritos; V).

\tit{KROPOTKIN}, Piotr. \emph{A Moral Anarquista.} Tradução de José Luiz de
Souza Pérez. Lisboa: Edições Sílabo, 2009.

\titidem. \emph{El Apoyo Mutuo --- Um Factor de La Evolucion.} Madrid:
Ediciones Tierra y Libertad, 1948.

\tit{REICH}, Wilhelm. \emph{A Revolução Sexual}. Tradução de Ary Blaustein.
Rio de Janeiro: Zahar Editores, 1982.

\titidem. \emph{Psicologia de Massa do Fascismo}. Tradução de Maria da
Graça M. Macedo. São Paulo: Martins Fontes, 1988.

\tit{STIRNER}, Max. \emph{O Único e a sua Propriedade.} Tradução de João
Barrento. Lisboa: Antígona, 2004.
\end{bibliohedra}

\chapter[Da anormalidade à normalidade doentia da espécie humana, \emph{por José Maria Carvalho Ferreira}]{Da anormalidade à normalidade\\ doentia da espécie humana}
\hedramarkboth{Da anormalidade à normalidade\ldots{}}{}

\begin{flushright}
\textsc{José Maria Carvalho Ferreira}
\end{flushright}

\section{contradições e limites das medidas\break sanitárias para extinguira \textsc{covid}"-19}

\noindent{}Desde os tempos imemoriais que se conhece a existência de calamidades
mortíferas no seio da espécie humana. Como sempre a sistematicidade
específica das suas consequências leva a que os problemas perversos,
entretanto criados, sejam mediatizados pelos governantes, classe
política, ciência e \emph{mass media} como um problema meramente
sanitário, de contato humano e de higienização forçada pelo
acantonamento generalizado nas aldeias, vilas e cidades, como se essas
medidas fossem as únicas profilaxias com valor genuíno para contrastar
as funções de equilíbrio imunológico entre a vida e a morte. Nos últimos
séculos, atravessados pelo progresso e a razão, a ciência biológica e
médica, em sintonia com a indústria farmacêutica e química, saindo do
pedestal dos poderes dos possidentes, tem"-nos dados respostas, cada vez
mais assentes na plenitude e potenciação da vida em detrimento da morte,
nas quais o consumo da saúde como espaço"-tempo dos grupos sociais mais
vulneráveis assume uma preponderância crucial, como é o caso dos velhos
e das velhas, dos pobres e dos miseráveis.

Se é fato inequívoco que essa normalidade doentia resultante das
diferentes epidemias na história foi sempre sujeita a uma cura padrão,
sem grandes convulsões sociais, econômicas, políticas e culturais, com a
ocorrência da pandemia gerada pela \textsc{covid}"-19 emergiu um conjunto de
situações imprevisíveis consideradas anormais que, por várias razões,
põem em risco todos os comportamentos"-padrão da espécie humana em nível
mundial. Enquanto a normalização da doença epidêmica resultou dos
ditames do progresso e da razão germinada pela ciência e os fautores dos
poderes instituídos do capitalismo e do Estado, qualquer epidemia era e
foi, na maioria dos casos, quase sempre ultrapassada pelo valor
heurístico da ciência, do Estado, do capitalismo e, sobretudo, da
indústria farmacêutica. É evidente que esses benfeitores do progresso e
da razão sempre identificados com a saúde da espécie humana, em função
da ignorância endêmica relativa à fisiologia orgânica do seu corpo,
transformaram"-se, progressivamente, em cobaias impotentes de compreensão
e vivência com vírus e bactérias que acompanham a sua vida quotidiana
miserável, circunscrita à destruição mortífera de si e de todas as
espécies animais e vegetais que fazem parte da sua existência no
planeta.

No caso atual da pandemia denominada \textsc{covid}"-19, em meu entendimento, as
reflexões ou análises possíveis que me são permitidas realizar são
perpassadas pela ignorância, o medo, o risco e a incerteza. Desde sempre
na sua historicidade normativa, as doenças que têm acompanhado a
evolução da espécie humana por meio dos ditames do progresso e da razão
sempre foram vistas e assimiladas como algo natural e inevitável, que
poderiam ser superadas pela ciência médica e pela indústria
farmacêutica; inclusive, nos últimos tempos, pela força superior da
integração das ciências nos mecanismos automáticos das novas
tecnologias, com especial incidência para os papéis estruturantes da
robótica, nanotecnologia, inteligência artificial, informática,
cibernética, etc., realidade virtual inaudita que passou a dar corpo e
força às ciências médicas e biológicas. Em qualquer das circunstâncias
desse processo histórico subsistia sempre uma certeza absoluta. Para
esse efeito, todos os subsistemas econômico, político, cultural, social
e biológico eram e são interdependentes e complementares de forma a
produzirem sínteses de prevenção, controle e erradicação da doença ou da
morte que afete a condição"-função da espécie humana, tanto no teatro da
guerra entre grupos, comunidades e países, como nunca se convencionou
denominar, trabalho fábrica, nas prisões, nos centro comerciais, no
mundo do lazer, ou até recentemente no teletrabalho da economia virtual.
Em relação estreita com esse mundo que caminha a passos largos para o
abismo, nos outros espaços"-tempos dos excluídos e dos miseráveis
encontramos os exércitos dos velhos que esperam pelas antecâmaras da
morte, os mendigos, os desempregados, os precários e os desempregados do
mundo do trabalho, os pobres e os sem abrigo, os refugiados e a toda a
panóplia de situações anormais que a são a expressão lógica dessa
normalidade doentia.

Por detrás dessa crença religiosa de solução normativa, a ciência médica
e a indústria farmacêutica sempre pautaram a sua ação como se estivessem
predestinadas a suprir ou erradicar qualquer tipo de doença do corpo
humano, baseadas no pressuposto de que era essa a sua função em qualquer
circunstância, sendo que a primazia da vida era insubstituível em face
da emergência de qualquer sintoma de morte. Sempre foi apanágio da
civilização judaico"-cristã associar os dons miraculosos de Deus e dos
seus acólitos terrenos em sintonia e associação estreita de paladinos da
cura do corpo humano, com especial incidência das funções da ciência, da
tecnologia, do capitalismo e do Estado. Cada uma dessas instituições
rapidamente criou o seu espaço de poder e de vocação profissional, razão
pela qual evoluíssem, naturalmente, para a um processo de
institucionalização e legitimação nas sociedades contemporâneas em que
só elas, com exceção de outras consideradas anormais, tenham lugar ao
exercício da medicina e da ciência, a partir de uma estrutura de
autoridade hierárquica formal, uma divisão social do trabalho, poder de
decisão e de liderança em relação ao que se convencionou denominar
saúde, cura, morte do corpo, desde o nascimento até à morte, ao ponto de
cada médico, cientista, biólogo, padre, cada qual a seu modo, poder
controlar, prevenir, curar todas as anomalias do corpo humano, entre as
quais vírus, bactérias, órgãos, etc., e desse modo justificar a sua
existência dominante sobre todos aqueles que ignoram, como eu, a
especificidade metabólica do seu corpo.

Na normalidade instituída, nada pode ser como antes, sobretudo se
tivermos presente o que ocorre em nível mundial gerado pela \textsc{covid}"-19. As
instituições e as organizações normativas em face das evidências da
generalização desse vírus demonstram"-se apáticas, ignorantes,
disfuncionais, perdidas no meio de um turbilhão de incapacidades
manifestas de controle, prevenção, de solução para a cura do corpo
humano que tem articulações biológicas e sociais com a economia, a
política, a cultura e a sociedade. No sentido amplo do termo, o problema
do vírus não se confina à condição"-função do corpo humano, mas ele tem
repercussões manifestas em todos os espaços"-tempos de copresença física
em que a divisão social do trabalho, os processos de socialização e de
sociabilidade estão diretamente articulados com as relações sociais e as
interações sociais decorrentes dos contatos humanos. Estamos
efetivamente a caminho de uma anormalidade anterior para uma normalidade
do vírus da \textsc{covid}"-19.

Nessa assunção, para chegarmos a evidências empíricas que possam
retratar um pouco, estatisticamente, segundo a
``worldometer"-estatísticas mundiais em tempo real'', baseando"-me na
informação veiculada, no dia 16 de setembro de 2020, para uma população
mundial de 7.812.213.084, as mortes causadas no domínio da saúde eram de
9.230.042 por doenças devido a transmissões contagiosas; 30.254.044 por
abortos; 5.406.273 de crianças menores de 5 anos; 1.185.241 de pessoas
infetadas pela \textsc{hiv}/\textsc{aids}; e 5.839.414 por câncer. Se alargarmos essas
estatísticas a outros domínios, constatamos que há 845.996.139 pessoas
desnutridas no mundo atual, ao mesmo tempo que outras subsistem com peso
a mais, como é o caso de 1.699.019.912 pessoas. Enquanto existem
786.570.809 pessoas obesas no mundo, outras 18.362 morrem de fome.

Se tivermos presente a especificidade das múltiplas contingências da
epidemia gerada pela \textsc{covid}"-19, está mais que demonstrado que a
ignorância, o medo, o risco e a incerteza ultrapassam qualquer
raciocínio ou prática de superar a anormalidade/normalidade habitual da
espécie humana em relação ao que se entende por vida, morte e saúde.
Assim, com confinamentos ou sem confinamentos, com quarentenas ou sem
quarentenas, com testes ou sem testes, com higienes ou sem higienes
diversificadas, com descobertas ou não de vacinas num horizonte temporal
próximo ou longínquo, não restam dúvidas de que, desde cientistas,
governantes políticos, chefes religiosos, passando por uma
multiplicidade de especialistas que se dizem representantes e defensores
dos povos, perante as perversões da \textsc{covid}"-19, todos se revelam
impotentes para agir e decidir de forma consequente sobre os desafios
biológicos e sociais que se apresentam à espécie humana na sua
globalidade.

Tendo presentes os dados estatísticos mundiais a que já referi, sabendo
de antemão das diferenças e especificidades de cada país, é fato
indubitável que, na altura em que escrevo este texto, a manifestação
objetiva da \textsc{covid}"-19 revela"-se assustadora e irreversível, com
29.788.767 casos, 940.353 mortos e 2.597.636 seres humanos infetados
recuperados.

Não escondendo nem tampouco enaltecendo a linguagem ideológica e
religiosa dos números que envolvem a quantidade de óbitos, o que é fato
inequívoco por detrás deles, persistem as supostas análises objetivas da
ciência e da política que são, invariavelmente, balizadas pela
incerteza, pelo risco e pela ignorância no que concerne a todos os
processos de sociabilidade e de socialização da espécie humana, seja em
nível do espaço"-tempo de aglomerados ou confinamentos de 5, 20, 100,
1000, 10.000, 100.000 ou até 1.000.000 pessoas. Se passarmos para a
sociedade do mundo com fronteiras e pátrias definidas pela atual
civilização judaico"-cristã, então, esse processo torna"-se muito mais
abstrato e complexo.

Na impossibilidade de encontrar uma explicação científica plausível em
termos das causas e efeitos que permitam perceber a natureza biológica
da pandemia gerada pela \textsc{covid}"-19, com uma facilidade extraordinária,
cientistas de todo o mundo circunscrevem a resolução normativa das
diferentes latitudes desse problema ao contato humano, com especial
incidência para as relações sociais de copresença física a diferentes
dimensões populacionais, de promiscuidade física, respiratória, oral e
sexual.

Essa leitura mundial da \textsc{covid}"-19 prescinde do conhecimento de outras
realidades prementes que estão na origem da criação de múltiplos vírus e
bactérias e de mutações do corpo humano, como são os casos emblemáticos
da cadeia alimentar da espécie humana decorrentes da destruição e
mutação de espécies animais e vegetais. Na mesma linha de raciocínio,
até que ponto as experiências laboratoriais científicas de espécies
animais e vegetais estarão na origem de novos vírus e bactérias. As
guerras químicas e as experiências nucleares, cada vez mais
sofisticadas, em correlação estreita com os avatares nefastos do mundo
da economia que faz dos seus objetivos a destruição do ambiente e o
aquecimento global do planeta Terra. Se pensarmos no planeta Terra
criador de oxigênio e água, poderíamos perguntar até que ponto a
destruição do oxigênio e da água só por si, por diferentes vias, explica
a criação de vírus e bactérias e, simultaneamente, é a progenitora da
incapacidade imunológica do corpo humano. Esse mundo anormal/normalizado
nunca poderá ser questionado nem, tampouco, inquirido como a origem da
\textsc{covid}"-19, daí que o mais fácil tenha sido, desde março de 2020,
encontrar o bode expiatório no metabolismo do aparelho respiratório do
corpo humano.

Partindo"-se desse pressuposto científico inquestionável no contexto
modelar anormal/normal negativo, para os que detinham o poder político e
sanitário, haveria que construir estratégias ideológicas políticas e
econômicas, biológicas e sociais credíveis para superar os malefícios
causados pela \textsc{covid}"-19. Nessa dimensão estratégica, duas posturas têm
sido seguidas pelos políticos mundiais. Uma que consiste na defesa da
impotência decisória tendo presente a natureza invisível e inconcebível
de um tipo de vírus que leva as sociedades contemporâneas a não poderem
reproduzir o mesmo esquema de controle e de prevenção a tudo a que se
refere a biologia e relações sociais do que antes se inferia
normativamente como contato humano. Para os paradigmas científicos
dominantes e o poder instituído, só subsiste uma solução credível, quase
sempre balizados por processos de acantonamento ou de quarentena. Estas
são semelhantes a qualquer prisão do Estado, só que tem que ser
assumidas, voluntariamente, por qualquer indivíduo, grupo, organização,
família ou comunidade.

Na falta de outras propostas credíveis ajustadas ao desconhecimento do
vírus da \textsc{covid}"-19, procura"-se a todo o custo encontrar uma vacina que
supere milagrosamente pela via da ciência os malefícios da pandemia que
afeta as sociedades contemporâneas. Com o intuito de capitalizar as
soluções de erradicação da atual pandemia, os países capitalistas mais
desenvolvidos fazem da descoberta de uma vacina contra a \textsc{covid}"-19 um
mercado de capitalização e de maximização do lucro, sendo que esses
desígnios estratégicos e objetivos estão na miríade de poder político e
econômico e centrados na estabilização normativa da espécie humana no
que concerne à plausibilidade clássica dos dilemas da vida e da morte,
em que a indústria da saúde atravessa sobremaneira a indústria
farmacêutica. Não restam dúvidas de que essas hipóteses não são
plausíveis para hoje nem para amanhã. A natureza abstrata, complexa e
invisível do vírus ao manter"-se, previsivelmente, numa zona escura de
ignorância, medo, especulação, instrumentalização, mesmo junto daqueles
cuja função é conhecer a assunção da perda de capacidade imunológica do
corpo humano em relação à \textsc{covid}"-19.

\section{das probabilidades de eliminar, condicionar\break ou até acabar com a atual pandemia}

Pode"-se opinar em diferentes sentidos em relação à \textsc{covid}"-19, mas o que
para mim faz algum sentido interpretativo e explicativo se resume à
possibilidade ou não de compreender até que ponto as medidas
preconizadas pelo poder político, os cientistas, o capitalismo e o
Estado, desde março de 2019 até agora, têm ou não qualquer valor
heurístico no que concerne à erradicação da \textsc{covid}"-19, tomando como
referências as medidas profiláticas de diferentes tipos, processos de
acantonamento, de quarentena como soluções plausíveis de erradicação da
pandemia provocada pelo novo coronavírus. Na historicidade da espécie
humana, é difícil, senão impossível, subsistir sem laços sociais
conducentes ao imperativo físico e biológico do contato humano. Sem
contato humano, não há amor, amizade, liberdade e criatividade. Sem
essas dimensões comportamentais humanas, não há hipóteses de
socialização e de sociabilidade de construção das famílias, grupos,
comunidades, organizações, instituições, povos, sociedades.

Nesse sentido, para que o contato humano seja diminuído, drasticamente,
em face das perversões criadas pelo novo coronavírus, na falta de outras
estratégias, para o poder instituído, torna"-se necessário virtualizar as
relações sociais e os processos de socialização e de sociabilidade em
diferentes níveis: interpessoais, intragrupais, intergrupais,
intraorganizacionais, interorganizacionais, intrassocietais e
intersocietais.

Perante a atual tragédia sanitária e à incapacidade manifesta de
solucionar a tempo e de forma eficaz, como sempre, acentuam"-se as
perversões da \textsc{covid}"-19, sem que se vislumbrem soluções credíveis por
parte do poder instituído no sentido de voltar ao modelo da ação
individual e da ação coletiva do ser humano no estrito termo do
anormal/normal, fazendo tábua rasa do que está na origem e efeitos de
qualquer vírus em interdependência e complementaridade com a natureza
imunológica de qualquer ser humano. Sem ainda vivermos os efeitos das
influências estruturantes das novas tecnologias como hoje, é um dado
indiscutível que, desde a década de 1970, vivemos as contingências da
virtualização das relações sociais, e isso tem sido um fato progressivo
no mundo do trabalho, com especial incidência nos setores da produção,
distribuição, troca e consumo de bens e serviços, mas também da vida
quotidiana em geral. Nesse domínio, antes dessa realidade, os vários
elementos estruturantes de uma vida quotidiana de miséria, escravidão e
alienação do operariado industrial, associados à inexistência de
criatividade e liberdade da espécie humana, eram plenamente assumidos na
sua base concreta existencial.

Hoje, em vez disso, emergem novas modalidades de escravidão, dominação e
controle associadas às vicissitudes da plasticidade social, inerentes
aos imperativos funcionais do mundo virtual assentes cujo papel crucial
é assumido pelas Tecnologias de Informação e de Comunicação (\textsc{tic}). Estas
não somente virtualizam a quase totalidade do processo do trabalho e da
organização do trabalho das organizações e instituições da sociedade,
como também estão presentes nas esferas dos espaços"-tempos de produção,
distribuição, troca e consumo de bens e serviços imateriais ou
analítico"-simbólicos. Daqui podemos deduzir que das 24 horas potenciais
que cada indivíduo dispõe por dia, uma parte substancial é assimilada,
capturada, instrumentalizada e controlada, pela ação virtual das \textsc{tic}s.
Desde que os órgãos sensoriais estejam suficientemente estimulados e
adaptados para esse efeito, as \textsc{tic}s preenchem os requisitos virtuais e
funcionais para desenvolver as qualificações e aprendizagens
profissionais do mercado de trabalho mundial. Sendo assim, a
virtualização da vida quotidiana dos indivíduos foi objeto de uma
potenciação inaudita no nível do lazer e do consumo de bens e serviços
analítico"-simbólicos de uma diversidade quase infinita.

Em presença emergencial das \textsc{tic}s, toda a economia real é,
diferentemente, afetada. Não obstante a economia real estar
correlacionada com os espaços"-tempos inevitáveis da copresença física e
do contato humano nos espaços"-tempos da produção, distribuição, troca e
consumo de mercadorias, as \textsc{tic}s virtualizam as estruturas e funções da
economia real. A anormalidade/normalidade também se expressa em todos os
domínios exteriores da economia real que não estão secularmente
confinados às atividades laborais do mundo do trabalho nos setores
industrial, comercial e agrícola. Seja nos múltiplos espaços do lazer e
do consumo, transportes rodoviários, fluviais e aéreos, viagens
turísticas, manifestações desportivas, recreativas e culturais e,
sobretudo na vida quotidiana das famílias e das comunidades locais, o
contato humano e as relações sociais de copresença física eram, em todos
os espaços"-tempos, determinados para os processos de socialização e de
sociabilidade biológica traduzida em altos índices de produtividade do
trabalho, no dispêndio de energia cooperativa, de fraternidade e de
solidariedade nos locais de trabalho e de consumo. A família é, por
excelência e por vocação milenar, o local do afeto, da amizade e do
amor. Esses atributos só são possíveis de realizar quando o contato
humano é extenso e intenso.

Numa segunda dimensão, essencialmente crítica, ao mesmo tempo que os
efeitos estruturantes das \textsc{tic}s se revelam hegemônicos na vida quotidiana
dos indivíduos, o espaço"-tempo de intervenção da economia real clássica
circunscrita ao processo de industrialização e de urbanização das
sociedades perde, paulatinamente, o seu significado e a sua hegemonia
centrada, exclusivamente, no crescimento econômico e na criação de
riqueza social de bens materiais de consumo corrente e duradouro. As
evidências empíricas desse processo histórico têm se revelado um
elemento de destruição da natureza, com incidência na morte das espécies
animais e vegetais, poluição atmosférica, morte dos oceanos, rios,
florestas e, consequentemente, da Terra. O aquecimento global do planeta
Terra associado à diminuição da camada de ozônio e destruição do clima
vêm pôr a nu os efeitos perversos dessa economia real como plataforma
básica da destruição da água, ar, fogo e terra, elementos vitais de vida
de todas as espécies animais e vegetais, incluindo a humana.

No tempo concreto da nossa existência, estamos na presença de uma
economia manifestamente negativa, cujas potência e expressão básica se
resumem no poder da produção de morte e extinção dos elementos vitais de
vida. Perante esse descalabro estimulado pelo poder instituído, gestores
e escravocratas de todas as matizes e defensores de modelos de sociedade
contrastantes, os arautos das soluções miraculosas, tornaram"-se
apologistas da economia verde ou da economia azul, enaltecendo estas
como a solução de superação de todos os problemas de miséria, pobreza,
desemprego, exclusão social e crise ambiental. Nada mais falacioso, pois
estas, no contexto das soluções do capitalismo, nada mais são de que uma
forma de expiação histórica ou de simulacro de um paradigma societal que
caminha a passos largos para o abismo. Essas alternativas deduzidas da
economia azul ou da economia verde não resolvem a essência identitária
da espécie humana com a natureza, na medida em que a perpetuação do
capitalismo, do Estado e da civilização judaico"-cristã, em quaisquer
circunstâncias, sempre foi e será, em última instância, essência de
morte e não de vida. No meu entendimento, em pleno século \textsc{xxi}, a
anarquia só pode ser ou assumir"-se como potência de negação da realidade
que acabei de descrever.

Se pensarmos bem na emergência da nova pandemia circunscrita aos efeitos
negativos do novo coronavírus, desde logo estamos sujeitos a situações
de medo, ignorância, risco e incerteza que só favorecem a ação
individual e coletiva das instituições e organizações do Estado e da
sociedade civil no sentido da atomização, controle e repressão da
liberdade e da vida quotidiana dos indivíduos, sendo estes encurralados
nos dilemas da sobrevivência perante as hipóteses de vida e de morte
provocados pela epidemia. Esses mesmos indivíduos tornam"-se também os
fautores normalizadores da sua própria tragédia biológica e social.
Daqui emergem novos poderes oriundos da medicina e da indústria
farmacêutica, sem esquecer todos os poderes clássicos dos governos, das
polícias, dos exércitos, dos cientistas de todo o tipo, cuja função
primeira consiste em educar e controlar os cidadãos comuns no sentido da
normalização de uma vida sem sentido, em acantonamentos de diferentes
tipos, que não são mais do que uma diversidade de prisões físicas,
mentais e psíquicas pandêmicas à medida do medo e da ignorância de cada
ser humano.

Num contexto histórico em que a emergência estruturante das \textsc{tic}s se
torna galopante, não admira que a sua visibilidade social se expresse em
todas as esferas das atividades econômicas, sociais, políticas e
culturais, após o apogeu de um passado longínquo dos trinta gloriosos
anos do capitalismo (1974--1975). Confinados a uma expressão identitária,
embora diferenciada em espaços"-tempos de produção, distribuição, troca e
consumo de bens e serviços analítico"-simbólicos, não existem mais
espaços"-tempos para a perpetuação do modelo de economia real.

Desde que emergiu a uma nova epidemia gerada pelo novo coronavírus,
novos desafios sociais, biológicos, políticos, econômicos, sociais,
culturais e sobretudo científicos foram objeto de avaliação normativa.
As reações e adaptações daqueles que governam o mundo, dos cientistas,
dos burocratas, dos gestores e da indústria farmacêutica não se fizeram
esperar. Medo, ignorância, incerteza e risco aparecem mesclados com a
necessidade de tomar decisões urgentes \emph{ad hoc}, só com o pretexto
de manter o poder e a profissão. Para responder às múltiplas
solicitações dos diferentes súditos que mais não são que consumidores
passivos da sua própria miséria existencial, emergiu uma bateria de
especialistas, como especial relevo para os \emph{mass media,} sempre
incrustados em paliativos informativos, cuja base informativa consistia
em transformar a epidemia num modelo de vida quotidiana cerceado por uma
série de acantonamentos. Essa evolução dos poderes normalizadores da
espécie humana, associados ao espectro e funções especulativas e
instrumentais dos \emph{mass media} para gáudio da satisfação das
grandes massas e multidões esfomeadas de soluções históricas atávicas
que evitem deixá"-las cair em uma tragédia biológica e social no sentido
da morte, não está, de modo algum, garantida.

Pela minha parte, assumo a condição de ignorante, de medo, incerteza e
risco em relação às consequências e origem da \textsc{covid}"-19. Todavia, como já
referi, nesse mundo de protagonistas compulsivos da especulação e da
mentira descarada sobre a natureza das origens e efeitos da \textsc{covid}"-19,
sobressaem com grande proficiência todos os jornalistas, cientistas,
políticos e capitalistas que acham que o capitalismo e o Estado têm uma
existência histórica eterna. Ao que é lícito denominar povo ou espécie
humana, a estes resta"-lhes seguir de forma acrítica e especulativa as
vicissitudes da sua anormalidade existencial sanitária em relação às
contingências da \textsc{covid}"-19, que virou uma normalidade doentia, como todas
as outras a que estão sujeitas. Para os possidentes da riqueza social,
grupos sociais privilegiados, patrões e burocratas de toda a espécie,
mesmo com a existência da \textsc{covid}"-19, é possível usufruir de modalidades
do confinamento social sem implicações do contato físico, o que,
potencialmente, reduz os níveis de afetados nesses grupos sociais e
também o número de mortos.

Em relação à situação privilegiada dessa minoria no cômputo geral dos
quase 8 bilhões de seres humanos que habitam o planeta Terra, o mesmo já
não se pode afirmar no que se reporta aos milhões da população mundial
que sobrevivem em condições infra"-humanas em termos de alimentação,
habitação, água potável, oxigênio e salários de miséria. Estes são
constelação de gente em que o confinamento e o contato humano só são
pacíficos de se concretizar por meio da promiscuidade, porque só assim
podem ser seres sociais e biológicos no sentido extenso e intenso da
palavra. Como é lógico, seja em que condições paupérrimas de existência
se encontrem, a probabilidade de evoluírem para a situação de infetados
pelo novo coronavírus é grande, ao mesmo tempo que o sentido do caminho
para morte é superior.

Essa anormalidade/normalidade doentia provocada pela atual epidemia
alarmou os governantes mundiais e, ao mesmo tempo, pôs em causa o modelo
vigente da economia real. Para poder sair das contingências do contato
humano prevalecentes da economia real havia que parar e reformular os
seus alicerces básicos nos espaços"-tempos da produção, distribuição,
troca e consumo de mercadorias.

Num contexto de interdependência e complementaridade geradas pelas
contingências das sociedades contemporâneas, tendo presentes os
diferentes desafios invisíveis que a pandemia encerra, é por demais
evidente que a funcionalidade do processo e da organização do trabalho,
assim como em todos os setores da atividade econômica, e sobretudo para
evitar os contatos humanos baseados na copresença física, só é passível
de se realizar com base na economia virtual. Daqui, em grande medida,
emerge a solução ajustada ao confinamento e quarentena requerida para
superação e atenuação da crise gerada pelo novo coronavírus. Essa
influência das \textsc{tic}s de superação dessa crise não se remete,
exclusivamente, à economia virtual, mas também à virtualização dos
processos reportados à cultura, à política, ao social, à espécie humana
nas relações com as outras espécies animais e espécies vegetais, com
incidência na biologia, no clima, no meio ambiente, etc.

Para analisar essa evolução, podemos extrair as ilações mais relevantes.
Todos os bens e serviços imateriais ou analítico"-simbólicos resultantes
dos processos de produção, distribuição, troca e consumo através das
\textsc{tic}s são efêmeros, instantâneos, permanentes, circulares, infinitos, sem
chegar a um fim. Os \emph{inputs} e os \emph{outputs} de qualquer bem ou
serviços analítico"-simbólicos. Como serviços resultantes de um processo
de criação e transformação humana e automática, integram sempre
informação, conhecimento e energia. Em qualquer circunstância, os
\emph{inputs} e os \emph{outputs} são da mesma natureza. Nessa latitude,
estamos longe das matérias"-primas e mercadorias básicas da economia real
constituídas pelo petróleo, carvão, gás natural, ferro e aço, automóvel,
eletrodoméstico, construção civil, estrada, aviação, etc. Acresce, a
esse panorama, uma outra dimensão por via da influência das \textsc{tic}s na
economia virtual e na vida quotidiana em geral que importa sobremaneira
sublinhar. Enquanto na economia real os processos de produção,
distribuição, troca e consumo são constituídos por fronteiras e
barreiras rígidas, por uma organização autônoma específica e não há
continuidade laboral nesses processos, a economia virtual não necessita
de fronteiras, de uma organização específica e subsiste a imperatividade
contínua dos processos de produção, distribuição, troca e consumo
vivenciados em frações de segundo, minuto, hora, dia, semana, mês, ano e
por aí afora. Acresce"-se a essas diferenças que, para o fator de
produção trabalho na economia real, prevalece o elemento físico em
detrimento do cognitivo e do emocional.

A economia virtual, pelo contrário, privilegia a cognição e a emoção em
detrimento do elemento físico. Nada mais lógico, na estrita medida em
que o fator de produção na economia virtual necessita de competências
para descodificar e codificar informação, conhecimento e energia humana
automatizada. Só o consegue realizar com a proficiência devida pela
vida, sobretudo da cognição e emoção, ainda que tenha que utilizar a sua
energia humana. Transformar \emph{inputs} e produzir os \emph{outputs}
da economia real exige mais energia e menos utilização dos órgãos
sensoriais, como os casos da visão e da audição. Além disso, uma parte
substancial dos processos da economia real já utiliza uma panóplia de
elementos das \textsc{tic}s. Vejam"-se os casos emblemáticos das
máquinas"-ferramentas de comando numérico, informática, inteligência
artificial, robótica, mecatrónica, etc.

Essas singularidades de virtualização da vida quotidiana da espécie
humana na escala planetária não se reduzem ao mundo do trabalho. A
influência espaço"-temporal das \textsc{tic}s traduz"-se numa virtualização
crescente dos comportamentos humanos nas esferas econômica, social,
política, cultural, educacional e religiosa, na estrita medida em que os
órgãos sensoriais de cada indivíduo \emph{per si} interagem com toda a
panóplia dos automatismos das \textsc{tic}s com maior margem de liberdade,
criatividade e autonomia. A capacidade e a possibilidade de descodificar
as linguagens das \textsc{tic}s de forma atempada e adequada fazem com que esses
mesmos indivíduos prescindam do poder discricionário e autoritário dos
pais, da multidão de chefes, burocratas e patrões de todo o tipo e que,
também, tenham uma margem de autonomia e capacidade para transgredirem
em relação à normalidade das leis das sociedades e, desse modo,
subtraírem"-se das sanções e criminalizações decorrentes dos valores
vigentes.

Não obstante a irreversibilidade dos efeitos estruturantes das \textsc{tic}s, é
por demais evidente que nesse processo meramente competitivo e bastante
visível no mercado da economia virtual, por sua vez, como reverso da
medalha, têm"-se as suas expressões mais significativas na criação
generalizada de situações dualizadas para a espécie humana: desemprego,
precarização da vinculação contratual, pobreza, exclusão social, etc. Em
confronto com essa situação, a ação individual e coletiva das
instituições e organizações do Estado e da sociedade civil evolui no
sentido da atomização, controle e repressão da liberdade e da vida
quotidiana dos indivíduos, estando estes encurralados nos dilemas da
sobrevivência em relação à vida e à morte, hipoteticamente, provocados
pela epidemia.

No que concerne às probabilidades de sair de crise pandêmica atual e de
retorno à anormalidade/normativa desde que as causas e efeitos da
propagação são da exclusiva responsabilidade da espécie humana, então
temos um problema iminentemente biológico e social que incide na matriz
sistêmica de quaisquer contatos humanos, o que leva a que sua
condição"-função existencial seja pautada por um confinamento baseado no
autocontrole e na construção de pequenas prisões com delimitações de
espaços"-tempos de autonomia, copresença física, práticas sanitárias de
todo o tipo. Essas prisões são fundamentalmente psíquicas e mentais,
porque os processos de socialização e de sociabilidade das relações
sociais, nessas condições, só são possíveis de serem realizados por
processos virtuais. Além disso, ninguém sabe ainda qual é a essência do
vírus da \textsc{covid}"-19. A especulação e a instrumentalização a que assistimos
todos os dias acerca do espectro da evolução da \textsc{covid}"-19 são
sintomáticas no dia a dia do poder político, da ciência e dos \emph{mass
media}, para não falar da indústria farmacêutica.

Para mim, com a ignorância manifesta que tenho da biologia do meu corpo,
tenho extrema dificuldade de compreender que por via do meu aparelho
respiratório ou por qualquer episódio padrão através de partículas
virais, você respira, \emph{tosse} e espirra e está mergulhado num
conjunto de interações e contatos humanos sempre sujeitos a causalidades
e efeitos de infecção, de vida e de morte, para os quais os únicos
antídotos passam por soluções higienistas e sanitárias de
responsabilidade exclusiva de cada ser humano.

As instituições estatais de diferentes tipos procuram a todo o custo
precaver e controlar todos os desvios e transgressões que possam colidir
com o modelo da ciência relativo a situações de uma pandemia como a que
estamos vivendo. Na história da espécie humana, o aparelho respiratório
sempre esteve associado ao oxigênio como elemento de vida crucial. Não
consigo conceber essas funções biológicas de expiração e de inspiração
sem a ação dos pulmões, faringe, laringe, traqueia, brônquios.
Pergunto"-me em que medida a prevenção e o controle exercido pela
utilização das máscaras em espaços fechados e abertos, a proibição de
relações sociais em espaços inferiores a dois metros, as medidas
higiênicas podem afetar o corpo humano, até que ponto existe ou não uma
capacidade objetiva de evitar a situação de afetado pelo novo
coronavírus, sabendo nós que a invisibilidade dessa pandemia e a
ignorância e a especulação informativa por parte da ciência são
manifestas e que, por essa razão, seja propalada uma propaganda
mediática para acalmar os espíritos com a criação de uma nova vacina,
como se esta tivesse o dom messiânico de acabar com os medos atávicos da
morte, se o meu corpo perdeu ou não capacidade imunológica no contexto
de uma espécie humana que caminha a passos largos para a sua extinção.

Na sua pequenez metabólica e existencial em relação a múltiplos aspetos
da sua vida quotidiana, não é de admirar que cada ser humano esteja numa
situação de ignorância relativa à essência biológica do seu corpo,
deixando que esse conhecimento fosse entregue aos especialistas e
burocratas da ciência médica e à indústria farmacêutica, burocratas do
bem"-estar social do Estado, capitalistas e religiões cuja função
prioritária serve para instrumentalizar os corpos como mercadorias de
compra e venda propiciadora de lucro. Desde logo devemos olhar para esse
panorama de impotência comportamental para superar a perduração da
essência da vida biológica e social de todas as espécies animais e
vegetais; persiste sempre um dilema irredutível porque a potência da
vida no planeta Terra só tem significado desde que persistam laços de
sociabilidade e socialização conjugados com a interdependência,
complementaridade e identidade sistemáticas entre os que integram a
espécie humana.

Desta análise, resulta que na impossibilidade de sair da atual situação
biológica e social provocada pelo novo coronavírus, a economia real
tende a afundar"-se, e a economia e o mundo virtual da política, da
cultura e da sociedade civil emergem como a única solução plausível da
anormalidade doentia personificada pelas sociedades contemporâneas. Seja
no nível da família, seja no das comunidades ou povos, pela via do mundo
virtual, é possível ultrapassar os contágios e os contatos provocados
pelas relações sociais. O trabalho nas fábricas, transportes ou nas
superfícies comerciais é possível de se realizar pela socialização do
conhecimento, informação e energia sem contato ou envolvimento humano
baseado na copresença física. O acantonamento em diferentes
espaços"-tempo com especial relevância para a vida quotidiana da família,
nos níveis local, regional, nacional e mundial, é passível de vivificar
sobretudo com base no mundo virtual impulsionado pelas \textsc{tic}s. Assim
sendo, neste momento histórico atribulado da pandemia gerada pelo novo
coronavírus, por mais paradoxal que pareça, no meu entendimento, o
capitalismo e o Estado não soçobraram, definitivamente, porque as \textsc{tic}s
personificam o último recurso de não generalização da pandemia e ainda
permitem que não se propague a falência generalizada dessa realidade
política, econômica e social insustentável.

Por isso, em vez de ficar sempre debaixo dos ditames da informação e de
conhecimento dos esquemas normativos da ciência, do capitalismo e do
Estado, questiono"-me se já não terá chegado o momento definitivo de pôr
em questionamento o valor heurístico dessas instituições como hipóteses
de cuidarem e controlarem a vida humana a seu bel prazer. Diante dos
vários fatores questionáveis que confirmam essa necessidade histórica,
pergunto"-me qual é a natureza e como se realiza a totalidade do processo
da cadeia alimentar da espécie humana, sabendo do conteúdo e das formas
das relações que mantém com as outras espécies animais e vegetais, uma
vez que, na batuta discricionária e escravocrata da civilização
judaico"-cristã, a espécie humana faz das outras espécies um instrumento
do seu nó górdio de prazer e de ostentação consumista, com consequências
biológicas inevitáveis no corpo humano, no qual a obesidade, cânceres de
todo o tipo, pandemias e uma multiplicidade de doenças são o corolário
lógico desses absurdos anormais/normais da cadeia alimentar da espécie
humana. Numa outra dimensão, a relação da espécie humana com a água e o
oxigênio só é explicável pela existência mortífera, miserável e estúpida
de uma espécie humana entorpecida pela mente, psique e mente, focada na
destruição e na morte, que não conhece os aspetos elementares da vida da
terra, oceanos, rios e florestas. Tanta estupidez e tanta ignorância
juntas só podem dar origem à produção de bactérias e vírus provocados
através da destruição da natureza; é lógico e plausível que o próprio
corpo humano sofra mutações metabólicas e, consequentemente,
incapacidades imunológicas relativas à emergência de novos vírus e de
bactérias. Se considerarmos o corpo humo como totalidade biológica, não
podemos esquecer que os elementos psíquico e mental podem estar nas
mudanças metabólicas do comportamento humano que se traduziu numa
situação de incapacidade imunitária diante da pandemia atual.

Sem ser totalmente utópico depois de traçar o que considero mais
relevante em vez de pensarmos que estamos a viver uma situação de
normalidade social, econômica, política, cultural, biológica sempre
vista em função da normalidade decorrente de uma estabilidade normativa
racional inculcada e aculturada pela versão de valores, de moral e de
ética judaico"-cristã. A pandemia não é uma anormalidade, ela é antes de
mais nada o resultado lógico de uma normalidade doentia que não para de
crescer, em que a vida é vencida e destruída a passos pela morte,
gerados pelo capitalismo, o Estado e a espécie humana aculturada pela
civilização judaico"-cristã.

Já, agora, se o vírus centrado na incapacidade imunológica do aparelho
respiratório do corpo humano é tão evidente em relação aos problemas que
se apresentam às sociedades contemporâneas, pergunto"-me até que ponto,
em termos de potências de saúde e de morte, os cientistas, classe
política e a multiplicidade de quejandos profissionais pretendem
controlar, prevenir e governar a espécie humana; os vírus e bactérias
até que ponto mais importantes subsistem e são esquecidos no nível da
incapacidade imunológica mental e psíquica da espécie humana. Estes são
mais do que manifestos na impotência imunológica do corpo humano, o que
se traduz na criação de um quadro negro de miséria, morte, doença,
violência, cuja expressão maior é espelhada pela existência de guerras.
Para anormalidade sanitária não há explicação de qualquer imunidade ou
de vírus e bactérias. Antes mesmo das contingências da \textsc{covid}"-19, já era
previsível que a espécie humana caminhasse para o abismo. Desse modo,
ainda com alguma veleidade de suspiros heterotópicos, função do que
acabo de referir, sem querer deificar ou ideologizar a anarquia, ela é
cada vez mais uma probabilidade histórica com um valor heurístico imenso
para impedir que a espécie humana e todas as outras espécies soçobrem no
planeta Terra. Daqui extraio a lição que me ajuda a caminhar nesse
sentido, considerando que: ``Anarquia é um caos auto"-organizado, sem
deuses e sem amos. É só poderá ser uma probabilidade não linear no
espaço"-tempo do universo''.



\chapterspecial{Um vírus a serviço da ordem}{Aforismo libertário em uma movimentação\\ pelo espaço de um corpo social pandêmico}{\textsc{l.i.m.a.}}

\epigraph{É preciso, pois, ser antissocial para realizar o\\
heroísmo sem par de enunciar as verdades}{\textsc{maria lacerda de moura}}

\noindent{}Diversas formas de se explicar o mundo,
ao longo da história, recorreram à analogia com o corpo humano. O mundo
dos homens, para os homens, e como este se organiza, ou é organizado
pelos homens, têm, não só na ideia, a forma de um corpo masculino,
robusto, saudável, idealizado. Nos diferentes campos da sociologia, há a
referência ao totalizante ``corpo social'', que engloba espaços, ruas,
arquiteturas, biomas, práticas, corpos, vidas, viveres, saberes,
comeres, beberes, costumes, percursos históricos, espiritualidades,
materialidades e um sem"-número de atributos humanos e não humanos, mas
sempre a partir do humano, em especial do \emph{humano masculino} (não
por acaso, fala"-se em ``bicho"-homem''). Tal ideia do social como um
corpo, que tem a sua própria saúde, do qual o humano que o produz é
somente célula, saudável ou adoecida, extrapola o campo da metáfora ao
influenciar materialmente no entendimento e nas relações intersubjetivas
em sociedade. Não se trata mais de interpretar o social como um corpo,
mas de entendê"-lo como e a partir de tal.

Quem nos ajuda a entender o funcionamento dessas vidas em célula que
compõem e mantêm a \emph{bio} desse \emph{corpus social} e seu
assujeitamento por intermédio do poder é o transgressor instrumental
analítico de Michel Foucault, para quem, desde o século \textsc{xviii},
assistimos à consolidação/emergência de duas tecnologias de poder: a
técnica \emph{disciplinar} e a técnica \emph{biopolítica}. A técnica
disciplinar ``é centrada no corpo, produz efeitos individualizantes,
manipula o corpo como foco de forças que é preciso tornar úteis e dóceis
ao mesmo tempo'' (\textsc{foucault}, 2010, p. 209). A técnica biopolítica incide
na ``população como um problema político, como um problema a um só tempo
científico e político, como problema biológico e como problema de
poder'' (Idem, p. 214). É pela biopolítica, portanto, que os
corpos"-célula passam a governar"-se a si mesmos e uns aos outros,
reproduzindo as instâncias de governança em seu comportamento e
paradoxalmente as reforçando e as tornando desnecessárias para ditar sua
conduta. A emergência dessa camada biopolítica do poder --- que se
interessa em fazer viver (gerenciar a vida) e deixar morrer (eliminar os
perigosos à saúde da comunidade) --- acontece com a consolidação da
medicina social. Referimo"-nos a um controle da saúde pelo controle da
vida e o agenciamento da morte. A produção de corpos"-vida, que nutrem
uma sociedade vigorosa e viril, e corpos matáveis, biomatéria a ser
absorvida e defecada pelo grande corpo, é onde a saúde e a segurança
pública não se unem, posto que não estão dissociadas em nenhum momento,
mas justamente revelam esses laços íntimos e interdependentes de
funcionamento para a continuidade da cadeia reprodutiva.

Para ditar a conduta dos corpos, evidentemente, é necessária uma conduta
das consciências. É pensando nisso que a \emph{governamentalidade},
outro conceito de Foucault (derivado da noção de governo, usada por ele
desde o final da década de 1970 em substituição ao conceito de poder),
seria a operação de condução das consciências de todos e de cada um e a
cada passo, operada através das tecnologias de poder (disciplinar e
biopolítica), apoiada, em linhas gerais, por três elementos: a condução
de consciências, as relações diplomático"-militares e o Estado Policial
Administrativo. A noção de governamentalidade assenta"-se, assim, sobre a
noção de governo como condução das condutas. Governar não significa dar
ordens diretas, que devem ser obedecidas de forma incondicional; o
governo pressupõe que o sujeito governado seja ``livre'', conduza"-se a
si mesmo, e é essa conduta individual que será coletivamente conduzida
pelo governo.

Na contemporaneidade, tempo que podemos supor como uma fase adulta do
corpo social, e de modo circunscrito ao Brasil, talvez faça mais sentido
tratarmos de uma \emph{governamentalidade democrática}\footnote{A
  perspectiva da governamentalidade democrática tem sido explorada
  conceitualmente por Gallo (2012, 2015, 2017a, 2017b).}: uma tecnologia
de poder centrada na chave das políticas públicas para as populações,
pela incitação de uma atitude cidadã de todos e a todo tempo e pelos
dispositivos de participação obrigatórios nos aparelhos de Estado
(saúde, escola, trabalho, previdência, registros, prisões, etc.) ---
princípios preconizados nos textos, programas, normativas e legislações
brasileiras produzidos após a ditadura civil militar. É dessa forma que
os assujeitamentos se dão, não apenas na instituição escolar, mas
educando e escolarizando os corpos a todo o momento, numa intersecção
entre as instituições. Para essa maquinaria, é sempre necessário o
Estado, que garante e determina a valoração de mercado e a geração de
riquezas, vendendo a ideia de segurança e liberdade, e não opressão e
submissão, como a panaceia que garante o assujeitamento perante a
organização. A partir dessas ferramentas, o corpo social, que é um corpo
velho, mantém"-se saudável, se reforçando a todo momento, embora jamais
se renovando, apresentando outras facetas, mas as mesmas funções e
dispositivos de poder.

Em uma organização do corpo em que a saúde é também aparelho de Estado,
logo ferramenta de governamentalidade democrática, a situação de
pandemia também surge como forma de assujeitamento e docilização dos
corpos. Ao ter toda a sua saúde convertida e convergida em um mecanismo
público ou privado (ambos configurando ainda Estado), a sociedade se vê
entregue aos protocolos impostos pelas figuras de poder e como estas
utilizam suas ferramentas de controle, sendo estes cumpridos ou não.
Para além do sentido puramente médico da palavra, podemos entender o
vírus como transmissor de uma doença, e a pandemia como a dimensão
política que toma a doença ao reordenar o mundo e os costumes, ao
potencializar o domínio e a relação de dependência às ferramentas
democráticas de governamentalidade. A pandemia é a peste, e a peste
organiza a função do Estado de agenciar a morte.

Ao entendermos a peste como mais um eventual mecanismo de poder,
encontramos eco na obra dramática do filósofo franco"-argelino Albert
Camus. Em seu \emph{Estado de Sítio} (2002), vemos alegorias dotadas de
personificação, tais como o Nada, a Morte e a Peste, simbioticamente
representando a pandemia da famigerada ``gripe espanhola'' e a ascensão
do regime fascista sobre a Espanha da década de 1930. Ao dominar o país,
subjugando seu povo, a ditadura da Peste e de sua secretária direta, a
Morte, desenvolve toda uma série de burocracias, documentos, filas,
departamentos, estatísticas, fichários e registros que ordenam as marcas
da doença que cada um há de receber, e o quão depressa sua vida acabará.
A Morte não é um instrumento de poder, mas uma refém escravizada pela
Peste, como se pode perceber no trecho que destacamos a seguir.

\begin{quote}\parindent=0em
\textsc{a peste} --- Em nome do que você discute minhas ordens?

\textsc{a secretária} ---
Em nome da memória. Guardo algumas lembranças. Eu era livre antes de
você e estava à mercê do acaso. Ninguém me detestava. Eu era a que
termina tudo, a que une os amores, a que dá sua forma a todos os
destinos. \textit{Era a equilibrada. Mas você me colocou a serviço da
lógica e do regulamento.} Endureci a mão que, às vezes, socorria
{[}\ldots{}{]}. Com eles, chegava a trabalhar com a aceitação, a meu modo eu
existia. Hoje, cometo violência contra eles e todos me renegam até o
último suspiro. Talvez seja por isso que eu amava este homem que você me
mandou matar. \textit{Ele me escolheu livremente. A seu modo, teve
piedade de mim. Amo aquele que me acolhe} {[}\ldots{}{]}. Quando digo que amo
este, estou dizendo que o invejo. Em nós, os conquistadores, \textit{o
amor tomou uma forma miserável} {[}\ldots{}{]}. Você sabe muito bem que,
forçados a matar, a gente acaba invejando a inocência daqueles que
matamos {[}\ldots{}{]}. A terra é doce para aqueles que amaram muito (idem,
p. 156--162, grifos nossos).
\end{quote}

A Peste, portanto, assim como outros mecanismos, os serviços de saúde,
as políticas de repressão, as práticas de extermínio exercidas pelo
monopólio da violência estatal, torna a imprevisível e muitas vezes
bem"-vinda Morte algo calculado, esperado, temido. A pandemia encaixa"-se
no corpo social de maneira saudável, garantindo o controle biopolítico
de suas células. Sob o regime da Peste, a Morte não é mais parte da
vida. A Peste obriga a Morte a ser feia. E, feia, ela é temida e odiada.
Não se submeter à ditadura da Peste é ousar reivindicar a beleza da
morte, insistindo na afirmação da vida e na insubmissão ao Estado.

Mirando"-nos em experiências e práticas de vida de anarquistas, podemos
encontrar não só diferentes estratégias para essa luta contra a fealdade
da morte e a sua libertação dos aparelhos de Estado, como também um
semelhante e comum espírito ingovernável que percebe a necessidade de
combate à saúde do corpo social em seus múltiplos aspectos. Médico, o
libertário Errico Malatesta enfrentara a epidemia de cólera, em Napoli,
nas últimas décadas do século \textsc{xix} (\textsc{crimethinc}, 2020a), não se aliando a
programas de saúde governamentais, mas organizando com os de baixo,
grupos autônomos, que se protegiam e tratavam. Da mesma forma, ao longo
de sua vida, ele também combateu outras doenças a serviço do Estado,
como o nascente fascismo, não só o institucionalizado, mas os contágios
da doença fascista que habitam as práticas das células do corpo social,
o microfascismo que grassa pelo corpo antes de instalar"-se no Estado.
``Digamos francamente, por mais doloroso que seja constatá"-lo. Fascistas
também existem fora do partido fascista, existem em todas as classes e
em todos os partidos: tem gente de todo o mundo que, não sendo
fascistas, inclusive sendo antifascistas, têm a alma fascista, o mesmo
desejo de abuso que distingue os fascistas {[}\ldots{}{]} \textit{suas
paixões más}. Desde o assassinato brutal a golpes, de incêndios e
devastações, a \textit{pequenas tiranias}, as pequenas vexações que
humilham, os insultos que ofendem o sentido de dignidade humana''
(\textsc{malatesta}, 2020, p. 6--7, grifos nossos).

Pensando as ``pequenas tiranias'' que combateu Malatesta, ou ecoando os
``microfascismos'' desvendados por Deleuze \& Guattari (1980)\footnote{Não
  podemos passar ao largo da bela análise feita pelos filósofos
  franceses em \emph{Mille Plateaux}, na qual mostram que o fascismo só
  pode instalar"-se no Estado por estar entranhado na micropolítica. Ecos
  de Malatesta\ldots{} São os microfacismos que alimentam o fascismo de
  Estado. Seu mecanismo de ação: fazer com que o desejo deseje sua
  própria repressão; aí reside a força do fascismo, sua atuação no nível
  desejante.}, entendemos que o combate à pandemia deve existir como uma
só luta também contra o fascismo. Devemos entender a dimensão social do
vírus e a medula do fascismo como ferramentas do Estado, senão corremos
o risco de uma captura pelos instrumentos de poder e submetermo"-nos à
feiura da morte. De volta a Camus, é o humano, vivo, sujeito e dotado de
nome que consegue enfrentar as alegorias e promover uma libertação.
Insubmisso e destemido, o apaixonado Diego rebela"-se e decide confrontar
a Morte de frente, escolhendo a morte, acolhendo a ela.

\begin{quote}\parindent=0em
\textsc{diego} ---
Vamos, \textit{acabe} logo! Acabe com esta comédia suja! O que está
esperando? Faça seu trabalho e não se divirta comigo, que sou maior do
que você {[}\ldots{}{]}. Você só leva em conta os conjuntos! Cem mil homens,
assim a coisa fica interessante. É uma estatística, e as estatísticas
são mudas! Com elas se traçam curvas e gráficos {[}\ldots{}{]}! Mas eu a
previno: um homem sozinho incomoda muito mais --- grita sua alegria ou
sua agonia. \textit{Enquanto viver, eu continuarei a bagunçar sua
ordem}. {[}\ldots{}{]}! Pertenço a uma raça que honrava tanto a morte como a
vida {[}\ldots{}{]}. \textit{Eu já entendi seu sistema. Vocês lhes deram a
dor da fome e das separações para distraí"-los da sua revolta. Vocês os
exaurem, devoram seu tempo e suas forças para que não tenham lazer nem
ânimo!} Eles se arrastam, fiquem contentes! Estão sozinhos, mesmo sendo
uma massa; eu também estou sozinho. \textit{Cada um de nós está sozinho
graças à covardia dos outros} {[}\ldots{}{]}. Vocês acreditaram que podiam
reduzir tudo a cifras e fórmulas! Mas, na sua bela nomenclatura,
esqueceram a rosa selvagem, os sinais no céu, os rostos do verão, a
grande voz do mar, o instante do dilaceramento e a cólera dos homens
{[}\ldots{}{]}!

\textsc{a secretária} ---
Vou lhe contar um pequeno segredo\ldots{} o sistema deles é excelente, você
tem razão, mas há uma falha na máquina {[}\ldots{}{]}. Pelo que eu saiba,
\textit{sempre bastou que um homem vença seu medo e se revolte para que
a máquina comece a falhar.} Não digo que ela pare, longe disso. Mas,
enfim, ela falha e, às vezes, degringola completamente {[}\ldots{}{]}.
\textit{Ainda está com medo?}

\textsc{diego} ---
\textit{Não}.

\textsc{a secretária} ---
\textit{Então, não posso fazer mais nada contra você}. Isto também está
no regulamento. Mas posso lhe dizer: é a primeira vez que este
regulamento tem a minha aprovação (\textsc{camus}, 2002, p. 127--131, grifos
nossos).
\end{quote}

Ao tomar"-se a Morte em nossa posse (não propriedade!) toma"-se a vida tal
e qual. O ato de coragem antissocial que atenta ao conformismo é capaz
de encontrar e de produzir falhas nos mecanismos de Estado que agenciam
e escravizam a morte. Acolher a Morte recusando a mortalidade e o
morticínio, resgatando"-a das pequenas e grandes tiranias, dos molares e
moleculares fascismos.

Sob regime democrático, se ainda se pode estranhar tratar de fascismos,
cabe reforçar que a Constituição Brasileira de 1988 e o \textsc{sus} são códigos
do aparelho de Estado e, enquanto tais, sempre que os torcermos sairá
sangue. Códigos e programas e políticas operam como articuladores de
megamáquinas sociais que submetem e fazem participar populações e
subjetividades maquínicas singulares. Entre as máquinas de captura pela
renda da terra, lucro da empresa e impostos do Estado, esses programas
(\textsc{sus}, \textsc{eca}, etc.) são os juízes macabros de uma luta assimétrica de
dominação. Operando por seletividade, esses programas fazem viver,
gerenciam as relações do viver e fazem morrer, a necessária estatística
do extermínio para garantir a ordem.

É preciso atentar contra o aparelhamento do corpo por máquinas de
captura estatais que sobrecodificam processos e pulsões do que é vivo.
Todo o corpo é um delírio e um agenciamento em variação contínua entre
conexões e cortes com presenças humanas e não humanas. Uma aglutinação
sempre em movimento e sem muito controle de coisas e pessoas e cheiros e
lances e pedradas e paredes e ruas e palavras e rolês em volta de nós
--- em nosso mundo.

O corpo do sujeito, e não o social, é talvez o encontro com o que está
fora, com a outra pessoa/território, com o que é intempestivo. O
agenciamento, a colaboração, a solidariedade entre coisas animadas e
inanimadas --- transversalmente interdependência interespécies, como vem
falando Ana Tsing (2019).

Sem dúvida, também, faz"-se urgente repensar nossa submissão aos ditames
da saúde fármaco"-política dos aparelhos de Estado e inaugurar no
presente práticas de apoio mútuo para o fortalecimento da nossa pulsão
de vida (ingovernável) e ampliar os encontros (possíveis) para práticas
solidárias de saúde. A saúde do corpo da gente está colocada
diametralmente oposta à saúde do velho corpo social, que devora órgãos,
sangue e saúde das pessoas, tal qual um bilionário de vida longeva
ordena o agenciamento de sua própria saúde, literalmente mutilando
outros a serviço de si e praticando uma antivida.

Se está a serviço da saúde do corpo social, o vírus, em sua dimensão
pandêmica, é paradoxalmente um anticorpo, contra tudo aquilo que atenta
à estabilidade e à ordem; ao governo e ao Estado. Um exemplo não
distante são as insurgências populares de enorme adesão e insubmissão
que tomaram o Chile, desde 2019, imparáveis pelo aparato de repressão
policial"-militar, finalmente esvaziadas pelos protocolos de saúde e
cuidado das células. Se é a insubmissão que atenta contra a saúde
social, contra o anticorpo pandêmico podemos identificar a anarquia, em
suas múltiplas formas, como o vírus a favor da vida e da beleza em morte
como parte da vida, lutando para infectar as partes saudáveis de uma
sociedade que não querem manter. O
vírus libertário busca combater o vírus literal e autoritário. Adoecer o
Estado é necessário para fazer fenecer a Peste, esteja ele manifestado
como instituições públicas ou disfarçado de livre"-mercado e propriedade
privada. Uma guerra virológica, vírus contra vírus\ldots{}

Referenciar a anarquia como uma doença social não é de todo uma
novidade. Se em algum momento essa figura foi usada para pejorar os
insurrectos, a alcunha foi abraçada e assumida por aqueles que, em sua
linguagem, produção de significado e leitura de mundo, respondem à ordem
e ao progresso: ``sim, eu quero o seu fim''. ``Para um anarquista a
linguagem pode ser um vírus estancando os modelos, suprimindo os
intelectuais"-profetas, arruinando as palavras de ordem, desmontando
histórias idealizadas de um passado remoto e sem se apartar, na
atualidade, de uma luta urgente da qual não se pode nem deve se
esquivar'' (\textsc{passetti} \& \textsc{augusto}, 2008, p. 57).\footnote{No romance
  ciberpunk \emph{Snow Crash} (traduzido no Brasil como \emph{Nevasca}),
  Neal Stephenson (2008) fala de um vírus muito antigo, que domina os
  humanos e que pode ser transmitido tanto pela linguagem e pelos
  códigos quanto pela troca de fluidos corporais. As religiões são
  formas de disseminar o vírus pela linguagem, controlando nossos corpos
  e nossas mentes. Uma chave interessante para a leitura do mundo
  pandemizado contemporâneo\ldots{}}

Se, no contexto de governamentalidade democrática, o Estado prefere
optar pela promessa de uma ideia de liberdade ao invés da repressão em
sua faceta mais crua, é de se esperar as soluções frias que os governos
tendem a oferecer para o problema do qual estes são parte e agente de
maquinações. A defesa da ordem e do bem comum é explícita como
ressonante de práticas cidadãs a serem perseguidas, bem como a família,
a solidariedade e a tolerância. No princípio básico do apoio mútuo,
comum entre os anarquismos, sabemos que a solidariedade proclamada pelo
Estado e pelo capital não é aquela entre semelhantes, que levará a
relações emancipatórias coletivas na concretude do real. É o canto da
sereia que nos chama aos rochedos, uma armadilha semântica que sequestra
e deforma pautas históricas, fazendo"-as lucrativas. A solidariedade do
Estado manifesta"-se na caridade, que é um gesto de ajuda a outra pessoa
que faz garantir somente a salvação do caridoso, é a esmola ou o dízimo.
A caridade não mascara e não esconde sua raiz cristã, a religião que
melhor soube apoderar"-se do Estado, tornando"-se seu aparelho.\footnote{Importante
  sublinhar que o cristianismo se tornou a religião oficial do Império
  Romano e, depois de sua queda, embrenhou"-se nos Estados monárquicos
  europeus, mantendo a aliança da Igreja Católica com os governos,
  situação seguida depois pelas Igrejas Protestantes, que também
  souberam bem aliar"-se aos Estados. Sabemos que a modernidade consagra
  a propagada ``separação'' entre Estado e Igreja. Mas a análise
  perspicaz de Foucault mostra que os Estados modernos
  governamentalizados guardam os moldes e as maquinarias do poder
  pastoral, tecnologia de poder forjada nas comunidades cristãs. A
  Igreja pôde sim separar"-se do Estado, pois a estrutura deste já estava
  totalmente tomada por aquela. Ver o curso \emph{Segurança, Território,
  População} (\textsc{foucault}, 2008), para entender em minúcias esse processo.}

O conceito de cidadania, portanto, produz um corpo (individual e
populacional) que vive a serviço do funcionamento saudável da máquina
pública. Um corpo obediente e dócil. A obrigação da participação inibe
diversas potencialidades de produção de subjetividades ingovernáveis e,
desse modo, não pode haver uma solidariedade libertária e anarquista por
meio da cidadania, a conclamação à participação e integração das ações
de Estado. Seja o voluntarismo e o empreendedorismo, seja em cada um
cumprir seu isolamento em quarentena, tal e qual mandam a cartilha e o
protocolo.

É preciso recusar as soluções frias do neoliberalismo (mesmo quando
mascaradas pela caridade democrática --- outra adversária histórica
acusada pelos libertários) que atropelam inúmeras vidas e seus contextos
sociais, operando um massacre de Estado. Massacre de vidas, tanto no
sentido biológico, quando milhares estão sendo condenados a morrer nas
malhas sucateadas dos sistemas de saúde, como no sentido
sociopolítico"-subjetivo, destruindo o acesso a formas de sociabilidade
coletivas e a afirmação de vidas insubmissas. Faz"-se urgente inventar
formas de experiência do vivo que explodam as gaiolas do Estado, do
capital, da biopolítica, da disciplina e do controle.

Constranger o fascismo como prática de Estado, inibir a pequena tirania
da saúde enquanto protocolo de controle biopolítico, e não se deixar
capturar por fórmulas prontas de ação e organização é o que levou e leva
anarquistas históricos ou histórico"-contemporâneos a desenvolverem seus
grupos de afinidade (\textsc{crimethinc}, 2020b), suas práticas de educação
(i)moral dentro e fora dos espaços escolares e proporem a emergência de
sujeitos renovados. Para Malatesta, a doença social do fascismo vence
porque ``não despertou a desaprovação, a indignação geral, o horror
moral que nos parece que deveria nascer espontaneamente em cada alma
gentil. E lamentavelmente não poderão se impor estas materialmente se
antes não houver uma revolta moral'' (\textsc{malatesta}, 2020, p. 6). A pandemia
não se envergonhará de obrigar a morte a ser inimiga da vida sem uma
prática cotidiana de educação e formação de si e do meio que seja oposta
ao corpo masculino, fálico, retilíneo e patriarcal da sociedade do
Estado. ``A educação dos anarquistas não caminha em linha reta; provoca
a descoberta de outros percursos, atiça coexistências, inova, gera
outros fluxos e outras possibilidades, que levam ao combate direto na
fronteira entre a derradeira reforma da sociedade e a \emph{morte da
sociedade}'' (\textsc{passetti} \& \textsc{augusto}, 2008, p. 92).

Diante de um corpo social velho, decrépito e, infelizmente, saudável,
que se reforça e reformula, o enfrentamento devido talvez seja a
constante renovação, reinvenção anárquica de si e do espaço,
espalhamento virótico de faíscas libertárias que acendam o fogo
ingovernável.

É justamente por temer a renovação, as novas formas de sociedade, as
formas antissociais nas quais os sujeitos realizam o anti"-heroísmo de
enunciar verdades, plurais e renováveis, que o Estado lança seu cajado
para uma educação retilínea, liberticida, sobretudo sobre os corpos
infantes, impondo a infância moral às crianças muito além do espaço
escolar. Corpos vividos, assujeitados como células saudáveis, tendem aos
vícios do funcionamento do todo, conseguindo muito dificilmente se
renovar, mesmo quando se acreditam insubmissos, ainda quando se
proclamam antifascistas. A infância só se agrupa por afinidade, ignora a
democracia e sua governamentalidade e incorpora a renovação em sua
própria acepção. Reforçado pela Peste, o Estado (\emph{de Sítio}
permanente) controla creches e praças, escolas e lares, subjetividades e
afetos, conduzindo coercitivamente olhares e produções de verdades.

O vírus está a serviço da ordem do capitalismo, que toma forma no corpo
homem, branco, hétero, \emph{cis}, capacitista e adulto. É preciso
reforçar que a pandemia nos expõe desigualdades já muito debatidas (mas
não superadas), entre elas, a opressão de idade (\textsc{santiago} \& \textsc{faria},
2016), ainda pouco debatida nos espaços anarquistas. Importante frisar
que as crianças também sofrem dinâmicas de opressão, em maior ou igual
medida que os adultos, portanto, falar de \emph{infância} significa
falar de \emph{infâncias} (\textsc{cohn}, 2014), uma vez que há modos de vida
distintos entre as crianças, na forma de opressões por raça, gênero,
classe.

Philippe Ariès discutiu na sua \emph{História social da criança e da
família} (1986) que as nossas concepções de infância e de família
nuclear são construções sociais relativamente recentes na história. O
anarquista Giovanni Rossi, nas experimentações libertárias na Colônia
Cecília, também defendia um argumento similar, mas revolucionário: era
necessário superar a família nuclear, pois ali residia um sustentáculo
do regime capitalista (\textsc{rossi} apud \textsc{peruzzo}, 2007, p. 8). No cenário de
superação da família nuclear, as crianças, então, sairiam da tutela
parental exclusiva dos genitores e integrariam a comunidade.

Durante a pandemia, o vírus trabalha fortemente a serviço da ordem
estrutural da família nuclear: as crianças são forçadas a conviver
unicamente com os pais, tendo seu corpo retirado da cidade, dos espaços
públicos. Aqui atentamos para o fato de que há diversas configurações
familiares, mães \emph{solo}, famílias com muitos membros, dinâmicas que
não podem ser invisibilizadas. Se as crianças não podem sair do convívio
pandêmico exclusivo com suas famílias, quão limitados podem ser a
experiência e o repertório infantil?

Esse contexto também nos leva ao questionamento: que lugares são
destinados para as crianças? O que há entre instituição familiar e
instituição escolar? Qual nosso posicionamento anarquista para a
superação da invisibilidade das crianças durante uma pandemia que nos
força ao confinamento?

O corpo infantil é instrumento de resistência e transgressão (\textsc{arenhart},
2015, p. 96). É o corpo inquieto, não submisso e não obediente, que está
no plano da curiosidade e das descobertas, do olhar atento que escapa do
modo institucionalizado de socialização. Agostinho (2018) defende o
argumento de que as crianças participam de corpo inteiro, e aqui
entendemos a infância a partir da sua \emph{inteireza} e
\emph{completude}, o que é refletido em seus corpos, contrariando o
argumento de que ``seu corpo está inacabado no nascimento e só será
concluído por meio de ação na sociedade'' (idem, p. 350). Essa
\emph{ação na sociedade} significa, também, a submissão infantil à
família nuclear.

Os anarquistas vêm se dedicando, entre outras frentes de militância, a
combater a hegemonia das instituições. Devemos nos questionar: como
combater a hegemonia institucional que permeia a infância? Nesse momento
que nos encontramos diante de limitações de contato físico, que redes de
apoio são pensadas para as crianças? O que reservamos para as crianças e
sua \emph{inteireza,} além de contato remoto e de tentativas de
submissão dos seus corpos inquietos ante as telas de computadores? Não
poderemos, enquanto anarquistas, defender uma infância remota. Se a
ordem é hegemonia institucional entre família e escola, emerge a
necessidade de pensarmos outro projeto de infância, para infância, não
\emph{sobre} crianças, mas \emph{com crianças}.

Talvez encontremos um caminho com o anarquista francês René Schérer, que
defende a ideia de um \emph{co"-ire} (\textsc{schérer} \& \textsc{hocquenghem},
1976), que pode ser entendido como \emph{ir junto}, caminhar ao lado.
Como nos colocamos ao lado das crianças quando as confinamos em
instituições? Nesse espaço, parece não haver horizontalidade possível,
apenas acima e abaixo\ldots{} Schérer, no seu \emph{Petit Alphabet
Impertinent} (2014), nos fala sobre a \textit{rua}\footnote{Sobre a
  importância da rua para uma concepção libertária de infância, ver:
  Gallo (2018).}. Nas ruas, as crianças formam seus bandos, em exercício
de autonomia. Adultos, podemos fazer agenciamentos com tais bandos,
caminhando juntos, sem imposições de uns sobre outros, construindo uma
vida coletiva libertária. A rua, esse espaço público que foi tirado das
crianças, a serviço do vírus, a serviço da ordem. Para \emph{caminhar
junto} das crianças, precisamos voltar à rua. Ocupar é a ordem.

\begin{bibliohedra}
\tit{AGOSTINHO}, Kátia. As crianças participam de corpo inteiro. In:
\emph{Conjectura: Filosofia e Educação.} Caxias do Sul, maio/ago. 2018,
vol. 23, n. 2, p. 347--362. Disponível em:
\emph{http://www.ucs.br/etc/revistas/index.php/conjectura/article/view/4384/pdf}.
Acesso em 04/10/2020.

\tit{ARENHART}, Deise. O corpo como expressão de culturas infantis na escola:
marcas de geração e classe social. In: \emph{Cadernos de Formação \textsc{rbce}}.
Rio de Janeiro, março 2015, vol. 6, n. 1, p. 91--102. Disponível em:
\emph{http://revista.cbce.org.br/index.php/cadernos/article/view/2085}.
Acesso em: 04/10/2020.

\tit{ARIÈS}, Philippe. \emph{História Social da Criança e da Família}.
Tradução de Dora Flaksman. Rio de Janeiro: Guanabara, 1986.

\tit{CAMUS}, Albert. \emph{Estado de Sítio}. Tradução de Alcione Araújo. Rio
de Janeiro: Civilização Brasileira, 2002.

\tit{COHN}, Clarice. Concepções de infância e infâncias: um estado da arte da
antropologia da criança no Brasil. In: \emph{Civitas --- Revista De
Ciências Sociais}, 2014, 13(2), p. 221--244. Disponível em:
\emph{https://revistaseletronicas.pucrs.br/ojs/index.php/civitas/article/view/15478}.
Acesso em: 04/10/2020.

\tit{CRIMETHINC}. \emph{Anarquistas contra a peste: Malatesta e a epidemia de
cólera de 1884}. 15 de agosto de 2020a. Disponível em:
\emph{https://pt.crimethinc.com/2020/07/15/anarquistas-contra-a-peste-malatesta-e-a-epidemia-de-colera-de-1884-1}.
Acesso em: 04/10/2020.

\titidem. \emph{Sobrevivendo ao vírus: um guia anarquista: Capitalismo
--- Totalitarismo crescente --- Estratégias de resistência}. 20 de março
de 2020b. Disponível em
\emph{https://pt.crimethinc.com/2020/03/20/sobrevivendo-ao-virus-um-guia-anarquista-capitalismo-em-crise-totalitarismo-crescente-estrategias-de-resistencia}.
Acesso em: 04/10/2020.

\tit{DELEUZE}, Gilles; \textsc{guattari}, Félix. \emph{Mille Plateaux}. Paris: Les
Éditions du Minuit, 1980.

\tit{FOUCAULT}, Michel. \emph{Em defesa da Sociedade}. Tradução de Maria
Ermantina Galvão. São Paulo: Martins Fontes, 2010.

\titidem. \emph{Segurança, Território, População}. Tradução de Eduardo
Brandão. São Paulo: Martins Fontes, 2008.

\tit{GALLO}, Sílvio. Da escola à rua: passos para uma filosofia da infância.
In: \textsc{rodrigues}, A. (org.). \emph{Crianças em dissidências --- narrativas
desobedientes da infância}. Salvador: Devires, 2018, p. 201--216.

\titidem. Políticas da diferença e políticas públicas em educação no
Brasil. In: \emph{Educação e Filosofia,} Uberlândia, set./dez. 2017a,
vol. 31, n. 63, p. 1497--1523.

\titidem. Biopolítica e subjetividade: resistência? In: \emph{Educar em
Revista}. Curitiba, Brasil, out./dez. 2017b, vol. 33, n. 66, p. 77--94.

\titidem. Governamentalidade democrática e ensino de filosofia no Brasil
contemporâneo. In: \emph{Cadernos de Pesquisa.} São Paulo: Fundação
Carlos Chagas, 2012, vol. 42, p. 48--64.

\titidem. ``O pequeno cidadão'': sobre a condução da infância em uma
governamentalidade democrática. In: \textsc{resende}, H. (org.). \emph{Michel
Foucault -- O Governo da Infância}. Belo Horizonte: Autêntica, 2015, p.
329--343.

\tit{MALATESTA}, Errico. Porque o fascismo vence. In: \textsc{malatesta}, Errico et al.
\emph{Os Grandes Escritos Antifascistas (vol. I)}. Tradução de Diana A.
Shravya. Editora Terra Sem Amos, 2020.

\tit{PASSETTI}, Edson; \textsc{augusto}, Acácio.
\emph{Anarquismos e Educação}. Belo Horizonte: Autêntica Editora, 2008.

\tit{PERUZZO}, Gabriel. A família nuclear sob as lentes libertárias de
Giovanni Rossi. In: \emph{Tempos Acadêmicos: Revista do Curso de
História}. Criciúma, 2007, vol. 5. Disponível em:
\emph{http://periodicos.unesc.net/historia/article/view/422}. Acesso em:
04/10/2020.

\tit{SANTIAGO}, Flávio; \textsc{de faria}, Ana Lúcia Goulart. Para além do
adultocentrismo: uma outra formação docente descolonizadora é preciso.
In: \emph{Educação e Fronteiras}, Dourados, maio 2016, vol. 5, n. 13, p.
72--85. Disponível em:
\emph{https://ojs.ufgd.edu.br/index.php/educacao/article/view/5184}.
Acesso em: 04/10/2020.

\tit{SCHÉRER}, René. \emph{Petit alphabet impertinant}. Paris: Hermann, 2014.

\tit{SCHÉRER}, René; \textsc{hocquenghem}, Guy. «Co"-Ire»: album systématique de
l'enfance. In: \emph{Recherches}. Paris, 1976, nº 22.

\tit{STEPHENSON}, Neal. \emph{Nevasca}. Tradução de Fábio Fernandes. São
Paulo: Aleph, 2008.

\tit{TSING}, Anna. \emph{Viver nas ruínas: paisagens multiespécies no
Antropoceno}. Tradução de Thiago Mota Cardoso e outros. Brasília: \textsc{ieb}
Mil Folhas, 2019.
\end{bibliohedra}

\chapterspecial{Infância e pandemia}{A anárquica imunologia infantil}{Marco Antônio Arantes}

\epigraph{Eu prometo nunca obedecer somente porque alguém\\
maior que eu em um terno está mandando}{\textsc{john seven e jana christy}, 2012}

\epigraph{Que riqueza não possui a criança no seu sorriso, nas suas
brincadeiras, na sua gritaria, em suma, na sua simples existência!}{\textsc{max stirner}, 1845}

\noindent{}Em maio de 2020, o escritor punk mexicano, Wenceslao Bruciaga fez a
seguinte pergunta no texto ``Días de pandemia: príncipes y niños
anarquistas'': não estariam as crianças zombando da tirania dos adultos
e das estruturas autoritárias da família com o seu sistema imunológico?
Bruciaga fala de uma anarquia infantil e imunológica ao vírus da
\textsc{covid}"-19, como se fosse uma resposta infantil às atitudes autoritárias e
insolentes dos adultos sobre elas. ``Meninos e meninas com sintomas
leves ou assintomáticos enviando a tirania dos adultos para a cama''
(\textsc{bruciaga}, 06/05/2020).

Desde o início da pandemia ficou perceptível o resgate de modelos de
intervenção e dispositivos de poder, que por aqui foram muito familiares
no início do século \textsc{xx}, caracterizando a medicina e o sanitarismo como
saberes institucionais da sociedade, aos poucos consolidados por suas
formas de atuação sob a tutela do Estado. Nunca é demais lembrar que,
desde a administração colonial, a classe médica estava encarregada de
fiscalizar portos, locais que eram vistos como vetores de
pestes.\footnote{Segundo Machado et al., em 1682, a Câmara de Salvador
  nomeou ``oficiais de saúde para fiscalizar as embarcações visando
  evitar a entrada de pestes'' (\textsc{machado} et al., 1978, p. 47).}

O recrudescimento das visões médicas sobre a pandemia conduziu a uma
série de pesquisas sobre os seus impactos na vida das crianças. Em sua
grande maioria, estão centradas nos efeitos do confinamento prolongado,
mas, por ora, muito inconclusivos e evasivos sobre as sequelas
psicológicas nas próximas décadas.

O estudo de Wang, Zhang, Zhao, Jiang (2020) aponta que o confinamento da
criança em duração prolongada poderia produzir impactos psicológicos,
estresse, irritabilidade, pesadelos, medo de infecção, angústia
psicológica, frustração, apatia, dificuldade de concentração e tédio.
Acrescem"-se, a isso, a falta de contato com os amigos no dia a dia, o
uso intensivo da internet,\footnote{Deslandes e Coutinho falam de um
  período de hiperconectividade no qual o ``isolamento social adotado
  para o enfrentamento da Pandemia do \textsc{covid}"-19 intensificou alguns
  elementos ligados à sociabilidade digital (hiperexposição, diluição de
  fronteiras público"-privadas"-íntimas, espetacularização de si) que
  criam condições para o acirramento da violência digital'' (\textsc{deslandes}
  \& \textsc{coutinho}, 2020, p. 2480).} a falta de lazer em espaços públicos e,
não menos comum, a falta de estrutura e espaço nas residências, além da
redução da renda de seus pais, quando não, a perda de emprego. No caso
do desemprego dos pais, o luto pela perda de uma estabilidade financeira
impactaria diretamente no planejamento econômico das famílias,
obrigando"-as a cortar contas e até mesmo o cancelamento das matrículas
dos filhos das escolas particulares, além do consequente afastamento de
seus colegas da turma.\footnote{Embora as primeiras notícias relatassem
  que as crianças eram consideradas menos ``vulneráveis'' a desenvolver
  os efeitos mais graves da \textsc{covid}"-19, elas estariam também propensas a
  se contaminar e transmitir o vírus para os mais velhos.}

De fato, a pandemia trouxe novos e grandes desafios para as famílias,
tais como: convivência próxima por longos períodos de tempo; ausência
física em casas de parentes, escolas, creches, parques públicos,
\emph{shoppings}, ginásios, campos de futebol, etc. Foi também preciso
um rearranjo do ambiente físico para acomodar as demandas de trabalho,
estudo e lazer doméstico. Ademais, houve uma sobrecarga de trabalho
doméstico; instabilidade no emprego, desemprego e problemas financeiros;
falta ou irregularidade do suporte regular dos serviços de saúde e
assistência social e comunitária à família, separação e morte de
familiares, entre outros.

Outros estudos falavam em transtornos psicológicos que oscilavam entre
confusão mental, pânico e raiva. Haveria, também, outros fatores de
riscos que, durante o confinamento, comprometeriam a saúde mental das
crianças, reforçando a ideia de que o ambiente é fundamental para o
bem"-estar delas, tais como a ``falta de estimulação adequada no nível de
desenvolvimento das crianças; violência, maus"-tratos, negligência e
conflitos, práticas parentais com disciplina abusiva e coercitiva,
desnutrição, baixa escolaridade, desemprego e instabilidade financeira,
alta densidade habitacional no lar, problemas de saúde mental dos pais,
entre outros'' (\textsc{linhares} \& \textsc{enumo}, 2020, p. 4).

Em uma reportagem no jornal \emph{El Pais}\footnote{Na pesquisa,
  constatou"-se que ``os hábitos também mudaram: 25\% das crianças
  passaram a comer mais do que o habitual, 73\% usaram dispositivos
  eletrônicos mais de 90 minutos por dia, em comparação com 15\% que
  faziam isso antes da quarentena, e apenas 14\% praticavam 60 minutos
  diários de atividade física, que é o recomendável segundo a
  Organização Mundial da Saúde'' (\textsc{portinari}, 06/06/2020).}, projeta"-se
uma quarta pandemia sanitária: a dos transtornos mentais em crianças e
adultos que não suportaram os efeitos do confinamento. O vírus também
mataria crianças de forma indireta, colocando em risco mães grávidas com
restrições de movimento para exames pré e pós"-natais. Contudo,
conclui"-se que as sequelas do confinamento não poderiam ser avaliadas em
um curto espaço de tempo.

A pandemia afetou o cotidiano de milhares de crianças numa escala
global. Se, antes, determinado perfil de criança estava acostumado a uma
vida movimentada com os seus pais e amigos da escola, à mobilidade em
espaços públicos, aos passeios com os pais, às viagens de férias, em um
sobressalto, elas passaram a considerar as suas casas, quintais, seus
quartos e brinquedos como referências física e social de suas vidas. O
processo de aprendizagem, antes socializado numa sala de aula, teria
agora que ser adaptado a um ensino a distância, com o uso diário dos
recursos tecnológicos da conectividade.

Mas há um outro aspecto que existia na pré"-pandemia: a desigualdade
entre as crianças. Durante uma pandemia, haveria situações que seriam
enfrentadas com mais facilidade por crianças de origem privilegiada,
revelando notáveis diferenças em relação à classe social. Trata"-se de
uma invisibilidade e de uma pandemia que não são viral e biológica, mas
econômica e social. Tais crianças ainda sofrem violências de todos os
tipos. Também são pobres e vivem em ambientes tensos. As suas casas são
apertadas, insalubres e sem privacidade. Elas não possuem quartos
privados e dividem os poucos cômodos com membros adultos da família em
ambientes apertados. Também não estudam em boas escolas e não têm acesso
à internet. Nunca viajaram para o exterior, e os seus pais desconhecem
outras línguas. Nunca compraram computadores e celulares avançados e não
possuem livros e internet em suas casas. Para agravar, seus pais não
conseguem ajudá"-los nas tarefas da escola, porque não possuem noções
básicas de matemática, ciência e português. Illich já comentava, em sua
obra \emph{Sociedade sem Escolas} (1973), que a classe social atravessa
a questão educacional. É fundamental pensar as questões de Illich em
tempos de pandemia. Para o autor, a aprendizagem escolar de uma criança
pobre não se nivela a de uma criança rica. Mas ele observava esse
problema noutro contexto, em que o estudante pobre está em desvantagem
quando depende da escola para o seu processo de aprendizagem, ao
contrário do estudante rico. ``essas vantagens vão desde a conversação e
livros {[}\ldots{}{]} e uma diferente idiossincrasia'' (\textsc{illich}, 1973, p. 29).

Na maioria das vezes, tais crianças vivem em situações de extrema
pobreza e são identificadas com ``vulnerabilidade'' física. Muitos de
seus amigos vivem os mesmos dilemas. Suas casas estão em locais sujos,
sem estrutura, sem água, sem esgoto, e vivenciam diariamente situações
em que o sim e o não são as suas únicas escolhas. No mais, são acuadas
pela violência e dificuldades financeiras dos pais, além da pecha de
``crianças marcadas'', desestruturadas em descompasso com o modelo
``ideal'' familiar.\footnote{Segundo Stirner, o amor familiar muitas
  vezes é calcado por representações de uma família ideal, a ponto de a
  família ser ``um conceito sagrado que o indivíduo isolado não pode
  ofender'' (\textsc{stirner}, 2004, p. 75).} Tais dificuldades incidem para o
trabalho infantil, aumentando a sua invisibilidade na sociedade. Para
essas crianças, a pandemia só veio a agravar a restrição alimentar, a
violência doméstica e a fragilidade física.

É notória a histórica relação dos anarquistas com questões ligadas à
saúde das crianças. Francisco Ferrer y Guardia, que incluiria o
relatório de seu amigo Domela Niewenhuis em sua obra \emph{A Escola
Moderna} de 1909, afirmava que ``nunca será feito o bastante em prol das
crianças. Quem não se interessa pelas crianças não é digno de que
ninguém por ele se interesse, porque as crianças são o futuro'' (\textsc{ferrer
y guardia}, 2014, p. 66). Uma das preocupações de Ferrer era com a
salubridade dos edifícios e com a instrução sanitária das crianças, de
forma que as habituará ``às práticas higiênicas, lavação das mãos, boca,
banhos, natação, limpeza das unhas, etc.'' (idem, p. 59). Ele sublinhava
a importância de ambientes limpos na formação das crianças. A sua
pedagogia, fortemente influenciada pelas ciências naturais, valorizava a
higiene e repugnava objetos, pessoas e animais sujos e causadoras de
doenças. A sujeira tem o poder ``de infeção indefinida até causar
epidemias'' (ibidem, p. 55).

Se as crianças são o futuro, devem ser educadas em ambientes saudáveis
como espíritos livres, pois elas são possuidoras de inteligências
nascentes.\footnote{O relatório de Domela Niewenhuis é incisivo e
  irônico sobre o papel dos pais na educação. ``Na educação das
  crianças, a coisa mais difícil do mundo, quase todo mundo acha que se
  tem competência para ela pelo fato de ser pai de família'' (\textsc{ferrer y
    guardia}, 2014, p. 67).} Das crianças, emergirá um novo mundo, um mundo
com menos doenças e mais limpo. Daí a importância da higiene em sua
formação, pois elas não nascem com ideias preconcebidas sobre doenças e
sujeiras. Elas brincam e se sujam, mas não entendem por si sós o que
está por trás da sujeira. As suas melhores roupas podem não significar
nada numa brincadeira que envolve sujeira. No entanto, elas têm a
compulsão pelo conhecimento, um impulso para novas descobertas, um
impulso maior do que a confirmação das coisas conhecidas. Um impulso
além da limpeza.

Nietzsche, em \emph{A genealogia da moral} (1887), dizia que as crianças
são espíritos livres que brincam com os mundos. São seres marcados pela
singularidade. São criadoras e desconcertantes. Nenhuma criança nasce e
brinca para desejar a morte e a doença. Querem a vida. Elas são
indiferentes à dor da morte cristã. Crianças dão gargalhadas, mas não
entendem por que o ``onipotente Deus'' é tão sério e despreza o riso.
Nietzsche sabe o quanto esse Deus cristão atemoriza as crianças. Deus é
proibição, angústia, silêncio e reprovações. Ele lança suas pragas, mas
as crianças querem brotar novos e livres mundos. São vontades livres que
nas manhãs afastam ``da fronte os sonhos terríveis'' (\textsc{nietzsche}, 1998,
p. 166). Na sua imaginação, um vírus não é pálio para sua imaginação
onírica. Ele morreria com uma espada colorida e algumas cambalhotas. Ele
se desmantelaria como um brinquedo, mas o vírus não será um mal
combatido com quarentenas e vacinas, até que alguém lhe diga como
combatê"-lo. ``A criança é inocência, e o esquecimento, um novo começar,
um brinquedo, uma roda que gira sobre si, um movimento, uma santa
afirmação'' (\textsc{nietzsche}, 2002, p. 37). A sua visão ilógica infantil da
vida cai"-lhe muito bem. Elas ignoram os valores morais dos adultos. Suas
verdades lhe são indiferentes. A criança não se desespera por uma
verdade que nunca existiu. Não lhe interessa o poder. Não quer prestígio
nem autoridade. Crianças não são rebanhos\footnote{Nietzsche mostra o
  quanto são nefastas relações construídas sob a égide do autoritarismo.
  ``As dissonâncias não resolvidas na relação entre o caráter e a
  atitude dos pais ressoam na natureza da criança e constituem a
  história íntima de seus sofrimentos'' (\textsc{nietzsche}, 2000, p. 132).}. Nem
a mentira lhe é mortal. Se ela mente, ``uma aversão à mentira lhe é
estranha e inacessível, e ela mente com toda a inocência'' (idem, p.
35).

A pandemia nos faz repensar sobre antigas práticas corriqueiras da
criança no ambiente familiar. O vírus não é um instrumento de controle
nas mãos dos pais. Até mesmo as crianças sabem o que as sufoca, que não
as deixa correr, cantar, brincar, gritar, pular e dormir em paz. Mas
isso não se chama vírus. Elas continuam altivas e alegres, mas o espaço
não é mais o mesmo no confinamento. Também nunca estiveram tanto tempo
próximos dos pais e longe dos amigos. Muitas coisas lhes vêm na cabeça:
medos e ansiedades. A pandemia também coloca em xeque muitos valores da
família tradicional envolvidos numa atmosfera conservadora, abrindo
brechas para cuidados disfarçados de autoritarismos do ``se ficar
bonzinho agora'', será mais amado e suportável para todos viverem na
quarentena. Práticas de liberdades infantis não são negociáveis. Roberto
Freire, em \emph{Utopia e Paixão} (\textsc{freire} \& \textsc{brito}, 1986), chamava isso
de ``violência amorosa às crianças'', uma mistura de amor, medo e
chantagem. Para Freire, elas precisam se sentir seguras e não esperam a
confirmação parental de que são ``pequenos dependentes'', encerradas nos
valores morais do protótipo pai, mãe e filho. Indo ao encontro de Reich,
Freire enfatizava a importância de ``criar formas alternativas de
convivência'', formas espontâneas de exercício de liberdade, em
confronto direto com um Estado repressor que se emaranhou na figura do
pai e da mãe, que respondem às demandas infantis com ``o chinelo, o
cinto, a surra'' (idem, p. 37).

Damo"-nos conta do que as crianças pensam, sentem, escrevem e leem, além
das exigências institucionais. Se a pandemia trouxe consigo muitas
restrições, ela também trouxe muitas informações e novidades. Elas estão
mais presentes, no corpo a corpo diário, ajudando a reinventar aquele
mundo onde estávamos acostumados a viver, agora reduzido a um pequeno
espaço físico.

Em tempos de pandemia, nada justifica o pátrio poder dos pais em nome da
vida, aquelas rédeas do poder incrustradas nas famílias, que conduzem as
crianças para a submissão e a obediência. Os pais não devem ser uma
projeção do Estado.\footnote{Para Despentes, ``Um Estado que se projeta
  como mãe todo poderosa é um Estado fascista'' (\textsc{despentes}, 2016, p.
  20).} Talvez seja uma oportunidade para se repensar o cuidado das
crianças e a gestão do espaço doméstico. É a vida da criança e não a
autoridade insolente que deve ser valorizada.\footnote{Segundo o
  relatório da \textsc{ong} (World Vision), escrito em maio de 2020, projeta"-se
  um aumento de 32\% de violências físicas, emocionais e sexuais contra
  as crianças, além do aumento do trabalho infantil e futuros casamentos
  infantis pós"-pandemia. ``As coisas ficam ainda piores para estas
  crianças, pois os sistemas e serviços que podem identificar, responder
  e prevenir tais ameaças e violência estão operando com pouca ou
  nenhuma capacidade durante a pandemia'' (\textsc{world vision}, 2020, p. 5).} E
as crianças, de forma natural, desafiam a autoridade a cada instante.
Cuidar da criança é cuidar ``profundamente uns dos outros --- agindo
como se a vida de todos tivesse um valor inerente e estivesse em risco;
como se a saúde de um fosse a saúde de todos'' (\textsc{milstein}, 2020).

Proteger uma criança não é o mesmo que inibir a sua liberdade. O amor e
a dependência dos pais não são uma arma para chantagear e inibir
manifestações de liberdade em situações de confinamento. Nenhum
autoritarismo contra as crianças é justificado pela pandemia. ``Bem
fraco é aquele que tem de recorrer à autoridade e erra"-se quando se
acredita que se melhora o insolente'' (\textsc{stirner}, 1979, p. 89).

As crianças precisam é de cuidado dos mais próximos, cuidado praticado
de forma imaginativa e inventiva, que reforce os laços afetivos. A
defesa da vida da criança na pandemia toma a forma de uma luta contra
todas as formas de autoritarismo e os valores tradicionais que cerceiam
a infância. A criança não é a responsável pelo ``achatamento da curva'',
mas, como diz Ferrer, ela é o futuro. Elas estão no mundo para serem
elas mesmas. A sua felicidade não deve estar amarrada a uma missão
familiar. ``Não se faz um filho nem para a sociedade, nem para a
perpetuação da existência coletiva, mas para si e para ele mesmo''
(\textsc{gauchet}, 2008, p. 1).

A reinvenção das práticas de convívio e aprendizagem é um ponto
importante para o anarquismo. Não será o Estado a instituição que dará
as diretrizes de saúde concernentes às crianças. Nem a soberania dos
pais será a diretriz comportamental nesses tempos de confinamento. A
pandemia sob o olhar anarquista nos conduz a uma necessária compreensão
da complexidade e das variações na infância. Ferrer tinha a compreensão
dessa complexidade infantil. ``Se temos órgãos, é preciso que se formem
e se desenvolvam; é preciso deixar às crianças a oportunidade de
desdobrar a natureza'' (\textsc{ferrer y guardia}, 2014, p. 69).

Numa perspectiva anarquista, entende"-se que este é um momento oportuno
para construir e reinventar as relações com as crianças. ``Nossos sonhos
mais selvagens estão enraizados nas relações cotidianas {[}\ldots{}{]}. A
\textsc{covid}"-19 mostra"-nos, mais do que nunca, como estamos interligados
{[}\ldots{}{]}. Todos nós merecemos ser amados, respeitados e conhecidos
pelas pessoas que cuidam de nós'' (\textsc{ffitch}, 2020).

Numa escala global, estaríamos falando em solidariedade. ``Se devemos
começar na prática organizando o atendimento àqueles que já são próximos
e íntimos --- nós mesmos, nossas famílias, amigos, vizinhos e entes
queridos ---, então parte desse esforço implica expandir continuamente a
organização e a coordenação do atendimento em qualquer escala a todos
que necessitarem'' (\textsc{paul}, 2020).

Para os anarquistas, a saúde não deve ser um instrumento político. Não
se confunde com individualismo ou desprezo pela saúde dos outros, mas
com o cuidado de si. A saúde das crianças, portanto, é, antes de tudo,
extensiva aos adultos, e vice"-versa.

Durante uma pandemia, o espaço da criança não deve ser hostil e
opressivo. Se os adultos não vivem sozinhos neste mundo, pode"-se dizer o
mesmo sobre as crianças. Se a pandemia está presente em todos os
lugares, ela é também parte de uma dimensão da sociabilidade infantil.
Ela tem uma dimensão ética que vai além da necessidade. Envolve também
relações alicerçadas no cuidado, carinho, coragem e diálogo com as
crianças. No plano social, o cuidado com as crianças é extensivo ao
cuidado coletivo e às práticas sociais solidárias. ``Devemos aprender
sobre nossas próprias necessidades e as necessidades daqueles de quem
somos capazes de cuidar'' (idem). Gauchet dizia que é importante mostrar
à criança sempre um caminho para a vida, e que não existe um mundo
ideal. Pois cada criança tem a sua singularidade, e ``tudo deve ser
feito para fazer com que a criança vá em direção a sua própria
autonomia'' (\textsc{gauchet}, 2008, p. 1).

A pandemia não destruiu o mundo da criança. Os seus amigos ainda
existem. Afastar"-se fisicamente dos amigos não é o mesmo que se
distanciar dos amigos e dos pais. É um momento de reinvenção e
experimentações horizontais que potencializem num processo inventivo sua
autonomia.

A ativação de um cuidado vai de encontro ao autoritarismo, proibição,
medo e pânico dirigido às crianças. Situações"-limite de desemprego e
perda de renda não exigirão mais respeito e obediência aos pais. Talvez
as ``rupturas na estrutura do trabalho refletem a estabilidade
estilhaçada da unidade"-lar e da unidade"-família'' (\textsc{bey}, s/d, p. 9).

Não é a morte o horizonte da criança. Crianças têm a vitalidade e a
potência da existência no seu pensamento e nos seus corpos. Suas vidas
são singulares, plurais e oníricas. Crianças reinventam espaços. Como
diz Foucault, as crianças ``levam muito tempo para saber que têm um
corpo'' (\textsc{foucault}, 2013, p. 15). No mais, elas reinventam os espaços. A
criança transforma e sobrepõem espaços. O espaço que a criança imagina é
onírico. A cama pode ser uma floresta, um celeiro, uma piscina, um
abismo. O seu quarto apertado pode ser infinito como o universo. Elas
não vivem sob modelos de existência. Em suas cabeças, instante é um
tempo eterno de liberdade. Elas são o aqui e o agora.

\begin{bibliohedra}
\tit{A UNION COMMUNISTE LIBERTAIRE}. Pandémie, crise: les classes populaires
sont toujours en lutte. Disponível em:
https://rebellyon.info/Pandemie-crise-les-classes-populaires-22263.
Acesso em: 10/09/2020.

\tit{BEY}, Hakim. \textsc{taz} --- \emph{Zona Autônoma Temporária}. Tradução de
Patricia Decia e Renato Resende. Coletivo Sabotagem, s/d. Disponível em:
http://www.mom.arq.ufmg.br/mom/02\_arq\_interface/4a\_aula/Hakim\_Bey\_TAZ.pdf.
Acesso em: 28/09/2020.

\tit{BRUCIAGA}, Wenceslao. Días de pandemia: príncipes y niños anarquistas.
06/05/2020. Disponível em:
https://www.milenio.com/opinion/wenceslao-bruciaga/el-nuevo-orden/dias-de-pandemia-principes-y-ninos-anarquistas.
Acesso em: 28/09/2020.

\tit{DESLANDES}, Suely Ferreira; \textsc{coutinho}, Tiago. O uso intensivo da internet
por crianças e adolescentes no contexto da \textsc{covid}"-19 e os riscos para
violências autoinflingidas. Rio de Janeiro: Ciênc. Saúde Coletiva junho
de 2020, vol. 25, supl. 1, p. 2479--2486.

\tit{DESPENTES}, Virginie. \emph{Teoria do King Kong}. Tradução de Márcia
Bechara, São Paulo: N"-1 edições, 2006.

\tit{FERRER Y GUARDIA}, Francisco. \emph{A Escola Moderna}. Tradução de Camilo
Alvares, São Paulo: Biblioteca Terra Livre. 2014.

\tit{FFITHC}, Madeline. Anarchism's Lessons for a Pandemic --- Afflicted
World. 28/05/2020. Disponível em:
https://lithub.com/anarchisms-lessons-for-a-pandemic-afflicted-world/.
Acesso em: 16/09/2020.

\tit{FOUCAULT}, Michel. \emph{O Corpo Utópico --- As Heterotopias}. Tradução
de Salma Tannus Muchail. São Paulo: N"-1 Edições, 2013.

\tit{FREIRE}, Roberto; \textsc{brito}, Fausto. \emph{Utopia e Paixão: a política do
cotidiano}. Rio de Janeiro: Rocco, 1986.

\tit{GAUCHET}, Marcel. De ``L'enfant du Désir'' à ``la crise de
l'individuation''. 2008. Disponível em:
https://www.meirieu.com/PATRIMOINE/gauchet\_individuation.pdf. Acesso
em: 10/09/2020.

\tit{ILLICH}, Ivan. \emph{Sociedade sem escolas}. Tradução de Lúcia Mathilde
Endlich Orth, Petrópolis: Ed. Vozes, 1973.

\tit{LINHARES}, Maria Beatriz Martins; \textsc{enumo}, Sônia Regina Fiorim. Reflexões
baseadas na Psicologia sobre efeitos da pandemia \textsc{covid}"-19 no
desenvolvimento infantil. \emph{Estud. psicol. (Campinas)} {[}online{]}.
2020, vol. 37.

\tit{MACHADO}, Roberto; \textsc{loureiro}, Angela; \textsc{luz}, Rogerio; \textsc{muricy}, Katia. \emph{A
Danação da Norma}. Rio de Janeiro: Edições Graal, 1978.

\tit{MILSTEIN}, Cindy. Collective Care Is Our Best Weapon Against \textsc{covid}"-19.
Disponível em: https://mutualaiddisasterrelief.org/collective-care/.
Acesso em: 16/09/2020.

\tit{NIETZSCHE}, Friedrich. Os Discursos de Zaratustra --- Das Três
Transformações. In: \emph{Assim Falou Zaratustra}. Tradução de José
Mendes de Souza. EBooksBrasil, 2002.

\titidem. Contribuição à História dos Sentimentos Morais. In:
\emph{Humano, Demasiado Humano}. Tradução de Paulo César de Souza. São
Paulo: Cia das Letras, 2000.

\titidem. Apêndice: Fado e História. In: \emph{Genealogia da Moral: uma
polêmica}. Tradução de Paulo César de Souza. São Paulo: Cia das Letras,
1998.

\tit{PAUL}, Ian Alan. Dez Premissas para uma Pandemia. Disponível em:
https://medium.com/@joaolucasxavier/dez-premissas-para-
uma-pandemia-5f4dfd36f144. Acesso em: 10/09/2020.

\tit{PORTINARI}, Beatriz. Os efeitos do Confinamento na Saúde Mental de
Crianças e Adolescentes. In: \emph{El Pais}, 06/06/2020. Disponível em:
\emph{https://bit.ly/3cHhMod}. Acesso em 29/09/2020.

\tit{SEVEN}, John; \textsc{christy}, Jana. \emph{A Rule is to Break: The Oath of
Anarchy.} San Francisco, \textsc{ca}: Manic D Press, 2012.

\tit{STIRNER}, Max {[}1845{]}. \emph{O Único e a sua Propriedade.} Tradução de
João Barrento, São Paulo: Martins Fontes, 2004.

\titidem. \emph{Textos Dispersos}. Tradução de José Bragança de Miranda,
Lisboa: Via Editora, 1979.

\tit{WANG}, G., \textsc{zhang}, Y., \textsc{zhao}, J., \textsc{zhang}, J., \& \textsc{jiang}, F. Mitigate the
effects of home confinement on children during the \textsc{covid}"-19 outbreak.
\emph{The Lancet}, 2020, 395(10228), p. 945--947. Disponível em:
\emph{https://bit.ly/31Do029}.
Acesso em: 25/09/2020.

\tit{WORLD VISION}. A Perfect Storm: millions more children at risk of
violence under lockdown and into the `new normal'. Disponível em:
https://www.wvi.org/sites/default/files/2020-05/Aftershocks\%20FINAL\%20VERSION\_0.pdf.
Acesso em: 25/09/2020.
\end{bibliohedra}

\chapterspecial{Instantâneo de uma pandemia\footnotemark}{}{Ronald Creagh}

\footnotetext{Tradução do francês por Martha Gambini.}

\hfill\ \emph{Aos meus camaradas de São Paulo e do Rio de Janeiro.}\bigskip

\noindent{}Um acontecimento mundial exige tempo para ser compreendido. Por vezes,
seus efeitos se estendem por vários séculos. Esse é o caso de certas
epidemias, e podemos prever que também será assim para a \textsc{covid}"-19,
surgida em 2019\footnote{\textsc{covid} é uma abreviação de \emph{corona vírus
  disease}, que significa textualmente doença do vírus em forma de
  coroa. Assim, os dois termos, que designam o mesmo acontecimento, são
  aqui utilizados. Este artigo se beneficiou das sugestões e correções
  de René Fugler, Didier e Marielle Giraud e de Danièle Haas. Quero
  agradecê"-los vivamente. Sou o único responsável por eventuais erros.}.
Ela se propagou por 185 países. Tem uma elevada taxa de contágio. Até
hoje é a mais vasta infecção da história. Ela é complexa e específica.

E é anunciada a possível chegada de vírus mais perigosos: ``Sabemos que
o degelo acelerado do \emph{permafrost}¸ numa profundidade cada vez
maior, faz com que vírus podendo ter 50.000 anos ou mesmo mais voltem à
superfície. E, nesse momento, há um trabalho grande sendo feito, sobre
os megavírus. E para nós, isso é extremamente perigoso, pois esses vírus
não são bem"-conhecidos: então, há uma grande ameaça. Podem existir
patologias que surgem da pré"-história'' (\textsc{patou"-mathis}, 2020). Entendemos
muito pouco o mundo atual, porque suas modificações se aceleram.
Entramos numa nova era ecológica, inclusive no que diz respeito à saúde
humana.

As epidemias precedentes, inclusive a \textsc{aids}, emergiram num mundo que,
desde então, mudou. Os meios médicos, as escolhas econômicas e sociais,
a política internacional da época atual, as vítimas, os recursos
mobilizados não têm nada a ver com aqueles dos períodos precedentes. A
sociedade está se recompondo. Estamos penetrando num futuro instável,
diferente e planetário. E, se a sociedade se decompõe e se reconstrói de
forma diferente, é preciso reagir, embora ninguém nos diga isso.

Compreender o novo coronavírus não significa apenas fazer seu balanço
estatístico. É indispensável extrair daí pistas para as catástrofes
futuras. Naturalmente, cada país, cada meio social, vive essa infecção
de maneira particular. Uma perspectiva geral seria agora prematura, mas
já podemos propor algumas possibilidades.

Quase não se fala em mudanças sociais. O sistema atualmente em vigor
modifica"-se rapidamente, e as diferenças por vezes são percebidas e
discutidas aqui e ali. Uma catástrofe mundial está recompondo a
configuração da sociedade. Não podemos cruzar os braços ou adotar apenas
uma visão parcial.

As reflexões propostas aqui também são escolhas subjetivas: portanto,
desejamos que despertem leituras críticas e construtivas.

Um movimento filosófico já antigo começa a reunir os espíritos abertos.
Ele lembra acontecimentos históricos precedentes como a revolta na
Inglaterra dos Ludistas, que durou de 1811 a 1816. Eis o que escreve o
psicólogo Tomás Ibañez: ``Hoje é essencial `reinventar' esse tipo de
revolta, fazendo"-a passar da esfera das reivindicações puramente
econômicas para a esfera mais diretamente política das lutas pela
liberdade e contra o totalitarismo de novo tipo que tem se instaurado já
há algum tempo e que encontra, na crise atual do \textsc{covid}"-19, um abundante
combustível para acelerar seu desenvolvimento'' (\textsc{ibañez}, 2020). O autor
propõe um reajuste, uma remodelagem de todos os pilares e fachadas da
sociedade, quer se trate dos formadores de opinião, dos serviços
médicos, da economia e do sistema político.

Para apreender o alcance e a lógica social de nossas relações com a
pandemia do novo coronavírus, chamaremos a atenção para certas
particularidades de algumas pandemias do passado. A seguir, iremos
abordar os meios concernidos, ou seja, os agentes, os meios médicos,
econômicos e políticos e as vítimas, muito diversificadas.

Essas posições críticas diferem de muitos movimentos de contestação
contemporâneos. Questioná"-los, sob a ótica de um pensador como Piotr
Kropotkin (1842--1941), talvez finalmente permita perceber melhor as
possibilidades oferecidas por múltiplas alternativas construtivas. Suas
posições, semelhantes e por vezes inspiradas por seus escritos,
valorizam a importância da ajuda mútua para a sobrevivência da espécie.
Posições aliás hoje retomadas de maneira científica por uma minoria de
biólogos (\textsc{garcia}, 2012).

Para começar, vamos dar um sobrevoo sobre algumas grandes diferenças com
o passado.

\section{a história das epidemias}

As doenças contagiosas e as epidemias existem pelo menos há sete mil
anos, na época da sociedade mesopotâmica. As primeiras populações
humanas eram nômades. Viviam da caça e do extrativismo. As que viviam
nas proximidades de um curso d'água, do mar ou de um oceano, também se
dedicavam à pesca. Progressivamente, uma grande parte dos grupos
fixou"-se territorialmente para se entregar à pecuária e aos trabalhos
agrícolas: ``Na época neolítica, há o início da pecuária, que provoca
uma promiscuidade entre o homem e o animal. Certo número de zoonoses
passou para o homem nesse momento. E nunca devemos esquecer que os vírus
são extraordinariamente adaptáveis. Eles sofrem incessantes mutações, o
que constitui o estágio último do parasitismo'' (\textsc{david}, 2020).

A proximidade com o rebanho e depois, pouco a pouco, a produção de
excedentes, a profissionalização do artesanato, o desenvolvimento do
comércio deram lugar à criação de aldeias e cidades. A multiplicação
dessas implantações provocou o crescimento demográfico. Mas essas
concentrações humanas também acabaram por facilitar o contágio.

Os deslocamentos de população continuaram, especialmente em razão das
mudanças climáticas e das guerras. Mais tarde, as caravanas que
transportavam alimentos, animais e produtos artesanais percorriam
distâncias inacreditáveis. Foi assim que três pandemias se sucederam de
541 a 767. Agrupadas sob o nome de ``peste de Justiniano'', elas
remodelaram a bacia do Mediterrâneo: enfraqueceram o Império Bizantino,
permitiram a expansão do Islã e a queda do Império Otomano.

Outro movimento humano gerou modificações em curso da ecologia do
planeta: as cruzadas nos séculos \textsc{xi} e \textsc{xii}. Elas contribuíram para a
propagação das infecções. Assim, pode"-se afirmar que no século \textsc{xiv} a
``unificação microbiana do mundo'' tinha começado (\textsc{le roy ladurie}, 1972,
p. 629). A partir de então, as epidemias em certo território corriam o
risco de se tornarem pandemias, espalhadas por todo um continente ou
mesmo pelo mundo inteiro. Assim, a época moderna viveu infecções que se
propagaram em numerosos territórios. Por exemplo, a gripe de 1899
assolou a Europa durante cinco anos.

Talvez nunca cheguemos a conhecer as circunstâncias da aparição do novo
coronavírus na China ou em outra parte. Ao menos dispomos de documentos
oficiais, mas eles também são discutíveis. Ele apareceu em 2019 e
propagou"-se muito rapidamente ao ponto de se tornar uma calamidade
mundial, uma pandemia, pois ele não se limitou a um país e nem mesmo a
um continente. Ele ainda grassa no momento em que estas linhas estão
sendo escritas.

\section{as origens do novo coronavírus}

Em 30 de dezembro de 2019, a Administração Médica do Comitê Municipal de
Wuhan, na província chinesa de Hubei, lançou na internet uma informação
urgente: uma pneumonia de causa desconhecida havia aparecido naquela
cidade. Na manhã do dia seguinte, um repórter do \emph{China Business
News} entrou em contato em linha direta com essa autoridade, que
confirmou que a causa era desconhecida (\textsc{promed}, 2020).

Nesse mesmo dia, o escritório chinês da Organização Mundial da Saúde
informou o escritório local do Regulamento Sanitário Internacional (\textsc{rsi})
e a Sociedade Internacional das Doenças Infecciosas ProMed, que
transmitiu a informação a seus cinquenta mil membros no mundo.

Em princípio foi denunciado o mercado de animais selvagens de Wuhan na
China. Pierre Jouventin, pesquisador em ecologia científica, escreveu:
``Há por volta de cem mercados locais em Wuhan e neles se misturam, no
calor e umidade, humanos e animais vivos ou mortos. Esse patógeno teria
partido de um morcego e teria sido veiculado por um pangolim para sofrer
uma mutação ou se recombinar e se adaptar a um novo hospedeiro, o homem.
É a terceira vez que uma transferência animal"-homem acontece nesses
mercados, e a importação de animais selvagens, hoje proibida, deveria
ter sido impedida antes. Paraíso das pandemias, a China está longe de
deter seu monopólio, pois muitas outras aparecem em outros lugares e
também em animais de criação, como o porco. Mas ¾ das doenças emergentes
vêm dos animais selvagens e seguem"-se à destruição dos habitats
naturais'' (\textsc{jouventin}, 2020).

A seguir, outras acusações foram formuladas, como veremos abaixo. Elas
provinham de fontes hostis a Pequim. De qualquer forma, Wuhan, a partir
de então, ficou proibida de consumir animais selvagens pelo menos
durante os próximos cinco anos.

\section{as vítimas: os trabalhadores e os pobres}

A pandemia provocou a crise das empresas e, em consequência, dos
assalariados e pequenos empregadores. O fechamento de certa agência ou
fábrica pertencendo a uma multinacional, ou sua transferência, levou
desemprego às populações dos territórios afetados. Setores inteiros
foram atingidos: transportes, turismo, esporte, restaurantes, editoras,
economia informal e muitos outros. Algumas profissões sofreram de forma
particular, como os trabalhadores temporários, mas também os atores, os
músicos e os profissionais eventuais de eventos culturais, que ficaram
desempregados.

A infecção propagou"-se em todos os continentes. Foi preciso, em primeiro
lugar, mobilizar os serviços médicos, mas também se impuseram escolhas
econômicas. Os Estados intervieram e tomaram decisões sobre os meios de
proteção em função de suas possibilidades e respectivas doutrinas. A
pandemia provocou flexões até mesmo na política internacional.

Os cuidados são muito custosos em determinados países. Além disso, os
mais pobres, condenados a viver com muitos numa mesma casa, fechados num
lugar pequeno, e muitas vezes insalubre, só têm acesso a cuidados de
menor qualidade. Seus bairros, por vezes colocados compulsoriamente em
quarentena, são frequentemente suspeitos de estarem infectados.

No entanto, nem tudo é negativo. A sobrevivência de uma coletividade
coloca em relevo o papel essencial dos empregos pouco remunerados:
cuidadores, varredores, pequenos vendedores de um mercado\ldots{} que
pertencem a etnias de culturas muito diferentes entre elas.

Os Estados, inicialmente sem vacinas e proteção para os rostos das
populações, utilizaram como soluções substitutivas recursos que remetem
principalmente à Idade Média: máscaras, confinamento, distância social,
quarentena. Após o eventual confinamento parcial ou geral, eles
escolheram para a economia a última oportunidade oferecida pela
tecnologia, o \emph{home office.}

Não era preciso salvar as empresas? Essa escolha apresenta a vantagem de
reduzir as responsabilidades legais dos poderes públicos e dos
empregadores. Ela também atrai a atenção para a função principal de um
governo: a proteção de sua população e de seus cidadãos expatriados.

E quanto aos ambientes médicos?

\section{laboratórios e especialistas}

A surpresa e o medo criados pela pandemia foram ainda maiores pelo fato
de a pesquisa médica celebrar diariamente suas proezas. Entretanto, com
exceção da China, onde a pandemia se declarou, os Estados poderiam ter
antecipado sua vinda.

A intervenção deles foi tardia, e suas decisões sobre medidas de
proteção, contestáveis. Mas elas fizeram cair uma chuva de ouro sobre os
laboratórios de pesquisa. Bem mais do que ocorreu nas epidemias
anteriores. Assistimos por toda parte a uma corrida infernal do pessoal
de institutos concorrentes, que se lançavam ferozmente à busca de uma
vacina lucrativa contra a \textsc{covid}"-19.

O trabalho científico sobre o vírus é essencial. Mas, embora seja
importante preservá"-lo e seguir de perto os progressos da pesquisa, o
conjunto da questão não se limita nem à máscara, nem ao confinamento,
nem à vacinação. Nem mesmo apenas ao aspecto médico. O rosto descoberto
mudou de sinal. ``Que o outro com quem cruzamos na rua não respeite os
gestos de barreira ou que o governo se mostre incapaz de fornecer os
bloqueios que supostamente me protegem, tudo isso é equivalente: em um
segundo, o mundo exterior revela"-se como meu inimigo íntimo'' (\textsc{fœssel} \&
\textsc{riquier}, 2020, p. 46).

\section{as escolhas econômicas}

O sistema econômico contemporâneo não cria somente uma linha de fratura
entre as classes sociais. Ele repercute politicamente sobre um grande
número de países atualmente colonizados pelas grandes potências que,
entre mil outras responsabilidades, sustentam e por vezes instauram
dirigentes totalitários e novas formas de tipos de tirania que cedo ou
tarde conduzem a guerras civis.

Nos países em que se praticou o confinamento, a oferta reduziu"-se e,
portanto, a criação de riqueza. E também a demanda, pois os custos
tinham às vezes aumentado e as famílias perdido seus rendimentos:
``{[}As pandemias{]} reduzem duravelmente a oferta de trabalho, que se
torna rara, aumentam a remuneração do trabalho, deprimem o rendimento do
capital que se torna excedentário, tornam mais lento o ritmo dos
investimentos e diminuem a capacidade de produção da economia {[}\ldots{}{]}
Devido às restrições sanitárias impostas pela continuação da luta contra
a epidemia, há grandes chances de que o aparelho de produção não possa
mais rodar em regime pleno, que uma parte das capacidades de produção
permaneça duravelmente subempregada e depois se torne obsoleta e que a
baixa do rendimento do capital não mais incite as empresas a
reconstituí"-las'' (\textsc{trainar}, 2020, p. 230)\footnote{Muitos comentários
  precedentes foram inspirados nesse artigo.}.

As antecipações dos empreendedores desabaram, o que repercutiu nas
empresas e na Bolsa. Os preços dos imóveis comerciais entraram em queda.
Acrescentemos a isso o endividamento das empresas e principalmente dos
pequenos empregadores, obrigados a continuar funcionando em um
escritório ou loja fantasmas ou mesmo a declarar falência.

O confinamento reduziu a poluição especialmente a das classes mais
favorecidas. Ele também baixou a renda dos governos locais e nacionais.
Pois, quanto mais rigoroso foi o confinamento, mais os consumidores
reduziram suas despesas e, por isso, o crescimento caiu. E na Itália,
entre outros, a aplicação do confinamento foi mais estrita nas
municipalidades que possuíam maior capacidade fiscal (\textsc{bonaccorsi} et al.,
2020).

\section{as escolhas sociais dos estados}

No Brasil, onde mais de três mil militares ocupam postos nos
ministérios, o presidente Jair Bolsonaro, antigo paraquedista, ergueu"-se
contra a Organização Mundial da Saúde e a utilização de máscaras. Em
contrapartida, vários estados brasileiros estabeleceram que os
residentes deveriam permanecer em suas casas (\textsc{saliba}, 2020).

Na Indonésia, em 5 de outubro de 2020, sete dos nove partidos do
parlamento votaram 79 emendas a uma lei já existente. O que constitui um
documento de mais de mil páginas. Sob pretexto de estimular os
investidores estrangeiros a criar novos empregos, os trabalhadores
perderão parte de seus ganhos sociais, as indenizações rescisórias serão
reduzidas, as demissões serão facilitadas e todos os empregados, sem
dúvida, farão horas extras. Milhares de manifestantes, sem máscara,
ocuparam as ruas em Jacarta e outras cidades\ldots{} (\textsc{ulfiana}, 11/10/2020).

Seja em ditaduras ou democracias, a igualdade só existe nos hinos e
promessas políticas. A utilização abstrata dessa palavra pertence ao
terreno da impostura. O pobre nunca foi igual ao rico. E, mesmo em
matéria de direitos, o cidadão médio não pode pagar advogados
prestigiosos e influentes.

A situação variou segundo os países, pois tanto as decisões quanto seus
impactos foram bem diversos. Que preço as políticas e os economistas
atribuem à vida humana? Eles preferem sacrificar uma parte da população?

A difusão do vírus revelou a incompetência dos poderes públicos: eles se
interessam mais por certas instituições econômicas do que pelas
necessidades de seus sujeitos.

Os riscos extremos, como as pandemias, são raros, mas cruciais. Qual o
grau de aversão que certa sociedade manifesta por eles? Qual parte eles
ocupam nas preocupações daqueles que tomam as decisões? A prioridade
está focada no consumo imediato ou se busca antecipar soluções e assim
reduzir os custos da prevenção?

Esses cuidados variam de acordo com as nações e cada leitor dará uma
resposta em função do país onde reside e também de sua profissão (um
militar raciocina de modo diferente de uma mãe de vários filhos).

A pandemia impôs uma hierarquia nas escolhas sociais. Ela obrigou os
Estados a colocarem em funcionamento certo número de ações e de meios.
Assim, eles embarcaram na política internacional, o que era inevitável.

Antes da pandemia, a França, por exemplo, estava mais interessada em
vigiar a população do que em sua segurança. Esse controle relaciona"-se
cada vez mais aos meios digitais. Pode"-se prever que os \emph{Big Data},
fontes de renda e sobretudo de poder, sejam mais ou menos retomados
pelos Estados que participam de alguma forma de hegemonia regional ou
internacional.

Estados haviam deixado desaparecer empresas que fabricavam máscaras; e
as autoridades tinham liquidado os estoques destinados à população. Daí
a penúria.

O governo declarou inicialmente que as máscaras eram inúteis; a seguir,
ele as tornou obrigatórias. O Ministro da Saúde pediu demissão de sua
função em plena crise, e o Presidente da República esperou um mês antes
de nomear um substituto. E decidiu confinar a população para evitar o
contágio. Finalmente, os Estados em concorrência conseguiram obter
máscaras.

Proprietários de \emph{Big Data} ligados à inteligência digital, os
poderes públicos conseguiram mudar a natureza de suas relações com os
``cidadãos''. A apropriação de uma tecnologia de controle permitiu que
eles otimizassem as telemanipulações da opinião do país. As decisões
``democráticas'' metamorfosearam"-se em imposições mais ou menos
camufladas. Foi assim que a segurança pública, garantida entre outros
por agentes da circulação, foi substituída por câmeras, utilizadas
principalmente por uma polícia concentrada na defesa dos poderes
instalados, mesmo que desprezíveis.

Muitos governos favorecem as empresas de produção em vez dos centros de
pesquisa, que eles orientam de acordo com suas necessidades militares,
ou que eles obrigam a recorrer a instâncias privadas. Assim, a pesquisa
filosófica ou científica é abandonada em proveito do desenvolvimento das
tecnologias. O que definitivamente se mostra muito mais benéfico para os
meios comerciais e financeiros.

Enfim, a geopolítica mudou em razão da espionagem. Hoje, não basta mais
descobrir o que as outras nações estão fazendo. Estão disponíveis meios
para influenciar a população de nações cortejadas ou rivais,
especialmente em período eleitoral.

\section{a política internacional}

Em 15 de junho de 2020, o presidente dos Estados Unidos, Donald Trump,
declarou, numa entrevista ao \emph{Wall Street Journal,} que a China
talvez tivesse lançado intencionalmente o vírus com o objetivo de
degradar as economias concorrentes. Mais tarde, ele fez circular a ideia
de que o novo coronavírus provinha do laboratório de Wuhan, na China
(\textsc{seib}, 2020). Assim, a pandemia vai marcar a política internacional.

No nível mundial, a volta da importância das fronteiras irá repercutir
inevitavelmente na geopolítica. E, então, quantos governos no mundo
aceitariam lutar contra uma coalisão de multibilionários?

Mais dissimuladas, as tentativas de captar as iniciativas farmacêuticas
mais promissoras também estão em curso, dessa vez com finalidades
políticas e mesmo eleitoreiras. Em 16 de março de 2020, Trump entra em
contato com um laboratório farmacêutico alemão. Ele lhe propõe um bilhão
de dólares, em caso de sucesso, para obter a exclusividade da vacina
para os Estados Unidos. A opinião alemã fica profundamente chocada com
isso. E o laboratório rejeita a atraente oferta.

Na Europa, no dia 4 de maio, a presidente da Comissão Europeia, Ursula
von der Leyen, lança uma iniciativa sem precedentes. Ela organiza um
teleton mundial visando a levantar 7,5 bilhões de euros para financiar
uma vacina contra o novo coronavírus que seria colocada a serviço de
todos. Angela Merkel, chanceler federal da Alemanha, e Emmanuel Macron,
presidente da República Francesa, entre outros, dão o seu acordo. Mas o
presidente dos Estados Unidos recusa a participação: ``American First''.
Alguns dias mais tarde, em 13 de maio, Paul Hudson, Diretor Geral do
laboratório Sanofi, anuncia que a China e os Estados Unidos seriam
prioritariamente abastecidos com a vacina contra a \textsc{covid}"-19, porque os
dois países teriam sido os primeiros a financiar essa pesquisa (\textsc{paton} et
al., 2020).

Foi um segundo escândalo na União Europeia.

No fim das contas, o Elisée lembra"-se, de repente, que o gigantesco
grupo farmacêutico francês Sanofi se beneficiava anualmente de dezenas
de milhões de crédito de imposto a título da pesquisa. O presidente
Macron sente"-se obrigado a intervir, e o gigante desiste.

Como explica Camille Magnard, há o risco de que a China e os Estados
Unidos se apropriem de todos os grandes laboratórios (\textsc{magnard}, 2019). A
corrida ao dinheiro poderia arrastar esses grupos farmacêuticos maiores
a apoiarem o ``nacionalismo vacinal''. Isso significaria que, como
aconteceu com o sarampo e a tuberculose, os países pobres não poderiam
ter acesso à vacina contra a \textsc{covid}"-19.

O regulamento sanitário mundial não se encontra harmonizado. Esse é o
caso da legislação sobre os \textsc{ogm}s e da ecomarcação dos produtos
alimentares, como o açúcar e o azeite de dendê. A dependência em
medicamentos é tão séria quanto na alimentação. ``Nos Estados Unidos,
80\% dos medicamentos são de origem chinesa {[}\ldots{}{]}. A Índia, a
farmácia do mundo, depende em 70\% de ingredientes chineses'' (\textsc{ithurbid}
\& \textsc{maillard}, 2020, p. 500). E ainda, ``aliás, a forte dependência da
União Europeia em relação ao comércio exterior torna"-a particularmente
exposta aos efeitos da coronadepressão {[}\ldots{}{]}'' (\textsc{heisbourg}, 2020, p.
535).

Quando estarão disponíveis estoques reais de máscaras, de luvas, de
testes e de medicamentos de base? (Idem). Os grandes riscos podem ser
raros, mas seus estragos são muito mais importantes que o custo
econômico dos estoques de reserva.

\section{a desglobalização está a caminho}

A globalização comercial e a disseminação digital não criam milagres. A
primeira encoraja a deslocalização, a queda de qualidade, os conflitos
comerciais. A segunda aumenta de forma inacreditável a espionagem
industrial e as intervenções políticas estrangeiras, especialmente no
momento das eleições e das guerras. Em contrapartida, será que é
aceitável continuar dessa forma o financiamento dos laboratórios
privados e não uma pesquisa pública de espírito internacionalista?

No entanto, no oposto da globalização, o nacionalismo serve de veículo
para um novo tipo de fascismo e, é claro, para a aparição de Estados
autoritários, para não dizer ditatoriais. É evidente que o
empobrecimento programado das classes médias, em certo número de países
entre os mais prósperos, irá necessariamente provocar múltiplos
protestos.

Nas casas de repouso para idosos ou hospitais, muitos moribundos
acabaram seus dias sem mesmo receber uma última assistência de suas
famílias; seres que lhes eram os mais queridos foram até mesmo proibidos
de comparecer a seus funerais.

Das várias opiniões formuladas durante a pandemia, podemos levantar uma
hipótese que será necessário verificar. Muitas pessoas estavam divididas
entre duas crenças fatalistas, a dupla necessidade do progresso técnico
e a de um Estado indispensável para garantir a segurança nacional.

Foi graças aos avanços técnicos que as mulheres não precisam mais lavar
as roupas da família em um rio; que a duração da vida humana se ampliou
enormemente; que podemos nos comunicar instantaneamente de um ponto a
outro da terra. Reduzir o sentido da palavra progresso apenas à técnica
é resultado de um espírito simplificador. Onde está o prazer de uma
troca entre Hong Kong e Rio de Janeiro, se o que vemos, em ambas as
partes, é apenas repressão e miséria?

Uma civilização avança quando os homens também participam disso. Quando
não se trabalha mais cinquenta horas por semana. Quando não se vive sob
a ameaça nuclear. Em outras palavras, quando os costumes mudam.

\section{o protesto social}

O aparecimento da \textsc{covid}"-19 colocou em evidência a falta de preparo dos
Estados, a seguir suas ordens contraditórias, e, finalmente, eles se
mostraram incapazes de garantir a segurança das populações, o que
constitui sua missão e o próprio fundamento de sua autoridade.

Os reis sempre estão nus, mas ninguém quer saber disso. Eles não são
intocáveis. Outros apertam as mãos em meio a multidões, e os felizes
eleitos lembrarão disso por toda a vida. As correntes hostis sempre
existiram nos corredores do poder. Mas, desde as revoluções, elas também
aparecem no povo. O maior acesso à educação para as classes médias criou
correntes de hostilidade. A gestão econômica, financeira e social é
colocada em questão.

Assistimos à recuperação de empresas em dificuldades com o dinheiro dos
contribuintes e, depois, à sua venda a sumidades da finança ou a
multinacionais. Os favores financeiros concedidos a grandes firmas ou a
magnatas são, aliás, apontados abertamente. Mesmo a União Europeia tem
seus próprios paraísos fiscais. Ela estabeleceu uma lista deles, mas sem
acrescentar a Irlanda, Luxemburgo, Malta e os Países Baixos.

Outra tragédia se amplifica. A destruição de um número inacreditável de
espécies, a exploração ilimitada de recursos naturais insubstituíveis, a
poluição do ar e dos mares, o aquecimento da terra evocam o colapso de
nosso planeta. Ora, as promessas e os compromissos para reduzir a
destruição ecológica do planeta traduziram"-se com frequência por ações
insuficientes, e mesmo apenas simbólicas.

Evidentemente, governos de toda parte multiplicaram as ocasiões de
ruptura com os rebeldes que são apresentados como terroristas. Para se
proteger de novas aparições dissidentes, tal ou tal serviço público,
inclusive a educação e as universidades, submeteram"-se à concorrência
com o setor privado. Cada vez mais frequentemente, os desafiantes acabam
numa prisão.

Reina a ilusão de que o Estado poderia descartar as classes dirigentes,
a rede capitalista, e que bastaria uma relação de forças favorável para
liberar o poder vigente de todas essas ligações. Por exemplo, por uma
revolução mais ou menos violenta. Isso significa esquecer que os meios
deixam sua marca sobre os fins. A nova situação também não escaparia de
ter que recorrer à violência.

O que pensar de seus combates? Eles desconfiam de tal ou tal partido
político ou do governo estabelecido. Mas não questionam a existência do
Estado enquanto instituição no ápice de uma pirâmide hierárquica social.
Assim, lança"-se veneno sobre aqueles que possuem grande poder de
decisão, sem desarticular um sistema social hierárquico.

A maioria dos militantes erra de alvo. Ataca uma casta política, e não
um sistema. Talvez eles desejem ocupar o posto\ldots{}\footnote{Espero
  apresentar meus argumentos numa obra que está em preparação e que, sem
  dúvida, será publicada em 2022.}

Devemos nos curvar às evidências. Toda uma propaganda e rituais alçam
implicitamente o Estado e seu chefe ao patamar do sagrado: a saudação à
bandeira, a comemoração das vitórias, a cerimônia de posse ou a coroação
real\ldots{} Os poderes que querem se estabelecer retomam, adaptando"-os, os
modos e rituais de reinos na Mesopotâmia que datam de vários milhares de
anos.

O dogma da necessidade do Estado não é analisado. Ora, em sua forma
atual, essa instituição data de apenas quatro séculos e resulta de
fronteiras estabelecidas por meio de guerras. As populações locais não
tiveram qualquer participação nas decisões: a regra é a lei do mais
forte.

\section{alternativas construtivas}

Nossa perspectiva geral será feita no espírito do ex"-príncipe russo
Piotr Kropotkin. Em 1854, ele tinha 12 anos. E já tendo lido publicações
em favor da república, ele renuncia, com essa idade, a seu título de
príncipe.

Tornou"-se geógrafo, mas também zoologista, e não se contentou em estudar
as montanhas e os vales: esforçou"-se por compreender o conjunto das
relações recíprocas de cada população com seu meio ambiente. Ele avalia
que a moda do pensamento darwinista, retocada por alguns economistas
muito lidos em sua época, não levava em conta toda a maneira de ver de
Darwin. Essa moda proclama que a conservação de uma espécie está fundada
na sobrevivência dos mais fortes. Mas, dessa maneira, escamoteia outra
vertente do pensamento de Darwin, que ressalta a necessidade da ajuda
mútua: ``existe uma quantidade enorme de lutas e de exterminação no seio
de diversas espécies. Mas também existe o mesmo tanto, e talvez mais, de
suporte recíproco, de ajuda e de defesa mútua'' (\textsc{kropotkin}, 1902). Ele
demonstra a importância do animal que se sacrifica para salvar sua
progenitura, e, de maneira geral, a necessidade incontornável da
solidariedade para evitar o desaparecimento da espécie e mesmo para
garantir seu progresso. O mesmo tema será retomado sob outra forma por
grandes biólogos contemporâneos, críticos daqueles que afirmam a
existência de um suposto gene egoísta (\textsc{garcia}, 2012, p. 235--236).

Com 30 anos, Kropotkin irá se declarar anarquista. Assim, participa do
surgimento do movimento em 1872 e publica duas obras muito notáveis,
\emph{A Conquista do pão} e, principalmente, \emph{Mutualismo. }

Ele se torna, em vários países, um dos pensadores mais influentes. Nos
Estados Unidos, por exemplo, inspirou outras grandes figuras da geração
seguinte, como o filósofo da tecnologia Lewis Mumford (1895--1990) ou
Paul Goodman (1911--1972), reputado autor da crítica social. Kropotkin
sugere alternativas ao Estado e, de fato, elas aparecem no mundo
inteiro.

\section{a ajuda mútua e a solidariedade coletivas}

Isso coloca a questão da influência da pandemia sobre a vida social. A
situação humana no mundo é dramática. A democracia nunca virá de um
Estado, mas sim de uma população que tenha compreendido a solidariedade
e que pratique a ajuda mútua. ``O momento presente, pelo surgimento de
uma doença fortemente contagiosa, lembra a fragilidade da vida e o risco
da morte. De um lado, os humanos experimentam a vulnerabilidade que se
expressa através de normas de vida brutalmente acidentadas e
transformadas: a referência ao sujeito soberano parece obsoleta. De
outro lado, a \textsc{covid}"-19 revela que somos interdependentes, contaminantes,
contaminados, contamináveis. Somos todos ligados uns aos outros, e uma
conduta individual pode ter um efeito considerável sobre outros
indivíduos: a autonomia, valor moral por excelência, encontra"-se
comprometida'' (\textsc{brugère} \& \textsc{le blanc}, 2020, p. 78).

Ações de solidariedade proliferaram por toda parte. Confecção de
máscaras para os vizinhos, visitas às pessoas isoladas, cooperações no
nível do bairro se multiplicaram. Sem falar dos milhares de voluntários
que, em cada país, auxiliam o próximo de maneira desinteressada.

Todos esses esforços, se forem discretos, permanecem desconhecidos. Por
que não falar deles e convidá"-los a se federar? Uma sociedade
igualitária é aquela em que cada um dá de acordo com suas aptidões e
seus recursos e recebe segundo suas necessidades.

Há, também, e principalmente, a questão ecológica, a de nossas relações
com a natureza e com os animais, a destruição das diversas espécies
animais, assim como o desmatamento. Podemos contar com os ecologistas
para agarrar essa oportunidade de estabelecer um diagnóstico das
relações mantidas pela humanidade com a natureza. Mas parece que não se
pode questionar nossas sociedades fundadas sobre uma pirâmide de
autoridades, fronteiras que muitas vezes foram criadas por guerras e as
crescentes influências das empresas em expansão.

Não passamos de elementos da natureza, mas nossos cérebros são talvez um
dos corpos mais complexos do cosmos, constituídos por bilhões de
neurônios. Por que não tentar mudar já nosso entorno imediato?

Também há recursos, trocas e gestões pouco visíveis e, no entanto, bem
reais. Nosso patrimônio cultural e milhares de reservas naturais,
criadas no curso de bilhões de anos, são mantidas por populações ou por
comunidades regionais ou locais. Essas coletividades lutam contra as
agressões de uma economia baseada na propriedade. Centenas de ações são
criadas e se mantêm por decisões coletivas. Elas podem ser encontradas
por toda parte.

O confinamento também revelou a inadequação do sistema econômico e suas
mentiras. Quando ele fala de programa social ou de reestruturação, é
preciso escutar demissão de pessoal ou deslocalização\footnote{Criticam"-se
  com razão as \emph{fake news} e as teorias da conspiração, mas
  perguntamos: essas formas de comunicação também não são mantidas por
  uma educação fundada ``naquilo que é preciso saber'', mais do que na
  arte de criticar as fontes de informação? E será que alguns não se
  lançam a isso porque os poderes públicos e as grandes mídias perderam
  a seus olhos qualquer credibilidade?}. Como todos os interessados
notaram, parece que certo número de chefes de Estado, como Trump ou
Bolsonaro, e outros ainda, entrou num concurso para ver quem lançaria
mais \emph{fake news}. Seria esse um sinal de obsolescência da estrutura
estatal? O sistema hierárquico que funda a sociedade data de sete mil
anos. Ele não estaria finalmente se revelando inadequado?

\section{nossos bens essenciais}


\begin{quote}
{[}O confinamento{]} é um bom momento para se refletir sobre a
desvalorização histórica do trabalho da reprodução social. Observamos
repentinamente que o único trabalho realmente necessário é o do cuidado,
da alimentação, da saúde e da coleta de lixo --- trabalho sub"-remunerado
e que foi amplamente feminizado e reduzido ao isolamento em casa. Hoje,
todos nós estamos vivendo assim, e as queixas sobre crianças que
atrapalham trabalhos importantes parecem charmosas e engraçadas até o
momento em que começamos a perguntar por que acreditamos que um trabalho
por dinheiro é mais proveitoso do que cuidar das crianças ou da casa
(\textsc{lance} \& \textsc{jun}, 2020).
\end{quote}

É claro que os cuidados exigem pessoas atentas, a alimentação pressupõe
atividades agrícolas, a saúde impõe a presença de médicos competentes, a
coleta do lixo exige uma repartição desse trabalho entre diversos
voluntários. E não seria preciso reservar um lugar para a inventividade
artesanal, musical, artística e filosófica? Esses bens essenciais
reclamam uma cooperação que se realiza na autonomia. Cada ser é múltiplo
e sonha desenvolver suas diversas possibilidades, em suma, realizar"-se.

Seria necessário seguir o exemplo de solidariedade cotidiana presente
nas miseráveis comunidades que são as favelas brasileiras, apesar ou em
razão mesmo da indiferença a seu respeito. E não é o mesmo que acontece
em certos bairros"-guetos no mundo, sobre os quais só se contam histórias
de droga e confrontos?

Uma outra economia, fundada sobre o dom, está longe de ter desaparecido.
Operárias mal"-remuneradas distribuíram gratuitamente em sua vizinhança
máscaras que elas haviam confeccionado. O mundo conta com dezenas de
milhares de voluntários, repórteres, médicos, enfermeiros e outros. Qual
é o país onde não se praticam trocas espontâneas de bens e de serviços
entre vizinhos, amigos e conhecidos? Quem nunca recorreu à ajuda mútua
para resolver um problema técnico que ia além da própria competência? E
lembremos das nações sem Estado, como as tribos de índios nas Américas
ou os clãs na África.

Não seria possível, no futuro, que todos os humanos fossem forçados a
adotar formas de vida coletivas diferentes das sociedades piramidais nas
quais vivemos? Jérome E. Roos e Léonidas Oikonomakis, do \emph{European
University Institute,} apresentam múltiplos exemplos de movimentos
autônomos, no mundo inteiro. Eles se inspiram em quatro linhas de ação:
1. Mudar o mundo sem tomar o poder, como sugere John Holloway. 2 ``O
horizontalismo'', ou seja, a democracia direta praticada por grandes
grupos de pessoas. 3. A ação direta, criar zonas autônomas temporárias
(\textsc{roos} \& \textsc{oikonomakis}; \textsc{kiersey} \& \textsc{vrasty}).

Todo mundo, ou quase, participou um dia ou outro de economias fundadas
sobre o dom. Vocês nunca viram moradores de rua compartilharem seus
alimentos? Todo esse universo alternativo não mostra a engenhosidade
cordial dos humanos e não é a prova de que outros mundos possíveis já
existem? Eles não prefiguram o futuro, que sempre permanece
desconhecido. Mas oferecem um leque de escolhas oferecidas a populações
em que cada um é diferente. Eles sempre poderão se federar livremente.
Charles Fourrier já o havia antecipado. Será preciso esperar tragédias
coletivas para se desfazer as sociedades piramidais por escolhas
horizontais? Apesar de tudo, hoje estas se multiplicam sob centenas de
formas e meios sociais.


\begin{bibliohedra}
\tit{BONACCORSI}, Giovanni, et al.. Economic and social consequences of human
mobility restrictions under \textsc{covid}"-19. In: \emph{\textsc{pnas}}, junho 2020.
Disponível em: \emph{https://bit.ly/39zhbmG}.

\tit{BRUGÈRE}, Fabienne; \textsc{le blanc}, Guillaume. Le lieu du soin. In:
\emph{Esprit}, maio 2020, p. 77--85.

\tit{DAVID}, Bruno. L'invité(e) des matins: Ce que le Covid fait à la science
(podcast in \emph{France Culture}). Entrevistador: Guillaume Erner,
7/10/2020. Disponível em:
\emph{https://www.franceculture.fr/emissions/linvitee-des-matins/ce-que-le-covid-fait-a-la-science}.
Acesso em 23/09/2020.

\tit{FŒSSEL}, Michaël; \textsc{riquier}, Camille. Déraison de guérir. In:
\emph{Études,} maio 2020, p. 43--53.

\tit{GARCIA}, Renaud. Nature humaine et anarchie: la pensée de Pierre
Kropotkine. Tese (Doutorado em Filosofia) --- École Normale Supérieure
de Lyon, Université de Lyon, École doctorale de philosophie, Lyon, 2012.

\tit{HEISBOURG}, François. La pandémie remodèle"-t"-elle la géopolitique? In:
\emph{Commentaire,} 2020, n. 171, p. 51--536.

\tit{IBÁÑEZ}, Tomás. Ludd, Hypermodernité et néototalitarisme en temps de
\textsc{covid}"-19. In: \emph{Le blog de Floréal}. Disponível em:
\emph{https://florealanar.wordpress.com/2020/05/02/ludd-hypermodernite-et-neototalitarisme-en-temps-de-covid-19/}.
Acesso em: 4/10/2020.

\tit{ITHURBIDE}, Philippe; \textsc{maillard}, Didier. \textsc{covid}"-19: le monde économique
d'après. In: \emph{Commentaire}. Paris, 2020, n. 171, p. 495--506.

\tit{JOUVENTIN}, Pierre. Un Constat d'échec. In: \textsc{latouche}, Serge; \textsc{jouventin},
Pierre et al., \emph{Ce que nous dit la crise du Coronavirus. Regards
croisés sur les responsabilités de notre société}. Libre et solidaire,
2020.

\tit{KIERSEY}, Nicholas; \textsc{vrasty}, Wanda. A Convergent genealogy? Space, time
and the promise of horizontal politics today. In: \emph{Capital \&
Class,} 2016, p. 1--20.

\tit{KROPOTKIN}, Piotr. \emph{Mutual Aid: A Factor of Evolution}. 1902.
Disponível em:
https://theanarchistlibrary.org/library/petr-kropotkin-mutual-aid-a-factor-of-evolution

\tit{LANCE}, Mark; \textsc{jun}, Nathan. Anarchist Responses to a Pandemic: The
\textsc{covid}"-19 Crisis as a Case Study in Mutual Aid. In: \emph{Kennedy
Institut of Ethics Journal.} Washington, \textsc{dc}: Georgetown University,
2020. Disponível em:
https://kiej.georgetown.edu/anarchist-responses-covid-19-special-issue/\#\_ftn7.
Acesso em: 1/10/2020.

\tit{LE ROY LADURIE}, E. Un concept: l'unification microbienne du monde
(\textsc{xiv}\textsuperscript{e}-\textsc{xvii}\textsuperscript{e} siècles). In:
\emph{Revue suisse d'histoire}, 1973, n° 23, p. 627--696.

\tit{MAGNARD}, Camille. Revue de Presse internationale: Vaccin: le français
Sanofi cède au ``America First'' de Donald Trump (podcast in \emph{France
Culture}), 14/5/2020. Disponível em:
https://www.franceculture.fr/emissions/revue-de-presse-internationale/la-revue-de-presse-internationale-emission-du-jeudi-14-mai-2020.
Acesso em: 27/8/2020.

\tit{PATON}, James, \textsc{griffin}, Riley; \textsc{koon}, Cynthia. U.S. Likely to Get Sanofi
Vaccine First If It Succeeds. In: \emph{Bloomberg}, 13/5/2020.
Disponível em:
https://www.bloomberg.com/news/articles/2020-05-13/u-s-to-get-sanofi-covid-vaccine-first-if-it-succeeds-ceo-says.
Acesso em: 27/8/2020.

\tit{PATOU-MATHIS}, Marylène. L'invité des matins: Ce que le Covid fait à la
science. (podcast in \emph{France Culture}). Entrevistador: Guillaume
Erner, 7/10/2020. Disponível em:
\emph{https://www.franceculture.fr/emissions/linvitee-des-matins/ce-que-le-covid-fait-a-la-science}.
Acesso em: 23/09/2020.

\tit{PROMED}. Undiagnosed pneumonia --- China (hubei): Request for
information. In: \emph{ProMed --- International Society for Infectious
Diseases}, 30/12/2019. Disponível em:
https://promedmail.org/promed-post/?id=6864153\%20\#COVID19. Acesso em:
24/9/2020.

\tit{ROOS}, Jérôme; \textsc{oikonomakis}, Leonidas. We are everywhere! In: \textsc{holloway},
John. \emph{Change the World without Taking Power}: \emph{The Meaning of
Revolution Today}. Londres: Pluto Press, 2002. \textsc{saliba}, Adonis.
Sansistema urĝostato kaj demokratia erozio en Brazilo {[}Urgência
sanitária e erosão democrática no Brasil{]}. In: \emph{Esperanto en
Brazilio,} 20/07/2020. Disponível em:
http://esperanto.brazilo.org/archives/16548. Acesso em: 20/10/2020.

\tit{SEIB}, Gerald F. Trump Writes Campaign Script With Three Big Themes. In:
\emph{The Wall Street Journal,} 15/06/2020. Disponível em:
\emph{https://www.wsj.com/articles/trump-writes-campaign-script-with-three-big-themes-11592232042}.
Acesso em: 23/09/2020.

\tit{TRAINAR}, Philippe. Les conséquences économiques du \textsc{covid}-19. In:
\emph{Commentaire}. Paris, 2020, n. 170, p. 255--264.

\tit{ULFIANA}, Ajeng Dinar. Indonezio: perfortaj alfrontiĝojções dum
manifestacioj kontraŭ laborleĝo {[}Indonesia, violentos confrontos por
ocasião de manifestações contra o direito do trabalho{]}. In:
\emph{Esperanto en Brasilio,} 9/10/2020. Disponível em:
http://esperanto.brazilo.org/archives/20141. Acesso em: 11/10/2020.
\end{bibliohedra}

\chapterspecial{Pensamento e práticas libertárias depois da \textsc{covid-19}\footnotemark}{}{Tomás Ibáñez}
\hedramarkboth{Pensamento e práticas libertárias\ldots}{}

\footnotetext{Tradução do espanhol por Beatriz S. Carneiro.}

\noindent{}No meu entender, o olhar libertário sobre a \textsc{covid}"-19 deveria servir
fundamentalmente para extrair lições e elementos de reflexão que
permitam enriquecer e renovar suas próprias abordagens teórico"-práticas.
Não como um mero exercício intelectual, mas sim para tornar mais efetiva
a luta, para impulsionar, ampliar e fortalecer as práticas de liberdade
ou, o que dá no mesmo, a luta contra todos os dispositivos de dominação.

Se a atual pandemia deve ser motivo de preocupação, é obviamente por
suas consequências letais, mas também por outras razões, que veremos ao
longo deste texto, e porque antecipa a sucessão de novos episódios que
envolverão um perigo semelhante, senão maior. Esses novos episódios
fazem parte do nosso futuro pois, embora seja certo que o risco
biológico faça parte da própria condição humana, também é verdade que
sua probabilidade de ocorrer e a magnitude de suas consequências são
aumentadas pelas atuais condições de vida, imputáveis em grande parte,
apesar de não exclusivamente, ao sistema capitalista.

Entre os múltiplos fatores que facilitam a emergência e o
desenvolvimento de pandemias, vale destacar as enormes aglomerações
humanas espremidas em cidades gigantescas; a expansão demográfica
claramente desmesurada; uma globalização que propicia trocas comerciais
constantes e rápidas que cruzam todo o planeta; os meios de transportes,
tão poluentes quanto rápidos e de relativo baixo custo, que favorecem o
deslocamento incessante de milhões de pessoas, por motivos seja de
trabalho, seja de lazer; uma redução brusca dos investimentos em
serviços públicos de saúde; uma degradação ambiental drástica. Ao que
convém ainda acrescentar a existência de extensos bolsões de
insalubridade, desnutrição e precariedade em amplas zonas da Terra.

Por si mesma, essa lista deixa muito evidente que a pandemia de \textsc{covid}"-19
é um fenômeno cuja etiologia, no sentido amplo do termo, é extremamente
diversa, e sabemos bem que, diante de fenômenos complexos, explicações
simples são inaceitáveis, assim como, no caso preciso da \textsc{covid}"-19, é
totalmente equivocada a atribuição de uma causalidade única a um
fenômeno que apresenta múltiplas arestas.

\section{o abraço entre o capitalismo e a pandemia,\break infelizmente, não destrói o capitalismo}

Mesmo sabendo que explicações simples são enganosas, estamos vendo como
uma parte não desprezível do anarquismo militante sucumbiu, em maior ou
menor grau, a essa tendência simplificadora, apontando o capitalismo
como a principal, senão a única, causa dos efeitos da pandemia, por
vezes, apresentando"-o como o principal responsável pela sua origem, ou,
pelo menos, pela sua disseminação e grau de letalidade. Curiosamente, o
fato de que pandemias muito mais letais do que a atual ocorreram em
tempos nos quais ainda não havia capitalismo não parece lançar dúvidas
sobre a relação postulada entre o capitalismo e a pandemia atual. No
entanto, é preciso lembrar, por exemplo, que a Peste Negra, iniciada em
1347, devastou nada menos que um quarto da população europeia em um
período em que o capitalismo ainda era totalmente incipiente e,
portanto, não poderia ser considerado como o fator determinante. Apesar
da comparação ser totalmente surreal devido às diferenças entre os
agentes infecciosos e, sobretudo, pelas enormes e múltiplas diferenças
entre as duas épocas, não obstante, chama a atenção o fato de que seria
preciso multiplicar por mil --- é fácil dizer isso --- o número atual de
mortes por \textsc{covid}"-19 na Europa para se chegar à cifra de 200 milhões de
mortos que corresponderia a um quarto de sua população atual.

Em muitas ocasiões, essa tendência à simplificação suscitou como
corolário a esperança de que a suposta ligação direta entre o
capitalismo e a pandemia provocaria uma intensa tomada de consciência
que, por mero instinto de sobrevivência, ergueria as populações contra o
capitalismo em uma luta radical para substituí"-lo por um sistema
econômico, social e político mais justo. Fica claro, nesse ponto do
desenrolar da situação, que tal esperança não atentou para o fato de que
a pandemia poderia provocar o resultado oposto e direcionar as
populações angustiadas para uma demanda por maior segurança e maior
presença do Estado, impulsionando"-as a buscar refúgio em uma
estabilidade conservadora, avessa a qualquer perspectiva de alteração da
ordem estabelecida.

É normal e, evidentemente, é para ser celebrado que a pandemia aguce o
olhar crítico sobre o capitalismo e seus estragos e intensifique a
consciência de que é indispensável lutar para destruí"-lo; porém, o
desejo de acabar com suas atrocidades não deve obscurecer nossa
capacidade de análise.

Nem o capitalismo pode ser considerado o principal fator dos danos que a
atual pandemia está produzindo, nem cabe pensar que a pandemia promoverá
um intenso ciclo de lutas capaz de transformar o mundo, tampouco é
sensato proclamar que o sistema capitalista está atingido de morte por
essa crise.

Embora possa parecer um paradoxo, resulta que essa forma de ver as
coisas enfraquece as lutas contra o sistema capitalista e suas
estruturas de dominação, retrocedendo"-as a tempos passados e a esquemas
antiquados.

\textls[-5]{As abordagens que se seguiram à grande revolta de maio de 1968
direcionaram as lutas para o desmonte, \textit{na atualidade,} dos
dispositivos de poder/dominação, tanto dos dispositivos diretamente
articulados pelo próprio capitalismo, como, por exemplo, a imposição da
forma e da lógica do mercado para todas as esferas da vida, ou daqueles
dispositivos que simplesmente se mantêm vigentes em seu interior, como o
patriarcado. Essa multiplicação e diversificação das frentes de
resistência e subversão deram início a avanços notáveis para as práticas
de liberdade e para a vida das pessoas, sem esperar pela grande explosão
revolucionária que, por sua própria definição, sempre fica fora do
presente enquanto não tenha acontecido.}

\textls[-10]{A partir da constatação de que nem mesmo a expansão catastrófica da
\textsc{covid}"-19 está produzindo uma revolta geral contra o capitalismo, o
anarquismo deveria tirar uma primeira lição que
consiste na urgência para
continuar multiplicando e diversificando as frentes de luta radical e
revolucionária, tendo o presente como horizonte, em vez de dedicar suas
energias a um remoto ataque final que parece cada vez mais ilusório e
distante.}

\section{a explosão populacional como um\break estímulo à pandemia e ao ecocídio}

Não há dúvida de que a destruição do equilíbrio ecológico e a devastação
dos espaços naturais do planeta têm desempenhado um papel relevante no
surgimento e expansão da pandemia, mostrando que o risco biológico e o
risco ecológico não são independentes um do outro. No entanto, o
enfoque, amplamente favorecido pela grande mídia, nesse inegável risco
ecológico pode nos fazer esquecer o importante papel que o
\textit{crescimento demográfico desenfreado} desempenha no próprio
incremento do risco ecológico. Se for admitido que a deterioração do
ecossistema é um dos fatores que promove o aumento dos riscos
biológicos, e se essa deterioração é função, entre outros fatores, do
aumento demográfico, fica logicamente estabelecida a relação entre
pandemias e expansão demográfica.

\textls[-5]{Quando eu nasci, há\ldots{} apenas! (perdoe"-me a ironia dessa
exclamação, não pude evitar\ldots{}) três quartos de século, aumentei em uma
unidade os 2,5 bilhões de congêneres que habitavam o planeta naquele
tempo. Hoje, esse número cresceu para 7,7 bilhões de pessoas e
continuará a aumentar em cerca de 2 bilhões nos próximos trinta anos,
aproximando"-nos dos 10 bilhões de seres humanos na Terra. Isso significa
que o aumento durante esses trinta anos por si só equivale a quase toda
a população que existia nos anos 1950, resultante do progressivo aumento
demográfico ao longo dos milhares de anos decorridos desde o início da
humanidade. Parece incrível, mas levará apenas trinta anos para produzir
o mesmo aumento demográfico que a humanidade tinha produzido ao longo de
uma existência milenar.}

\textls[-15]{Diante desses dados, é difícil entender por que o crescimento
demográfico não suscita tanto medo nem instila tanta preocupação, e não
estimula tantas consciências quanto o risco ecológico; especialmente se
levarmos em conta o fato de que, em nosso atual sistema de produção,
\textit{o crescimento populacional desencadeia inevitavelmente o próprio
risco ecológico,} por razões óbvias.}

Sem dúvida, devem ser numerosos e muito poderosos os interesses
econômicos e as crenças atávicas e religiosas que impedem alertar sobre
os riscos do aumento demográfico com a mesma contundência com que se
questiona a degradação ambiental.

Diante dessa resistência, é sabido que um setor do movimento anarquista
defendeu, historicamente, certas teses eugenistas e enfatizou o
necessário e autorresponsável controle da natalidade. Nas circunstâncias
atuais, quando, de um lado, o efeito conjunto do aumento demográfico, de
outro lado, a concentração populacional e, em terceiro lugar, os grandes
fluxos migratórios, que já começaram e que aumentarão rapidamente no
futuro próximo, pressagiam o aumento do risco biológico, parece"-nos que,
sensibilizado pela \textsc{covid}"-19, o movimento anarquista deveria retomar,
renovando e intensificando, o trabalho de conscientização eugenista,
evitando, é claro, os desvios transumanistas, e focando na procriação
consciente e exclusivamente voluntária, mas também no \textit{perigo que
o aumento da população acarreta}.

\textls[-15]{Essa tarefa implica, entre outras coisas, acentuar ainda mais o já
considerável e necessário envolvimento do anarquismo no movimento
feminista e desenvolver uma atividade de conscientização transgênero
dirigida a homens e mulheres do ponto de vista do \textit{feminismo
ácrata} que, sem desqualificar a maternidade, destaque suas repercussões
sociais e políticas mais perniciosas.}

\section{o biopoder e a medicalização da\break existência no espelho da \textsc{covid}"-19}

Sabendo o que é a quintessência do pensamento, da sensibilidade e das
práticas libertárias, é óbvio que a atenção dada aos dispositivos e
mecanismos de poder"-dominação precisa ser colocada no mais alto nível. É
por isso que o olhar anarquista não poderia deixar de perceber que a
atual pandemia ilustra de forma espetacular o sucesso que Michel
Foucault teve ao desenvolver, há pouco mais de quarenta anos, seu
conceito de \textit{biopoder} para caracterizar a nova forma de
governamentalidade articulada pelo neoliberalismo. Sem dúvida, algumas
das novas modalidades de exercício de poder a que ele se referiu na
época, como a gestão da vida, a biossegurança e o controle das
populações que representam, passaram a ocupar um lugar preferencial na
agenda do capitalismo digital neoliberal próprio de nosso tempo.

Hoje, o exercício do poder"-dominação deslocou"-se do modelo tradicional
da lei e da sanção, ou seja, de um modelo baseado principalmente no
dever, na punição e na força, para um modelo baseado na \textit{gestão
da vida} e no \textit{controle produtivo e normalizador das populações}.
O biopoder coloca \textit{a vida} no centro dos procedimentos de poder,
tornando o cuidado e a gestão dela poderosas fontes de recursos para
promover a livre submissão dos sujeitos e para controlar e administrar
as populações.

Com as ferramentas proporcionadas pela revolução da informação, que
permitem ir além da própria biopolítica e implantar uma
\textit{biopolítica digital}, verifica"-se que o extraordinário
desenvolvimento da medicalização da vida e a desmedida importância
adquirida pelo lucrativo \textit{complexo técnico"-médico} que integra
tanto a dinâmica indústria"-farmacológica quanto os onerosos instrumentos
diagnósticos e cirúrgicos, cuja renovação deve ser tão rápida e
constante quanto conveniente para a indústria médica, mostram"-se
fundamentais para, entre outras coisas, fazer recair sobre o sujeito a
responsabilidade de cuidar da própria saúde, assim como a dos demais,
mediante uma tríplice faceta: por um lado, o autocontrole, por outro
lado, a vigilância contínua que deve exercer sobre as pessoas ao seu
redor e, em terceiro lugar, o olhar vigilante pelo qual é observado
pelos outros.

A lista de condutas saudáveis converteu"-se no breviário que cada pessoa
deve interiorizar e respeitar, não só para preservar a sua saúde, mas
também para preservar a saúde dos seus concidadãos, e com o qual o
sentimento de culpa por negligenciar a própria saúde se multiplica. Está
claro que infundir preocupação com os perigos que ameaçam a saúde,
incitar o medo e promover a autoculpabilização são algumas das
ferramentas úteis para o exercício do biopoder. E resulta que a gestão
da atual pandemia mostra que essas ferramentas funcionam à perfeição,
encurralando e enfraquecendo, sem ter que exercer uma repressão notável,
as veleidades de infringir as pautas traçadas e impostas pelas
instituições.

Além disso, a pandemia está servindo como um \textit{grande banco de
ensaios} para experimentar procedimentos de controle massivo das
populações por meio de, entre outras coisas, coleta de seus dados em
massa, desenvolvimento de conhecimento especializado sobre suas
características e dinâmica, bem como do grau em que aceitam ser
submetidas sem opor demasiada resistência ou, inclusive, se ofereçam
elas mesmas para serem dirigidas de forma ainda mais rigorosa, vigiadas
ainda mais minuciosamente e punidas com mais severidade ainda (para o
seu próprio bem, claro\ldots{}).

Embora faça lustros que o anarquismo deveria já ter incorporado de uma
forma muito mais decisiva as novas concepções sobre as relações de poder
elaboradas fundamentalmente por Foucault, parece claro que a \textsc{covid}"-19
fornece novos argumentos para que o pensamento libertário \textit{renove
e enriqueça sua análise crítica do poder}, incorporando plenamente a
reflexão sobre o biopoder.

\section{o fulgurante avanço do totalitarismo de um novo tipo}

Byun"-Chul Han, o pensador norte"-coreano radicado na Alemanha, alertou
recentemente que, além de virologistas e epidemiologistas, são sobretudo
os cientistas da computação e especialistas em \emph{Big Data} que lutam
contra as pandemias. A \textsc{covid}"-19 não demorou muito para tornar isso
evidente, mas também estimulou o desenvolvimento de sofisticadas medidas
de controle social graças à demanda por biossegurança gerada pelo medo
da população diante do risco biológico.

Independentemente de as unidades de saúde e de os cuidados médicos serem
infinitamente superiores aos que havia quando ocorreram as pandemias dos
séculos anteriores, as similitudes dos modelos implantados para impedir
sua disseminação não deixam de ser impressionantes. Por exemplo, durante
a Peste Negra, que fustigou a Europa no final da Idade Média, as pessoas
buscavam localizar os infectados; confinavam"-nos em suas casas com uma
proibição estrita de sair; marcavam as residências deles para que
ninguém se aproximasse; aumentavam a vigilância para detectar novos
casos; desinfetavam as casas (claro que as queimando, ao contrário de
hoje, mas apenas porque era o melhor desinfetante disponível); zonas
inteiras das aldeias eram isoladas e, às vezes, a totalidade de uma
aldeia, impedindo a entrada e saída; todas as atividades nas áreas
infectadas eram suspensas, etc. É surpreendente que, tanto ontem como
hoje, os princípios básicos de contenção da pandemia conformam um modelo
muito semelhante. No entanto, também uma diferença capital é avaliada no
que se refere às modalidades de vigilância, bem como em relação à coleta
e processamento das informações. Obviamente, essa diferença,
propriamente abismal, deve"-se basicamente aos instrumentos
proporcionados pela \textit{revolução digital.}

Não é necessário detalhar aqui o uso que está sendo feito das novas
tecnologias digitais no âmbito da \textsc{covid}"-19, a mídia frequentemente as
menciona; no entanto, na medida em que essa pandemia fornece combustível
abundante para acelerar o desenvolvimento dos mais sofisticados
instrumentos de controle social, vale a pena nos deter no que a \textsc{covid}"-19
está ajudando a implementar agora, mas que vem sendo gestado há muito
tempo, graças à revolução digital.

Essa revolução reforçou ainda mais o estreito vínculo, típico da
modernidade, entre, de um lado, a razão científica, de outro lado, as
tecnologias e, em terceiro, o poder econômico e político. O resultado
foi a transformação do capitalismo, que agora se tornou um capitalismo
digital dotado de uma estrutura sofisticada de vigilância e de captação
e processamento de dados que não dizem respeito apenas a indivíduos e
grupos, mas também a todos os processos e atividades que acontecem no
espaço social. Essa nova forma de capitalismo está avançando rapidamente
na esfera política para um \textit{totalitarismo de um novo tipo} que já
mostra suas presas nos cinco continentes.

Agora, ao contrário dos regimes totalitários anteriores, são os próprios
sujeitos que proporcionam constantemente, mediante todos e cada um de
seus comportamentos, sistematicamente, coletados e processados por
sofisticados algoritmos, os elementos que permitem uma sujeição ainda
mais integral, visto que é a própria vida de milhares de pessoas que
alimenta os dispositivos de controle e normalização. Resulta, ainda, que
o capitalismo digital não se satisfaz em aproveitar a \textsc{covid}"-19 para
refinar e ampliar os dispositivos de controle social, senão que também
dela se aproveita para modificar o âmbito trabalhista, promovendo o
\textit{teletrabalho} com uma intensidade nunca vista anteriormente.
Além de isolar fisicamente os trabalhadores e evitar qualquer relação
que não seja puramente profissional, essa reestruturação do trabalho,
também, expande o instrumento digital em todo o tecido social,
tornando"-o totalmente \textit{imprescindível} e, dessa forma,
assegurando a possibilidade de um controle constante e detalhado dos
trabalhadores.

Não se trata aqui de esboçar uma distopia de feitio orwelliano, mas
basta pensar, por exemplo, que mesmo as máscaras não são um obstáculo
para que milhões de rostos por segundo possam ser identificados em
manifestações e aglomerações. O controle policial por meio do
reconhecimento facial requer que agentes equipados com óculos providos
de um \emph{hardware} de realidade aumentada enviem dados a um centro de
controle e recebam as informações e as instruções pertinentes quase
instantaneamente graças às \textit{redes 5G.} É óbvio que, se esse novo
tipo de totalitarismo conseguir enraizar"-se, as possibilidades de luta e
resistência contra a dominação e a exploração serão anuladas ou
reduzidas à insignificância.

A \textsc{covid}"-19 tem tornado mais evidente, aos olhos das pessoas, a
sofisticação das medidas de controle que estão nas mãos do Estado e que
vão continuar a ser aperfeiçoadas em ritmo acelerado; urge, portanto,
que o anarquismo tome ciência desse fato e não perca a oportunidade,
nesse momento, de enfatizar a ameaça que muitas pessoas têm percebido
mais ou menos claramente na raiz da pandemia.
Estou convencido de que o
anarquismo deveria se apressar a colocar em um lugar preponderante de
sua agenda a necessidade de lutar por todos os meios contra o
totalitarismo de novo tipo que paira sobre a humanidade.

Hoje, torna"-se imprescindível reinventar o tipo de revolta que os
Luditas protagonizaram, quando, no século 19, destruíram parte da nova
maquinaria têxtil, cuja instalação na Inglaterra estava eliminando
empregos e condenando parte da população à miséria. Entre as práticas de
resistência que o anarquismo deveria incentivar estão, por exemplo, as
práticas dos \emph{hackers}; a sabotagem do 5G, como está acontecendo na
Inglaterra; o incitamento a prescindir ao máximo do uso do celular e da
intervenção nas redes sociais; a criação de oficinas de defesa contra
vigilância computacional, etc.

Junto ao desenvolvimento de práticas de combate aos sistemas de controle
digital, os quais, na sua dimensão policial, atenazam as veleidades
subversivas e, na sua dimensão econômica, garantem o lucro das grandes
plataformas globais graças às informações que lhes prestamos, é
imprescindível realizar uma ampla campanha de sensibilização para a
grande ameaça que representa o novo tipo de totalitarismo e desmantelar,
na imaginação das pessoas, os argumentos que procuram legitimá"-lo a
partir do medo suscitado pela \textsc{covid}"-19 e por futuras pandemias.

\section{preservar a subversão em tempos adversos}

Em situações extremas, como as causadas por terremotos, grandes
inundações, tsunamis, erupções vulcânicas, etc., costuma ocorrer que
\textit{iniciativas populares auto"-organizadas} se antecipem e
substituam as medidas governamentais. Não obstante, uma situação como a
criada pela \textsc{covid}"-19 parecia impossibilitar totalmente esse tipo de
iniciativa popular, devido ao medo do contágio e ao férreo isolamento
imposto à população. Contra toda previsão, essa impossibilidade foi
posta em xeque, embora seja verdade que as iniciativas populares tiveram
um alcance muito menor do que o obtido em outros tipos de situações
extremas.

Não se pode esquecer que as fases mais difíceis das medidas decretadas
para deter a pandemia foram semelhantes, pelo menos na Espanha, às que
se tomam quando o estado de sítio é proclamado: proibição de reuniões,
de aglomerações e de manifestações, a imposição de um confinamento
estrito que impediu até mesmo de ir às casas de parentes e amigos, etc..
Apesar disso, foram surgindo focos de resistência espontânea, e
desenvolveu"-se uma dinâmica de auto"-organização e solidariedade social
que não deixava de evocar as considerações de Kropotkin sobre o
\textit{apoio mútuo} e incutir um certo otimismo quanto à capacidade de
reação da população. Assim, surgiram brigadas de solidariedade popular
animadas por coletivos dispostos a fornecer alimentos, cuidados e todo
tipo de auxílio material e psicológico aos mais necessitados, redes de
autodefesa de saúde, coletivos de bairro que ousaram realizar saídas
clandestinas para encher as paredes e os muros de grafites, de frases
pichadas e de pasquins denunciando, por exemplo, as consequências letais
dos cortes na saúde. E, ao mesmo tempo, lançou"-se mão das tecnologias
digitais para formar grupos de discussão e de troca de informações, a
fim de manter aberta a capacidade de análise crítica da situação e de
formular propostas para que a pandemia não arrasasse a atividade
política de caráter antagônico.

Além dessas iniciativas, geralmente localizadas nos setores mais
politizados e militantes, também era produzida de forma espontânea, em
determinados locais, uma reação da vizinhança contra o isolamento
claustrofóbico mediante à comunicação com os moradores mais próximos,
seja no bloco de suas próprias residências, seja com os blocos
adjacentes, se houvesse varandas disponíveis. Dessa forma, presenciou"-se
uma espécie de descoberta repentina de que pessoas que habitavam o andar
contíguo, até então desconhecidas, também existiam.

Portanto, embora tenha provocado a manifestação de reações nada
solidárias, como uma hostilidade que podia chegar à denúncia de quem não
demonstrava atitude ou comportamento suficientemente submisso, a
\textsc{covid}"-19 também pôs em evidência fontes de solidariedade e resistências
que surgiram basicamente das relações interpessoais e de pequenos grupos
previamente existentes. Essa circunstância sugere que o anarquismo
deveria voltar a colocar em primeiro plano a criação de \textit{vínculos
de afinidade,} que são os que permitem manter pequenos núcleos de trocas
e relações impregnadas de \textit{confiança mútua}. São esses núcleos de
afinidade que podem assegurar a sobrevivência de projetos e práticas de
combate quando as condições se tornam mais adversas. Isso indica a
importância de multiplicar, no tecido social, a inserção de tantos
núcleos impregnados de sensibilidade insubordinada e ação subversiva
quanto possível, em vez de apostar tudo na criação de organizações
extensas.

Além disso, o anarquismo deveria aumentar a importância que reveste a
atuação no âmbito populacional mais próximo, ou seja, no bairro onde as
pessoas moram, na rua onde vivem, no prédio onde habitam. A criação de
vínculos de afinidade em espaços geograficamente próximos é, entre
outras coisas, a melhor forma de manter a capacidade de resistência em
situações extremas e em momentos quando as comunicações eletrônicas são
neutralizadas ou bloqueadas pelos poderes.

Por fim, e para concluir, está claro que a capacidade de interferir na
realidade depende do grau em que nossa maneira de a entender captura
efetivamente suas características e do grau em que nossas ações têm a
capacidade de afetar tais características. A \textsc{covid}"-19 trouxe à tona, ou
deu maior visibilidade, a uma série de aspectos da realidade atual cuja
análise deveria entrar no caderno de bordo do anarquismo para encarar as
ações pertinentes nos tempos da pós"-\textsc{covid}"-19.

%\emph{Setembro de 2020}

%\textbf{Acácio Augusto} é pesquisador no Nu"-Sol (Núcleo de Sociabilidade
%Libertária) na \textsc{puc"-sp}, professor no Departamento de Relações
%Internacionais da \textsc{unifesp}, onde coordena o \textsc{lasi}nTec (Laboratório de
%Análise em Segurança Interacional de Tecnologias de Monitoramento) e no
%Programa de Pós"-Graduação em Psicologia Institucional da \textsc{ufes}. Contato:
%acacio.augusto@unifesp.br.

%\textbf{Adriana Martinez} é Doutora em Ciências Sociais pela \textsc{puc"-sp.}
%Contato:
%\href{mailto:drimartinez.5@gmail.com}{\nolinkurl{drimartinez.5@gmail.com}}
%
%\textbf{Allan Antliff} é professor Titular de História da Arte na
%University of Victoria, no Canadá. Autor de \emph{Anarchy and Art: From
%the Paris Commune to the Fall of the Berlin Wall} (Arsenal Pulp Press,
%2007/Trad. português: Madras, 2009); \emph{Joseph Beuys} (Londres:
%Phaidon Press, 2014), entre outros. Contato:
%\href{mailto:allan@uvic.ca}{\nolinkurl{allan@uvic.ca}}
%
%\textbf{André Liohn} é fotógrafo. Publica em jornais e revistas ao redor
%do planeta. Premiado com a medalha Robert Capa, o Prix Bayeux"-Calvados
%des Correspondants de Guerre, o Picture of the year International, entre
%outros. Em 2016, foi indicado para o prêmio Jabuti pelo livro
%\emph{Correspondente de Guerra}, escrito em parceria com Diogo Schelp.
%Contato:
%\href{mailto:andreliohn1@gmail.com}{\nolinkurl{andreliohn1@gmail.com}}
%
%\textbf{Claire Auzias} é uma historiadora de história social
%contemporânea hoje aposentada, e socióloga que dedicou muitos anos
%especialmente aos Roms e aos Ciganos da Europa e, particularmente, junto
%ao Socius, do \textsc{iseg} de Lisboa, em 2009--2012. Anarquista individualista e
%autora de vários livros em língua francesa. Vive atualmente em Paris..
%Contato:
%\href{mailto:claireauzias@gmail.com}{\nolinkurl{claireauzias@gmail.com}}


%\textbf{Eliane Carvalho} é pesquisadora no Nu"-Sol e doutora em Ciências
%Sociais pela \textsc{puc"-sp}. Contato:
%\href{mailto:eliane@riseup.net}{\nolinkurl{eliane@riseup.net}}.
%
%\textbf{Flávia Lucchesi} é pesquisadora no Nu"-Sol e doutoranda no
%Programa de Estudos Pós"-Graduados em Ciências Sociais da \textsc{puc"-sp.}
%Contato:
%\href{mailto:flalucchesi@gmail.com}{\nolinkurl{flalucchesi@gmail.com}}
%
%\textbf{Gustavo Ramus} é mestre em Ciências Sociais pela \textsc{puc"-sp.}
%Atualmente é músico e pertence às bandas Trupe Chá de Boldo e Fera
%Neném. Contato:
%\href{mailto:gustavoramus@gmail.com}{\nolinkurl{gustavoramus@gmail.com}}
%
%\textbf{Ilana Viana do Amaral} é inimiga da ordem atual das coisas,
%insurrecionalista, amante, mãe, confuseira, professora de Filosofia na
%Universidade Estadual do Ceará, associada em formação permanente do
%Corpo Freudiano, Escola de Psicanálise -- Seção Fortaleza e faz,
%atualmente, pós"-doutoramento na Universidade de Lisboa. Contato:
%\href{mailto:ilana.amaral@bol.com.br}{\nolinkurl{ilana.amaral@bol.com.br}}



%\textbf{L.I.M.A. ---} O Laboratório Insurgente de Maquinarias
%Anarquistas é um coletivo que ocupa a Faculdade de Educação da
%Universidade Estadual de Campinas desde 2019. O texto desta coletânea é
%uma experimentação de escrita coletiva maquinada por alguns de seus
%membros: Olivia Pires Coelho (doutoranda na \textsc{fe}"-Unicamp); Rafael
%Limongelli (doutorando na \textsc{fe}"-Unicamp); Renato Mendes (mestrando na
%\textsc{fe}"-Unicamp) e Sílvio Gallo (professor titular da \textsc{fe}"-Unicamp). Contato:
%\href{mailto:gallo@unicamp.br}{\nolinkurl{gallo@unicamp.br}}
%
%\textbf{Marco Antônio Arantes} é professor associado do curso de
%Ciências Sociais da Unioeste --- Campus de Toledo/\textsc{pr}. Pesquisa relações
%entre a literatura, o teatro e as Ciências Sociais. Contato:
%\href{mailto:marcoaarantes@uol.com.br}{\nolinkurl{marcoaarantes@uol.com.br}}
%
%\textbf{Ronald Creagh} é sociólogo franco"-britânico; escreveu
%especialmente sobre comunidades utópicas e anarquistas nos Estados
%Unidos. Seu último livro é \emph{Os Estados Unidos de Élisée Reclus}
%(Lyon: Atelier de Création Libertaire, 2019). Contribui para a edição da
%revista anarquista semestral \emph{Réfractions}. Contato:
%\href{mailto:ronaldcreagh@orange.fr}{\nolinkurl{ronaldcreagh@orange.fr}}
%
%\textbf{Tomás Ibáñez}, ativo no anarquismo francês em \emph{maio 68}, na
%luta libertária contra a ditadura franquista e na reconstrução da \textsc{cnt}, é
%autor de vários livros, traduzidos em línguas diversas, dos quais
%\emph{Anarquismo é movimento} foi publicado no Brasil. Atualmente é
%membro dos comitês editoriais das revistas \emph{Réfractions} (França) e
%\emph{Libre Pensamiento} (Espanha). Contato:
%\href{mailto:toms.ibez@gmail.com}{\nolinkurl{toms.ibez@gmail.com}}
