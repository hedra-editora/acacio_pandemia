\chapter{Sobre os autores}

\bigskip
\noindent\textbf{Acácio Augusto} é pesquisador no Nu"-Sol (Núcleo de Sociabilidade
Libertária) na \textsc{puc"-sp}, professor no Departamento de Relações
Internacionais da \textsc{unifesp}, onde coordena o \textsc{lasi}nTec (Laboratório de
Análise em Segurança Interacional de Tecnologias de Monitoramento) e no
Programa de Pós"-Graduação em Psicologia Institucional da \textsc{ufes}. Contato:
acacio.augusto@unifesp.br.

\bigskip
\noindent\textbf{Adriana Martinez} é Doutora em Ciências Sociais pela \textsc{puc"-sp.}
Contato: drimartinez.5@gmail.com.

\bigskip
\noindent\textbf{Allan Antliff} é professor Titular de História da Arte na
University of Victoria, no Canadá. Autor de \emph{Anarchy and Art: From
the Paris Commune to the Fall of the Berlin Wall} (Arsenal Pulp Press,
2007/Trad. português: Madras, 2009); \emph{Joseph Beuys} (Londres:
Phaidon Press, 2014), entre outros. Contato:
allan@uvic.ca.

\bigskip
\noindent\textbf{André Liohn} é fotógrafo. Publica em jornais e revistas ao redor
do planeta. Premiado com a medalha Robert Capa, o Prix Bayeux"-Calvados
des Correspondants de Guerre, o Picture of the year International, entre
outros. Em 2016, foi indicado para o prêmio Jabuti pelo livro
\emph{Correspondente de Guerra}, escrito em parceria com Diogo Schelp.
Contato: andreliohn1@gmail.com.

\bigskip
\noindent\textbf{Claire Auzias} é uma historiadora de história social
contemporânea hoje aposentada, e socióloga que dedicou muitos anos
especialmente aos Roms e aos Ciganos da Europa e, particularmente, junto
ao Socius, do \textsc{iseg} de Lisboa, em 2009--2012. Anarquista individualista e
autora de vários livros em língua francesa. Vive atualmente em Paris.
Contato: claireauzias@gmail.com.


\bigskip
\noindent\textbf{Eliane Carvalho} é pesquisadora no Nu"-Sol e doutora em Ciências
Sociais pela \textsc{puc"-sp}. Contato: eliane@riseup.net.

\bigskip
\noindent\textbf{Flávia Lucchesi} é pesquisadora no Nu"-Sol e doutoranda no
Programa de Estudos Pós"-Graduados em Ciências Sociais da \textsc{puc"-sp.}
Contato: flalucchesi@gmail.com.

\bigskip
\noindent\textbf{Gustavo Ramus} é mestre em Ciências Sociais pela \textsc{puc"-sp.}
Atualmente é músico e pertence às bandas Trupe Chá de Boldo e Fera
Neném. Contato: gustavoramus@gmail.com.

\bigskip
\noindent\textbf{Ilana Viana do Amaral} é inimiga da ordem atual das coisas,
insurrecionalista, amante, mãe, confuseira, professora de Filosofia na
Universidade Estadual do Ceará, associada em formação permanente do
Corpo Freudiano, Escola de Psicanálise -- Seção Fortaleza e faz,
atualmente, pós"-doutoramento na Universidade de Lisboa. Contato:
ilana.amaral@bol.com.br.



\bigskip
\noindent\textbf{\textsc{l.i.m.a.}} --- O Laboratório Insurgente de Maquinarias
Anarquistas é um coletivo que ocupa a Faculdade de Educação da
Universidade Estadual de Campinas desde 2019. O texto desta coletânea é
uma experimentação de escrita coletiva maquinada por alguns de seus
membros: Olivia Pires Coelho (doutoranda na \textsc{fe}"-Unicamp); Rafael
Limongelli (doutorando na \textsc{fe}"-Unicamp); Renato Mendes (mestrando na
\textsc{fe}"-Unicamp) e Sílvio Gallo (professor titular da \textsc{fe}"-Unicamp). Contato:
gallo@unicamp.br.

\bigskip
\noindent\textbf{Marco Antônio Arantes} é professor associado do curso de
Ciências Sociais da Unioeste --- Campus de Toledo/\textsc{pr}. Pesquisa relações
entre a literatura, o teatro e as Ciências Sociais. Contato:
marcoaarantes@uol.com.br.

\bigskip
\noindent\textbf{Ronald Creagh} é sociólogo franco"-britânico; escreveu
especialmente sobre comunidades utópicas e anarquistas nos Estados
Unidos. Seu último livro é \emph{Os Estados Unidos de Élisée Reclus}
(Lyon: Atelier de Création Libertaire, 2019). Contribui para a edição da
revista anarquista semestral \emph{Réfractions}. Contato:
ronaldcreagh@orange.fr.

\bigskip
\noindent\textbf{Tomás Ibáñez}, ativo no anarquismo francês em \emph{maio 68}, na
luta libertária contra a ditadura franquista e na reconstrução da \textsc{cnt}, é
autor de vários livros, traduzidos em línguas diversas, dos quais
\emph{Anarquismo é movimento} foi publicado no Brasil. Atualmente é
membro dos comitês editoriais das revistas \emph{Réfractions} (França) e
\emph{Libre Pensamiento} (Espanha). Contato:
toms.ibez@gmail.com.



