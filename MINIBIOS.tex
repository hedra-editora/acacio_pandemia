\chapter{Sobre os autores}

\textsc{acácio augusto} é professor no curso de Relações Internacionais da
  Universidade Federal de São Paulo (\versal{EPPEN-UNIFESP}) e coordenador do
  \versal{LASI}n\versal{T}ec/\versal{UNIFESP} (Laboratório de Análise em Segurança Internacional e
  Tecnologias de monitoramento). É pesquisador no Nu"-Sol
  (Núcleo de Sociabilidade Libertária) e autor de
  \emph{Anarquia y lucha antipolítica -- ayer y hoy} (Barcelona:
    NoLibros, 2019), \emph{Política e polícia: cuidados, controles e
  penalizações de jovens} (Rio de Janeiro: Lamparina, 2013), dentre
  outros.

\bigskip
\noindent\textsc{adriana f. martinez} é graduada em ciências sociais pela \textsc{puc-sp}, com mestrado em Integração da América Latina pela Universidade de São Paulo (\textsc{usp}) e doutoranda em ciências sociais pela \textsc{puc-sp}.

\bigskip
\noindent\textsc{allan antliff} é professor pesquisador de história da arte na Universidade de Victoria (Canadá), diretor do arquivo anarquista nessa mesma instituição e um dos fundadores da \textit{Toronto Anarchist Free School}.

\bigskip
\noindent\textsc{andré liohn}

\bigskip
\noindent\textsc{claire auzias} é historiadora, feminista e libertária francesa. Na década de 1980 defendeu seu doutorador sobre a história oral dos movimentos libertários na Universidade de Lyon e, desde então, ensina história e sociologia nessa instituição.

\bigskip
\noindent\textsc{eliane carvalho}

\bigskip
\noindent\textsc{flávia lucchesi} possui graduação e mestrado em ciências sociais pela \textsc{puc-sp} e atualmente desenvolve seu doutorado em ciências sociais pela mesma instituição com o projeto ``Queer: práticas e embates libertários''.

\bigskip
\noindent\textsc{gustavo ramus} é mestre em ciências sociais pela \textsc{puc-sp} e é pesquisador do  Núcleo de sociabilidade libertária (Nu-Sol).

\bigskip
\noindent\textsc{ilana viana do amaral} é doutora em filosofia pela \textsc{puc-sp} e professora efetiva da Universidade Estadual do Ceará (\textsc{uece}), onde leciona nos cursos de graduação e mestrado. Coordena o \textsc{pet}-Humanas da \textsc{uece} com o projeto: ``Cultura contemporânea em uma perspectiva inter (ou) trans disciplinar''.

\bigskip
\noindent\textsc{l.i.m.a.}

\bigskip
\noindent\textsc{marco antônio arantes} possui graduação em ciências sociais pela Universidade Estadual Paulista (\textsc{unesp}), mestrado em ciência política pela \textsc{puc-sp} e doutorado em ciência política pela \textsc{puc-sp}.
Atualmente é professor associado do Curso de Ciências Sociais da Universidade Estadual do Oeste do Paraná (\textsc{unioeste}).

\bigskip
\noindent\textsc{ronald creagh} é sociólogo e historiador francês. Com mais de nove livros publicados sobre anarquismo, experiências libertárias e comunidades intencionais, é professor de Civilização Americana na Universidade de Montpellier, França.

\bigskip
\noindent\textsc{tomás ibáñez} é psicólogo e teórico do anarquismo espanhol. Filho de anarquistas exilados na França, participou das revoltas de maio de 1968 e, entre as décadas de 1960 e 1980, participou de movimentos contra o franquismo e dedicou"-se à reconstrução da Confederação Nacional do Trabalho (\textsc{cnt}).