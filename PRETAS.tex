
\textbf{Pandemia e anarquia} reúne quinze ensaios de teóricos e pesquisadores do anarquismo e das práticas libertárias que refletem sobre as implicações sociopolíticas do coronavírus e sua relação com novos modos de existência. Além de pensadores da Somaterapia e do Nu"-Sol (Núcleo de Sociabilidade Libertária), participam da obra historiadores e cientistas políticos franceses, portugueses e espanhóis. Perpassando diversas esferas das relações humanas, da economia e da ciência às relações amorosas e ao ser criança durante a pandemia, os escritos deste livro insurgem"-se contra a falsa ruptura com o mundo dado antes da Covid"-19 --- com a pretensa reinvenção de estruturas hierárquicas e autoritárias que, ao final, continuam em ação --- para criticar a própria racionalidade liberal e, dessa crise, vislumbrar as
linhas de fuga para uma nova configuração planetária.


\textbf{Edson Passetti} é professor livre"-docente no Departamento e no
Programa de Estudos Pós"-Graduados em Ciências Sociais e coordena o
Nu"-Sol na \textsc{puc"-sp}/ Brasil; e é membro da editoria de \emph{verve,} revista autogestionária semestral. Contato: edson.passetti@uol.com.br.

\textbf{João da Mata} é psicólogo, doutor em Psicologia (\textsc{uff}); doutor.
em Sociologia Econômica e das Organizações (Universidade de Lisboa) e
pós"-doutor em História (\textsc{uff}). Trabalha com a Soma --- uma terapia
anarquista há trinta anos, em grupos no Brasil e no exterior. Para saber
mais sobre a Soma:
www.somaterapia.com.br. Contato:
jodamata@hotmail.com.

\textbf{José Maria Carvalho Ferreira} é sociólogo e professor
catedrático aposentado do \textsc{iseg} --- Universidade Técnica de Lisboa,
atualmente, integrada na Universidade de Lisboa. Tem publicado vários
livros e artigos em revistas e editoras nacionais e internacionais. Foi
diretor do jornal \emph{A Batalha}, diretor da revista \emph{Utopia} e
membro da Associação Cultural A Vida. Contato:
jmcf@iseg.ulisboa.pt.


