
\textbf{Pandemia e anarquia} reúne quinze ensaios de teóricos e pesquisadores do anarquismo e das práticas libertárias que refletem sobre as implicações sociopolíticas do coronavírus e sua relação com novos modos de existência. Além de pensadores da Somaterapia e do Nu-Sol (Núcleo de Sociabilidade Libertária), participam da obra historiadores e cientistas políticos franceses, portugueses e espanhóis. Confrontando a categoria do ``novo normal'', que pretende reinventar o cotidiano tal como estava dado antes da pandemia --- com suas estruturas hierárquicas e autoritárias, a conformidade da gestão governamental e ambiental ---, os escritos desse livro insurgem"-se contra a própria racionalidade liberal e perscrutam, na crise atual, as linhas de fuga para uma nova configuração planetária.


\textbf{Edson Passetti} é professor livre-docente no Departamento e no
Programa de Estudos Pós-Graduados em Ciências Sociais e coordena o
Nu-Sol na PUC-SP/ Brasil; e é membro da editoria de \emph{verve,} revista autogestionária semestral. Contato: edson.passetti@uol.com.br.

\textbf{João da Mata} é psicólogo, doutor em Psicologia (UFF); doutor.
em Sociologia Econômica e das Organizações (Universidade de Lisboa) e
pós-doutor em História (UFF). Trabalha com a Soma --- uma terapia
anarquista há trinta anos, em grupos no Brasil e no exterior. Para saber
mais sobre a Soma:
www.somaterapia.com.br. Contato:
jodamata@hotmail.com.

\textbf{José Maria Carvalho Ferreira} é sociólogo e professor
catedrático aposentado do ISEG --- Universidade Técnica de Lisboa,
atualmente, integrada na Universidade de Lisboa. Tem publicado vários
livros e artigos em revistas e editoras nacionais e internacionais. Foi
diretor do jornal \emph{A Batalha}, diretor da revista \emph{Utopia} e
membro da Associação Cultural A Vida. Contato:
jmcf@iseg.ulisboa.pt.


